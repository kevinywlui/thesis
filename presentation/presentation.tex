\documentclass{beamer}

\usepackage{amsthm}
\newtheorem{proposition}[theorem]{Proposition}
\newtheorem{conjecture}[theorem]{Conjecture}
\newcommand{\QQ}{\mathbf{Q}}
\newcommand{\QQbar}{\overline{\mathbf{Q}}}
\newcommand{\RR}{\mathbf{R}}
\newcommand{\FF}{\mathbf{F}}
\newcommand{\TT}{\mathbf{T}}
\newcommand{\F}{\mathbf{F}}
\newcommand{\I}{\mathcal{I}}
\renewcommand{\H}{\mathcal{H}}
\newcommand{\ZZ}{\mathbf{Z}}
\newcommand{\CC}{\mathbf{C}}
\newcommand{\SL}{\mathrm{SL}}
\renewcommand{\Im}{\mathrm{Im}}
\newcommand{\tor}{\mathrm{tor}}
\newcommand{\num}{\mathrm{num}}
\newcommand{\Cl}{\mathrm{Cl}}
\newcommand{\Hom}{\mathrm{Hom}}
\newcommand{\Frob}{\mathrm{Frob}}
\newcommand{\End}{\mathrm{End}}
\newcommand{\p}{\mathfrak{p}}

\usepackage{graphicx}
\usetheme[block=fill, progressbar=frametitle, numbering=fraction]{metropolis}
\definecolor{mPurple}{HTML}{4b2e83}
\setbeamercolor{palette primary}{%
    bg = mPurple
}

% \usefonttheme{professionalfonts} % required for mathspec
% \usepackage{mathspec}
% \setsansfont[BoldFont={Fira Sans},
% Numbers={OldStyle}]{Fira Sans Light}
% \setmathsfont(Digits)[Numbers={Lining, Proportional}]{Fira
% Sans Light}
\title{%
    Arithmetic of Totally Split Modular Jacobians and Enumeration of Isogeny
    Classes of Prime Level Simple Modular Abelian Varieties
}
\author{Kevin Lui\\Adviser: William Stein}
\date{May 30, 2019}
\institute{Final Exam}

\begin{document}
\frame{\titlepage}

\section{How I got started in my research}

\begin{frame}{Google Summer of Code}
    \begin{itemize}
        \item 
            Once upon a time, William implemented a lot of functionality
            for modular abelian varieties into MAGMA.
        \item
            William started Sage but some of the functionality never made it
            into Sage.
        \item
            I spent a summer working on improving some of the modular abelian
            variety code. Thanks Google!
    \end{itemize} 
\end{frame}

\begin{frame}{Table Making}
    \begin{itemize}
        \item
            Working on the code required me to study a paper that Hao and
            William were working on. I started working on this paper too.
        \item
            This paper had an outline of a table at the end. I tried to fill it
            in.
        \item
            It was hard.
   \end{itemize} 
\end{frame}

\begin{frame}{Research Goal}
    \Huge{Compute as many invariants as possible for modular abelian
    varieties}
\end{frame}

\begin{frame}{Thesis overview}
    I created methods to attack
    \begin{itemize}
        \item
            computing the rational torsion subgroup of modular
            Jacobian $J_0(N)$.
        \item
            enumerating the isogeny class of simple abelian subvarieties of
            $J_0(N)$ with $N$ prime.
    \end{itemize}
\end{frame}

\section{Some background on modular abelian varieties}

\begin{frame}{Modular Jacobian}
    \begin{itemize}
        \item
            The modular curve $X_0(N)$ is an algebraic curve defined over $\QQ$
            which parameterizes elliptic curves with additional $N$-torsion
            data.
        \item 
            Throughout $J=J_0(N)$ is the Jacobian of the modular curve
            $X_0(N)$.
    \end{itemize} 
\end{frame}

\begin{frame}{Defining Data}
    here is defining data
\end{frame}

\begin{frame}{Things we can do easily}
    \begin{itemize}
        \item
            Hecke operators, star-involution, degeneracy maps -- these can all
            be expressed using modular symbols.
        \item
            A decomposition of any modular abelian variety into simple factors
            -- the new factors appear with multiplicity one, the old factors
            are images of degeneracy maps.
        \item 
            Shimura and cuspidal subgroup -- Shimura is the kernel of a
            difference of degeneracy maps, cuspidal can be obtain via
            integration pairing.
    \end{itemize}
\end{frame}

\section{Computing rational torsion subgroup for $J_0(N)$}

\begin{frame}{Rational Torsion Subgroup}
    The \emph{cuspidal} subgroup, $C$, of $J_0(N)$ is the subgroup generated by
    degree-0 divisors of $X_0(N)$ of the form $(\alpha)-(\beta)$, where
    $\alpha,\beta$ are cusps.

    The rational points of $C$ form the \emph{rational cuspidal} subgroup
    $C(\QQ)\subseteq J_0(N)(\QQ)_\tor$. Ogg conjectured they are equal.

    \begin{theorem}[Mazur '77]
        If $N$ is prime, then
        \[
            C(\QQ)=J_0(N)(\QQ)_\tor
        \]
    \end{theorem}
\end{frame}

\begin{frame}{Generalized Ogg Conjecture}
    \begin{conjecture}[Generalized Ogg Conjecture]
        Let $N$ be any positive integer and $C$ be the cuspidal subgroup of
        $J_0(N)$. Then
        \[
            C(\QQ) = J_0(N)(\QQ)_\tor.
        \]
    \end{conjecture}
\end{frame}

\begin{frame}{Progress of Generalized Ogg Conjecture}

    \begin{theorem}[Mazur '77]
        When $N$ is prime, $J_0(N)(\QQ)_\tor =C(\QQ)$.
    \end{theorem}
    \begin{theorem}[Ohta '14]
        When $N$ is a positive squarefree integer,
        \[
            J_0(N)(\QQ)[q^\infty]=C(\QQ)[q^\infty]
        \]
        for $q\nmid 6$.
    \end{theorem}
    \begin{theorem}[Yoo '15]
        When $p>3$,
        \[
            J_0(3p)(\QQ)[3^\infty] = C(3p)(\QQ)[3^\infty]
        \]
        unless $p\equiv 1 \pmod{9}$ and $3^{\frac{p-1}{3}} \equiv 1
        \pmod{9}$.
    \end{theorem}
\end{frame}

\begin{frame}{Progress of Generalized Ogg Conjecture}
    \begin{theorem}[Ling '97]
        When $p\geq 5$ is a prime and $r$ a positive integer,
        \[
            J_0(p^r)(\QQ)[q^\infty] = C(p^r)(\QQ)[q^\infty]
        \]
        for any prime $q\nmid 6p$.
    \end{theorem}
    \begin{theorem}[Ren '18]
        When $N$ is any positive integer,
        \[
            J_0(N)(\QQ)[q^\infty]=C(\QQ)[q^\infty]
        \]
        for $q\nmid 6N\pi(N)$, where $\pi(N) = \prod_{p\mid N}
        (p^2-1)$.
    \end{theorem}
\end{frame}

\begin{frame}{Goal Generalized Ogg Conjecture}
    \begin{itemize}
        \item 
            Verify $[J_0(N)(\QQ)_\tor:C(\QQ)]=1$ for any many $J_0(N)$ as
            possible.
        \item
            Bound $[J_0(N)(\QQ)_\tor:C(\QQ)]$ as best as possible.
    \end{itemize}
\end{frame}

\begin{frame}{Current Sage approach}
    \begin{itemize}
        \item
            Compute a lower bound -- rational cuspidal subgroup.
        \item
            Compute an upper bound -- reduction modulo primes.
        \item
            Hope they agree. 
        \item
            They probably won't.
        \item
            The first example is $J_0(30)$ which is
            isogenous to a product of 3 elliptic curves.
    \end{itemize}
\end{frame}

\begin{frame}{Upper bound}
    \begin{itemize}
        \item 
            Let $A$ be an abelian variety defined over a number field $K$.
            Suppose $\p$ is an unramified prime of $K$ of odd residue
            characteristic. Then $A(K)_\tor \hookrightarrow A_\p(\FF_\p)$.
        \item
            Applying to our case,
            \[
                \#J_0(N)(\QQ)_\tor \mid  \gcd_{p\nmid 2N, p\leq 50} \#J_0(N)_p
                (\FF_p),
            \]
            where $\# J_0(N)_p(\FF_p)=\mathrm{charpoly}(\Frob_p)(1)
            =\mathrm{charpoly}(T_p)(p+1)$.
        \item
            The right hand side is an isogeny invariant so we expect it not to
            be tight. In fact, there's an abelian variety isogenous to
            $J_0(30)$ with strictly larger torsion order.
    \end{itemize}
\end{frame}

\begin{frame}{$J_0(30)$}
    \Large{$J_0(30)$ is the motivating example for the first half of my thesis}
\end{frame}

\begin{frame}{Shimura subgroup}
    \begin{definition}
        The Shimura subgroup, $\Sigma$, of $J_0(N)$ is the kernel of the
        natural degeneracy map $J_0(N)\to J_1(N)$.
    \end{definition}
\end{frame}

\begin{frame}{Degeneracy maps and the old-subvariety}
    \begin{itemize}
        \item 
            Let $L$ be a divisor of $N$. For every divisor $t_1,\ldots,t_r$ of
            $N/L$, there exists a morphism $d_t:J_0(L)\to J_0(N)$, called the
            degeneracy map. This map has finite kernel.
        \item
            Let $\Phi_L ^r: J_0(L)^r \to J_0(N)$ be given by
            \[
                \Phi_L(x_1,\ldots,x_r)=d_{t_1}(x_1)+\cdots+d_{t_r}(x_r).
            \]
            The image of this map is called the $N/L$-old subvariety.
        \item
            Shimuracondition-here
    \end{itemize}
\end{frame}

\begin{frame}{$J_0(30)$}
    \begin{itemize}
        \item
            $J_0(30)$ is isogenous to $E_{15} ^2\times E_{30}$, where $E_{15}$
            and $E_{30}$ are elliptic curves contained the unique isogeny class
            of conductor 15 and 30.
        \item
            Using the previous notation, 
        \item
            okayokay
    \end{itemize}
\end{frame}

\begin{frame}{The totally split hope}
    \begin{itemize}
        \item 
            The rational torsion points of elliptic curves can be determined
            using Nagell-Lutz.
        \item
            Using some Galois cohomology, we can try to reconstruct the
            rational torsion of $J_0(N)$ from its elliptic factors.
    \end{itemize}
\end{frame}


\begin{frame}{Totally split $J_0(N)$}
    \begin{definition}
        A Jacobian $J_0(N)$ is \emph{totally split} if it is isogenous over
        $\QQ$ to a product of elliptic curves.
    \end{definition}
\end{frame}

\begin{frame}{Enumerating totally split $J_0(N)$}
    \begin{theorem}[L.]
        There are 71 nontrivial totally split $J_0(N)$. We can provably
        enumerate them all. The largest is $J_0(1200)$.
    \end{theorem}
\end{frame}

\begin{frame}{Sketch of proof}
    \begin{block}{Lemma}
    The only possible primes $p$, where $J_0(p)$ is totally split are
    \[
        2, 3, 5, 7, 11, 13, 17, 19, 37.
    \]
    \end{block}
    \begin{proof}
        \begin{itemize}
            \item 
                The rational torsion order of $J_0(p)$ is $n=\num((p-1)/12)$.
            \item
                If a prime $\ell$ divides $n$ then $\ell$ divides the rational
                torsion order of some elliptic curve of prime conductor.
            \item
                Elliptic curves of prime conductor don't have much rational
                torsion.
        \end{itemize}
   \end{proof}
    Call an integer $N$ \emph{good} if $J_0(N)$ is totally split and
    \emph{bad}, otherwise. There is an isogeny from $J_0(N)$ to $J_0(NM)$. So
    if $N$ is bad so are all the multiples.

    So we do a breadth-first search on the divisible tree of positive integers
    supported on the good prime. The previous condition allows us to prune
    branches.
\end{frame}


\begin{frame}{Upper bound}
    \begin{itemize}
        \item 
            Let $A$ be an abelian variety defined over a number field $K$.
            Suppose $\p$ is an unramified prime of $K$ of odd residue
            characteristic. Then $A(K)_\tor \hookrightarrow A_\p(\FF_\p)$.
        \item
            In our case,
            \[
                \#J_0(N)(\QQ)_\tor \mid  \gcd_{p\nmid 2N} \#J_0(N)_p (\FF_p),
            \]
            where we can compute the term on the right by explicitly computing
            the Frobenius polynomial using the Eichler-Shimura relations which
            relates Frobenius with Hecke operators.
        \item
            The right hand side is an isogeny invariant so we expect it not to
            be tight. In fact, there's an abelian variety isogenous to
            $J_0(30)$ with strictly larger torsion order.
        \item
            A better upper bound is coming!!
    \end{itemize} 
\end{frame}


\begin{frame}{A better upper bound}
    \begin{itemize}
        \item 
            Recall that for any prime $\ell\nmid 2N$,
            \[
                J_0(N)(\QQ)_\tor\hookrightarrow J_\ell(\F_\ell).
            \]
        \item
            (Stein) Let $\eta_\ell = T_\ell - (\ell+1)$. For any prime
            $\ell\nmid 2N$,
            \[
                J_0(N)(\QQ)_\tor\subseteq J_0(N)[\eta_\ell].
            \]
        \item
            Let $\mathcal{I} = \langle \eta_\ell:\ell\nmid 2N\rangle$. When $N$
            this prime, Mazur calls this the Eisenstein ideal.
            \[
                J_0(N)(\QQ)_\tor \subseteq J_0(N)[\I]
            \]
    \end{itemize} 
\end{frame}

\begin{frame}{A better upper bound cont.}
    \begin{itemize}
        \item
            \[
                J_0(N)(\QQ)_\tor \subseteq J_0(N)[\I]
            \]
        \item
            The cuspidal and Shimura subgroups are contained on the right so
            the group on the right contains some obviously not rational points.
        \item
            The star involution, $s$, is the map corresponding to complex
            conjugation so
            \[
                J_0(N)(\CC)[s] = J_0(N)(\RR).
            \]
        \item
            Let $\I^* = \langle \I, s \rangle$. William calls this the real
            Eisenstein ideal.
   \end{itemize} 
\end{frame}

\begin{frame}{Approximating real Eisenstein kernel}
    \begin{itemize}
        \item
            Let $\I_r ^*= \langle \eta_\ell:\ell\nmid 2N, \ell \leq r \rangle$.
        \item
            Let $E_r = J_0(N)[\I_r]$.
        \item 
            We have
            \[
                C(\QQ)\subseteq J_0(N)(\QQ)_\tor \subseteq J_0(N)[\I^*]
                \subseteq E_r.
            \]
    \end{itemize}
\end{frame}

\begin{frame}{Reducing to subvarieties}
    \begin{itemize}
        \item
            Let $x_1,\ldots,x_k\in E_r$ be a set of representatives for
            $E_r/C(\QQ)$ with $x_1\in C(\QQ)$.
        \item
            To verify the Generalize Ogg Conjecture, it suffices to show
            $x_i\notin J_0(N)(\QQ)_\tor$ for $i=2,\ldots,k$.
        \item
            Or equivalently, $x_i$ is in an abelian variety $A_i$ where we have
            verified the Generalized Ogg Conjecture.
    \end{itemize}
\end{frame}

\section{Enumerating the odd-isogeny class of simple abelian subvarieties of
$J_0(N)$ with $N$ prime}

\begin{frame}{Working on paper with Hao and William}
    \begin{itemize}
        \item 
            One of the sections in this paper is giving methods for enumerating
            the rational isogeny class of modular abelian varieties. 
        \item
            In general, this seems hopelessly difficult.
        \item
            But Frank Calegari had an idea for a special case!
    \end{itemize} 
\end{frame}

\begin{frame}{F. Calegari's Theorem}
    Let $A\subseteq J_0(N)$ be a simple abelian variety with $N$ prime and
    $\TT_A = \Im(\TT\to \End(A))$ integrally closed. Let $\H$ be a set of
    odd integral representatives for $\Cl(\TT_A)$.
    \begin{theorem}[F. Calegari]
        Suppose $\phi:A\to A'$ be an isogeny with $\ker\phi$ supported only on the
        non-Eisenstein primes of odd residue characteristic. Then
        \[
            A'\cong A/A[C]
        \]
        for some $C\in \H$.
    \end{theorem}
    Though there are infinitely many non-Eisenstein primes away from 2, this
    theorem reduces the isogenies supported on these primes to just a finite
    list!
\end{frame}

\begin{frame}{Some experimental data}
    \begin{itemize}
        \item 
            There are [[this many]] simple abelian subvarieties, $A$, of
            $J_0(N)$ with $\dim(A)\leq [[bound]]$ and $N\leq [[bound]]$
        \item
            For all of these abelian subvarieties, $\Cl(\TT_A)$ is trivial.
    \end{itemize} 
\end{frame}

\begin{frame}{Restating in terms Galois submodules}
    \begin{itemize}
        \item 
            The image of an isogeny is determined by its kernel up to isomorphism. 
        \item
            Define an equivalence relation on the set of finite odd-order
            $G_\QQ$-submodules of $A(\QQbar)$ given by $M_1\sim M_2$ if and only if
            $A/M_1\cong A/M_2$.
        \item
            So the goal is to enumerate the odd-order $G_\QQ$-submodules of
            $A(\QQbar)_\mathrm{odd}$.
    \end{itemize}
\end{frame}

\begin{frame}{Making sense of F. Calegari's theorem}
    \begin{proposition}
        Suppose $M$ is a finite odd-order $G_\QQ$-submodule of $J_0(N)$, with $N$
        prime. Then $M$ is a $\TT[G_\QQ]$-module.
    \end{proposition}

    In particular, we can refer to the support of $M\subseteq A$ as a
    $\TT_A$-module.
\end{frame}

\begin{frame}{General Idea}
    \begin{itemize}
        \item 
            Enumerate, up to equivalence, the finite odd-order submodules
            supported on the non-Eisenstein primes. Done by F. Calegari!
        \item
            Enumerate, up to equivalence, the finite odd-order submodules
            supported on the non-Eisenstein primes. To come!
        \item
            Put them together to enumerate the odd-degree isogeny class.
    \end{itemize}
\end{frame}

\begin{frame}{Eisenstein Isogenies}
    this was hard
    \begin{itemize}
        \item 
            $\mathbb{Q}$
        \item
            $\mathbf{Q}$
    \end{itemize}
\end{frame}

    
\end{document}
