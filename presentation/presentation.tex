\documentclass{beamer}

\usepackage{booktabs}
\usepackage{amsthm}
\newtheorem{proposition}[theorem]{Proposition}
\newtheorem{conjecture}[theorem]{Conjecture}
\newcommand{\QQ}{\mathbf{Q}}
\newcommand{\QQbar}{\overline{\mathbf{Q}}}
\newcommand{\RR}{\mathbf{R}}
\newcommand{\ZZ}{\mathbf{Z}}
\newcommand{\FF}{\mathbf{F}}
\newcommand{\TT}{\mathbf{T}}
\newcommand{\Jac}{\mathrm{Jac}}
\newcommand{\F}{\mathbf{F}}
\newcommand{\I}{\mathcal{I}}
\renewcommand{\H}{\mathcal{H}}
\renewcommand{\ZZ}{\mathbf{Z}}
\newcommand{\CC}{\mathbf{C}}
\newcommand{\SL}{\mathrm{SL}}
\renewcommand{\Im}{\mathrm{Im}}
\newcommand{\tor}{\mathrm{tor}}
\newcommand{\Ann}{\mathrm{Ann}}
\newcommand{\num}{\mathrm{num}}
\newcommand{\old}{\mathrm{old}}
\newcommand{\new}{\mathrm{new}}
\newcommand{\Cl}{\mathrm{Cl}}
\newcommand{\Hom}{\mathrm{Hom}}
\newcommand{\Frob}{\mathrm{Frob}}
\newcommand{\End}{\mathrm{End}}
\newcommand{\p}{\mathfrak{p}}

\usepackage{tikz-cd}

\usepackage{graphicx}
\usetheme[block=fill, progressbar=frametitle, numbering=fraction]{metropolis}
\definecolor{mPurple}{HTML}{4b2e83}
\setbeamercolor{palette primary}{%
    bg = mPurple
}

% \usefonttheme{professionalfonts} % required for mathspec
% \usepackage{mathspec}
% \setsansfont[BoldFont={Fira Sans},
% Numbers={OldStyle}]{Fira Sans Light}
% \setmathsfont(Digits)[Numbers={Lining, Proportional}]{Fira
% Sans Light}
\title{%
    Arithmetic of Totally Split Modular Jacobians and Enumeration of Isogeny
    Classes of Prime Level Simple Modular Abelian Varieties
}
\author{Kevin Lui\\Adviser: William Stein}
\date{May 30, 2019}
\institute{Final Exam}

\begin{document}
\frame{\titlepage}

\section{How I got started in my research}

\begin{frame}{Google Summer of Code}
    \begin{itemize}
        \item
            Once upon a time, William implemented a lot of functionality
            for modular abelian varieties into MAGMA.
        \item
            William started Sage but some of the functionality never made it
            into Sage.
        \item
            I spent a summer working on improving some of the modular abelian
            variety code. Thanks Google!
    \end{itemize}
\end{frame}

\begin{frame}{Table Making}
    \begin{itemize}
        \item
            Working on the code required me to study a paper that Hao and
            William were working on. I started working on this paper too.
        \item
            This paper had an outline of a table at the end. I tried to fill it
            in.
        \item
            It was hard.
   \end{itemize}
\end{frame}

\begin{frame}{Research Goal}
    \Huge{Compute as many invariants as possible for modular abelian
    varieties}
\end{frame}

\begin{frame}{Thesis overview}
    I created methods to attack
    \begin{itemize}
        \item
            computing the rational torsion subgroup of modular
            Jacobian $J_0(N)$.
        \item
            enumerating the isogeny class of abelian subvarieties of
            $J_0(N)$.
    \end{itemize}
\end{frame}

\section{Some background on modular abelian varieties}

\begin{frame}{Modular Jacobian}
    \begin{itemize}
        \item
            The modular curves $X_0(N), X_1(N)$ is an algebraic curve defined
            over $\QQ$ which parameterizes elliptic curves with additional
            $N$-torsion data.
        \item
            The Jacobians $J_0(N)=\Jac(X_0(N))$ and $J_1(N)=\Jac(X_1(N))$ are
            abelian varieties over $\QQ$.
    \end{itemize}
\end{frame}

\begin{frame}{Modular Abelian Varieties}
    \begin{definition}[Modular Abelian Variety]
        An abelian variety $A$ is a \emph{modular} if there is a finite-degree
        morphism $\varphi:A\to J_1(N)$ for some $N$.
    \end{definition}
\end{frame}

\begin{frame}[fragile]{Defining Data}
    Let $\varphi:A\to J_1(N)$ be a finite-degree morphism. Let $B$ be the image
    of $A$ in $J$. Then there is an isogeny $B$ to $A$ with kernel $G$ such
    that $A\cong B/G$.
    \begin{equation*}
        \label{eq:defining_data}
        \begin{tikzcd}
            0 \arrow[r]
            &
            G \arrow[r]
            &
            B \arrow[r]\arrow[rd]
            &
            A \arrow[d, "\varphi"]
            \\
            &
            &
            &
            J
        \end{tikzcd}
    \end{equation*}
    So we can represent any modular abelian variety $J$ by giving $G\subseteq
    B\subseteq J$ all defined over $\QQ$.
\end{frame}

\begin{frame}{Modular Symbols}
    \begin{itemize}
        \item
            Modular symbols provide a finite presentation for $H_1(J, \ZZ)$.
        \item
            We'll take it as a blackbox.
    \end{itemize}
\end{frame}

% \begin{frame}{Defining ambient Jacobian $J$}
%     Recall $A\cong B/G$, $G\subseteq B \subseteq J$.

%     \begin{itemize}
%         \item
%             Modular symbols provide a finite presentation for $H_1(J, \ZZ)$.
%             We'll take this as a blackbox.
%         \item
%             $H_1(J, \ZZ)\cong \ZZ^{2\mathrm{dim}(J)}$
%         \item
%             $H_1(J, \QQ) = H_1(J, \ZZ)\otimes \QQ\cong \QQ^{2\mathrm{dim}(J)}$
%         \item
%             $J(\QQbar)_\tor$ is given by $H_1(J, \QQ)/H_1(J, \QQ)$.
%     \end{itemize}
% \end{frame}

% \begin{frame}{Defining abelian subvariety $B$}
%     Recall $A\cong B/G$, $G\subseteq B \subseteq J$.
%     \begin{itemize}
%         \item
%             The embedding of $H_1(B, \QQ)$ into $H_1(J, \QQ)$
%     \end{itemize}
% \end{frame}

\begin{frame}{Defining Data cont.}
    Recall $A\cong B/G$, $G\subseteq B \subseteq J$.
    \begin{itemize}
        \item
            (Defining $J$) Give a modular symbol basis for $H_1(J, \ZZ)\cong
            \ZZ^{2\mathrm{dim}(J)}$, $H_1(J, \QQ) = H_1(J, \ZZ)\otimes
            \QQ\cong \QQ^{2\mathrm{dim}(J)}$. So $J(\QQbar)_\tor$ is given as
            $H_1(J, \QQ)/H_1(J, \QQ)$.
        \item
            (Defining $B$) Give a basis for the subspace $V:=H_1(B,
            \QQ)\subseteq H_1(J, \QQ)$. Then $H_1(B, \ZZ)=V\cap H_1(J, \QQ)$.
        \item
            (Defining $G$) Give a basis for the lattice $L$, $H_1(B, \ZZ)
            \subseteq L \subseteq H_1(B, \QQ)$ such that $G\cong \Lambda/H_1(B,
            \ZZ)$.
    \end{itemize}
\end{frame}

\begin{frame}{Example}
    We give an example of the defining data of some elliptic curve factor of
    the 3-dimensional abelian variety $J_0(30)$. Here $B$ is the image of
    $J_0(15)\to J_0(30)$ under the degeneracy map corresponding to $1$ and $G$
    is the Shimura subgroup of $B$.
    \begin{itemize}
        \item
            (Defining $J$) A basis for our representation of $H_1(J, \ZZ)$ is
            given by modular symbols.
        \item
            (Defining $B$) A basis for $H_1(B, \ZZ)$ is $\{(1,0,-1,1,0,1),
            (0,1,0,-1,1,0)\}$.
        \item
            (Defining $G$) A basis for $G$ is $\{(1/4, 1/2, -1/4, -1/4, 1/2,
            1/4), (0,1,0,-1,1,0)\}$.
    \end{itemize}
\end{frame}

\begin{frame}{Defining data for groups, morphisms}
    \begin{itemize}
        \item
            (Define groups) Suppose $A\sim (L, V, J)$. Then
            $A(\QQbar)_\tor=V/L$

            A subgroup $H$ of
            $A(\QQbar)_\tor$ is specified by giving a lattice $\Lambda$ so that
            $H = \Lambda/L$.
        \item
            (Define morphisms) Suppose $A\sim (L, V, J)$ and $B\sim (L', V', J')$.
            Then $A\to B$ can be specified by giving a map $V\to V'$ sending
            $L\to L'$.
    \end{itemize}
\end{frame}

\begin{frame}{Things we can do}
    \begin{itemize}
        \item
            Hecke operators, star-involution, degeneracy maps -- these can all
            be expressed using modular symbols.
        \item
            A decomposition of $A$ into a sum of its simple subvarieties. The
            simple new subvariety appear with multiplicity one, the simple old
            subvarieties are images of degeneracy maps.
        \item
            Shimura and cuspidal subgroup -- Shimura is the kernel of a
            difference of degeneracy maps, cuspidal can be obtain via
            integration pairing.
        \item
            Essentially, linear algebra.
    \end{itemize}
\end{frame}

\begin{frame}{Things that are hard}
    Galois action is hard.
    \begin{itemize}
        \item
            Consider $J_0(11)$. This is the elliptic curve
            \[
                y^2 + y = x^3 - x^2 - 10x - 20.
            \]
        \item
            In our presentation, a basis for $H_1(J, \ZZ)$ is given by
            $\mathcal{B}=\{(1,8), (1,9)\}$.
        \item
            Consider the 5-torsion point $x$ with $[x]_\mathcal{B}=(1/5, 2/5)$.
            What's the Galois action?
        \item
            $(11/5 \mu_5^3 + 11/5 \mu_5^2 + 3/5, 22/5 \mu_5 ^3 + 11/5 \mu_5^2 +
            33/5 \mu_5 + 14/5)$
    \end{itemize}
\end{frame}

\section{Computing rational torsion subgroup for $J_0(N)$}

\begin{frame}{Why did I start with rational torsion subgroup?}
    \begin{itemize}
        \item
            I understood the word rational.
        \item
            I understood the word torsion.
    \end{itemize}
\end{frame}


\begin{frame}{Rational Torsion Subgroup}
    The \emph{cuspidal} subgroup, $C$, of $J_0(N)$ is the subgroup generated by
    degree-0 divisors of $X_0(N)$ of the form $(\alpha)-(\beta)$, where
    $\alpha,\beta$ are cusps.

    The rational points of $C$ form the \emph{rational cuspidal} subgroup
    $C(\QQ)\subseteq J_0(N)(\QQ)_\tor$. Ogg conjectured they are equal.

    \begin{theorem}[Mazur '77]
        If $N$ is prime, then
        \[
            C(\QQ)=J_0(N)(\QQ)_\tor.
        \]
    \end{theorem}
\end{frame}

\begin{frame}{Generalized Ogg Conjecture}
    \begin{conjecture}[Generalized Ogg Conjecture]
        Let $N$ be any positive integer and $C$ be the cuspidal subgroup of
        $J_0(N)$. Then
        \[
            C(\QQ) = J_0(N)(\QQ)_\tor.
        \]
    \end{conjecture}
\end{frame}

\begin{frame}{Progress of Generalized Ogg Conjecture}
    \begin{theorem}[Mazur '77]
        When $N$ is prime, $J_0(N)(\QQ)_\tor =C(\QQ)$.
    \end{theorem}
    \begin{theorem}[Ohta '14]
        When $N$ is a positive squarefree integer,
        \[
            J_0(N)(\QQ)[q^\infty]=C(\QQ)[q^\infty]
        \]
        for $q\nmid 6$.
    \end{theorem}
    \begin{theorem}[Yoo '15]
        When $p>3$,
        \[
            J_0(3p)(\QQ)[3^\infty] = C(3p)(\QQ)[3^\infty]
        \]
        unless $p\equiv 1 \pmod{9}$ and $3^{\frac{p-1}{3}} \equiv 1
        \pmod{9}$.
    \end{theorem}
\end{frame}

\begin{frame}{Progress of Generalized Ogg Conjecture cont.}
    \begin{theorem}[Ling '97]
        When $p\geq 5$ is a prime and $r$ a positive integer,
        \[
            J_0(p^r)(\QQ)[q^\infty] = C(p^r)(\QQ)[q^\infty]
        \]
        for any prime $q\nmid 6p$.
    \end{theorem}
    \begin{theorem}[Ren '18]
        When $N$ is any positive integer,
        \[
            J_0(N)(\QQ)[q^\infty]=C(\QQ)[q^\infty]
        \]
        for $q\nmid 6N\pi(N)$, where $\pi(N) = \prod_{p\mid N}
        (p^2-1)$.
    \end{theorem}
\end{frame}

\begin{frame}{Goal for Generalized Ogg Conjecture}
    \begin{itemize}
        \item
            Verify $[J_0(N)(\QQ)_\tor:C(\QQ)]=1$ for any many $J_0(N)$ as
            possible.
        \item
            Bound $[J_0(N)(\QQ)_\tor:C(\QQ)]$ as best as possible.
    \end{itemize}
\end{frame}

\begin{frame}{Current Sage approach}
    \begin{itemize}
        \item
            Compute a lower bound -- rational cuspidal subgroup.
        \item
            Compute an upper bound -- reduction modulo primes.
        \item
            Hope they agree.
        \item
            They probably won't.
        \item
            The first example is $J_0(30)$ which is
            isogenous to a product of 3 elliptic curves.
    \end{itemize}
\end{frame}

\begin{frame}{Upper bound}
    \begin{itemize}
        \item
            Upper bound is obtained by reduction modulo primes.
        \item
            The bound obtained is an isogeny invariant.
        \item
            There is an abelian variety isogenous to $J_0(30)$ with strictly
            larger torsion order.
    \end{itemize}
\end{frame}

% \begin{frame}{Upper bound}
%     \begin{itemize}
%         \item
%             Let $A$ be an abelian variety defined over a number field $K$.
%             Suppose $\p$ is an unramified prime of $K$ of odd residue
%             characteristic. Then $A(K)_\tor \hookrightarrow A_\p(\FF_\p)$.
%         \item
%             Applying to our case,
%             \[
%                 \#J_0(N)(\QQ)_\tor \mid  \gcd_{p\nmid 2N, p\leq 50} \#J_0(N)_p
%                 (\FF_p),
%             \]
%             where $\# J_0(N)_p(\FF_p)=\mathrm{charpoly}(\Frob_p)(1)
%             =\mathrm{charpoly}(T_p)(p+1)$.
%         \item
%             The right hand side is an isogeny invariant so we expect it not to
%             be tight. In fact, there's an abelian variety isogenous to
%             $J_0(30)$ with strictly larger torsion order.
%     \end{itemize}
% \end{frame}

\begin{frame}{$J_0(30)$}
    \Large{$J_0(30)$ is the motivating example for the first half of my thesis}
\end{frame}

\begin{frame}{Computing $J_0(30)(\QQ)_\tor$}
    \begin{itemize}
        \item
            Rank 0 so $J_0(30)(\QQ)=J_0(30)(\QQ)_\tor$.
        \item
            (Lower bound) $\#C(\QQ)=192$
        \item
            (Upper bound) Reduction modulo $7$, gives $\# J_0(30)(\QQ)_\tor\leq
            768$.
        \item
            (Upper bound)/(Lower bound) = 4.
    \end{itemize}
\end{frame}

\begin{frame}{Decomposition of $J_0(30)$}
    \begin{itemize}
        \item
            $J_0(30)$ decomposes as a sum of 3 of its elliptic subvarieties
            $E_1+E_2+E_3$.
        \item
            We have $E_1, E_2$ are isomorphic to the elliptic curve $J_0(15)$
            and $E_3$ is isomorphic to an elliptic curve of conductor 30. They
            are all rank 0.
        \item
            Let $J_\old=E_1+E_2$ and $E_3=J_\old$.
    \end{itemize}
\end{frame}

\begin{frame}{Old subvariety}
    \begin{itemize}
        \item
            We'll start by trying to compute $J_\old(\QQ)_\tor$.
        \item
            Rational torsion order is not multiplicative on exact sequences.
        \item
            But it is multiplicative on direct sums.
        \item
            We start by showing $J_\old$ is a direct sum of 2 elliptic
            subvarieties.
    \end{itemize}
\end{frame}

\begin{frame}{Shimura subgroup}
    \begin{definition}
        The Shimura subgroup, $\Sigma$, of $J_0(N)$ is the kernel of the
        natural degeneracy map $J_0(N)\to J_1(N)$.
    \end{definition}
\end{frame}

\begin{frame}{Degeneracy maps and the old-subvariety}
    \begin{itemize}
        \item
            Let $L$ be a divisor of $N$. For every divisor $t_1,\ldots,t_r$ of
            $N/L$, there exists a morphism $d_t:J_0(L)\to J_0(N)$, called the
            degeneracy map. This map has finite kernel.
        \item
            Let $\Phi_L ^r: J_0(L)^r \to J_0(N)$ be given by
            \[
                \Phi_L(x_1,\ldots,x_r)=d_{t_1}(x_1)+\cdots+d_{t_r}(x_r).
            \]
            The image of this map is called the $N/L$-old subvariety.
    \end{itemize}
\end{frame}

\begin{frame}{Direct sum decomposition}
    \begin{itemize}
        \item
            Let $\Phi_L ^r: J_0(L)^r \to J_0(N)$ be given by
            \[
                \Phi_L(x_1,\ldots,x_r)=d_{t_1}(x_1)+\cdots+d_{t_r}(x_r).
            \]
        \item
            Let $\Sigma(L)_0 ^r = \{(x_1,\ldots,x_r)\in \Sigma(L)^r: \sum
            x_i=0\}\subseteq J_0(L)^r$.
        \end{itemize}
\end{frame}

\begin{frame}{Direct Sum Decomposition}
    \begin{proposition}[L.]
        Suppose that $\ker\Phi_L ^r = \Sigma(L)_0 ^r$ and $E=J_0(L)$ is an elliptic
        curve. Then there is a $\QQ$-isomorphism
        \[
            \Im\Phi_L ^N \cong E \times F^{r-1},
        \]
        where $F$ is the $J_1(L)$-optimal curve in the isogeny class of $E$.
    \end{proposition}
    \begin{theorem}[Ribet '90]
        If $N/L=p$ is a prime coprime to $L$, then $\ker\Phi_L ^r = \Sigma(L)_0
        ^r$.
    \end{theorem}
    Later generalized. (Ling '95)
\end{frame}

\begin{frame}{Determining $J_\old(\QQ)_\tor$}
    \begin{itemize}
        \item
            $30/15=2$ is a prime coprime to $15$, and $J_0(15)$ is an
            elliptic curve.
        \item
            So by previous proposition, $J_\old$ is a direct sum of elliptic
            curves.
    \end{itemize}
\end{frame}

\begin{frame}[standout]{Galois Cohomology}
    \Huge{Whiteboard time}
\end{frame}

\begin{frame}{The totally split hope}
    \begin{itemize}
        \item
            The rational torsion points of elliptic curves can be determined
            using Nagell-Lutz.
        \item
            Using some Galois cohomology, we can try to reconstruct the
            rational torsion of $J_0(N)$ from its elliptic factors.
    \end{itemize}
\end{frame}

\begin{frame}{Totally split $J_0(N)$}
    \begin{definition}
        A Jacobian $J_0(N)$ is \emph{totally split} if it is isogenous over
        $\QQ$ to a product of elliptic curves.
    \end{definition}
\end{frame}

\begin{frame}{Enumerating totally split $J_0(N)$}
    \begin{theorem}[L.]
        There are 71 nontrivial totally split $J_0(N)$. We can provably
        enumerate them all. The largest is $J_0(1200)$.
    \end{theorem}
\end{frame}

\begin{frame}{Lemma on totally split $J_0(p)$}
    \begin{block}{Lemma}
    The only possible primes $p$, where $J_0(p)$ is totally split are
    \[
        2, 3, 5, 7, 11, 13, 17, 19, 37.
    \]
    \end{block}
    \begin{proof}
        \begin{itemize}
            \item
                The rational torsion order of $J_0(p)$ is $n=\num((p-1)/12)$.
            \item
                If a prime $\ell$ divides $n$ then $\ell$ divides the rational
                torsion order of some elliptic curve of prime conductor.
            \item
                Elliptic curves of prime conductor don't have much rational
                torsion.
        \end{itemize}
   \end{proof}
\end{frame}

\begin{frame}{Search for all totally split}
    \begin{itemize}
        \item
            Call an integer $N$ \emph{good} if $J_0(N)$ is totally split and
            \emph{bad}, otherwise.
        \item
            There is are finite-degree morphisms $J_0(N)\to J_0(NM)$ given by
            degeneracy maps.
        \item
            If $N$ is bad so are all its multiples.
        \item
            We do a breadth-first search on the divisible tree of positive integers
            supported on the good prime.
        \item
            If we find a bad node, we can prune that branch.
    \end{itemize}
\end{frame}

\begin{frame}{Results}
    There are 46 rank-0 totally split $J_0(N)$. The Generalized Ogg Conjecture
    has been verified for all but 9 of them. In those cases, the index by
    bounded by a power of 2.
\begin{table}%
    \label{tab:rank_zero_bound}
    \caption{Bound on $[J_0(N)(\QQ)_\tor:C(N)(\QQ)]$}
    \centering
    \begin{tabular}{rr}
        \toprule
        $N$ & bound \\
        \midrule
        84 & 8 \\
        90 & 16 \\
        96 & 64 \\
        120 & 512 \\
        132 & 32 \\
        \bottomrule
    \end{tabular}
    \begin{tabular}{rr}
        \toprule
        $N$ & bound \\
        \midrule
        144 & 64 \\
        150 & 16 \\
        168 & 512 \\
        180 & 128  \\
            &  \\
        \bottomrule
    \end{tabular}
\end{table}
\end{frame}

\section{Enumerating the odd-isogeny class of simple abelian subvarieties of
$J_0(N)$ with $N$ prime}

\begin{frame}{Working on paper with Hao and William}
    \begin{itemize}
        \item
            One question, we hoped to answer was: given a modular abelian
            variety, can we enumerate its rational isogeny class?
        \item
            In general, this seems hopelessly difficult.
        \item
            But Frank Calegari had an idea for a special case!
    \end{itemize}
\end{frame}

\begin{frame}{Restating in terms Galois submodules}
    \begin{itemize}
        \item
            Let $A$ be a modular abelian variety and $\varphi:A\to A'$ be
            $\QQ$-isogeny.
        \item
            The image $A'$ is determined, up to isomorphism, by $\ker\varphi$
            which is a finite $G_\QQ$-submodule of $A(\QQbar)$.
        \item
            Every finite $G_\QQ$-submodule of $A(\QQbar)$ yields an isogeny.
        \item
            Define an equivalence relation on the set of finite odd-order
            $G_\QQ$-submodules of $A(\QQbar)$ given by $M_1\sim M_2$ if and only if
            $A/M_1\cong A/M_2$.
        \item
            We can enumerate the $\QQ$-isogeny class of $A$ by enumerating the
            $G_\QQ$-submodules of $A(\QQbar)$, up to equivalence.
    \end{itemize}
\end{frame}

\begin{frame}{Background}
    Let $A\subseteq J_0(N)$ be a simple abelian variety with $N$ prime and
    $\TT_A = \Im(\TT\to \End(A))$ integrally closed.
    \begin{itemize}
        \item
            The endomorphism ring and Hecke algebra of $A$ isomorphic to an
            order of a number field of degree equal to $\dim A$.
        \item
            The Eisenstein ideal, $\mathcal{I}\subseteq \TT$, is the ideal
            generated by $T_\ell-(1+\ell)$ for $\ell\neq N$ prime. Let
            $\I=\Im(\I\to \End(A))$.
        \item
            A prime of $\TT$ is Eisenstein if it divides $\I$ and
            non-Eisenstein otherwise. Same in $\TT_A$.
        \item
            Let $\p\subseteq \TT_A$ be a prime. Then $A[\p]$ is irreducible as
            a $G_\QQ$-module if and only if $\p$ is non-Eisenstein.
    \end{itemize}
\end{frame}


\begin{frame}{F. Calegari's Theorem}
    Let $A\subseteq J_0(N)$ be a simple abelian subvariety with $N$ prime. Let
    $\H$ be a set of odd integral representatives for $\Cl(\TT_A)$.
    \begin{theorem}[F. Calegari]
        Suppose $\phi:A\to A'$ be an isogeny with $\ker\phi$ supported only on the
        non-Eisenstein primes of odd residue characteristic. Then
        \[
            A'\cong A/A[C]
        \]
        for some $C\in \H$.
    \end{theorem}
    \begin{itemize}
        \item
            The $\TT_A$-support of a $G_\QQ$-module?!??!?!?!
    \end{itemize}
\end{frame}

\begin{frame}{Making sense of F. Calegari's theorem}
    \begin{proposition}
        Suppose $M$ is a finite odd-order $G_\QQ$-submodule of $J_0(N)$, with $N$
        prime. Then $M$ is a $\TT[G_\QQ]$-module.

        In particular, if $A\subseteq J_0(N)$, we can refer to the support of
        $M\subseteq A$ as a $\TT_A$-module.
    \end{proposition}
\end{frame}

\begin{frame}{Finite odd-order Galois modules are Hecke}
    Let $M$ be a finite odd-order Galois module of $J_0(N)$. Let $\TT$ be the
    Hecke algebra of $J_0(N)$. Want to show $\TT M = M$.
    \begin{itemize}
        \item
            It suffices to prove this on $\ell$-primary parts for prime
            $\ell>2$. It suffices to prove this for each each
            $G_\QQ$-composition factor of $M$.
        \item
            So assume $M$ is $\ell$-torsion.
        \item
            Ribet ('91) shows that $\TT/\ell \TT$ is generated by $T_p$ for
            $p\nmid \ell N$. So it suffices to show $T_p M \subseteq M$.
        \item
            By Eichler-Shimura, $T_p = F+p/F\in \End(J_{/\F_p})$, where $F$ is
            Frobenius. So we derivate the $T_p$-stability from its
            $G_\QQ$-stability.
    \end{itemize}
\end{frame}

\begin{frame}{Non-eisenstein implies kernel of Hecke}
    \begin{theorem}[Helm]
        Let $M$ be a finite odd-order $G_\QQ$-module supported, as a $\TT$-module,
        only on the non-Eisenstein primes. If $I=\Ann_\TT(M)$, then $M=J[I]$.
    \end{theorem}
\end{frame}

\begin{frame}{Back to F. Calegari's theorem}
    \begin{theorem}[F. Calegari]
        Suppose $\phi:A\to A'$ be an isogeny with $\ker\phi$ supported only on the
        non-Eisenstein primes of odd residue characteristic. Then
        \[
            A'\cong A/A[C]
        \]
        for some $C\in \H$, where $\H$ is a set of odd integral representatives
        for $\Cl(\TT_A)$.
    \end{theorem}
    \begin{itemize}
        \item
            By Helm, $\ker\phi=A[D]$ for some ideal $D$.
        \item
            There exists $a,b\in \TT_A$ such that $aC=bD$.
    \end{itemize}
\end{frame}

\begin{frame}[fragile]{Back to F. Calegari's theorem cont.}
    \[
        \begin{tikzcd}
            A/A[bD]
            \arrow[r, "\sim", "b"']
            \ar[equal]{d}
            &
            A/A[D]\cong A'
            \ar[d]
            \\
            A/A[a C]
            \arrow[r, "\sim", "a"']
            &
            A/A[C]
        \end{tikzcd}
    \]
\end{frame}


\end{document}
