\documentclass{beamer}

\usepackage{amsthm}
\newtheorem{proposition}[theorem]{Proposition}
\newcommand{\QQ}{\mathbf{Q}}
\newcommand{\RR}{\mathbf{R}}
\newcommand{\FF}{\mathbf{F}}
\newcommand{\F}{\mathbf{F}}
\newcommand{\ZZ}{\mathbf{Z}}
\newcommand{\CC}{\mathbf{C}}
\newcommand{\SL}{\mathrm{SL}}
\newcommand{\tor}{\mathrm{tor}}
\newcommand{\p}{\mathfrak{p}}

\usepackage{graphicx}
\usetheme[block=fill, progressbar=frametitle, numbering=fraction]{metropolis}
\definecolor{mPurple}{HTML}{4b2e83}
\setbeamercolor{palette primary}{%
    bg = mPurple
}

\title{%
    Arithmetic of Totally Split Modular Jacobians and Enumeration of Isogeny
    Classes of Prime Level Simple Modular Abelian Varieties
}
\author{Kevin Lui\\Adviser: William Stein}
\date{May 30, 2019}
\institute{Final Exam}

\begin{document}
\frame{\titlepage}

\section{How I got started in my research}

\begin{frame}{Google Summer of Code}
    \begin{itemize}
        \item 
            Once upon a time, William implemented a lot of functionality
            for modular abelian varieties into MAGMA.
        \item
            William started Sage but some of the functionality never made it
            into Sage.
        \item
            I spent a summer working on improving some of the modular abelian
            variety code. Thanks Google!
    \end{itemize} 
\end{frame}

\begin{frame}{Table Making}
    \begin{itemize}
        \item
            Working on the code required me to study a paper that Hao and
            William were working on. I started working on this paper too.
        \item
            This paper had an outline of a table at the end. I tried to fill it
            in.
        \item
            It was hard.
   \end{itemize} 
\end{frame}

\begin{frame}{Research Goal}
    \Huge{Compute as many as invariants as possible for modular abelian
    varieties}
\end{frame}

\begin{frame}{Broad overview}
    I created methods to attempt to
    \begin{itemize}
        \item
            compute the rational torsion subgroup of totally split modular
            Jacobian $J_0(N)$.
        \item
            enumerate the odd-isogeny class of a simple abelian subvarieties of
            $J_0(N)$ with $N$ prime.
    \end{itemize}
\end{frame}

\section{Some background on modular abelian varieties}


% \begin{frame}{Elliptic curves}
%     \begin{itemize}
%         \item
%             An elliptic curve, $E$, over a characteristic zero field is the
%             solution space to $y^2=f(x)$, where $f(x)$ is a nonsingular cubic
%             in $x$.
%         \item
%             Through the Weierstrass $P$ function, the complex points of $E$ is
%             isomorphic to $\CC/\Lambda$.
%         \item
%             A morphism $\CC/\Lambda\to \CC/\Lambda'$ is a complex
%             linear map $\varphi:\CC\to \CC$ such that
%             $\varphi(\Lambda)\subseteq \Lambda'$.
%         \item
%             After some isomorphisms, every elliptic curve is isomorphic to
%             $\CC/\Lambda_\tau$, where $\Lambda_\tau = \langle 1, \tau \rangle$.
%         \item
%             $\CC/\Lambda_\tau \cong \CC/\Lambda_{\tau'}$ if and only if
%             there exists $M\in \SL_2(\ZZ)$ such that $M\tau = \tau'$.
%     \end{itemize}
% \end{frame}

% \begin{frame}{Modular Curve}
    
% \end{frame}

\begin{frame}{Things we can do easily}
    \begin{itemize}
        \item
            Hecke operators, star-involution, degeneracy maps -- these can all
            be expressed using modular symbols.
        \item
            A decomposition of any modular abelian variety into simple factors
            -- the new factors appear with multiplicity one, the old factors
            are images of degeneracy maps.
        \item 
            Shimura and cuspidal subgroup -- Shimura is the kernel of a
            difference of degeneracy maps, cuspidal can be obtain via
            integration pairing.
    \end{itemize}
\end{frame}

\section{Computing rational torsion subgroup for totally split $J_0(N)$}

\begin{frame}{Current Sage approach}
    \begin{itemize}
        \item
            Compute a lower bound -- rational cuspidal subgroup.
        \item
            Compute an upper bound -- reduction modulo primes.
        \item
            Hope they agree. 
        \item
            Sometimes they don't! The first example is $J_0(30)$ which
            isogenous to a product of 3 elliptic curves.
    \end{itemize}
\end{frame}

\begin{frame}{Totally split $J_0(N)$}
    \begin{definition}
        A Jacobian $J_0(N)$ is \emph{totally split} if it is isogenous over
        $\QQ$ to a product of elliptic curves.
    \end{definition}

    \begin{theorem}[K.L.]
        There are 71 nontrivial totally split $J_0(N)$. We can provably
        enumerate them all. The largest is $J_0(1200)$.
    \end{theorem}
\end{frame}

\begin{frame}{The totally split hope}
    \begin{itemize}
        \item 
            The rational torsion points of elliptic curves can be computed
            using Nagell-Lutz.
        \item
            Using some Galois cohomology, we can try to reconstruct the
            rational torsion of $J_0(N)$ from its elliptic factors.
    \end{itemize}
\end{frame}

\begin{frame}{Lower bound}
    \begin{itemize}
        \item    
            The \emph{cuspidal subgroup} 
            \[
                C(N) = \langle (\alpha)-(\beta):\alpha,\beta~\text{cusps}
                \rangle
            \]
            of $J_0(N)$ is the subgroup of
            degree-zero divisors supported on the cusps of $X_0(N)$.
            \pause
        \item
            Integration against cusp forms
            gives a map $i:C\to \Hom(S_2(N), \CC)$ given by 
            \[
                (\alpha)-(\beta) \mapsto \int_\alpha ^\beta \bullet
            \]
            In our presentation of $J_0(N)$, the cuspidal subgroup is
            quotient of $i(C)$ by $H_1(X_0(N), \ZZ)$. 
            \pause
        \item
            Glenn Stevens determined the Galois action on $C(N)$ so we can
            explicitly determine the \emph{rational cuspidal subgroup}
            $C(N)(\QQ)$.
    \end{itemize}
\end{frame}

\begin{frame}{Upper bound}
    \begin{itemize}
        \item 
            Let $A$ be an abelian variety defined over a number field $K$.
            Suppose $\p$ is an unramified prime of $K$ of odd residue
            characteristic. Then $A(K)_\tor \hookrightarrow A_\p(\FF_\p)$.
        \item
            In our case,
            \[
                \#J_0(N)(\QQ)_\tor \mid  \gcd_{p\nmid 2N} \#J_0(N)_p (\FF_p),
            \]
            where we can compute the term on the right by explicitly computing
            the Frobenius polynomial using the Eichler-Shimura relations which
            relates Frobenius with Hecke operators.
        \item
            The right hand side is an isogeny invariant so we expect it not to
            be tight. In fact, there's an abelian variety isogenous to
            $J_0(30)$ with strictly larger torsion.
        \item
            A better upper bound is coming!!
    \end{itemize} 
\end{frame}

\begin{frame}{Ogg Conjecture}
    \begin{itemize}
        \item 
            (Ogg) The cuspidal subgroup $C$ of $J_0(p)$ is cyclically generated
            by $(0)-(\infty)$ and has order $\mathrm{num} (p-1)/12$. This
            subgroup is rational so 
            \[
                C=C(\QQ)\subseteq J_0(p)(\QQ)_\tor.
            \]
            Ogg conjectured this is an equality.
            \pause
        \item
            (Mazur) $C=J_0(p)(\QQ)_\tor$.
    \end{itemize}
\end{frame}

\begin{frame}{Generalized Ogg Conjecture}
    \begin{conjecture}[Generalized Ogg Conjecture]
        Let $N$ be any positive integer and $C$ be the cuspidal subgroup of
        $J_0(N)$. Then
        \[
            C(\QQ) = J_0(N)(\QQ)_\tor.
        \]
    \end{conjecture}
\end{frame}

\begin{frame}{Progress of Generalized Ogg Conjecture}
    \begin{itemize}
        \item 
            (Mazur 77) When $N$ is prime, $J_0(N)(\QQ)_\tor =C(N)(\QQ)$.
            \pause
        \item
            (Ohta 14) When $N$ is a positive squarefree integer,
            \[
                J_0(N)(\QQ)[q^\infty]=C(N)(\QQ)[q^\infty]
            \]
            for $q\nmid 6$.
        \item
            (Yoo 15) When $p>3$,
            \[
                J_0(3p)(\QQ)[3^\infty] = C(3p)(\QQ)[3^\infty]
            \]
            unless $p\equiv 1 \pmod{9}$ and $3^{\frac{p-1}{3}} \equiv 1
            \pmod{9}$.
    \end{itemize}
\end{frame}

\begin{frame}{Progress of Generalized Ogg Conjecture}
    \begin{itemize}
        \item
            (Ling 97) When $p\geq 5$ is a prime and $r$ a positive integer,
            \[
                J_0(p^r)(\QQ)[q^\infty] = C(p^r)(\QQ)[q^\infty]
            \]
            for any prime $q\nmid 6p$.
        \item
            (Ren 18) When $N$ is any positive integer,
            \[
                J_0(N)(\QQ)[q^\infty]=C(N)(\QQ)[q^\infty]
            \]
            for $q\nmid 6N\pi(N)$, where $\pi(N) = \prod_{p\mid N}
            (p^2-1)$.
    \end{itemize}
\end{frame}

\begin{frame}{Restatement of goal}
    \large{Bound $[J_0(\QQ)_\tor:C(N)(\QQ)]$ for totally split $J_0(N)$}
\end{frame}

\begin{frame}{A better upper bound}
    \begin{itemize}
        \item 
            Recall that for any prime $\ell\nmid 2N$,
            \[
                J_0(N)(\QQ)_\tor\hookrightarrow J_\ell(\F_\ell).
            \]
        \item
            (Stein) Let $\eta_\ell = T_\ell - (\ell+1)$. For any prime
            $\ell\nmid 2N$,
            \[
                J_0(N)(\QQ)_\tor\subseteq J[\eta_\ell].
            \]
        \item
            Let $\mathcal{I}
    \end{itemize} 
\end{frame}

\section{Enumerating the odd-isogeny class of simple abelian subvarieties of
$J_0(N)$ with $N$ prime}

    
\end{document}
