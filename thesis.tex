\documentclass[11pt, proquest]{uwthesis}
\setcounter{tocdepth}{1}  % Print the chapter and sections to the toc

\usepackage{url}
\usepackage{hyperref}
\hypersetup{colorlinks=true}
\usepackage{booktabs}
\usepackage{microtype}
\usepackage{tikz-cd}
\tikzset{labl/.style={anchor=south, rotate=90, inner sep=.5mm}}
 

% macros.tex
\usepackage{amsmath}
\usepackage{amsfonts}
\usepackage{amssymb}
\usepackage{amsthm}

\usepackage{url}


% You change everything, by adding \usepackage{times} to the document
% Preamble. Now all the roman letters will be set in times and all the
% sans serif stuff will be set in Helvetica. If you don't like times,
% you can try the packages: palatcm, charter, helvet, palatino, avant,
% newcent and bookman
% If you want to change explicitly to a certain font, use the command
% \fontfamily{XYZ}\selectfont whereby XYZ can be set to: pag for Adobe
% AvantGarde, pbk for Adobe Bookman, pcr for Adobe Courier, phv for
% Adobe Helvetica, pnc for Adobe NewCenturySchoolbook, ppl for Adobe
% Palatino, ptm for Adobe Times Roman, pzc for Adobe ZapfChancery
\newcommand{\courier}{\fontfamily{pcr}\selectfont}



\newcommand{\edit}[1]{\footnote{[[#1]]}\marginpar{\hfill {\sf[[\thefootnote]]}}}
%\newcommand{\edit}[1]{{\sl\small [[Todo: #1]]}}


%\author{William~A. Stein}

\newcommand{\Hbar}{\overline{H}}

\newcommand{\myhead}[3]{
\par\noindent
{Version #2}
\vspace{10ex}
\par\noindent
{\bf \LARGE #1}\\
\vspace{3ex}
\par\noindent
{\large W.\thinspace{}A. Stein}\\
{\small Department of Mathematics, Harvard University}\vspace{1ex}\\
#3     
\vspace{2ex}\par
}

\newcommand{\myheadauth}[3]{
\par\noindent
{Version #2}
\vspace{10ex}
\par\noindent
{\bf \LARGE #1}\\
\vspace{3ex}
\par\noindent
#3     
\vspace{5ex}\par
}

\usepackage{xspace}  % to allow for text macros that don't eat space 
\newcommand{\SAGE}{{\sf Sage}\xspace}
\newcommand{\sage}{\SAGE}
\newcommand{\gzero}{\Gamma_0(N)}
\newcommand{\esM}{\overline{\sM}}
\newcommand{\sM}{\boldsymbol{\mathcal{M}}}
\newcommand{\sS}{\boldsymbol{\mathcal{S}}}
\newcommand{\sB}{\boldsymbol{\mathcal{B}}}       
\newcommand{\bA}{\mathbb{A}}
\newcommand{\cK}{\mathcal{K}}
\newcommand{\Adual}{A^{\vee}}
\newcommand{\Bdual}{B^{\vee}}
\newcommand{\kr}[2]{\left(\frac{#1}{#2}\right)}

\newcommand{\defn}[1]{{\em #1}}
\newcommand{\solution}[1]{\vspace{1em}%
  \par\noindent{\bf Solution #1.} }
\newcommand{\todo}[1]{\noindent$\bullet$ {\small \textsf{#1}} $\bullet$\\}
\newcommand{\done}[1]{\noindent {\small \textsf{Done: #1}}\\}
\newcommand{\danger}[1]{\marginpar{\small \textsl{#1}}}
\renewcommand{\div}{\mbox{\rm div}}
\DeclareMathOperator{\GCD}{GCD}
\DeclareMathOperator{\Supp}{Supp}
\DeclareMathOperator{\CH}{CH}
\DeclareMathOperator{\sss}{ss}
\renewcommand{\ss}{\sss}
\DeclareMathOperator{\red}{red}
\DeclareMathOperator{\xgcd}{xgcd}
\DeclareMathOperator{\Kol}{Kol}
\DeclareMathOperator{\can}{can}
\DeclareMathOperator{\Cl}{Cl}
\DeclareMathOperator{\Mod}{Mod}
\DeclareMathOperator{\chr}{char}
\DeclareMathOperator{\charpoly}{charpoly}
\DeclareMathOperator{\cris}{cris}
\DeclareMathOperator{\dR}{dR}
\DeclareMathOperator{\Fil}{Fil}
\DeclareMathOperator{\ind}{ind}
\DeclareMathOperator{\im}{im}
\DeclareMathOperator{\oo}{\infty}
\DeclareMathOperator{\abs}{abs}
\DeclareMathOperator{\lcm}{lcm}
\DeclareMathOperator{\cores}{cores}
\DeclareMathOperator{\coker}{coker}
\DeclareMathOperator{\image}{image}
\DeclareMathOperator{\prt}{part}
\DeclareMathOperator{\proj}{proj}
\DeclareMathOperator{\Br}{Br}
\DeclareMathOperator{\Ann}{Ann}
\DeclareMathOperator{\End}{End}
\DeclareMathOperator{\Tan}{Tan}
\DeclareMathOperator{\Eis}{Eis}
\newcommand{\unity}{\mathbb{1}}
\DeclareMathOperator{\Pic}{Pic}
\DeclareMathOperator{\Tate}{Tate}
\DeclareMathOperator{\Vol}{Vol}
\DeclareMathOperator{\Vis}{Vis}
\DeclareMathOperator{\Reg}{Reg}
%\DeclareMathOperator{\myRes}{Res}
%\newcommand{\Res}{\myRes}
\DeclareMathOperator{\Res}{Res}
\newcommand{\an}{{\rm an}}
\DeclareMathOperator{\rank}{rank}
\DeclareMathOperator{\Sel}{Sel}
\DeclareMathOperator{\Mat}{Mat}
\DeclareMathOperator{\BSD}{BSD}
\DeclareMathOperator{\id}{id}
\DeclareMathOperator{\dz}{dz}
%\DeclareMathOperator{\Re}{Re}
\renewcommand{\Re}{\mbox{\rm Re}}
\DeclareMathOperator{\Imm}{Im}
\renewcommand{\Im}{\Imm}
\DeclareMathOperator{\Selmer}{Selmer}
\newcommand{\pfSel}{\widehat{\Sel}}
\newcommand{\qe}{\stackrel{\mbox{\tiny ?}}{=}}
\newcommand{\isog}{\simeq}
\newcommand{\e}{\mathbf{e}}
\newcommand{\bN}{\mathbf{N}}

% ---- SHA ----
\DeclareFontEncoding{OT2}{}{} % to enable usage of cyrillic fonts
  \newcommand{\textcyr}[1]{%
    {\fontencoding{OT2}\fontfamily{wncyr}\fontseries{m}\fontshape{n}%
     \selectfont #1}}
\newcommand{\Sha}{{\mbox{\textcyr{Sh}}}}

%\font\cyr=wncyr10 scaled \magstep 1
%\font\cyr=wncyr10

%\newcommand{\Sha}{{\cyr X}}
\newcommand{\Shaan}{\Sha_{\mbox{\tiny \rm an}}}
\newcommand{\TS}{Shafarevich-Tate group}

\newcommand{\Gam}{\Gamma}
\newcommand{\X}{\mathcal{X}}
\newcommand{\cH}{\mathcal{H}}
\newcommand{\cA}{\mathcal{A}}
\newcommand{\cF}{\mathcal{F}}
\newcommand{\cG}{\mathcal{G}}
\newcommand{\cJ}{\mathcal{J}}
\newcommand{\cL}{\mathcal{L}}
\newcommand{\cV}{\mathcal{V}}
\newcommand{\cO}{\mathcal{O}}
\newcommand{\cQ}{\mathcal{Q}}
\newcommand{\cX}{\mathcal{X}}
\newcommand{\ds}{\displaystyle}
\newcommand{\M}{\mathcal{M}}
\newcommand{\E}{\mathcal{E}}
\renewcommand{\L}{\mathcal{L}}
\newcommand{\J}{\mathcal{J}}
\DeclareMathOperator{\new}{new}
\DeclareMathOperator{\Morph}{Morph}
\DeclareMathOperator{\old}{old}
\DeclareMathOperator{\Sym}{Sym}
\DeclareMathOperator{\Symb}{Symb}
%\newcommand{\Sym}{\mathcal{S}{\rm ym}}
\newcommand{\dw}{\delta(w)} 
\newcommand{\dwh}{\widehat{\delta(w)}}      
\newcommand{\dlwh}{\widehat{\delta_\l(w)}} 
\newcommand{\dash}{-\!\!\!\!-\!\!\!\!-\!\!\!\!-} 
\DeclareMathOperator{\tor}{tor}  
\newcommand{\Frobl}{\Frob_{\ell}}
\newcommand{\tE}{\tilde{E}}
\renewcommand{\l}{\ell}
\renewcommand{\t}{\tau}
\DeclareMathOperator{\cond}{cond}
\DeclareMathOperator{\Spec}{Spec}
\DeclareMathOperator{\Div}{Div}
\DeclareMathOperator{\Jac}{Jac}
\DeclareMathOperator{\res}{res}
\DeclareMathOperator{\Ker}{Ker}
\DeclareMathOperator{\Coker}{Coker}
\DeclareMathOperator{\sep}{sep}
\DeclareMathOperator{\sign}{sign}
\DeclareMathOperator{\unr}{unr}
\newcommand{\sat}{\mathrm{sat}}
\newcommand{\N}{\mathcal{N}}
\newcommand{\U}{\mathcal{U}}
\newcommand{\Kbar}{\overline{K}}
\newcommand{\Lbar}{\overline{L}}
\newcommand{\gammabar}{\overline{\gamma}}
\newcommand{\q}{\mathbf{q}}
%\renewcommand{\star}{\times}
\newcommand{\gM}{\mathfrak{M}}
\newcommand{\gA}{\mathfrak{A}}
\newcommand{\gP}{\mathfrak{P}}
\newcommand{\bmu}{\boldsymbol{\mu}}
\newcommand{\union}{\cup}
\newcommand{\Tl}{T_{\ell}}
\newcommand{\into}{\rightarrow}
\newcommand{\onto}{\twoheadrightarrow}%  Surjection arrow

\newcommand{\meet}{\cap}
\newcommand{\cross}{\times}
\DeclareMathOperator{\md}{mod}
\DeclareMathOperator{\toric}{toric}
\DeclareMathOperator{\tors}{tors}
\DeclareMathOperator{\Frac}{Frac}
\DeclareMathOperator{\corank}{corank}
\newcommand{\rb}{\overline{\rho}}
\newcommand{\ra}{\rightarrow}
\newcommand{\xra}[1]{\xrightarrow{#1}}
\newcommand{\hra}{\hookrightarrow}
\newcommand{\la}{\leftarrow}
\newcommand{\lra}{\longrightarrow}
\newcommand{\riso}{\xrightarrow{\sim}}
\newcommand{\da}{\downarrow}
\newcommand{\ua}{\uparrow}
\newcommand{\con}{\equiv}
\newcommand{\Gm}{\mathbb{G}_m}
\newcommand{\pni}{\par\noindent}
\newcommand{\set}[1]{\{#1\}}
\newcommand{\iv}{^{-1}}
\newcommand{\alp}{\alpha}
\newcommand{\bq}{\mathbf{q}}
\newcommand{\cpp}{{\tt C++}}
\newcommand{\tensor}{\otimes}
\newcommand{\bg}{{\tt BruceGenus}}
\newcommand{\abcd}[4]{\left(
        \begin{smallmatrix}#1&#2\\#3&#4\end{smallmatrix}\right)}
\newcommand{\mthree}[9]{\left(
        \begin{matrix}#1&#2&#3\\#4&#5&#6\\#7&#8&#9
        \end{matrix}\right)}
\newcommand{\mtwo}[4]{\left(
        \begin{matrix}#1&#2\\#3&#4
        \end{matrix}\right)}
\newcommand{\vtwo}[2]{\left(
        \begin{matrix}#1\\#2
        \end{matrix}\right)}
\newcommand{\smallmtwo}[4]{\left(
        \begin{smallmatrix}#1&#2\\#3&#4
        \end{smallmatrix}\right)}
\newcommand{\twopii}{2\pi{}i{}}  
\newcommand{\eps}{\varepsilon}
\newcommand{\vphi}{\varphi}
\newcommand{\gp}{\mathfrak{p}}
\newcommand{\W}{\mathcal{W}}
\newcommand{\oz}{\overline{z}}
\newcommand{\Zpstar}{\Zp^{\star}}
\newcommand{\Zhat}{\widehat{\Z}}
\newcommand{\Zbar}{\overline{\Z}}
\newcommand{\Zl}{\Z_{\ell}}
\newcommand{\comment}[1]{}
\newcommand{\Q}{\mathbb{Q}}
\newcommand{\QQ}{\mathbb{Q}}
\newcommand{\GQ}{G_{\Q}}
\newcommand{\R}{\mathbb{R}}
\newcommand{\RR}{\mathbb{R}}
\newcommand{\PP}{\mathbb{P}}
\newcommand{\D}{{\mathbf D}}
\newcommand{\cC}{\mathcal{C}}
\newcommand{\cD}{\mathcal{D}}
\newcommand{\cP}{\mathcal{P}}
\newcommand{\cS}{\mathcal{S}}
\newcommand{\Sbar}{\overline{S}}
\newcommand{\K}{{\mathbb K}}
\newcommand{\C}{\mathbb{C}}
\newcommand{\CC}{\mathbb{C}}
\newcommand{\Cp}{{\mathbb C}_p}
\newcommand{\Sets}{\mbox{\rm\bf Sets}}
\newcommand{\bcC}{\boldsymbol{\mathcal{C}}}
\renewcommand{\P}{\mathbb{P}}
\newcommand{\Qbar}{\overline{\Q}}
\newcommand{\QQbar}{\overline{\Q}}
\newcommand{\kbar}{\overline{k}}
\newcommand{\dual}{\bot}
\newcommand{\T}{\mathbb{T}}
\newcommand{\TT}{\mathbb{T}}
\newcommand{\calT}{\mathcal{T}}
\newcommand{\cT}{\mathcal{T}}
\newcommand{\cbT}{\mathbb{\mathcal{T}}}
\newcommand{\cU}{\mathcal{U}}
\newcommand{\Z}{\mathbb{Z}}
\newcommand{\ZZ}{\mathbb{Z}}
\newcommand{\F}{\mathbb{F}}
\newcommand{\FF}{\mathbb{F}}
\newcommand{\Fl}{\F_{\ell}}
\newcommand{\Fell}{\Fl}
\newcommand{\Flbar}{\overline{\F}_{\ell}}
\newcommand{\Flnu}{\F_{\ell^{\nu}}}
\newcommand{\Fbar}{\overline{\F}}
\newcommand{\Fpbar}{\overline{\F}_p}
\newcommand{\fbar}{\overline{f}}
\newcommand{\Qp}{\Q_p}
\newcommand{\Ql}{\Q_{\ell}}
\newcommand{\Qell}{\Q_{\ell}}
\newcommand{\Qlbar}{\overline{\Q}_{\ell}}
\newcommand{\Qlnr}{\Q_{\ell}^{\text{nr}}}
\newcommand{\Qlur}{\Q_{\ell}^{\text{ur}}}
\newcommand{\Qltm}{\Q_{\ell}^{\text{tame}}}
\newcommand{\Qv}{\Q_v}
\newcommand{\Qpbar}{\Qbar_p}
\newcommand{\Zp}{\Z_p}
\newcommand{\Fp}{\F_p}
\newcommand{\Fq}{\F_q}
\newcommand{\Fqbar}{\overline{\F}_q}
\newcommand{\Ad}{Ad}
\newcommand{\adz}{\Ad^0}
\renewcommand{\O}{\mathcal{O}}
\newcommand{\A}{\mathcal{A}}
\newcommand{\Og}{O_{\gamma}}
\newcommand{\isom}{\cong}
\newcommand{\ncisom}{\approx}   % noncanonical isomorphism
\DeclareMathOperator{\ab}{ab}
\DeclareMathOperator{\alg}{alg}
\DeclareMathOperator{\Aut}{Aut}
\DeclareMathOperator{\Frob}{Frob}
\DeclareMathOperator{\Fr}{Fr}
\DeclareMathOperator{\Ver}{Ver}
\DeclareMathOperator{\Norm}{Norm}
\DeclareMathOperator{\Ind}{Ind}
\DeclareMathOperator{\norm}{norm}
\DeclareMathOperator{\disc}{disc}
\DeclareMathOperator{\ord}{ord}
\DeclareMathOperator{\GL}{GL}
\DeclareMathOperator{\PSL}{PSL}
\DeclareMathOperator{\PGL}{PGL}
\DeclareMathOperator{\Gal}{Gal}
\DeclareMathOperator{\SL}{SL}
\DeclareMathOperator{\SO}{SO}
\DeclareMathOperator{\WC}{WC}
\newcommand{\galq}{\Gal(\Qbar/\Q)}
\newcommand{\rhobar}{\overline{\rho}}
\newcommand{\cM}{\mathcal{M}}
\newcommand{\cB}{\mathcal{B}}
\newcommand{\cE}{\mathcal{E}}
\newcommand{\cR}{\mathcal{R}}
\newcommand{\et}{\text{\rm\'et}}

\newcommand{\sltwoz}{\SL_2(\Z)}
\newcommand{\sltwo}{\SL_2}
\newcommand{\gltwoz}{\GL_2(\Z)}
\newcommand{\mtwoz}{M_2(\Z)}
\newcommand{\gltwoq}{\GL_2(\Q)}
\newcommand{\gltwo}{\GL_2}
\newcommand{\gln}{\GL_n}
\newcommand{\psltwoz}{\PSL_2(\Z)}
\newcommand{\psltwo}{\PSL_2}
\newcommand{\h}{\mathfrak{h}}
\renewcommand{\a}{\mathfrak{a}}
\newcommand{\p}{\mathfrak{p}}
\newcommand{\m}{\mathfrak{m}}
\newcommand{\trho}{\tilde{\rho}}
\newcommand{\rhol}{\rho_{\ell}}
\newcommand{\rhoss}{\rho^{\text{ss}}}
\DeclareMathOperator{\tr}{tr}
\DeclareMathOperator{\order}{order}
\DeclareMathOperator{\ur}{ur}
\DeclareMathOperator{\Tr}{Tr}
\DeclareMathOperator{\Hom}{Hom}
\DeclareMathOperator{\Mor}{Mor}
\DeclareMathOperator{\HH}{H}
\renewcommand{\H}{\HH}
\DeclareMathOperator{\Ext}{Ext}
\DeclareMathOperator{\Tor}{Tor}
\newcommand{\smallzero}{\left(\begin{smallmatrix}0&0\\0&0
                        \end{smallmatrix}\right)}
\newcommand{\smallone}{\left(\begin{smallmatrix}1&0\\0&1
                        \end{smallmatrix}\right)}

\newcommand{\pari}{{\sc Pari}}
\newcommand{\magma}{{\sc Magma}}
\newcommand{\hecke}{{\sc Hecke}}
\newcommand{\lidia}{{\sc LiDIA}}

%%%% Theoremstyles
\theoremstyle{plain}
\newtheorem{theorem}{Theorem}[section]
\newtheorem{proposition}[theorem]{Proposition}
\newtheorem{corollary}[theorem]{Corollary}
\newtheorem{claim}[theorem]{Claim}
\newtheorem{lemma}[theorem]{Lemma}
\newtheorem{hypothesis}[theorem]{Hypothesis}
\newtheorem{conjecture}[theorem]{Conjecture}

\theoremstyle{definition}
\newtheorem{definition}[theorem]{Definition}
\newtheorem{question}[theorem]{Question}
\newtheorem{idea}[theorem]{Idea}
\newtheorem{project}[theorem]{Project}
\newtheorem{problem}[theorem]{Problem}
\newtheorem{openproblem}[theorem]{Open Problem}
\newtheorem{challenge}[theorem]{Challenge}

%\theoremstyle{remark}
\newtheorem{goal}[theorem]{Goal}
\newtheorem{remark}[theorem]{Remark}
\newtheorem{remarks}[theorem]{Remarks}
\newtheorem{example}[theorem]{Example}
\newtheorem{exercise}[theorem]{Exercise}

\numberwithin{equation}{section}
\numberwithin{figure}{section}
\numberwithin{table}{section}


% bulleted list environment
\newenvironment{bulletlist}
   {
      \begin{list}
         {$\bullet$}
         {
            \setlength{\itemsep}{.5ex}
            \setlength{\parsep}{0ex}
            \setlength{\parskip}{0ex}
            \setlength{\topsep}{.5ex}
         }
   }
   {
      \end{list}
   }
%end newenvironment

% bulleted list environment
\newenvironment{dashlist}
   {
      \begin{list}
         {---}
         {
            \setlength{\itemsep}{.5ex}
            \setlength{\parsep}{0ex}
            \setlength{\parskip}{0ex}
            \setlength{\topsep}{.5ex}
         }
   }
   {
      \end{list}
   }
%end newenvironment

% numbered list environment
\newcounter{listnum}
\newenvironment{numlist}
   {
      \begin{list}
            {{\em \thelistnum.}}{
            \usecounter{listnum}
            \setlength{\itemsep}{.5ex}
            \setlength{\parsep}{0ex}
            \setlength{\parskip}{0ex}
            \setlength{\topsep}{.5ex}
         }
   }
   {
      \end{list}
   }
%end newenvironment

\newcommand{\hd}[1]{\vspace{1ex}\noindent{\bf #1} }
\newcommand{\nf}[1]{\underline{#1}} 
\newcommand{\cbar}{\overline{c}}

\DeclareMathOperator{\rad}{rad}

\theoremstyle{definition}
\newtheorem{algor}[theorem]{Algorithm}
\newenvironment{algorithm}[1]{%
\begin{algor}[#1]\index{{\bf Algorithm}!#1}
}%
{\end{algor}}

\newenvironment{steps}%
{\begin{enumerate}\setlength{\itemsep}{0.1ex}}{\end{enumerate}}

\usepackage{color}
\usepackage{cprotect}
\usepackage{listings} 
\lstdefinelanguage{Sage}[]{Python}
{morekeywords={True,False,sage,singular},
sensitive=true}
\lstset{
  showtabs=False,
  showspaces=False,
  showstringspaces=False,
  commentstyle={\ttfamily\color{dredcolor}},
  keywordstyle={\ttfamily\color{dbluecolor}\bfseries},
  stringstyle ={\ttfamily\color{dgraycolor}\bfseries},
  language = Sage,
  basicstyle={\small \ttfamily},
  aboveskip=1em,
  belowskip=1em,
  backgroundcolor=\color{lightyellow},
  frame=single
}
\definecolor{lightyellow}{rgb}{1,1,.86}
\definecolor{dblackcolor}{rgb}{0.0,0.0,0.0}
\definecolor{dbluecolor}{rgb}{.01,.02,0.7}
\definecolor{dredcolor}{rgb}{0.8,0,0}
\definecolor{dgraycolor}{rgb}{0.30,0.3,0.30}
\definecolor{graycolor}{rgb}{0.35,0.35,0.35}
\newcommand{\dblue}{\color{dbluecolor}\bf}
\newcommand{\dred}{\color{dredcolor}\bf}
\newcommand{\dblack}{\color{dblackcolor}\bf}
\newcommand{\gray}{\color{graycolor}}

\newcommand{\dbd}[1]{\langle#1\rangle}   % make a diamond bracket d symbol

%%% Local Variables: 
%%% mode: latex
%%% TeX-master: t
%%% End: 



\renewcommand{\q}{\mathfrak{q}}
\renewcommand{\old}{\mathrm{old}}
\renewcommand{\tor}{\mathrm{tor}}
\newcommand{\f}{\mathfrak{f}}
\newcommand{\I}{\mathcal{I}}
\newcommand{\odd}{\mathrm{odd}}





\usepackage[inline]{showlabels}

\begin{document}
\prelimpages

\Title{%
    Arithmetic of Totally Split Modular Jacobians and Enumeration of Isogeny
    Classes of Prime Level Simple Modular Abelian Varieties
}
\Author{Kevin Lui}
\Year{2019}
\Program{Mathematics}

\Chair{William Stein}{Professor}{Mathematics}
\Signature{Ralph Greenberg}
\Signature{Bianca Viray}
\Signature{Yen-Chi Chen}

\copyrightpage

\titlepage
\tableofcontents
\listoftables
\abstract{%
In this thesis, we aim to give algorithms for computing two key invariants of
the modular Jacobians $J_0(N)$.

We first give methods for computing the rational torsion order of rank-0
Jacobians $J_0(N)$ that are isogenous to a product of elliptic curves. We call
these the rank-0 totally split Jacobians. The rational torsion is an important
invariant of the Generalized BSD conjecture so being able to compute the
rational torsion order will provide evidence towards this conjecture. We will
provably enumerate the set of totally split $J_0(N)$, give an algorithm for
computing the rational torsion subgroup, and later give techniques for
computing the rational torsion order for rank-0 totally split Jacobians
$J_0(N)$. 

Next we will give an algorithm for computing the rational odd-isogeny class of a
simple abelian subvariety $A$ of $J_0(N)$ for prime $N$, under mild conditions.
In particular, when the Hecke algebra of $A$ is isomorphic to a maximal order
of a number field, we are able to enumerate the odd-isogeny class. This will allow
us to build a table of isogeny graphs of simple abelian subvarieties of
$J_0(N)$.
}


\textpages

\chapter*{Preface}
\addcontentsline{toc}{chapter}{Preface}

I've have written this thesis with some accessibility in mind. In particular,
my goal was to write a thesis assessable to myself as a second graduate
student. I will assume the reader has taken a graduate level algebra sequence
and have had some exposure to complex abelian varieties and modular forms. For
learning about complex abelian varieties, I recommend the article of Rosen
in~\cite{mr89b:14029}, and for learning modular forms, I recommend the
combination of the books by Stein~\cite{stein:modform} and, Diamond and
Shurman~\cite{diamond-shurman}.

After reading the introduction, an interested expert can skip to
Chapters~\ref{chap:totally_split} and~\ref{chap:isogeny_class}, and
back-reference as needed.

Chapter~\ref{chap:algorithms} has been largely taken from the source code of
\sage~\cite{sage} and a forthcoming paper by Hao Chen, myself, and William Stein. For the
most part, the algorithms were created and implemented into \sage and \magma~by
William Stein. This chapter is included in this thesis so I am not relying on
currently unpublished work.

The code used to generate the data of this thesis was done in \sage~\cite{sage}
will be available here~\url{kevinlui.org/thesis}.

\chapter{Introduction}%
\label{chap:intro}

Let $X_0(N)$ be the modular curve whose non-cuspidal points parameterize
complex elliptic curve with some additional $N$-torsion data. Let
$J_0(N)=\Jac(X_0(N))$ be the Jacobian variety of $X_0(N)$. The study of
$J_0(N)$ has yielded many deep results in arithmetic geometry.

\section{Rational torsion points}%
\label{sec:rational_torsion_points}

% The rational torsion subgroup is an 

\section{Rational isogeny class}%
\label{sec:rational_isogeny_class}

%TODO 

\chapter{Theoretical Preliminaries}%
\label{chap:prelim}

\section{Modular Abelian Varieties}

In Chapter~\ref{chap:isogeny_class}, we will discuss enumerating the isogeny
class of simple abelian subvarieties of $J_0(N)$. This leads us to define a
class of abelian variety containing the subquotients of $J_0(N)$.

\begin{definition}
    \label{defn:modabvar}
    Let $A$ be an abelian variety over $\QQ$. Then $A$ is \emph{modular} if
    there exists some $N$ and a finite degree morphism $\phi:A\to J_1(N)$.

    Note that modularity is closed under isogenies, subvarieties, and products.
\end{definition}

The natural quotient of $X_1(N)\to X_0(N)$ induces a finite degree map
$J_0(N)\to J_1(N)$. The kernel of this map, $\Sigma(N)$, is finite and is the
Shimura subgroup of $J_0(N)$. Therefore, all abelian varieties isogenous to a
simple subvariety of $J_0(N)$ are modular.


\section{Galois Representations}

\subsection{$\ell$-adic representation}%
\label{sub:_ell_adic_representation}

Let $B$ be a simple abelian variety over $\QQ$ with an embedding of a number
field $E$ into its endomorphism algebra $\End(B)\otimes \QQ$. By functoriality,
$E$ acts on the on the $\dim B$-dimensional vector space $\mathrm{Lie}(B)$.
Therefore, there is a divisibility bound $\deg E\mid \dim B$. The simple
abelian varieties that achieve this bound are called
\emph{$\GL_2$-type}~\cite[\S 2]{ribet:abvars}. To motivate this definition,
observe that $B[n]\cong (\ZZ/n)^{2\dim B}$ so $V_\ell(B)$ is a rank-2 $E\otimes
\QQ_\ell$-module. Therefore, the $G_\QQ$-action on $V_\ell(B)$ gives the
representation $G_\QQ\to \GL_2(E\otimes \QQ_\ell)$. More generally, we say an
abelian variety is \emph{$\GL_2$-type} if it is isogenous to a product of
simple $\GL_2$-type abelian varieties.

Modular abelian varieties provide a large class of examples of such varieties.
Let $f=\sum a_n q^n$ be a newform of $S_2(\Gamma_1(N))$ and
$K_f=\QQ(\ldots,a_n,\ldots)$ be the \emph{Hecke eigenvalue field} of $f$.
Shimura~\cite[Theorem 7.14]{shimura:intro} associates to $f$ an abelian variety
$A_f$ defined over $\QQ$ and an embedding of $K_f$ into the endomorphism
algebra $\End(A_f)\otimes \QQ$ with $\deg K_f = \dim A_f$.

This shows that every modular abelian varieties is of $\GL_2$-type so a natural
question is whether every $\GL_2$-type abelian variety is modular. Assuming Serre's
modularity conjecture~\cite[3.2.4]{serre:conjectures}, Ribet~\cite[Thm.
4.4]{ribet:abvars} shows that the every $\GL_2$-type abelian variety is
modular. The Serre's modularity has now been proved~\cite{}, so
\begin{theorem}[Khare, Wintenberg]
    Every $\GL_2$-type abelian variety over $\QQ$ is modular.
\end{theorem}

\section{Natural maps}%
\label{sec:natural_maps}

In this section, we will describe some common maps encountered in the study of
modular abelian varieties. We will only give the theoretical description of
these maps. An explicit description that is more readily used for computations
will be described in Section~\ref{sec:alg_natural_maps}.

\subsection{Degeneracy Maps}%
\label{sub:degeneracy_maps}

Let $L\mid N$, $LM=N$, and $t_1,\ldots,t_r$ be the divisors of $M$ in
increasing order. Then, for each $t_i$, there are degeneracy maps relating the
modular curves, forms, and Jacobians of level $L$ and $N$. For ease of
exposition, we will present the $\Gamma_0(N)$ case. The same arguments
generalize easily to $\Gamma_1(N)$.

We first give the more algebraic construction. Recall that the non-cuspidal
points of $X_0(N)$ correspond to elliptic curves with some $N$-torsion data.
The degeneracy map to $X_0(L)$ will forget some of this data. More precisely,
on the non-cuspidal points
\begin{equation}
    \label{eq:degen_moduli}
    \begin{split}
        d_{L,t}: Y_0(N)  \to Y_0(L) \\
        [ E, C_N ]       \mapsto [ E/C_t, C_L ' ],
    \end{split}
\end{equation}
where $C_N$ is a cyclic subgroup of $E$ of order $N$, $C_t$ is the unique
cyclic subgroup of $C_N$ of order $t$, and $C_L '\subseteq C_N/C_t$ is the
unique subgroup of order $L$. By~\cite[Chap. 1, Prop. 6.8]{hartshorne},
$d_{L,t}$ extends uniquely to a map (which we give the same name). By Pic and
Alb functoriality, this induces the maps $d_{L,t} ^*:J_0(L)\to J_0(N)$, $d_{L,
t*} :J_0(N)\to J_0(L)$, respectively.

There is a more complex-analytic construction. Here we will view the
non-cuspidal points of $X_0(N)$ as $\Gamma_0(N)\setminus \h$. Under this
interpretation, \eqref{eq:degen_moduli} becomes
\begin{equation}
    \label{eq:degen_analytic}
    \begin{split}
        d_{L,t}: \Gamma_0(N)\setminus \h \to \Gamma_0(L)\setminus \h \\
        \Gamma_0(N)z\mapsto \Gamma_0(L) tz.
    \end{split}
\end{equation}

There is both a contravariant (Picard) and covariant (Albanese) construction of
the Jacobian. Consequently, we have the following diagram
\[
    here is a diagram.
\]
The bottom map is $p+1$ because of this!! %TODO

% TODO
% \subsection{Hecke operators}%
% \label{sub:hecke_operators}

% The Hecke operators are given by this thing.

\subsection{Endomorphism Ring}

Let $A$ be a simple abelian variety so $A\sim A_f$ for some newform $f=\sum
a_n q^n$. Let $K_f=\QQ(\ldots,a_n,\ldots)$ be the Hecke eigenvalue field of
$f$. By~\cite[Prop. 7.14]{shimura:intro}, the endomorphism algebra
$\End(A_f)\otimes \QQ$ is isomorphic to $K_f$ and is generated as a
$\QQ$-vector space by the Hecke operators. It follows that both the
endomorphism ring and the Hecke algebra of $A_f$ are isomorphic to orders of
$K_f$. Since $A\sim A_f$, the endomorphism ring of $A$ is also isomorphic to an
order of $K_f$.


\subsection{Optimal Elliptic Curves}%
\label{sub:optimal_elliptic_curves}

Let $E$ be an elliptic curve of conductor $N$. By the Modularity Theorem for
elliptic curves~\cite{breuil-conrad-diamond-taylor}. There exists a surjection
$\phi:J_0(N)\onto E$. In general, $\ker\phi$ is not connected. There does
$\phi':J_0(N)\onto E'$ with $E'$ isogenous to $E$ so that $\ker\phi'$ is
connected. By multiplicity one, this $E'$ is unique in its isogeny
class. This leads to the definition of optimal quotients.

\begin{definition}%
    \label{defn:optimal_quotient}
    Let $J$ be the Jacobian of a curve. Then an abelian variety $A$ is an
    \emph{optimal quotient} of $J$, if there exists a surjective morphism $J
    \twoheadrightarrow A$ with connected kernel. Equivalently, $A$ is an
    optimal quotient of $J$ is it is the quotient of $J$ by an abelian
    subvariety.
\end{definition}

If $A$ is an optimal quotient of $J$, then $A^\vee$ can be embedded as a
subvariety of $J$. This particularly useful in the case where $J$ is totally
split because both elliptic curves and Jacobians are self-dual.

\begin{proposition}
    \label{prop:optimal}
    Let $A$ be an abelian variety and $J$ a modular Jacobian. Then $A$ is an
    optimal quotient of $J$ if and only if there exists an injection of
    $A^\vee$ into $J$.

    Suppose $A$ is self-dual. Then
    $A$ is an optimal quotient of $J$ if and only if it is an abelian
    subvariety of $J$.
\end{proposition}
\begin{proof}
    Suppose $A$ is an optimal quotient of $J$. Then by dualizing the sequence
    of abelian varieties~\cite[Proposition 2.4.2]{lange-birkenhake:complex}
    \[
        0 \to C \to J \to A \to 0 \qquad
        0\to A^\vee \to J^\vee \to C^\vee \to 0,
    \]
    there is an injection of $A^\vee\hookrightarrow J^\vee$.

    Conversely, if $A^\vee$ injects into $J$. Then by the Poincar\`{e} Reducibility
    Theorem, the quotient of $J$ by $A^\vee$ is again an abelian variety $C$.
    Then by dualizing the sequence of abelian varieties
    \[
        0 \to A^\vee \to J \to C \to 0 \qquad
        0 \to C^\vee \to J^\vee \to A \to 0,
    \]
    there is a surjection of $J^\vee\cong J\twoheadrightarrow A$ with connected
    kernel $C^\vee$.
\end{proof}

Using the machinery of modular symbols, Cremona has created large databases of
elliptic curves and their invariants, including the $J_0(N)$-optimal curve in
each isogeny class. In this database, the curves are labeled $NXT$, where $N$
is the conductor, $X$ is the isogeny class within that conductor, and $T$ is
the isomorphism class with that isogeny class. When $T=1$, then that curve is
the $J_0(N)$-optimal curve with its isogeny class. For example, the curve
$15a1$ is the $J_0(N)$-optimal curve within the isogeny class $15a$.

We will soon also be interested in $J_1(N)$-optimal elliptic curves. Though the
$J_0(N)$-optimal and $J_1(N)$-optimal curves often agree, this is not always
the case. We can use Algorithm~\ref{alg:weierstrass} to determine the
Weierstrass equation of the $J_1(N)$-optimal curve. Alternatively, we can use
Stevens Conjecture~\cite[Conjecture II]{stevens:param} which states that within
an isogeny class, the $J_1(N)$-optimal curve is the curve of minimal Faltings
height. This conjecture is still open but some progress has been made. Stein
and Watkins~\cite[\S 3]{stein-watkins:ns} have proved Stevens conjecture for
isogeny classes of prime conductor. Vatsal~\cite[Thm.
1.11]{vatsal:multiplicative} has proved Stevens conjectures for isogeny classes
containing an elliptic curve $E$ such that for some prime $\ell\geq 7$,
$E[\ell]$ is reducible and $E$ is ordinary at $\ell$.

\section{Old subvariety}%
\label{sec:old_subvariety}

Let $L$ be a proper divisor of $N$ and $t_1,\ldots,t_r$ be divisors of $N/L$ in
some order. For each divisor $t_i$, the degeneracy map
(\ref{sub:degeneracy_maps} $\delta_i$ gives an
embedding $\delta_{i} ^*:J_0(L)\to J_0(N)$. We now gather all the degeneracy
maps coming from $J_0(N)$ to define
\[
    \Phi_L ^N = \prod_{i=1} ^r \delta_{t_i} ^* : J_0(L)^r \to J_0(N).
\]
The \emph{old subvariety} of $J_0(N)$ is $\sum_{L\mid N} \Im\Phi_L ^N$ and the
\emph{$N/L$-old subvariety} is $\Im\Phi_L ^N$.

There is a strong relationship between $\ker \Phi_L ^N$ and
\[
    \Sigma(L)_0 ^r =\{(x_1,\ldots,x_r)\in \Sigma(L)^r: \sum x_i = 0\}.
\]
Because degeneracy maps agree on the Shimura subgroup~\cite[Theorem
4]{ling-oesterle:shimura}, $\Sigma(L)_0 ^r \subseteq \ker\Phi_L ^N$. The reverse
equality was established by Ribet when $M$ a prime coprime to $L$ and was
generalized by Ling to:
\begin{theorem}[{\cite[Prop. 1]{ribet:raising}\cite[Thm. 3]{ling:shimura}}]
    \label{thm:ribet-ling}
    Let $L$ and $M$ be relatively prime integers with $M$ squarefree. Let
    \[
        \Phi_L ^N = \prod_{i=1} ^r d_{L,t_i} ^* : J_0(L)^r \to J_0(LM)
    \]
    be as defined above. Then
    \begin{enumerate}
        \item
            If $L$ is odd or $M$ is prime, then $\Sigma(L)_0 ^r=\ker\Phi_L ^N$.
        \item
            If $L$ is even and $M$ is not a prime, then $[\ker\Phi_L ^N:
            \Sigma(L)_0 ^r]$ is a power of 2.
    \end{enumerate}
\end{theorem}

However, when $\ker\Phi_L ^N = \Sigma(L)_0 ^r$, there is an useful direct sum
decomposition of $\Im\Phi_L ^N$, especially when $J_0(L)$ is an elliptic curve.

\begin{corollary}
    \label{cor:elliptic_decomp}
    Suppose that $\ker\Phi_L ^N = \Sigma(L)_0 ^r$ and $E=J_0(L)$ is an elliptic
    curve. Then there is a $\QQ$-isomorphism
    \[
        \Im\Phi_L ^N \cong E \times F^{r-1},
    \]
    where $F$ is the $J_1(L)$-optimal curve in the isogeny class of $E$.
\end{corollary}
\begin{proof}
    This follows from Proposition~\ref{prop:decomp} and
    Proposition~\ref{prop:optimal}.
\end{proof}

This decomposition is particularly useful because the $J_0(L)$-optimal curves
are identified in Cremona's table and the $J_1(L)$-optimal curves can be
identified using Steven's conjecture. For example, using this decomposition,
$J_0(22) = J_0(22)_\old \cong E\times F$, where $E$ is the $J_0(11)$-optimal
curve and $F$ is the $J_1(11)$-optimal curve (in this case, $E=J_0(11)$ and
$F=J_1(11)$). So computing the BSD invariants of $J_0(22)$ amounts to computing
the BSD invariants of the elliptic curves $E$ and $F$.

\begin{proposition}
    \label{prop:decomp}
    Suppose $\ker\Phi_L ^N=\Sigma(L)_0 ^r$. Then there is a $\QQ$-isomorphism
    \[
        \Im\Phi_L ^N \cong J_0(L)\times \Im(J_0(L)\to J_1(L))^r.
    \]
\end{proposition}
\begin{proof}
    Let $e$ be the exponent of $\Sigma(L)$. For $i=1,\ldots,r-1$, let $m_i$ be
    integers so that $m_i \equiv 1 \pmod{e}$. Define $D_i:J_0(L)\to J_0(N)$ by
    \[
        D_i =
        \begin{cases}
            \delta_i                    & \text{if } i = 1\\
            \delta_i - m_i \delta_{i-1} & \text{if } 2\leq i \leq r.
        \end{cases}
    \]
    We first show $\Im\Phi_L ^N = \bigoplus_{i=1} ^r \Im D_i$.

    Define $\Phi_L ^N =\prod_{i=1} ^r D_i : J_0(L)^r \to J_0(N)$. We have that
    $\Phi_L ^N = \Phi_L ^N \circ T$, where
    \begin{align*}
        T:J_0(L)^r & \to J_0(L)^r \\
        (x_1,x_2,\ldots,x_r)&\mapsto (x_1-m_2x_2,x_2-m_3 x_3,\ldots,x_{r-1}-m_r
        x_r, x_r).
    \end{align*}
    The matrix associated to $T$ is consisted of $1$'s along the diagonal and
    $m_i$'s along the superdiagonal. The determinant is 1 so
    \[
        \sum_{i=1} ^r \Im D_i = \Im \Phi_L ^N.
    \]

    The goal is to now show that this sum is direct. Let $y_1,\ldots,y_r\in
    J_0(L)$. Suppose
    \begin{equation}
        \label{eq:phiLprime}
        \Phi_L ^N (y_1,\ldots,y_r) =D_1(y_1)+ \cdots +D_r(y_r)=0.
    \end{equation}
    Then that $T(y_1,\ldots,y_r)\in \ker\Phi_L ^N=\Sigma(L)_0 ^r$.  This
    immediately implies that $y_r\in \Sigma(L)$ and then, by repeated
    back-substitution, $y_i\in \Sigma(L)$ for $i=1,\ldots,r$. Since $m_i \equiv
    1 \pmod{e}$,
    \[
        T(y_1,\ldots,y_r)=(y_1-y_2,y_2-y_3,\ldots,y_{n-1}-y_n,y_n).
    \]
    Since $T(y_1,\ldots,y_r)\in \Sigma(L)_0 ^r$,
    \[
        (y_1-y_2)+(y_2-y_3)+\cdots+(y_{n-1}-y_n)+y_n =0.
    \]
    This implies that $y_1=0$ so $D_1(y_1)=0$. Moreover, for $i=2,\ldots,r$,
    \begin{equation*}
        D_i(y_i)
            = \delta_{i-1}(-y_i) + \delta_i (y_i)
            = \Phi_L ^N (0,\ldots,0,-y_i,y_i,0,\ldots,0)
            = 0,
    \end{equation*}
    where the last equality follows from the fact
    $(0,\ldots,0,-y_i,y_i,0,\ldots,0)\in \Sigma(L)_0 ^r$. Therefore, the terms
    in~\eqref{eq:phiLprime} are trivial so $\sum \Im D_i$ is direct.

    It remains to show $D_1(J_0(L))\cong J_0(L)$ and $D_i(J_0(L))\cong
    \Im(J_0(L)\to J_1(L))$ for $i\geq 2$. Notice that
    $D_1(x)=\Phi_L ^N(x,0,\ldots,0)$ and $D_i(x)=\Phi_L ^N(\ldots,-m_ix,x,\ldots)$.
    Since $\ker\Phi_L ^N=\Sigma(L)_0 ^r$, $\ker D_1=0$ and $\ker D_i=\Sigma(L)$
    for $i\geq 2$, as desired.
\end{proof}


\section{Connectedness of Hecke Algebra}%
\label{sec:connectedness_of_hecke_algebra}

Mazur~\cite[Prop. 10.6]{mazur:eisenstein} proves that the Hecke Algebra
$\TT(N)$ for $J_0(N)$ with $N$ prime is connected. This was done by showing any
direct product decomposition of $J_0(N)$ with $N$ prime contradicts the
irreducibility of the $\theta$-divisor. This was surprising to the author
as~\ref{cor:elliptic_decomp} gives a direct sum decomposition of $J_0(22)$. In
this section, we dissect Mazur's proof and give a mild generalization.
Moreover, we will explain why Mazur's argument fails in the composite case.

\subsection{General results for semistable Jacobians}

We will begin with some fairly general results. So assume $J$ is a semistable
Jacobian defined over $\QQ$ that is possibly not $J_0(N)$. By~\cite[Corollary
1.4]{ribet:endo}, isogenies, endomorphisms, and abelian subvarieties (a priori
defined over $\QQbar$) are defined over $\QQ$. In this section, we use this
fact freely and will make no reference to the field of definition.

\begin{theorem}
    \label{thm:theta_irred}
    Any Jacobian $J$ taken with its principal polarization cannot be decomposed
    into a nontrivial direct sum of principally polarized abelian varieties.
\end{theorem}
\begin{proof}
    Any such decomposition will give a decomposition of the $\Theta$-divisor
    attached to $J$ which contradicts the irreducibility of the
    $\Theta$-divisor~\cite[\S 4(a)]{kempf:riemann}.
\end{proof}

\begin{lemma}
    \label{lem:decomp_isogeny}
    Suppose $J$ decomposes nontrivially as the direct sum of abelian subvarieties
    $A\oplus B$. Then $A$ must share an isogenous factor with $B$.
\end{lemma}
\begin{proof}
    We proceed via contradiction. Suppose $A$ and $B$ share no isogenous
    factors. Let $\lambda:J\to \hat{J}$ be the principal polarization induced
    by its $\Theta$-divisor. Since $A$ shares no isogenous factors with $B$,
    $\lambda(A)=\hat{A}$ so $\lambda|_A$ is a polarization of $A$. Similarly,
    $\lambda|_B$ is a polarization of $B$. This now contradicts
    Theorem~\ref{thm:theta_irred}.
\end{proof}

\begin{lemma}
    \label{lem:faithful}
    Let $R$ be a ring acting faithfully on $J$. Let $S=\{A_1,\ldots,A_k\}$ be a
    set of representatives of the isogeny class of subvarieties of $J$.
    Suppose that for all $A\in S$, and every idempotent $r\in R$, either
    $rV_A=0$ or $rV_A=V_A$. Then $\Spec R$ is connected.
\end{lemma}
\begin{proof}
    Recall that $\Spec R$ is connected if and only if $R$ contains an
    idempotent $r$ different from $0$ or $1$. We proceed via by contradiction.
    Let $r\in R$ be an idempotent different from $0$ or $1$. Then $K = rK
    \oplus (1-r) K$ is a decomposition of $K$ into subvarieties. Moreover, this
    decomposition is nontrivial because $R$ acts faithfully.

    Let $S_1=\{A\in S:rV_A=V_A\}$ and $S_2=\{A\in S:(1-r)V_A=V_A\}$. Observe
    that $r$ and $1-r$ kill every element of $S_2$ and $S_1$, respectively, so
    $S=S_1\sqcup S_2$. We can rewrite the previous decomposition as
    \[
        J
        = rJ \oplus (1-r)J
        = \left(\sum_{A\in S_1} V_A \right)
        \oplus \left(\sum_{A\in S_2} V_A \right).
    \]
    The big summands share no isogenous factors which contradicts
    Lemma~\ref{lem:decomp_isogeny}.
\end{proof}

\subsection{Application to modular abelian varieties}

The goal now is to apply Lemma~\ref{lem:faithful} to the case of semistable
$J$ and subrings of the Hecke algebra $\TT$. Recall that the action of the
Hecke algebra on $J$ is faithful.

\begin{proposition}\label{prop:new_jacobian_connected}
    Suppose that $J=J_{\new}$. Then $\Spec\TT$ is connected.
\end{proposition}
\begin{proof}
    When all subvarieties are new, they appear with multiplicity 1 so the
    conditions of Lemma~\ref{lem:faithful} are automatic.
\end{proof}

\begin{proposition}
    Suppose $\TT'$ is the anemic Hecke algebra for $J$. Then $\Spec\TT'$ is
    connected.
\end{proposition}
\begin{proof}
    By Lemma~\ref{lem:faithful}, it suffices to show that for any newform
    $f$, and $r\in \TT'$, $rV_f=V_f$ or $rV_f=0$. Fix a newform $f$ of level
    $L$ and $r\in \TT'$. We now abuse notation by overloading $T_\ell$ and $r$
    as operators on $A_f$, $A_f^s$, and $J$. In the Formulaire section
    of~\cite{ribet:old}, for $\ell\nmid N$, $T_\ell\circ \phi=\phi\circ T_\ell$
    when $N$ is prime. However, this is also true with $N$ squarefree since
    $T_\ell$ commutes with the pushforward and pullback of any degeneracy map
    $\delta_d$ with $d\mid N$. So for any prime $\ell\nmid N$,
    \[
        T_\ell(\Phi_f(x_1,\ldots,x_s))
        = \Phi_f(T_\ell(x_1),\ldots,T_\ell(x_s)).
    \]
    It follows that
    \[
        r(\Phi_f(x_1,\ldots,x_s))
        = \Phi_f(rx_1,\ldots,rx_s).
    \]
    Therefore,
    \[
        rV_f = r(\Phi_f(A_f)) = \Phi_f(rA_f,\ldots,rA_f)
    \]
    but $A_f$ is simple so either $rA_f=A_f$ or $rA_f=0$. Therefore, by
    Lemma~\ref{lem:faithful}, $\Spec\TT'$ is connected.
\end{proof}

\begin{example}
    When $N$ is prime, the argument Mazur~\cite[Prop. 10.6]{mazur:eisenstein}
    gives is essentially the same as
    Proposition~\ref{prop:new_jacobian_connected}. We see that this fails for
    $J=J_0(22)$ since $J=d_1(J_0(11))+d_2(J_0(11))$ so the factors appear with
    multiplicity greater than 1.

    Using \sage, we can show that the Hecke algebra of $J_0(22)$ is isomorphic
    to $\ZZ[i]$ and is thus connected. On the other hand,
    by~\ref{cor:elliptic_decomp}, $J_0(22)$ is the direct product of elliptic
    curves so the endomorphism ring is not connected.
\end{example}


\section{Subvarieties of $J_0(N)$}%
\label{sec:subvarieties_of_j_0_n_}

In this section, we will prove that every abelian subvariety of $J_0(N)$ is the
image of degeneracy map and discuss some interesting questions arising from
this.

\begin{proposition}
    \label{prop:integral_degen}
    Let $A$ be a simple subvariety of $J_0(N)$. There exists a divisor $L$ of
    $N$ and a newform $f$ of level $L$ such that $A_f \sim A$. Let
    $d_1,\ldots,d_r$ be the full collection of degeneracy maps from $J_0(L)$ to
    $J_0(N)$. Then there exists integers $n_1,\ldots,n_r$ such that $S:=\sum
    n_i d_i|_{A_f}: A_f\to A$ is an isogeny from $A_f$ to $A$. Note that $S$ is
    defined over $\QQ$.
\end{proposition}
\begin{proof}
    Let $V_f=\sum_{i=1} ^r d_i(A_f)$ and $\Phi:A_f ^r \to V_f$ be defined by
    $D(x_1,\ldots,x_r) = d_1(x_1)+\cdots+d_r(x_r)$. Let $K_f$ be the Fourier
    coefficient field of $f$. Since $A$ is an abelian subvariety of $V_f$, there
    exists $M\in \End_0(V)\cong M_r(\End_0(A_f)) = M_r (K_f)$ such that $\Im M
    = A$. Let $i:A_f\to A_f ^r$ be the inclusion map into the first coordinate.
    Then there exists $U\in \Aut(A_f ^r)=\GL_r (K_f)$ such that,
    \[
        \begin{tikzcd}
            A_f \arrow[r,"i"] &
            A_f ^r \arrow[r, dotted, "U"] &
            A_f ^r \arrow[r, "D"] &
            V_f \arrow[r, "M"] &
            A,
        \end{tikzcd}
    \]
    the map $T:=M \circ \Phi \circ U\circ i:A_f\to A\in \Hom_0(A_f, A)$ is
    nonzero. Since degeneracy maps are $K_f$-linear, there exists coefficients
    $a_1,\ldots,a_r\in K_f$ such that $T = \sum a_i d_i$. Now there exists
    $b\in \ZZ^*$ such that $T':=bT\in \Hom(A_f, A)$ is nonzero and hence an
    isogeny. Since $T'(\Lambda_{A_f})\subset \Lambda_A$, $T'=\sum q_i d_i$
    for $q_i\in \QQ$. Finally, there exists $w\in \ZZ^*$ such that
    $S:=wT'=\sum n_i d_i$ with $n_i\in \ZZ$.
\end{proof}


\begin{proposition}
    \label{prop:star_eisenstein}
    Suppose $(L, N)$ satisfies the $\star$-condition. Then $\ker S\subseteq
    \Sigma(L)$. In particular, this isogeny is Eisenstein.
\end{proposition}
\begin{proof}
    Suppose $x\in \ker S$. Then $(n_1 x, \ldots, n_r x)\in \ker\Phi_L ^N =
    \Sigma(L)_0 ^r$. Hence, $n_i x\in \Sigma(L)$ for all $i$. Since
    $\gcd(n_1,\ldots,n_r)=1$, $x\in \Sigma(L)$.
\end{proof}

\begin{question}
    \label{question:iso}
    Let $A$ be a simple subvariety of $J_0(N)$ for some $N$. Let $\mathcal{I}$
    be the isogeny class of $A$. For all $B\in \mathcal{I}$, is $B$ a
    subvariety of $J_0(M)$ for some $M$?
\end{question}

\begin{example}
    Suppose $I$ is a Neumann-Setzer isogeny class of elliptic curves of
    conductor $p$. Then there are exactly 2 elliptic curves $E_0$ and $E_1$,
    where $E_0$ is $J_0(p)$-optimal and $E_1$ is $J_1(p)$-optimal
    (Stein-Watkins). Viewing $E_0$ as a subvariety of $J_0(p)$ and taking $q$
    to be a prime distinct from $p$, we have
    \[
        E_1 \isom \Im (d_1-d_q)(E_0)\subseteq J_0(pq),
    \]
    where $d_1, d_q:J_0(p)\to J_0(pq)$ are the natural degeneracy maps. This
    gives an affirmative answer to Question~\ref{question:iso} in the
    Neumann-Setzer isogeny class case.
\end{example}


\begin{question}
    Let $M$ be a $\Gal(\QQbar/\QQ)$-submodule of $J_0(L)$. Does there exists
    $N$ and an integral linear combination of degeneracy maps $S:=\sum n_i d_i$
    such that $\ker S=M$?
\end{question}

\begin{question}
    Is $\ker S$ always Eisenstein? This is true under the $\star$-condition.
\end{question}

\chapter{Algorithms for modular abelian varieties}%
\label{chap:algorithms}

In this chapter, we will review algorithms on modular abelian varieties. For
the most part, the goal is to reduce the computation problems to linear algebra
on modular symbol spaces.

\section{Defining Data}%
\label{sec:defining_data}

We begin by giving an explicit description of modular abelian varieties that is
suitable for linear algebraic computations. We represent modular abelian
varieties explicitly as follows. Let $A$ be a modular abelian variety and
$\varphi:A\to J$ a finite degree morphism. Let $B$ be the image of $A$ in $J$.
By dualizing, there is an isogeny $B$ to $A$ with kernel $G$ such that $A\isom
B/G$.
\begin{equation*}
    \label{eq:defining_data}
    \begin{tikzcd}
        0 \arrow[r]
            &
            G \arrow[r]
            &
            B \arrow[r]\arrow[rd]
            &
            A \arrow[d, "\varphi"]
            \arrow[r]
            &0
            \\
            &
            &
            &
            J
            &
    \end{tikzcd}
\end{equation*}
So we can represent any modular abelian variety $J$ by giving $G\subseteq
B\subseteq J$ all defined over $\QQ$.

\begin{itemize}
    \item
        We will represent $J$ by giving a modular
        symbol basis for $H_1(J, \ZZ)$ and $H_1(J, \QQ) = H_1(J, \ZZ)\otimes
        \QQ$ (Section~\ref{sec:modular_symbols}).
    \item
        We will represent $B$ as an abelian subvariety of $J$ as follows. The
        inclusion of $B\subseteq J$ induces an inclusion of rational homology
        $V=H_1(B, \QQ)$ into $H_1(J, \QQ)$ and $B$ is determined by this
        inclusion. Therefore, we specify $B$ by a basis in reduced echelon form
        for the subspace $V$. Of course, not every subspace of $H_1(J, \QQ)$
        corresponds to an abelian subvariety. In
        Algorithm~\ref{sec:decomp_verify}, we give a method for determining
        exactly when a subspace of $H_1(J, \QQ)$ corresponds to the rational
        homology of an abelian subvariety.
    \item
        We will represent $G$ as a finite subgroup of $B$ as follows. Let
        $\Lambda=V\cap H_1(J, \ZZ)$. Then the torsion subgroup of $B$ is given
        by $B(\CC)_\tor=V/\Lambda$. Therefore, we represent $G$ as specifying a
        Hermite normal form basis for the lattice $L$ with $\Lambda \subseteq L
        \subseteq V$.
\end{itemize}

Therefore, we can represent any modular abelian variety $A$ with the triplet
$(L, V, J)$, denoted $A\sim (L, V, J)$, with the properties $J=J_1(N)$
for some $N$, $V\subseteq H_1(J, \QQ)$, $L$ a lattice containing $V\cap H_1(J,
\ZZ)$. Since the $\QQ$-span of $L$ is $V$, $V$ can be recovered from $L$ so $A$
can also be specified by $(L, J)$, denote $A\sim (L, J)$.

In this Chapter, it'll be convenient to allow other Jacobian besides $J_1(N)$.
If $H$ is congruence subgroup with $\Gamma_1(N)\subseteq H\subseteq
\Gamma_0(N)$, we will denote $\Jac(X_H(N))$ by $J_H(N)$. For the purposes of
this thesis, we usually take $J_H(N)$ to mean $J_0(N)$ or $J_1(N)$.

\section{Modular Symbols}
\label{sec:modular_symbols}

The computational tools presented in this chapter will be built on top of the
theory of modular symbols. As mentioned in Section~\ref{sec:defining_data},
modular symbols give a finite presentation of $H_1(J, \ZZ)$. For instance, a
$\ZZ$-basis for $H_1(J_0(15), \ZZ)$ is given by the set of Manin symbols
$\{(1,8), (1,9)\}$. For this thesis, we will not need to delve into the
underlying theory of modular symbols and we will take it as a blackbox. We
instead refer the reader to~\cite[\S 3, \S 8, \S 9]{stein:modform}.

Similarly, the Hecke operators~\cite[\S 8.3]{stein:modform}, the degeneracy
maps~\cite[\S 8.6]{stein:modform}, and the star involution~\cite[\S
8.5]{stein:modform} can all be defined via modular symbols and thus induce maps
on modular Jacobians.

\section{Finite Subgroups}

The goal of this section is to establish some background for computing with
finite subgroups of modular abelian varieties. We begin by explaining how we
will present the data of a finite subgroup. We then go over some basic
arithmetic performed on finite subgroup. Lastly, we will discuss computing the
cuspidal and Shimura subgroup which will be used extensively in the upcoming
chapters.

\subsection{Defining Data}

Let $A=(V, L, J)$ be a modular abelian variety. A finite subgroup $G$ of $A$
can be specified by giving a defining lattice $\mathcal{L}$ such that
$\mathcal{L}/L = G$.

Given 2 finite subgroups $G_1=(\mathcal{L}_1, A)$ and $G_2=(\mathcal{L}_2, A)$,
a map $\varphi: G_1\to G_2$ can be given as a map on the defining lattices.

\subsection{Intersection of Finite Subgroups}
\label{sec:finitegroup_intersection}

Let $G=(\mathcal{L}_1, A)$ and $H=(\mathcal{L}_2, A)$ be finite subgroups of an
modular abelian variety $A=(L, V, J)$. Let $\mathcal{L}_1 ' = \mathcal{L}_1+L$
and $\mathcal{L}_2 ' = \mathcal{L}_2 + L$. Then the intersection $G\cap H$ is
the group $(\mathcal{L}, A)$, where $\mathcal{L}=\mathcal{L}_1\cap
\mathcal{L}_2 \cap V$.

\subsection{Sums of Finite Subgroups}

Let $G=(\mathcal{L}_1, A)$ and $H=(\mathcal{L}_2, A)$ be finite subgroups of
$A=(L, V, J)$. The sum is given by $G+H=(\mathcal{L}_1 + \mathcal{L}_2+L, A)$.

\subsection{Quotients of Finite Subgroups}

Let $G=(\mathcal{L}_1, A)$ and $H=(\mathcal{L}_2, A)$ be finite subgroups of
$A=(L, V, J)$ with $H\subseteq G$ so $\mathcal{L}_2\subseteq \mathcal{L}_1$.
The quotient is given by $G/H=(\mathcal{L}_1/\mathcal{L}_2, A/H)$, where the
computation of $A/H$ is given in
Section~\ref{sub:quotienting_by_finite_subgroup}.

\subsection{Cuspidal subgroup and rational cuspidal subgroup}%
\label{sub:cuspidal}

In this section, we will review the computation of the cuspidal and rational
cuspidal subgroups. 

% The Generalized Ogg Conjecture states that the rational
% cuspidal subgroup of $J_0(N)$ is equal to $J_0(N)(\QQ)_\tor$. Consequently, the
% computation of the rational cuspidal subgroup will be a crucial input to
% Chapter~\ref{chap:totally_split} where we attempt to verify the Generalized Ogg
% Conjecture for totally split $J_0(N)$. Moreover, along with the Shimura
% subgroup (Section~\ref{sec:shimura_subgroup}), when $N$ is prime, the cuspidal
% subgroup will be cyclic Galois subgroups of $J_0(N)(\QQbar)_\tor$ and will
% hence correspond to nontrivial isogenies. This idea will be applied in
% Chapter~\ref{chap:isogeny_class}, where we attempt to enumerate the odd-degree
% isogeny of simple subvarieties of $J_0(N)$ for $N$ prime.

The cusps of $X_H(N)$ are the equivalence classes of $\P^1(\QQ)$ under
$\Gamma_H(N)$. We will denote the elements by $[x/y]_{\Gamma_H(N)}$. The
\defn{cuspidal subgroup} $C_N$ of $J_H(N)$ is the subgroup of degree-0 divisors
$(\alpha)-(\beta)$ where $\alpha, \beta$ are cusps on $J$. More generally, if
$A=(L, V, J)$ is a modular abelian variety that is the quotient of $B$ by $L$.
Then the cuspidal subgroup of $A$ is defined to be $(C_N\cap B)/L$, where $C_N$ is
the cuspidal subgroup of $J_H(N)$.

The computational of $C_N$ is given in~\cite[\S 3.8]{stein:phd} and we will now
discuss the computation of the rational cuspidal subgroup. The Galois structure
of the cuspidal subgroup is well-understood~\cite[\S 1.3]{stevens:thesis}. The
points of the cuspidal subgroup are $\QQ(\mu_N)$-rational under the following
Galois action. There is an abstract group homomorphism
$\Gal(\QQ(\mu_N)/\QQ)\cong \ZZ/N\ZZ)^*$, let
\[
    \sigma_d\in \Gal(\Q(\mu_N)/\Q):\mu_N\mapsto \mu_N ^d
\]
then
\[
    \sigma_d([x/y]_{\Gamma_1})=[x/d'y]_{\Gamma_1},
\]
where $dd'\equiv 1 \mod{N}$. This explicit description of the Galois action
\emph{rational cuspidal subgroup} $C_N:=C_N ^{\Gal(\QQ(\mu)/\QQ)}$.

\subsection{Shimura subgroup}%
\label{sub:shimura}

In this section, we will explain how to compute the Shimura subgroup of
$J_0(N)$. The \defn{Shimura subgroup}, $\Sigma_N$, is the kernel of the natural
map $J_0(N)\to J_1(N)$. This will be useful in
Chapter~\ref{chap:totally_split}, where Moreover, the Shimura subgroup is a
Galois subgroup of $J_0(N)(\QQ)_\tor$. This fact will be used in
Chapter~\ref{chap:isogeny_class} where we enumerate the rational isogeny
classes.

Let $\Sigma_N$ be the Shimura subgroup of $J_0(N)$ so $\Sigma_N$ is the kernel
of the natural map $J_0(N)\to J_1(N)$. To compute $\Sigma_N$, we will use a
theorem by Ribet. We will choose some odd prime $p$ coprime to $N$,
by~\cite[Prop. 1]{ribet:raising}, $\Sigma_N=\ker(d_1-d_p)$, where $\delta_1
^*,\delta_p ^*:J_0(N)\to J_0(pN)$ are the degeneracy maps corresponding to
$1,p$. The computation of the kernel is given in~\ref{sub:kernel}.

\section{Morphisms}%
\label{sec:morphisms}

The goal of this section is to establish some background for computing with
morphisms between modular abelian varieties.

\subsection{Defining Data}%
\label{sub:defining_data}

Let $A=(L, V, J)$ and $B=(L', V', J')$ be modular abelian varieties and
$\varphi:A\to B$ a map of abelian varieties. Then $\varphi$ induces a map on rational
homology and is also completely determined by the map on rational homology. So
we will defined $\varphi:A\to B$ by giving $\varphi_V:V\to V'$. For brevity, we will
use $\varphi$ to denote the maps on defining lattices, rational homology, and
abelian varieties.

\subsection{Kernel}%
\label{sub:kernel}

Let $\varphi:A\to B$ be a morphism between modular abelian varieties $A=(L, V, J)$
and $B=(L', V', J')$. Let $V_K=\ker \varphi_V$ and $L_K=L\cap V_K$. Then the
kernel $K$ of $\varphi$ is the extension of the abelian variety $K^0 = (L_K, V_K,
J)$ by the finite component group $\varphi^{-1}(L')/L$.

\subsection{Image}%
\label{sub:image}

Let $\varphi:A\to B$ be a morphism between modular abelian varieties $A=(L, V, J)$
and $B=(L', V', J')$. The image of $\varphi$ in $B$ is the abelian subvariety
$\varphi(A)=(L'', \varphi(V), J')$, where $L''=\varphi(V)\cap L'$.


\subsection{Quotienting by finite subgroup}%
\label{sub:quotienting_by_finite_subgroup}

Let $A=(L, V, J)$ be a modular abelian variety with a finite subgroup
$G=\mathcal{L}, A)$. There is an isogeny $\vphi_G: A\to A'=(L+\mathcal{L}, V,
J)$ with kernel exactly $G$ given by the identity map on rational homology.


\section{Decomposition and Verification of Abelian Subvarieties}
\label{sec:decomp_verify}

\subsection{Decomposition of $J_H(N)$}%
\label{sub:decomposition_of_j_}

The decomposition of $J_H(N)$ is equivalent to the decomposition of the
cuspidal modular symbol space of $\Gamma_H(N)$. This is discussed in~\cite[\S
9]{stein:modform}.

\subsection{Simple abelian subvarieties}%
\label{sub:simple_abelian_subvarieties}

\begin{algorithm}{Simple abelian subvarieties as image of degeneracies}%
    \label{alg:simple_degen}
    Given a simple abelian subvariety $A$ of $J=J_H(N)$, this algorithm returns
    a newform $f$ of level $L$, and an isogeny $\varphi:A_f ^\vee\to A$, where
    $A_f ^\vee\subseteq J_H(L)$ is the optimal subvariety attached to $f$.
    \begin{enumerate}
        \item{} [Decompose $J$]
            Use Algorithm~\ref{alg:decomp_J} to obtain the decomposition
            $J=\sum_f V_f$, where the sum runs over newforms $f$ of level
            dividing $N$ and $V_f$ is the sum of all abelian subvarieties of
            $J_H(N)$ isogenous to $A_f$.
        \item{} [Determine newform]
            Find the newform $f$ of level $L$ so that $A\subseteq V_f$.
        \item{} [Basis under degeneracy]
            Let $\{b_1,\ldots,b_r\}$ be a $\ZZ$-basis for the defining lattice
            of $A_f^\vee$ and $\{\delta_1,\ldots,\delta_s\}$ be the set of degeneracy
            maps $\delta_j:J_H(L)\to J_H(N)$.
        \item{} [Solve system]
            Let $x$ be any nonzero element of the defining lattice of
            $A_f^\vee$. Find $c_{ij}\in \QQ$ such that $x=\sum c_{ij}
            d_j(b_i)$.
        \item{} [Clear denominators]
            Let $i', j'$ be such that $c_{i'j'}\neq 0$. Let
            $(n_1,\ldots,n_s)\in \ZZ^s$ be the vector obtained by
            clearing denominators from $(c_{i'1},\ldots,c_{i's})$.
        \item{} [Output]
            The desired isogeny is now given by $\varphi=\sum_{j=1} ^s n_j
            \delta_j|_{A_f ^\vee}$.
    \end{enumerate}
\end{algorithm}

Let $B$ be an abelian subvariety of $J$. The defining data of $B$ is given by
$(L, V, J)$, where $V=H_1(B,\QQ)$ and $L = H_1(B, \ZZ)$. Since $L = V\cap
H_1(J, \ZZ)$, the abelian subvariety $B$ is completely determined by the
subspace $V\subseteq H_1(J, \QQ)$, denoted $B\sim (V, J)$. As mentioned in
Section~\ref{sec:defining_data}, not every subspace $V$ of $H_1(J, \QQ)$
corresponds to an abelian subvariety. The following algorithm will determine
when $V$ does correspond to an abelian subvariety. When $V$ does correspond to
an abelian subvariety $B$, we will also give a decomposition of $B$ into simple
abelian subvarieties.

\begin{algorithm}{Decomposing and Verifying Abelian Subvarieties}
    \label{alg:decomp_and_verify_subvarieties}
    Let $J=J_H(N)$ and $V$ be a subspace of $H_1(J, \QQ)$. If $V$ corresponds
    to a subvariety $B$ of $J$, then this algorithm will return a decomposition
    into simple abelian subvarieties $X_i=(V_i, J)$ of $B$, otherwise, this
    algorithm will return `not a subvariety'.
    \begin{enumerate}
        \item{} [Decompose $J$]
            Use Algorithm~\ref{alg:decomp_J} to decompose $J$ as $J=\sum_f
            X_f$, where $X_f=(W_f, J)$ is the sum of all abelian subvarieties
            of $J$ isogenous to $A_f$, where $A_f$ is the optimal quotient
            attached to $f$.
        \item{} [Intersect with $V$]
            Set $V_f=V\cap W_f$. We have that $V$ corresponds to an abelian
            subvariety if and only if each $V_f$ corresponds to an abelian
            subvariety. So we will explain the rest of our algorithm for just a
            single $V_f$ with $f$ a newform of level $L$.
        \item{} [Build up to $V_f$]
            If $V_f$ does correspond to an abelian subvariety $B_f$, then $B_f$
            must decompose as a product of simple abelian varieties isogenous
            to $A_f$. Using Proposition~\ref{prop:integral_degen}, $V_f$
            corresponds to an abelian subvariety if and only if $V_f=\sum V_i$,
            where each $V_i$ is the image of an integral linear combination of
            $\delta_j:A_f ^\vee \to J_0(N)$.
            \begin{enumerate}
                \item{} [Initiate]
                    Set $S_0=0$ and $i=0$.
                \item{} [Add a $V_i$]
                    Pick some $x\in V_f\setminus S_i$. Use Algorithm~\ref{} to
                    determine if there exists $V_i$ such that $x\in V_i$, where
                    $V_i$ is the image of an integral linear combination of
                    $\delta_j:A_f ^\vee \to J_0(N)$. If not, then $V_f$ does
                    not correspond to an abelian subvariety and we are done.
                \item{} [Done?]
                    Set $S_{i+1} = S_i + V_i$. If $S_{i+1}=V_f$, we are done
                    and we output $X_i=(V_i, J)$. Otherwise, increment $i$ and
                    return to the last step.
            \end{enumerate}
    \end{enumerate}
\end{algorithm}

% TODO: abelian subvariety need not be Hecke stable...
% \begin{algorithm}{Decomposing and Verifying Abelian Subvarieties}
%     \label{alg:decomp_and_verify_subvarieties}
%     Let $J=J_H(N)$ and $V$ be a subspace of $H_1(J, \QQ)$. If $V$ corresponds
%     to a subvariety $B$ of $J$, then this algorithm will return a decomposition
%     into simple abelian subvarieties $X_i=(V_i, J)$ of $B$, otherwise, this
%     algorithm will return `not a subvariety'.
%     \begin{enumerate}
%         \item{} [$\ZZ$-basis for Hecke algebra]
%             Let $\TT$ be the Hecke algebra of $J$ and $s$ be the Sturm bound of
%             $\TT$. By~\cite[Appendix]{lario-schoof}, $\{T_1,T_2,\ldots, T_s\}$
%             spans $\TT$ as a $\ZZ$-module. So we can compute a $\ZZ$-basis
%             $\mathcal{B}$ of $\TT$.
%         \item{} [Decompose into potential Hecke eigenspaces]
%             We first decompose $V$ into a direct sum $V=W_1\oplus \cdots \oplus
%             W_r$ so that, for each $W_i$, there exists a newform $f_i$ of level
%             dividing $N$, such that either $W_i$ either corresponds to an
%             abelian subvariety isogenous to a power of $A_{f_i}$ or is not an
%             abelian subvariety and we can return `not a subvariety'.
%             \begin{enumerate}
%                 \item{} [Initialize]
%                     Set $j=1$.
%                 \item{} [Decompose into eigenspaces]
%                     Compute the simultaneous eigenspaces $U_{j,1},\ldots,U_{j,
%                     n_j}$ of $\{T_1,\ldots,T_j\}$. If $\sum_{k=1} ^{n_j}
%                     U_{j, k}\neq V$, return `not a subvariety'.
%                 \item{} [Compare with newforms]
%                     Let $S_{j,k}$ be the set of newforms whose $\ell$th Fourier
%                     coefficient agrees with the eigenvalue of $T_\ell$ on
%                     $U_{j, k}$ for $l=1,\ldots,j$. If $S_{j, k}$ is empty for
%                     any $k=1,\ldots,n_j$, return `not a subvariety'. If $S_{j,k}$ is not
%                     singleton for any $k=1,\ldots, n_j$, increment $j$ and
%                     return to (b). Otherwise, $S_{j,k}$ is singleton for all
%                     $k$ so let $S_{j,k}=\{f_k\}$ and $W_k=U_{j,k}$.
%             \end{enumerate}
%         \item{} [Is isogenous to power?]
%             We now verify that each $W_i$ is isogenous to a power of $A_{f_i}$
%             and give its decomposition. The $W_i$'so are pairwise non-isogenous
%             so we will do this individually for each $W_i$ and consider just a
%             single pair $W$ and $A_f$ with $f$ being a newform of level $L$.
%             Let $d_1,\ldots,d_r:J_H(L)\to J_H(N)$ be the full collection of
%             degeneracy maps from $J_H(L)$ to $J_H(N)$. Set $U=\{0\}$ and 
%             \begin{enumerate}
%                 \item{} [Image of $\QQ$-combination of $d_j$'s?]
%                     Choose any $v\in V\setminus U$. By
%                     Proposition~\ref{prop:integral_degen}, if $W$ is an abelian
%                     subvariety then there exists $q_1,\ldots,q_r\in \QQ$
%                     such that $v\in \Im \left(\sum_{i=1} ^r q_i
%                     \delta_i\right)$. If this is not the case, return
%                     `NO'. Let $R = \Im \left(\sum_{i=1} ^r q_i
%                     \delta_i\right)$. If $R\not\subset V$, return `not a
%                     subvariety'.
%                 \item{} [Full space?]
%                     Replace $U$ with $U+R$. If $V=U$, we are done. Otherwise,
%                     return to (B).
%             \end{enumerate}
%         \item{} [Output]
%     \end{enumerate}
% \end{algorithm}


\section{Simple abelian subvarieties as image of degeneracies}

Let $A$ be a simple modular abelian variety isogenous to $A_f$, where $f$ is a
newform of level $L$. The goal of this section is to explicitly give an isogeny
from $A_f ^\vee\subseteq J_0(L)$ to $A$. 

% In the case of a elliptic curve subvariety $E$ of $J_0(N)$, we can also return
% the Weierstrass equation for $E$. This is 

% \begin{algorithm}{Weierstrass equation of 1-dimension abelian subvariety}%
%     \label{alg:weierstrass}
%     Given an elliptic curve subvariety $E$ of $J_0(N)$. This algorithm
%     returns the Weierstrass equation for $E$.
%     \begin{enumerate}
%         \item{}
%             [Isogeny from optimal subvariety] Use
%             Algorithm~\ref{alg:simple_degen} to find a newform $f$ of level
%             $L$ and isogenies $\varphi:A_f\to E$.
%         \item{}
%             [Weierstrass of optimal subvariety] The Cremona tables contain the
%             Weierstrass equations for optimal subvarieties of $J_0(L)$ so in
%             particular, for $A_f$.
%         \item{}
%             [Kernel] Compute the kernel $M$ of $\varphi$ and use the complex
%             exponential map to identify $M$ as a set of points, $M'$ on the
%             Weierstrass equation for $A_f$.
%         \item{}
%             [Velu's formulas] Output the Weierstrass equation of $A_f/M'$ using
%             Vela's formulas.
%     \end{enumerate}
% \end{algorithm}

\section{Homomorphism spaces}

Let $A, B$ be simple modular abelian varieties. In this section, we give
algorithms for computing with homomorphism spaces between $A, B$. This
algorithms will be crucial for determining when $A$ and $B$ are isomorphic. 

We begin by giving an algorithm for computing a $\ZZ$-basis for $\Hom(A,B)$.
By applying this algorithm to the case when $A=B$, we obtain an algorithm for
computing a $\ZZ$-basis for $\End(A)$. Of course, $\End(A)$ also has the
structure of a ring that is isomorphic to an order in a number field. So we
will also give an algorithm for determining an order $\O$ isomorphic to
$\End(A)$, as well as maps to and from $\O$.

\subsection{Extending by $\QQ$}%
\label{sub:extending_by_qq_}

Let $A, B$ be simple modular abelian varieties both of dimension $d$. If $A$ is
not isogenous $B$, then $\Hom(A, B)=0$ and we are done. If $A$ is isogenous to
$B$, then $\Hom(A,B)\otimes \QQ \cong \End(A)\otimes \QQ\cong K_f$, where $K_f$
is the Hecke eigenvalue field of a newform $f$ associated to the simple modular
abelian variety $A$ (~\cite{shimura}). The goal of this section is to prove
Proposition~\ref{prop:extend_QQ} which allows us to recover $\Hom(A,B)$ from
$\Hom(A,B)\otimes \QQ$.

After choosing a basis for $\Lambda_1 = H_1(A, \ZZ)$ and $H_1(B, \ZZ)$, 
\[
    \Hom(\Lambda_1, \Lambda_2) \cong (\ZZ^{(2d)})^2.
\]
\begin{proposition}%
    \label{prop:extend_QQ}
    Let $A, B$ be simple abelian varieties over $\QQ$, let $\Lambda_1=H_1(A,
    \ZZ)$, and let $\Lambda_2=H_1(B, \ZZ)$. Embed $\Hom(A, B)$ into
    $\Hom(\Lambda_1, \Lambda_2)$ by the action on homology. Then
    \[
        \Hom(A, B) = 
        (\Hom(A, B) \otimes \QQ)\cap \Hom(\Lambda_1, \Lambda_2)
    \]
    where the intersection takes place in $\Hom(A,B)\otimes \QQ$.
\end{proposition}

We first prove a lemma that will be used in the proof of
Proposition~\ref{prop:extend_QQ}.

\begin{lemma}%
    \label{lem:aut}
    If $x\in \CC$ is fixed by every element of $\Aut(\CC/\QQ)$, then $x\in
    \QQ$.
\end{lemma}
\begin{proof}
    Suppose $x$ is transcendental, then there is a field automorphism
    $\sigma:\QQbar(x)\to\QQbar(x)$ given by $x\mapsto x+1$. This automorphism
    extends to an automorphism of $\CC$ that does not fix $x$. Therefore, $x$
    must be algebraic and by standard Galois theory, $x\in \QQ$.
\end{proof}

\begin{proof}[Proof of Proposition~\ref{prop:extend_QQ}]
    An element of $\Hom(A, B)$ is certainly an element of $\Hom(A, B)\otimes
    \QQ$. Moreover, an element of $\Hom(A, B) \subseteq \Hom_\CC (A,B)$ is a
    complex linear map from $\Tan(A_\CC)\to \Tan(B_\CC)$ that sends $\Lambda_1$
    to $\Lambda_2$ so it must be in $\Hom(\Lambda_1,\Lambda_2)$. This
    establishes the forward inclusion.

    Conversely, suppose $\varphi\in (\Hom(A, B)\otimes \QQ)\cap \Hom(\Lambda_1,
    \Lambda_2)$. Then there exists a positive integer $n$ such that $n\varphi n
    \Hom(A, B)$. Hence, $n\varphi\in \Hom(A,B)\subseteq \Hom_\CC(A, B)$ is a
    complex linear map from $\Tan(A_\CC)$ to $\Tan(B_\CC)$. Hence,
    $\varphi=(1/n)n\varphi$ is also a complex linear map from $\Tan(A_\CC)$ to
    $\Tan(B_\CC)$. By assumption, $\varphi$ also maps $\Lambda_1$ to
    $\Lambda_2$ so $\varphi\in \Hom_\CC(A,B)$. It remains to show that
    $\varphi$ is defined over $\QQ$. Let $\sigma\in \Gal(\CC/\QQ)$. Since
    $[n]\varphi\in \Hom(A,B)$, $\sigma([n]\varphi)-[n]\varphi=0$. By
    rearranging, $[n](\sigma\varphi-\varphi)=0$. The image of
    $\sigma\varphi-\varphi$ is either infinite or 0 and the kernel of $[n]$ is
    finite so we must have $\sigma\varphi=\varphi$. By Lemma~\ref{lem:aut},
    $\varphi\in \Hom(A,B)$.
\end{proof}

\subsection{Endomorphism Algebra as Field}%
\label{sub:endomorphism_algebra_as_field}

\begin{algorithm}{Endomorphism Algebra as Field}
    Given a simple abelian variety $A$ over $\QQ$, this algorithm computes a
    number field $F$ and an isomorphism from $\End(A)\otimes \QQ$ to $F$.
    \begin{enumerate}
        \item{} [Isogeny to $A_f ^\vee$]
            Use Algorithm~\ref{simple_deg} to compute an optimal subvariety
            $A_f ^\vee$ and an isogeny $\varphi:A\to A_f ^\vee$. This isogeny
            induces an isomorphism $\End(A)\otimes\QQ$ to $\End(A_f)\otimes
            \QQ$.
        \item{} [Random endomorphism]
            Use Algorithm~\ref{todo} to compute a $\ZZ$-basis $B$ for the Hecke
            algebra, $\TT'$, of $A_f ^\vee$. We then generated a random
            element, $T$, of $\End(A_f)$ by taking a random rational linear
            combination of the elements of $B$.
        \item{} [Is random element primitive?]
            Let $g$ be the minimal polynomial of $T$. If $\dim A = \deg g$,
            then $g$ will be a primitive generator for $\End(A)\otimes \QQ$ as
            a field and we proceed to the next step. Otherwise, return to the
            last step.
        \item{} [Output]
            Let $F$ be a number field generated a root, $\alpha$, of $g$. Let
            $\Psi:\End(A)\otimes \QQ\to F$ be the unique map sending $T$ to
            $\alpha$. We then precompose $\Psi$ with the isomorphism
            $\End(A)\otimes \QQ \to \End(A_f)\otimes \QQ$ to obtain the
            desired isomorphism.
    \end{enumerate}
\end{algorithm}
%TODO: proof

\subsection{Homomorphism space}%
\label{sub:_hom_a_b_}

Let $A, B$ be isogenous simple abelian varieties. The goal of this section is
to compute $\Hom(A,B)$.

\begin{algorithm}[Compute $\Hom(A,B)$]
    Given simple modular abelian varieties $A, B$, this algorithm computes
    $\Hom(A,B)$.
    \begin{enumerate}
        \item{} [Isogenous?]
            Use Algorithm~\ref{alg:isogenous} to determine if $A$ is isogenous
            to $B$. If not, then $\Hom(A,B)$ is trivial and we are done. If so,
            let $\varphi:A\to B$ be an isogeny.
        \item{} [Compute $\End(A)$]
            Use Algorithm~\ref{alg:end_A} to compute the endomorphism ring of
            $A$.
        \item{} [Image of $\End(A)$]
            Compute the image, $H$, of $\End(A)$
        \item{} [Saturate]
            Compute the saturation of $H$ in $\Hom(L_1, L_2)$.
    \end{enumerate}
\end{algorithm}
\begin{proof}
    
\end{proof}



\section{Isogeny and isomorphism testing}

In this section, we give an algorithm for determining when a pair of simple
modular abelian varieties are isomorphic.

\begin{algorithm}{Isomorphism testing}%
    \label{alg:isomorphism_testing}
    Given simple modular abelian varieties $A, B$, this algorithm determine if
    $A, B$ are isomorphic. If so, this algorithm will also return an
    isomorphism between $A, B$.
    \begin{enumerate}
        \item{} [Isogenous?]
            Use Algorithm~\ref{alg:isogenous} to determine if $A$ is isogenous
            to $B$. If not, then $A$ and $B$ are not isomorphic and we are
            done. If so, let $\varphi:A\to B$ be an isogeny.
        \item{}
            [Square degree?] The composition $\varphi^\vee \circ \varphi:B\to B$ is
            the multiplication by $d$ map where $d$ is the degree of $\varphi$. If
            $d$ is not square, return `not isomorphic'.
        \item{}
            [Endomorphism algebra] Use Algorithm~\ref{} to find a number field
            $K$, an order $\O\subseteq K$, and an isomorphism $\tau:\End(A)\to
            \O$.
        \item{}
            [Homomorphism space] Use Algorithm~\ref{} to compute $\Hom(A,B)$.
        \item{}
            [Image under $\varphi$] Compute the image $H_f$ of $\Hom(A, B)$ in
            $\End(A)$ by composing with $f$.
        \item{}
            [Norm equation] Find solutions $x_1,\ldots,x_r$, up to units in
            $\O$, to the norm equation $\Norm_{\O} (x) = \pm \sqrt{d}$. If
            there are no solutions, return `not isomorphic'.
        \item{}
            [Isomorphic?] For each solution $x_i$, determine if $x_i\in H_f$,
            if so, $x_i \circ f^{-1}$ is an isomorphism from $A\to B$. If
            $x_i\notin H_f$ for all $x_i$, then return `not isomorphic'.
    \end{enumerate}
\end{algorithm}

\chapter{Totally Split Jacobians}%
\label{chap:totally_split}

A Jacobian is said to be \emph{totally split} if it is $\QQ$-isogenous to a
product of elliptic curves. As mentioned in the introduction, the first $N$ for
which \sage fails to compute the rational torsion subgroup is $J_0(30)$ which
happens to be a product of 3 elliptic curves. The author and his adviser were
able to compute the rational torsion subgroup using the fact that $J_0(30)$ is
totally split, the fact that rational torsion subgroups of elliptic curves can
be computed, and Galois cohomology. The goal of this chapter is to see how far
we can push these techniques. In particular, we will show that there are
finitely many totally split $J_0(N)$, give a general (but totally impractical)
method for computing the rational torsion subgroup, and present some more
practical techniques for computing the rational torsion subgroup.


\section{Provably enumerating the set of totally split $J_0(N)$}
\label{sec:provably_enumerating}

The modular Jacobian $J_0(N)$ is totally split if and only if all newforms of
level dividing $N$ have rational Hecke coefficients. We expect this to be rare.
In fact, there are only finitely many totally split $J_0(N)$. Ralph Greenberg
quickly gave an argument on the way to a University of Washington Number
Theory Seminar lunch proving the set of totally split $J_0(N)$ is finite but
his argument did not give an effective method of enumeration. Moreover, at Sage
Days 87, Alyson Dienes suggested proving this using asymptotic bounds.
\begin{proposition}%
    \label{prop:totally_split}
    The set of totally split $J_0(N)$ is finite.
\end{proposition}
\begin{proof}
    If $J_0(N)$ is a product of elliptic curves then the dimension is
    exactly equal to the number of elliptic factors.
    \begin{itemize}
        \item
            By~\cite[Thm. 6]{martin:dimension}, the dimension is bounded below
            by $\frac{1}{12}N + O(\sqrt{N}\log\log N)$.
        \item
            By~\cite[Cor. 2]{brumer-silverman:number}, the number of elliptic
            curve factors of $J_0(N)$ is bounded above by $O(N^{1/2+\epsilon})$
            for any $\epsilon>0$.
    \end{itemize}
    Asymptotically, the lower bound will surpass the upper bound so there are
    finitely many totally split $J_0(N)$.
\end{proof}
Neither of their arguments gave an effective method of enumeration. The goal of
this section is to provably enumerate the set of $J_0(N)$ that are totally
split. There are 71 of totally split $J_0(N)$ that are nontrivial (see
Table~\ref{tab:split}).

We have that $J_0(N)$ is totally split if and only if its dimension is equal to
the number of (modular) elliptic curves of conductor $N$. If $N$ is less than
the upper limit of conductors in the Cremona Database, then the hard work has
been done and determining whether $J_0(N)$ is totally split is a quick
computation since there is a closed-form formula for $\dim J_0(N)$.

Call a positive integer $N$ \emph{good}, if $J_0(N)$ is totally split and
\emph{bad} otherwise. The simple factors of $J_0(N)$ are isogenous to simple
factors of $J_0(MN)$ for any positive integer $M$. Therefore, if $N$ is bad,
then so is any multiple of $N$. 

In Lemma~\ref{lem:good_primes}, we will enumerate all good primes. We can now
do a search on the divisibility tree of the positive integers supported on the
good primes. Moreover, we can prune a branch whenever we encounter a bad
integer during our search. Theorem~\ref{thm:finite_totally_split} asserts that
all branches are eventually pruned so this search yields all good integers.

\begin{lemma}
    \label{lem:good_primes}
    The only possible primes $p$ where $J_0(p)$ is totally split are
    \[
        2, 3, 5, 7, 11, 13, 17, 19, 37.
    \]
\end{lemma}
\begin{proof}
    Let $J=J_0(p)$ be a totally split Jacobian of prime level. If $\dim J=0$,
    then $J$ is clearly totally split so assume $\dim J>0$. Then $J\cong
    \prod_f E_f$ with $E_f$'s elliptic curves of conductor $p$. Let $n$ denote
    the order of the rational cuspidal subgroup of $J$ which is, as mentioned
    above, known to be the numerator of $(p-1)/12$. By Emerton's proof of
    Stein's refined Eisenstein conjecture~\cite[Theorem B]{emerton:optimal}, if
    $l$ divides $n$, then $l$ divides the order of $E_f(\QQ)$ for some elliptic
    factor of $J$. But elliptic curves of prime conductor do not have much
    rational torsion.

    In particular, Miyawaki~\cite{miyawaki:ell_prime} enumerates all curves of prime
    power conductor with odd-order rational torsion. The largest prime
    conductor here being 37. This implies that if $p>37$, then $\#C(\QQ)$ must
    be a power of 2 so $p=2^a 3^b + 1$ for some $a\geq 0$ and $b\in \{0,1\}$.
    We now show $a\leq 2$.

    If $a>2$, then $C(\QQ)$ has an order 2 element which implies some $E_i$ has
    a rational 2-torsion point. As a result of Setzer~\cite[Theorem
    2]{setzer:ell_prime}, $p=17$ or $p=u^2+64$ for some integer $u$. We split into 2
    cases to show that $p$ is never $u^2+64$.

    Suppose $b=0$. Then $p$ is a Fermat prime and thus a Fermat number. Outside
    of $3$ and $5$, the recursive formula for Fermat numbers and an induction
    argument shows that the last digit of Fermat numbers is always $7$. But the
    only possible last digits of $u^2+64$ are $0, 3, 4, 5, 8, 9$.

    Suppose $b=1$. Then $2^a\cdot 3 = u^2+63$. This implies $3$ divides $u$ so
    $9$ divides $u^2$. But now the right-hand side is divisible by $9$ while the
    left is not.

    In conclusion, we know that if $J_0(p)$ is a totally split Jacobian, then $p\leq
    37$. A computer search then determines which primes less than or equal to
    $37$ are totally split.
\end{proof}

The following procedure is a breath-first search.
\begin{algorithm}{Enumerating Good Integers}
    \label{alg:find_split}
    Let $S$ be the set of primes found in Lemma~\ref{lem:good_primes}. This
    procedure will halt (see Theorem~\ref{thm:finite_totally_split}) and return
    the list of good integers.
\end{algorithm}
\begin{enumerate}
    \item{} [Initialize]
        \label{step:initialize}
        Set $i=1$ and $M_1=S$. Here $M_i$ will represent the set of all good
        integers with $i$ prime factors, counting multiplicity.
    \item{} [Find prime multiples of $M_i$ are that good]
        Set
        \[
            M_{i+1}=\{pN: p\in S, N\in M_i, pN \text{ good}\}.
        \]
        If $M_{i+1}$ is non-empty, increment $i$ and repeat this step.
    \item{} [Return]
        Return $\bigcup_i M_i$.
\end{enumerate}
\begin{theorem}
    \label{thm:finite_totally_split}
    There are 71 integers $N$ for which $J_0(N)$ is a totally split Jacobian of
    positive dimension. They are given in Table~\ref{tab:split}.
\end{theorem}
\begin{proof}
    We run Algorithm~\ref{alg:find_split} and it terminates after find 71
    integers $N$ for which $J_0(N)$ is totally split (luckily before reaching
    the end of the Cremona database). This proves the finiteness of totally
    split $J_0(N)$ without using Proposition~\ref{prop:totally_split} but with
    the added benefit of provably enumerating the totally split $J_0(N)$.
\end{proof}

\section{Enumerating rational torsion is algorithmic}
\label{sec:enumerating_is_algorithmic}

In this section, we give a completely impractical algorithm to compute the
rational torsion subgroup of a totally split Jacobian $J_0(N)$ just to show it
is algorithmic.
\begin{proposition}
    Suppose $A$ is a totally split abelian subvariety of $J=J_0(N)$ Then we can
    compute the following data:
    \begin{enumerate}
        \item
            A number field containing $\QQ(A[n])$ for positive integer $n$.
        \item
            Let $n$ be a positive integer and $L$ be a number field containing
            $\QQ(A[n])$. The action of $\Gal(L/\QQ)$ on $A[n]$.
        \item
            The $K$-rational torsion points of $A(K)_\tor$.
    \end{enumerate}
    In particular, we can compute the rational torsion subgroup of any totally
    split Jacobian.
\end{proposition}
\begin{proof}
    Recall that the abelian subvarieties are represented by giving a submodule
    of the integral homology~\ref{sec:defining_data}. Suppose $A$ is
    1-dimensional subvariety of $J$. Then by~\ref{alg:weierstrass}, we can
    compute an elliptic curve $E_A$ given in Weierstrass defining equation and
    an isomorphism $\Phi_A:A_\tor\to (E_A)_\tor$.

    We will proceed by induction on the dimension, $d$, of $A$. We first
    consider the case $d=1$.
    \begin{enumerate}
        \item
            Let $n$ be a positive integer. Then using division polynomials, we
            can compute a number field $L$ containing $\QQ(A[n])$.
        \item
            Let $n$ be a positive integer and $L$ be a number field containing
            $\QQ(A[n])$. The Galois action on the points of $E_A[n]$ is given
            by applying the Galois action to each coordinate. Using $\Phi_A$
            and the action of $\Gal(L/\QQ)$ on $E_A$, we can explicit determine
            the action of $\Gal(L/\QQ)$ on $A[n]$.
        \item
            Let $K$ be a number field. Using reduction mod $p$, there exists an
            integer $m$, such that that $A(K)_\tor \subseteq A[m]$. Using (1),
            we can define a number field $L$ that contains $\QQ(A[m])$. Then
            using (2), we can compute
            \[
                A(K)_\tor = A[m]^{\Gal(L/K)}.
            \]
    \end{enumerate}

    Now assume we can compute (1)-(3) for any totally split abelian subvariety
    of dimension less than $k$.

    Let $A$ be of dimension $k+1$ and write $A=B+C$, where $B$ is of
    dimension $k$ and $C$ is of dimension $1$.  We have the exact sequence
    \[
        0\to B\cap C \to B\times C \to A,
    \]
    where we identify $B\cap C$ as a subgroup of $B\times C$ via the
    anti-diagonal embedding. So $A=(B\times C)/(B\cap C)$. Let $r$ be the
    exponent of $B\cap C$.
    \begin{enumerate}
        \item
            For any integer $n$, $A[n] \subseteq B[nr]+C[nr]$. So a number
            field containing $\QQ(A[n])$ is the compositum  of the number
            fields containing $\QQ(B[nr])$ and $\QQ(C[nr])$ which can be
            computed by the inductive hypothesis.
        \item
            The Galois action can be determined on $A[n]$ by viewing $A[n]$ as
            a subgroup of $(B[nr]\times C[nr])/(B\cap C)$.
        \item
            Using reduction mod $p$, there exists an integer $m$ such that
            $A(K)_\tor\subseteq A[m]$. Using (1), we can find a number field
            $L$ that contains $\QQ(A[m])$. Then we use (2), to compute
            \[
                A(K)_\tor = A[m]^{\Gal(L/K)}.
            \]
    \end{enumerate}
\end{proof}

\section{Strategies for computing the rational torsion subgroup}
\label{sec:strategies_for_rational_torsion}

The Generalized Ogg Conjecture is equivalent to the assertion that
$[J_0(N)(\QQ)_\tor:C_N(\QQ)]=1$. In this section, we give strategies for
bounding this index when $J_0(N)$ is a rank-0 totally split Jacobian. We are
able to compute generators for $C_N(\QQ)$ in our presentation
(\ref{sub:cuspidal}) and the group order. Therefore, when
$[J_0(N)(\QQ):C_N(\QQ)]=1$, we are able to compute generators for
$J_0(N)(\QQ)_\tor$ in our presentation.

Using these strategies, we are able to able to verify the Generalized Ogg
Conjecture for all but 9 rank-0 $J_0(N)$. In these 9 cases, we are
able to bound the index $[J_0(N)(\QQ)_\tor:C_N(\QQ)]$ by a power of 2
(Table~\ref{tab:rank_zero_bound}).

\subsection{Upper bound}%
\label{sub:upper_bound}

In this subsection, we give two techniques for giving an upper bound for
$J_0(N)(\QQ)_\tor$. The first technique uses reduction modulo $p$,
Eicher-Shimura, and the Hecke polynomial to obtain a bound on the order. The
second technique gives an ideal, $I^*$, similar to Mazur's Eisenstein ideal, so
that $J_0(N)(\QQ)_\tor\subseteq J_0(N)[I]$. We will be primarily using the
second technique for our computations but we give the first technique to
motivate the second and we will be using the first technique

\subsubsection{Reduction modulo primes}%
\label{ssub:reduction_modulo_primes}
Let $A$ be a modular abelian variety and $K$ be a number field. A common
technique for bounding $A(K)_\tor$ is to reduce modulo completely split primes
$p$ of $K$ of good reduction for $A$. In particular,
by~\cite[Appendix]{katz:torsion}, if $\mathcal{A}$ is the Neron model for $A$
and $p$ a prime completely split of $K$ of good reduction for $A$, then
\[
    A(K)_\tor \hookrightarrow \mathcal{A}_{/p}(\Fp)
\]
so $\# A(K)_\tor \mid \#\mathcal{A}_{/p}(\Fp)$.

The expression $\#\mathcal{A}_{\p}(\Fp)$ is an isogeny invariant and
multiplicative on direct products. So it suffices to describe how to compute
$\mathcal{A}_{/p}(\Fp)$ when $A\subseteq J_0(N)$ is a simple abelian subvariety
of the new subvariety of $J_0(N)$ (for $A\subseteq J_1(N)$, see~\cite[\S
3.5]{agashe-stein:bsd}). Let $F_p$ be the absolute Frobenius at $p$. By the
Eicher-Shimura relations, $T_p = F_p + p/F_p\in \End(\mathcal{A}_{/p})$. Then
\begin{align}
    \label{eq:katz_bound}
    \begin{split}
    \#\mathcal{A}_{/p}(\Fp)
    & = \deg(1-F_p) \\
    & = |\det(1-F_p)| \\
    & = \mathrm{charpoly}(F_p)(1) \\
    & = \mathrm{charpoly}(T_p)(p+1).
    \end{split}
\end{align}
The last term is the characteristic polynomial of the matrix associated to the
Hecke operator $T_p$ which can be computed using
Section~\ref{sec:modular_symbols}.

We now return to the case of $K=\QQ$ and will apply the more general number
field case in Chapter~\ref{chap:isogeny_class}. So in the case where $K=\QQ$
and $A=J_0(N)$, we have that for any finite set $S$ of odd prime not dividing
$N$,
\[
    \# J_0(N)(\QQ)_\tor \mid \gcd_{p\in S} \#\mathcal{J}_{/p}(\Fp).
\]
We now give the current \sage (Version 8.7) implementation of this idea by
William Stein.
\begin{algorithm}{Upper bound on rational torsion order}%
    \label{alg:upper_bound_stable}
    Given a modular Jacobian $J_0(N)$, this algorithm outputs an upper bound
    for $\# J_0(N)(\QQ)_\tor$. This algorithm computes successive GCD's until
    it stabilizes for 3 iterations. Of course, increasing the stability
    threshold could lead to be a better bound.
    \begin{enumerate}
        \item{} [Initialize]
            Let $i=0$, $p_0$ be the smallest prime greater than 2 not dividing
            $N$.
        \item{} [Add another factor]
            Use Section~\ref{sec:modular_symbols} to compute the matrix
            associated to $T_{p_0}$. Let $m_i = \mathrm{charpoly}(T_p)(p+1)$.
            If $i=0$, set $B_i=m_i$. Otherwise, set $B_i=\gcd(B_{i-1}, m_i)$.
        \item{} [Stable?/Output]
            If $i\leq 2$, increment $i$ and return to Step 2. Otherwise, if
            $B_i\neq B_{i-2}$, return to Step 2. Otherwise, $B_i$ has been
            stable for 3 iterations and we output $B_i$.
    \end{enumerate}
\end{algorithm}
There are two disadvantages to Algorithm~\ref{alg:upper_bound_stable}.

The first is that is isogeny invariant so we expect the bound to not be tight.
In Section~\ref{sec:example}, we show that there exists an abelian variety isogenous to
$J_0(30)$ with strictly greater rational torsion order. So
Algorithm~\ref{alg:upper_bound_stable} will never give a tight bound even if we
increase the stability threshold. This Jacobian is the first $J_0(N)$ that
\sage (Version 8.7) cannot compute and is the motivating example for this
totally split Jacobian project of this thesis.

The second is the loss of information of the group structure. For example,
suppose for some $A$ and odd primes $p,q$ of good reduction, we have the
group isomorphisms $\mathcal{A}_{/p}(\Fp)\cong \ZZ/2\times \ZZ/2$ and
$\mathcal{A}_{/q}(\F_q)\cong \ZZ/4$. If we look only at the orders, we can only
deduce $\# A_f(\QQ)_\tor\mid 4$. However, the group structure tells us that as
a group $A_f(\QQ)_\tor$ is trivial or $A_f(\QQ)_\tor \cong \ZZ/2$.

\subsubsection{Real Eisenstein Kernel}%
\label{ssub:real_eisenstein_kernel}

This next technique avoids both of the mentioned disadvantages of the first
technique. This technique was discovered by William Stein and will be
elaborated on as part of a forthcoming paper by Hao Chen, the author, and
William Stein. The idea give to construct an ideal $I^*$ so that
$J_0(N)(\QQ)_\tor\subseteq J_0(N)[I^*]$.

For any odd prime $\ell$ of good reduction (so $\ell\nmid 2N$), let
$\eta_\ell = T_\ell - (\ell+1)$.
\begin{lemma}[William Stein]%
    \label{lem:rational_is_eta}
    For any odd prime $\ell$ of good reduction, $J(\QQ)_\tor\subseteq
    J[\eta_\ell]$.
\end{lemma}
\begin{proof}
    Let $\ell$ be an odd prime of good reduction.
    By~\cite[Appendix]{katz:torsion}, there the reduction modulo $\ell$ map
    yields the inclusion $\tau:J(\QQ)_\tor \hookrightarrow
    \mathcal{J}_{/\ell}(\F_\ell)$. Let $F_\ell$ be the absolute Frobenius of of
    $\mathcal{J}_{/\ell}$. By Eicher-Shimura, $T_\ell=F_\ell+\ell/F_\ell$. Let
    $x\in J(\QQ)_\tor$ so $F_\ell(\tau(x))=\tau(x)$. We have
    \[
        \tau(\eta_\ell(x)) = (T_\ell-(\ell+1))\tau(x) =
        (F_\ell+\ell/F_\ell-(\ell+1))\tau(x) = 0.
    \]
    Since $\tau$ is injective $x\in J[\eta_\ell]$.
\end{proof}

Let $I= \langle \eta_\ell:\ell\nmid 2N \rangle$. By
Lemma~\ref{lem:rational_is_eta}, $J(\QQ)_\tor\subseteq J[I]$. Now let $\star$
be the star-involution so $J(\CC)(1-\star)=J(\RR)$. and $I^*=I +
\langle 1-\star \rangle$. William Stein calls this the Real Eisenstein Ideal.
We have that $J(\CC)[1-\star]=J(\RR)$ so combined with
Lemma~\ref{lem:rational_is_eta}, $J(\QQ)_\tor\subseteq J[I^*]$. Extending $I$
to $I^*$ is an important step in our strategy because it is often the case that
$J[I]$ is strictly larger than $J[I^*]$. For instance, if $J=J_0(N)$ with $N$
prime, then by~\cite[Cor. 16.3]{mazur:eisenstein}, $J[I]$ contains the Shimura
subgroup $\Sigma_N$ which is a finite subgroup of $\mu$-type and order
$\numerator((N-1)/12)$.

To approximate $J[I^*]$, let $I_r ^* = \langle \eta_\ell:\ell\leq r \rangle$
and define $E_r = J[I_r ^*]$.
We have
\begin{equation}%
    \label{eq:eta_equation}
    C_N(\QQ)\subseteq J(QQ)_\tor \subseteq E_r \subseteq J[I^*].
\end{equation}

\section{General approach to bound}%
\label{sec:galois_cohomology_bounds}

\subsection{Galois cohomology}%
\label{sub:galois_cohomology}

Let $A, E$ be abelian subvarieties of $J$ with $E$ an elliptic curve and
$E\not\subseteq A$. Then
\begin{equation}%
    \label{eq:ses_ab}
    \begin{tikzcd}
        0 \arrow[r] &
        A\cap E \arrow[r,"d"] &
        A\times E \arrow[r,"s"] &
        A+E \arrow[r] &
        0,
    \end{tikzcd}
\end{equation}
where $d(x)=(x,-x)$ and $s(x,y)=(x+y)$. By applying Galois cohomology, we
obtain the long exact sequence:
\begin{equation}%
    \label{eq:les_ab}
    \begin{tikzcd}
        0 \rar 
        &
        (A\cap E)(\QQ) \rar 
        &
        (A\times E)(\QQ) \rar
        \arrow[phantom, ""{coordinate, name=W}]{d}
        &
        (A+E)(\QQ)
        \arrow[
        rounded corners,
        to path={%
            -- ([xshift=2ex]\tikztostart.east)
            |- (W) [near end]\tikztonodes
            -| ([xshift=-2ex]\tikztotarget.west)
            -- (\tikztotarget)
        }
        ]{dll} \\
        &
        H^1(\QQ,A\cap E) \rar["\gamma"] 
        &
        H^1(\QQ,A\times E) \rar 
        & \ldots.
    \end{tikzcd}
\end{equation}
We now have
\begin{equation*}%
    \label{eq:galois_cohomology_order}
    \# (A+E)(\QQ) =
    \frac{\# (A\times E)(\QQ) \# \ker\gamma}{\# (A\cap E)(\QQ)}.
\end{equation*}
We will use this equality to inductively compute the rational torsion order of
$J$. Suppose we are able to compute the rational torsion order of $A$ for some
abelian subvariety $A$. The rational torsion of $E$ can be identified using
Nagell-Lutz. So we are able to compute $\#(A\times E)(\QQ)$ and $\#(A\cap
E)(\QQ)$. The hard part is $\#\ker\gamma$ and in general, we will only be able
to bound this term.

Let $\beta:H^1(\QQ, A\cap E)\to H^1(\QQ, E)$ be the map induced by the
inclusion of $A\cap E$ into $E$. Then $\ker\gamma\subseteq \ker\beta$. We have
that $A\cap E$ is some finite Galois group of $E$ and it is often the case that
$A\cap E=E[n]$ for some positive integer $n$. In this case, by Kummer Theory,
we have that
\[
    \#\ker\gamma\leq \#\ker\beta = \frac{\# E(\QQ)}{\# (nE(\QQ))}.
\]
Putting this altogether, when $A\cap E=E[n]$ for some positive integer $n$,
\begin{equation}
    \label{eq:galois_cohomology_bound}
    \# (A+E)(\QQ) \leq
    \frac{\# (A\times E)(\QQ)\#E(\QQ)}{\# (A\cap E)(\QQ)\#(nE(\QQ))}. 
\end{equation}

This now yields an inductive algorithm for computing the rational torsion order
of any rank-0 abelian totally split subvariety $X$ of $J_0(N)$.

\subsection{Reducing to smaller abelian subvarieties}%
\label{sub:reducing_to_smaller_abelian_subvarieties}

We now take advantage to \eqref{eq:eta_equation} to show that it often suffices
to verify the Generalized Ogg Conjecture on some abelian subvarieties of $J$.
\begin{proposition}
    Let $E$ be finite subset of $J_0(N)(\QQ)_\tor$ containing $C_N(\QQ)$ (for
    example $E=E_r$~\eqref{eq:eta_equation}). Let $x_1,\ldots,x_r\in E$ be a
    set of representatives of $E/(C_N(\QQ))$. Suppose for each $i$, there
    exists an abelian subvariety $A_i$ of $J_0(N)$ satisfying the Generalized
    Ogg Conjecture and containing $x_i$. Then $C_N(\QQ)=J_0(N)(\QQ)_\tor$ so
    $J_0(N)$ satisfies the Generalized Ogg Conjecture.
\end{proposition}
\begin{proof}
    We already have $C_N(\QQ)\subseteq J_0(N)(\QQ)_\tor$. To show the
    reverse inclusion, let $x\in J_0(N)(\QQ)_\tor\subseteq E$. Then $x\in
    x_i+C_N(\QQ)$ for some representative $x_i$. This implies $x_i\in
    J_0(N)(\QQ)_\tor$. But $A_i$ satisfies the Generalized Ogg Conjecture so
    $x_i\in C_N(\QQ)$. Hence, $x\in C_N(\QQ)$.
\end{proof}

% We will give an example of using this trick in Example~\ref{}.

% \subsection{Summary of techniques}%
% \label{sub:summary_of_techniques} 

% \subsubsection{Elliptic curves case}%
% \label{ssub:elliptic_curves_case}

% \begin{proposition}
%     The $N$ for which $J_0(N)$ is an elliptic curve are:
%     \[
%         11, 14, 15, 17, 19, 20, 21, 24, 27, 32, 36, 49.
%     \]
% \end{proposition}

% E50/C gives it.
% [22, 26, 28, 50, 38, 44, 54, 52, 76]

% 3 dim stuff
% [30, 33, 34, 40, 45, 48, 64]
% 64 is when E is in cuspidal.

\begin{theorem}%
    \label{thm:verification_rank_zero}
    There are 45 totally split rank-0 Jacobians $J_0(N)$ (See
    Table~\ref{tab:split}). The Generalized Ogg Conjecture has been verified
    for all but 9 such $N$'s given by
    \[
        84,90,96,120,132,144,150,168,180
    \]
    In these cases, $[J_0(N)(\QQ)_\tor:C_N(\QQ)]$ bounded by a power of 2. See
    Table~\ref{tab:rank_zero_bound}.
\end{theorem}
\begin{proof}
    We use Section~\ref{sec:galois_cohomology_bounds} to bound
    $[J_0(N)(\QQ)_\tor:C_N(\QQ)]$.

    If $N$ is not one of 
    \[
        84,90,96,120,132,144,150,168,180,
    \]
    then we use Algorithm~\ref{alg:final_bound} to show that the index is 1.
\end{proof}
\begin{algorithm}{}%
    \label{alg:final_bound}
    Given a totally split rank-0 Jacobian $J=J_0(N)$ of dimension $k$, this
    algorithm will output a bound on $[J(\QQ)_\tor:C_N(\QQ)]$, where $C_N$ is
    the cuspidal subgroup of $J$.
    \begin{enumerate}
        \item{} [Upper bound by $E_{50}$]
            Set $r=50$ in~\eqref{eq:eta_equation} and compute $E=E_{50}$.
        \item{} [Set of representatives]
            Compute a list of elements $x_1,\ldots,x_s\in E_{50}$ that form a
            set of representatives for $E_{50}/C(N)$.
        \item{} [Decompose $J$]
            Decompose $J$~\ref{sub:decomposition_of_j_} into $J=\sum_{i=1} ^k
            E_j$, where $E_j$ are elliptic curves. 
        \item{} [Reduce to smaller abelian varieties]
            Let $V$ be the set of abelian subvarieties obtained by taking sums
            of $E_j$'s. Let each $x_i$, find $A_i\in V$ of minimal dimension
            such that $x_i\in A_i$.
        \item{} [Galois cohomology]
            Use Subsection~\ref{sub:galois_cohomology} to verify the
            Generalized Ogg Conjecture on each $A_i$.
    \end{enumerate}
\end{algorithm}

\section{Example}
\label{sec:example}

Let $J=J_0(30)$. This is the first level for which \sage (Version 8.7) cannot
compute the order of the rational torsion subgroup. 

\subsection{Rational Torsion Subgroup}

This variety has dimension
3 and Mordell-Weil rank 0.
\begin{sagecode}
\begin{sagecell}
sage: J = J0(30)
sage: J.decomposition()
\end{sagecell}
\begin{sageout}
[
Simple abelian subvariety 15a(1,30) of dimension 1 of J0(30),
Simple abelian subvariety 15a(2,30) of dimension 1 of J0(30),
Simple abelian subvariety 30a(1,30) of dimension 1 of J0(30)
]
\end{sageout}
\end{sagecode}
\begin{sagecode}
\begin{sagecell}
sage: L = J.lseries()
sage: L.vanishes_at_1()
\end{sagecell}
\begin{sageout}
False
\end{sageout}
\end{sagecode}

The rational cuspidal subgroup $C(\QQ)$ provides a lower bound on the rational
torsion subgroup. The group $C(\QQ)$ is of order 192.
\begin{sagecode}
\begin{sagecell}
sage: J.rational_cuspidal_subgroup()
\end{sagecell}
\begin{sageout}
Finite subgroup with invariants [2, 4, 24] over QQ of Abelian variety J0(30) of dimension 3
\end{sageout}
\end{sagecode}

Let $A$ and $B$ be the old subvariety and new subvariety of $J_0(30)$. The
exact sequence
\begin{equation}
    \label{j30_ses}
    \begin{tikzcd}
        0 \arrow[r] &
        A\cap B \arrow[r,"d"] &
        A\times B \arrow[r,"s"] &
        J \arrow[r] &
        0,
    \end{tikzcd}
\end{equation}
where $d(x)=(x,-x)$ and $s(x,y)=(x+y)$ yields the sequence
\begin{equation}\label{j30_gal_co}
        \begin{tikzcd}
            0 \rar &
            (A\cap B)(\QQ) \rar &
            (A\times B)(\QQ) \rar
                 \ar[draw=none]{d}[name=X, anchor=center]{}
            & J(\QQ) \ar[rounded corners,
                to path={ -- ([xshift=2ex]\tikztostart.east)
                          |- (X.center) \tikztonodes
                          -| ([xshift=-2ex]\tikztotarget.west)
                          -- (\tikztotarget)}]{dll}[at end]{} \\
                          &
            H^1(\QQ,A\cap B) \rar["\gamma"] &
            H^1(\QQ,A\times B) \rar & \ldots
        \end{tikzcd}
\end{equation}

The old subvariety $A$ decomposes
as a product of two elliptic curves coming from $J_0(15)$ and the new
subvariety $B$ is the elliptic curve with Cremona label $30a1$.

Since $30/15$ is prime, the hypothesis of Corollary~\ref{cor:elliptic_decomp} is
satisfied so the old subvariety is isomorphic over $\QQ$ to $E\times F$, where
the Cremona labels are $E:15a1$, $F:15a8$. We can compute the rational torsion
order of elliptic curves so $\#A(\QQ)=32$ and $\#B(\QQ)=6$. The intersection
$A\cap B$ is is isomorphic as groups to $\ZZ/2\times \ZZ/2$ so $A\cap B= B[2]$.
This means $(A\cap B)(\QQ)\cong \ZZ/2$. 
\begin{sagecode}
\begin{sagecell}
sage: A = J.old_subvariety()
sage: B = J.new_subvariety()
sage: A.intersection(B)[0].invariants()
\end{sagecell}
\begin{sageout}
[2, 2]
\end{sageout}
\end{sagecode}

Moreover, 
\begin{equation}
    \begin{split}
        \ker (H^1(\QQ, A\cap B) \to&H^1(\QQ, A\times B))
    \subseteq 
    \\
    \ker (&H^1(\QQ, B[2]) \to H^1(\QQ,  B))
    =
    B(\QQ)/2B(\QQ).
    \end{split}
\end{equation}

This implies $\#\ker\gamma\leq 2$ in~\eqref{j30_gal_co}. Piecing everything
together
\[
    \# J_0(30)(\QQ) =
    \frac{\#A(\QQ)\cdot \#B(\QQ) \cdot \#\ker\gamma}{\#(A\cap B)(\QQ)}
    \leq 192.
\]
It follows that $J_0(30)(\QQ)=C(\QQ)$ and is isomorphic to $\ZZ/2\times
\ZZ/4\times \ZZ/24$.

\subsection{Reduction mod $p$ is not enough}

We have the isogeny
\[
    J_0(30) \sim E\times E\times F,
\]
where $E:15a1$ and $F:30a2$ are elliptic curves with $\#E(\QQ)=8$ and
$\#F(\QQ)=12$. This means that $U(S)\geq 8\cdot 8 \cdot 12=768$. This means
that the reduction mod $p$ strategy alone is not enough to show that
$J(\QQ)_\tor$ is cuspidal.

\documentclass[thesis.tex]{subfiles}

\begin{document}
    
\chapter{Enumerating the Odd Isogenies Class of Prime Level Subvarieties}%
\label{chap:isogeny_class}

Let $N$ be a prime number so that $J_0(N)$ is non-trivial so $N=11$ or $N\geq
17$. Let $A$ be a simple abelian subvariety of $J_0(N)$. The goal of this
chapter is to, under certain conditions, enumerate the $\QQ$-isomorphism
classes of abelian varieties isogenous to $A$ by an odd-degree $\QQ$-isogeny.
We will call this the \emph{odd-degree isogeny class} of $A$.

More precisely, let $\TT$ be the Hecke algebra of $J_0(N)$ and $\TT_A$ be the
image of $\TT$ in $\End(A)$. There exists a newform $f=\sum a_n q^n$ associated
$A$. By~\cite[Prop. 7.14]{shimura:intro}, $\TT_A$ is isomorphic to an order of
the Hecke eigenvalue field, $K_f=\QQ(\ldots,a_n,\ldots)$. The goal of this chapter
is to enumerate the odd-degree isogeny class of $A$ when $\TT_A$ is integrally
closed and (when Proposition~\ref{} works).

Unless otherwise stated, in this chapter, all abelian varieties, isogenies, and
isomorphisms are defined over $\QQ$.

The image of an isogeny is determined by its kernel up to isomorphism. So we
define an equivalence relation on the set of finite odd-order
$G_\QQ$-submodules of $A(\QQbar)$ given by $M_1\sim M_2$ if and only if
$A/M_1\isom A/M_2$. For any odd-order $G_\QQ$-submodule $X$ of $A(\QQbar)$, let
$\M(X)$ be the set of equivalence classes of finite odd-order
$G_\QQ$-submodules of $A(\QQbar)$ with a representative that is a submodule of
$X$. Let $A(\QQbar)_\odd$ be the odd-torsion part of $A(\QQbar)$. The goal of
this chapter can then be restated as enumerating a set of representatives for
$\M(A(\QQbar)_\odd)$ when $\TT_A$ is integrally closed and (when
Proposition~\ref{} works). 

We begin by showing every finite $G_\QQ$-submodule of $A(\QQbar)_\odd$ is a
$\TT_A$-module (Proposition~\ref{prop:G_modules_are_hecke}). This is useful
because the $\TT_A$-structure is more easily understood than the
$G_\QQ$-structure. For instance, we are now able to talk about the
$\TT_A$-support of finite odd-order $G_\QQ$-submodules. We then show that every
finite odd-order $G_\QQ$-submodule is equivalent to one support on $S$
(\ref{prop:bound_support}). Now we construct an ideal $Q$ of $\TT_A$ such that
$S$ is the set of primes dividing $Q$. So $\M(A[Q^\infty])=\M(A(\QQbar)_\odd$.
We will then give an method for determining an integer $k$ such that
$\M(A[Q^k])=\M(A[Q^\infty])=\M(A(\QQbar)_\odd)$
(Proposition~\ref{prop:bound_support}). Lastly, we give an algorithm for
enumerating the finite $G_\QQ$-submodules of $\M(A[Q^k])$. In summary, we have
the following algorithm.

\begin{algorithm}{Odd isogeny class}%
    \label{alg:odd_isogeny_class}
    Let $A$ be a simple abelian subvariety with $\TT_A$ integrally closed. This
    algorithm will enumerate the class of abelian varieties isomorphic
    isogenous to $A$ by an odd-degree isogeny.
    \begin{enumerate}
        \item{} [Class group representatives]
            Compute a set of odd integral representatives $H=\{C_i\}$ of
            $\Cl(\TT_A)$. Let $Q = (\prod_{C_i\in H} C_i)(\prod_{\p\in E} \p)$,
            where $E$ is the set of primes dividing the Eisenstein ideal.
        \item{} [Initialize search]
            Set $r=1$ and $X_{r-1}=\emptyset$.
        \item{} [$G_\QQ$-submodules of $A[Q^r]$]
            Use Algorithm~\ref{alg:enumerating_AX} to give a set, $X_r$, of
            representatives for $\M(A[Q^r])$.
        \item{} [Done?]
            By Corollary~\ref{prop:stop_looking}, if for all $x\in X_r$, $x\sim
            y$ for some $y\in X_{r-1}$, then $X_r$ is a set of representatives
            for $\M(A(\QQbar)_\odd)$. If this is not the case, increment $r$
            and repeat the last step.
        \item{} [Quotient and output]
            Output $A/M$ for $M\in X_r$.
    \end{enumerate}
\end{algorithm}

\section{Finite odd-order Galois Modules are Hecke}%
\label{sec:finite_odd_order_galois_modules_are_hecke}

The goal of this section is to prove every finite odd-order $G_\QQ$-submodule $M$
of $A(\QQbar)$ is a Hecke module (Proposition~\ref{prop:G_modules_are_hecke}). The
Galois action of $J(\QQbar)$ has been extensively studied by
Mazur~\cite{mazur:eisenstein}, so we weaken our hypothesis to $M$ a
$G_\QQ$-submodule of $J(\QQbar)_\odd$. This is allowed because $A$ is
$\TT[G_\QQ]$-stable.

It suffices to prove $M$ is $\TT$-stable for each $G_\QQ$-composition factor $V$
of $M[\ell^\infty]$ for $\ell>2$. The irreducibility of $V$ implies that it is
$\ell$-torsion. Ribet~\cite[Proposition 6.1]{ribet:semistable_gal} shows that
$\TT/\ell \TT$ is generated by $T_p$ for primes $p\nmid \ell N$. So we reduce
modulo $p$ for $p\nmid \ell N$, and use Eichler-Shimura to derived its
$\TT$-stability from its $G_\QQ$-stability.

\begin{proposition}\label{prop:G_modules_are_hecke}
    Suppose $M$ is a finite odd-order $G_\QQ$-submodule of $J_0(N)$, with $N$
    prime. Then $M$ is a $\TT[G_\QQ]$-module.
\end{proposition}
\begin{proof}
    It suffices to show $M$ is $\TT$-stable for each $\ell$-primary part. Let
    $\ell>2$ and assume $M\subseteq J[\ell^\infty]$. Let
    \[
        0 = M_0 \subsetneq \ldots \subsetneq M_n = M
    \]
    be an $G_\QQ$-composition series of $M$ with composition factors $X_i =
    M_i/M_{i-1}$. We proceed by induction on $n$ with the base
    case being the trivial $n=0$ case.

    Assume $M_{s-1}$ is an $\TT[G_\QQ]$-module. We will show $M_s$ is an
    $\TT[G_\QQ]$-module. Since $M_{s-1}$ is an $\TT[G_\QQ]$-module, for each
    $t\in \TT$, we have a well-defined map $t:X_s\to J(\QQbar)/M_{s-1}$. The
    goal is to show $t(X_s)\subseteq X_s$ for all $t\in \TT$.
    By~\cite[Proposition 2]{ribet:mult_p_finite}, $\TT/\ell \TT$ is generated
    by $T_p$ for $p\nmid \ell N$ so it suffices to show $T_p(X_s)\subseteq X_s$
    for prime $p\nmid \ell N$.

    Fix a prime $p\nmid \ell N$. Since $p$ does not divide $N$, $J$ has good
    reduction at $p$. Fix a place $\p$ over $p$. The reduction map yields an
    isomorphism~\cite[Theorem 1, Lemma 2]{serre-tate}
    \[
        \tau:J(\QQbar)[\ell^\infty] \riso J_{/\F_p} (\Fpbar)[\ell^\infty]
    \]
    sending $\Frob_\p$ to $F_p$, where $F_p$ is the absolute Frobenius on
    $J_{/\F_p}$. Under this isomorphism the natural $\TT$-action on $J(\QQ)$
    maps to the natural $\TT$-action on $J_{/\F_p}$~\cite[\S
    5.2]{ribet-stein:serre}. By Eicher-Shimura, $T_p = F+p/F\in
    \End(J_{/\F_p})$ so
    \[
        \tau(T_p X_s)
        = T_p\tau(X_s)
        = (F+p/F)\tau(X_s)
        = \tau((\Frob_\p+p/\Frob_\p)X_s)
        \subseteq \tau(X_s)
    \]
    hence, $T_p X_s\subseteq X_s$, as desired.
\end{proof}


\section{Non-Eisenstein modules are kernels of Hecke}%
\label{sec:non_eisenstein_modules_are_kernels_of_hecke}

In light of Section~\ref{sec:finite_odd_order_galois_modules_are_hecke}, all
finite odd-order $G_\QQ$-submodules of $J(\QQbar)$ are $\TT$-modules. We will
now use this fact freely and will often refer to the $\TT$-support of a finite
odd-order $G_\QQ$-submodule.

The goal now is to identify the finite odd-order $G_\QQ$-submodules of
$J(\QQbar)_\odd$ supported, as a $\TT$-module, on the non-Eisenstein primes by
their $\TT$-annihilators. Using Proposition~\ref{prop:G_modules_are_hecke}, we
can already do this in the irreducible case.

\begin{proposition}
    Let $\m$ be a non-Eisenstein prime of odd residue characteristic, if $M$ is
    a nonzero finite irreducible $G_\QQ$-submodule of $J(\QQbar)$ supported
    only on $\m$ as a $\TT$-module, then $M=J[\m]$.
\end{proposition}
\begin{proof}
    By Lemma~\ref{lem:cherry_street}, the $\TT$-annihilator of $M$ is $\m$.
    Hence, $M\subseteq J[\m]$ but $J[\m]$ is an irreducible
    $G_\QQ$-module~\cite[Proposition 14.2]{mazur:eisenstein} so $M=J[\m]$.
\end{proof}

The general case will follow from a mild adaptation of the work of David
Helm~\cite{helm:jacobian}. Helm considers the case of Jacobians, $J$, of
Shimura curves. One of the key inputs into Helm's proof is that for the maximal
ideals $\m$ in question is that if $T_\m J$ is the contravariant $\m$-adic Tate
module, then $T_\m J/\m T_\m J\cong J[\m]^\vee$ is dimension two over $\TT/\m$
and irreducible as a $G_\QQ$-module. By~\cite[Prop. 14.2]{mazur:eisenstein},
this is also the case for $J=J_0(N)$ with $N$ prime and $\m$ a non-Eisenstein
prime of odd residue characteristic.

\begin{theorem}[{\cite[Corollary 4.8]{helm:jacobian}}]%
    \label{thm:non_eisenstein_kernel_hecke}
    Let $M$ be a finite odd-order $G_\QQ$-module supported, as a $\TT$-module,
    only on the non-Eisenstein primes. If $I=\Ann_\TT(M)$, then $M=J[I]$.

    Moreover, since $A$ is both $\TT[G_\QQ]$-invariant, if $M$ is a finite
    odd-order $G_\QQ$-submodule of $A(\QQbar)$ supported, as a $\TT_A$-module,
    only on the non-Eisenstein primes. If $I=\Ann_{\TT_A}(M)$, then $M=A[I]$.
\end{theorem}
\begin{proof}
Since $M\subseteq J[I]$ and $\Supp_{\TT} M = \Supp_{\TT} J[I]$, it suffices to
prove $J[I]_\m = M_\m$ for each non-Eisenstein prime of odd residue
characteristic. So let $\m$ be a non-Eisenstein prime of odd residue
characteristic. We start by reviewing the contravariant Tate modules of $J$ and
proving Lemma~\ref{lem:finite_index}.

Let $T_\m J\isom \Hom(J[\m^\infty], \QQ_\ell/\ZZ_\ell)$ be the contravariant
Tate module at $\m$ and $\overline{\rho}_\m$ be the Galois representation
associated to $J[\m]^\vee$. Since $\m$ is an odd non-Eisenstein prime,
$\overline{\rho}_\m$ is an irreducible $G_\QQ$-representation of dimension 2
over $k_\m$ that is isomorphic to $J[\m]^\vee$~\cite[Prop.
14.2]{mazur:eisenstein}.

\begin{lemma}[{\cite[Lemma 4.6]{helm:jacobian}}]\label{lem:finite_index}
    Let $R$ be a $G_\QQ$-stable submodule of $T_\m J$ of finite index. Then
    $R=IT_\m J$ for some ideal $I$ of $\TT$.
\end{lemma}
\begin{proof}
    We proceed by induction on the maximal $G_\QQ$-composition series of $T_\m J/R$
    with the base case being the trivial length zero case. Let
    \[
        R = R_n \subsetneq R_{n-1} \subsetneq \cdots \subsetneq R_0 = T_\m J
    \]
    be a $G_\QQ$-composition series. By induction, $R_{n-1} = I'T_\m J$ for some
    $I'\subseteq \TT$.

    Consider $\m R_{n-1} + R$. This is a $G_\QQ$-module sitting between $R$ and
    $R_{n-1}$. By Nakayama's lemma, if $\m R_{n-1} R + R = R_{n-1}$, then
    $R=R_{n-1}$ which is a contradiction. Hence, $\m R_{n-1}+R=R$ so $R$
    contains $\m R_{n-1}$ and we can form the quotient.

    The module $R_{n-1}/\m R_{n-1}$ is $G_\QQ$-isomorphic to $(I'/\m
    I')\otimes_{\TT/\m} (T_\m J/\m T_\m J) \cong (I'/\m I')\otimes_{\TT/\m}
    J[\m]^\vee$, where $G_\QQ$ acts trivially on $I'/\m I'$. Let $V$ be the image
    of $R$ in $R_{n-1}/\m R_{n-1}$. Since $V$ is $G_\QQ$-invariant, and
    $J[\m]^\vee$ is irreducible, $V$ is given by $\hat{V}\otimes J[\m]^\vee$
    for some $\TT/\m$-subspace $\hat{V}$ of $I'/\m I'$. Let $I$ be the preimage
    of $\hat{V}$ in $I'$. Then $IT_m J = R$, since both contain $\m R_{n-1}$
    and map to $V$ modulo $\m R_{n-1}$.
\end{proof}


Let $B=J/M$ be the quotient abelian variety. Since $M$ is a $\TT$-module,
we may equipped with a $\TT$-action. So the projection $\phi:J \to B$ is an
$\TT[G_\QQ]$-isogeny with $\ker\phi = M$. For any odd non-Eisenstein prime
$\m$, $\phi$ induces the exact sequence
\[
    0 \to T_\m B \to T_\m J \to M^\vee _\m \to 0.
\]
In particular, the image of $T_\m B$ under $\phi$ is a finite index
$\TT[G_\QQ]$-submodule of $T_\m A$. By Lemma~\ref{lem:finite_index}, we can find
an ideal $I'$ of $\TT$ such that the image of $T_\m B$ is $I' _\m T_\m A$
for all odd non-Eisenstein primes $\m$. We have $M^\vee _\m = T_\m J / I'
T_\m J\cong J[I']_\m ^\vee$. Therefore, by taking annihilators of the dual,
we have that $I_\m = I'_\m$ and then by taking duals $M_\m = J[I]_\m$, as
desired.

\end{proof}
%\section{Enumerating representatives of $A[X]$}

%Assume $\TT_A$ is integrally closed and let $X$ be an odd ideal of $\TT_A$. The
%goal of this section is to give a set of representatives for $\M(A[X])$ by
%first enumerating the $G_\QQ$-submodules of $A[X]$. We
%have the decomposition $X=\bigoplus_{\p\in \Supp_{\TT_A} (X)} X[\p^{v_\p(X)}]$
%so it suffices to enumerate the $G_\QQ$-submodules of $A[\p^s]$ for any $s\geq
%1$. This will now follow from Mazur's study of the Galois action on torsion
%points~\cite[\S 14]{mazur:eisenstein}.

%\begin{proposition}[Mazur]\label{prop:all_G_subs}
%    Let $\p$ be a prime of residue characteristic $\ell>2$.
%    \begin{enumerate}
%        \item
%            If $\p$ is Eisenstein, then $A[\p]=C_A[\ell]\oplus \Sigma_A[\ell]$,
%            where $C_A=C\cap A$ and $\Sigma_A=\Sigma \cap A$ with $C$ and
%            $\Sigma$ the cuspidal and Shimura subgroups of $J$. The
%            $G_\QQ$-submodules of $A[\p]$ are the direct sum of $\ZZ$-submodules of
%            $C_A[\ell]$ and $\ZZ$-submodules of $\Sigma_A[\ell]$.
%        \item
%            if $\p$ is non-Eisenstein, $A[\p]$ is irreducible~\cite[Prop
%            14.2]{mazur:eisenstein} as as a $G_\QQ$-module so the only
%            $G_\QQ$-submodules are $0$ and $A[\p]$.
%    \end{enumerate}
%\end{proposition}
%\begin{proof}
%    The Eisenstein case is~\cite[Corollary 16.3]{mazur:eisenstein} along with
%    the fact that $C[\ell]\isom \ZZ/\ell$ and $\Sigma[\ell]\isom \mu\ell$.

%    The non-Eisenstein case is~\cite[Propositon 14.2]{mazur:eisenstein}.
%\end{proof}

%Let $a\in \p^{s-1}\setminus \p^s$ so $a$ generates $\p^{s-1}/\p^s$ as a
%$k_\p$-space. There exists a $\TT_A[G_\QQ]$-injection given by
%$\phi_s:A[\p^s]/A[\p^{s-1}]\to A[\p]$. The $G_\QQ$-submodules, $M$, of
%$A[\p^s]$ are then the $\ZZ$-submodules of $A[\p^s]$ such that $M\cap
%A[\p^{s-1}]$ and $\phi_s(M)$ are $G_\QQ$-submodules. This yields the following
%algorithm.

%\begin{algorithm}{Enumerating $G_\QQ$-submodules of {$A[X]$}}%
%    \label{alg:enumerating_AX}
%    Given an odd ideal $X$ of $\TT_A$. This algorithm will output a set of
%    representatives for $\M(A[X])$.
%    \begin{enumerate}
%        \item{} [Factor $X$]
%            Compute the factorization $X=\prod \p_i ^{e_i}$.
%        \item{} [Non-Eisenstein part]
%            For each non-Eisenstein $\p_i$, let $W_i=\{A[\p^j]: 0 \leq j
%            \leq e_i\}$.
%        \item{} [Eisenstein part]
%            For each Eisenstein $\p_i$, set $V_1$ to be the set of
%            $G_\QQ$-submodules of $A[\p]$. For $j=2,\ldots,e_i$, set $V_j$ to
%            be the set of $\ZZ$-submodules, $M$, of $A[\p^j]$ such that $M\cap
%            A[\p^{j-1}]\in V_{j-1}$ and $\phi(M)\in V_1$.
%        \item{} [Combine]
%            Combine the $W_i$'s to form $W=\{\sum_i M_i: M_i \in W_i\}$.
%        \item{} [Output representatives]
%            Use isomorphism testing 
%            %TODO
%            % (Algorithm~\ref{alg:ism_testing})
%            to produce a set of representatives of $\M(A[X])$ of elements of
%            $W$.
%    \end{enumerate}
%\end{algorithm}

\section{Bounding non-Eisenstein part}%
\label{sec:non_eisenstein_part}

In this section, we will show how to how to bound the odd-degree non-Eisenstein
isogenies by the class group of $\TT_A$.

\begin{proposition}
    \label{prop:acbd_iso}
    Let $A\subseteq J_0(N)$ be a simple abelian subvariety with $N$ prime.
    Suppose $\TT_A$ is integrally closed. Suppose $aC=bD$. Then there is an
    isomorphism $\varphi:A/A[D] \to A/A[C]$ with $b\varphi = a$.
\end{proposition}
\begin{proof}
    We will first establish the isomorphisms $b:A/A[bD]\to A/A[D]$ and
    $a:A/A[aC]\to A/A[C]$.
    \[
        \begin{tikzcd}
            0
            \arrow[r]
            &
            (A/A[D])[b]
            \arrow[hookrightarrow]{r}
            &
            A/A[D]
            \arrow[twoheadrightarrow]{r}{b}
            &
            A/A[D]
            \arrow[r]
            &
            0
        \end{tikzcd}.
    \]
    We have that $x\in (A/A[D])[b]$ if and only if $bx \in A[D]$ if and only if
    $x \in A[bD]$. Hence, $(A/A[D])[b]=A[bD]/A[D]$ so $b:A/A[bD]\xra{\sim}
    A[D]$. A similar argument applies for $a$. Now using $aC=bD$, we have
    \[
        \begin{tikzcd}
            A/A[bD]
            \arrow[r, "\sim", "b"']
            \ar[equal]{d}
            &
            A/A[D]
            \ar[d, "\sim" labl, "\phi"]
            \\
            A/A[a C]
            \arrow[r, "\sim", "a"']
            &
            A/A[C],
        \end{tikzcd}
    \]
    as desired.
\end{proof}

\begin{theorem}[Frank Calegari]
    \label{thm:frank}
    Let $A\subseteq J_0(N)$ be a simple abelian subvariety with $N$ prime.
    Suppose $\TT_A$ is integrally closed. Suppose $\varphi:A\to A'$ is an
    isogeny with $\ker\varphi$ supported on the non-Eisenstein primes of odd
    residue characteristic. Then
    \[
        A' \isom A/A[C]
    \]
    for some $C\in \H$.
\end{theorem}
\begin{proof}
    Let $M=\ker\varphi$. Then $M$ is a finite odd-order $G_\QQ$-module. By
    Proposition~\ref{prop:G_modules_are_hecke}, $M$ is a $\TT_A[G_\QQ]$-module.
    By Theorem~\ref{thm:non_eisenstein_kernel_hecke}, there exists an ideal
    $D$ of $\TT_A$. Now there exists $a,b\in \TT_A$ and $C\in H$, such that
    $aD=bC$. By Proposition~\ref{prop:acbd_iso},
    \[
        A'\isom A/A[D]\isom A/A[C].
    \]
\end{proof}


\section{Bounding support and valuations}%
\label{sec:bounding_support_and_valuations}

The goal of this section is to allow us to change the non-Eisenstein part
without missing up the Eisenstein part too much.


\begin{proposition}
    Let $A\subseteq J_0(N)$ be a simple abelian subvariety with $N$ prime.
    Suppose $\TT_A$ is integrally closed. Suppose $X\subseteq A(\QQbar)_\odd$.
    Let $X_{ne} = \sum_{\p \in \P_{ne}} X[\p^\infty]$. For each $\p_j\in \P_e$,
    let $e_j$ be the minimal integer such that $X[\p_j ^\infty]\subseteq X[\p_j
    ^{e_j}]$. $X\sim A[C]\oplus Y_e$, where $Y_e\subseteq \sum X[\p_j ^{e_j}]$.
\end{proposition}
\begin{proof}
    By Proposition~\ref{prop:G_modules_are_hecke}, $M$ is a
    $\TT_A[G_\QQ]$-module. By Theorem~\ref{thm:non_eisenstein_kernel_hecke},
    there exists an ideal $D$ of $\TT_A$ such that $X_{ne}=A[D]$. Now there
    exists $a,b\in \TT_A$ and $C\in H$, such that $aD=bC$. By
    Proposition~\ref{prop:acbd_iso}, there is an isomorphism $\varphi:A/X_{ne}\to
    A/A[C]$ satisfying $b\varphi=a$.

    Now consider the isogeny $\psi:A/X_e\to A/(X_{ne}+X_e)$ given by
    quotienting by $X_e$. Under $\varphi$, this
    induces the isogeny $\psi':A/A[C]\to A/(A[C]+\varphi(X_e))$ given by
    quotienting by $\varphi(X_e)$. So we have the following exact sequence.
    \[
        \begin{tikzcd}
            A/X_{ne}
            \arrow[r, "\psi"]
            \arrow[d, "\varphi"]
            &
            A/(X_{ne}+X_e)
            \arrow[d, "\varphi'"]
            \\
            A/A[C]
            \arrow[r, "\psi'"]
            &
            A/(A[C]+\varphi(X_e)).
        \end{tikzcd}
    \] 

    Let $Y_e = \varphi(X_e)\cap A[\I^\infty]$. We will show that
    \begin{equation}
        \label{eq:ac_ye}
        A[C]+\varphi(X_e) = A[C]+Y_e.
    \end{equation}
    Since $A[C]$ is supported away from $\P_e$, we have that \eqref{eq:ac_ye}
    holds on $\p$-primary parts for $\p\in \P_e$. Let $\p\in \P_{ne}$. To show
    \eqref{eq:ac_ye} holds on $\p$-primary parts for $\p \in \P_{ne}$, it
    suffices to show that $\varphi(X_e)[\p^\infty]\subseteq A[C]$. We have that
    \[
        v_\p(\varphi) = v_\p(a)-v_\p(b) = v_\p(D) - v_\p(C).
    \]
    Since $X_e$ is supported away from $\p$, by Lemma~\ref{},
    $\varphi(X_e)[\p^\infty] \subseteq $ so by Lemma, we are done.
\end{proof}

\section{Eisenstein part}%
\label{sec:eisenstein_part}

\begin{proposition}
    Suppose $A\subseteq J_0(N)$ is a simple abelian subvariety with $N$ prime.
    Let $\p$ be an Eisenstein prime of $\TT_A$ of odd residue characteristic
    $p$. Let $C$ and $\Sigma$
    be the cuspidal and Shimura subgroups of $J_0(N)$. Suppose $A[\p]\neq 0$,
    then $A[\p]=J_0(N)[\p]=C[p^\infty]\oplus \Sigma[p^\infty]$. Moreover, the
    $G_\QQ$-submodules of $A[\p]$ are of the form $H_C\oplus H_\Sigma$, where
    $H_C\subseteq C[p^\infty]$ and $H_\Sigma\subseteq \Sigma[p^\infty]$ are any
    subgroups.
\end{proposition}
\begin{proof}
    %TODO
\end{proof}

\begin{corollary}
    Suppose $A\subseteq J_0(N)$ is a simple abelian subvariety with $N$ prime.
    Let $\p$ be an Eisenstein prime of $\TT_A$ with residue characteristic
    $p>2$ with $p$ dividing $\mathrm{num}((N-1)/12)$ exactly. Suppose $\p$ is
    principally generated by $\alpha$. If there exists $L$ such that things are
    true then blah.
\end{corollary}

\begin{proposition}{{\cite[Prop. 4.5]{klosin-papikian:ribet}}}%
    \label{prop:eisenstein_cyclic}
    Let $A\subseteq J_0(N)$ be a simple abelian subvariety with $N$ prime. Let
    $\p$ be an Eisenstein prime of $\TT_A$ of odd residue characteristic $p$. 
\end{proposition}
\begin{proof}
    We may assume $M\subseteq A[\p^2]$.

    \begin{lemma}[{\cite[Lem. 4.6]{klosin-papikian:ribet}}]
        The number field $K=\QQ(M)$ is an abelian extension unramified away from $p, N$.
    \end{lemma}
    \begin{proof}
        We begin by localizing at $\p$ to show $M\cong $ Since $M$ is finite as a set, $M=M_\p$ and is
        finite as a $(\TT_A)_\p$-module. By assumption, $(\TT_A)_\m$ is a DVR\@, so
        \[
            M\cong (\TT_A)_\p / \p^{s_1} \times\cdots \times (\TT_A)_\p /
            \p^{s_r}.
        \]
        Since $\dim_{\TT_A/\p} A[\p]=2$ and $M[\p]\cong (\TT_A / \p)^r
        \subsetneq A[\p]$ so $r=1$. Therefore, $M\cong \TT_A/\p^s$ for some
        $s\geq 0$.

        Recall the elements of $\TT_A$ are defined over $\QQ$ so they commute
        with elements of $G_\QQ$, therefore,
        \[
            \Gal(K/\QQ)\subseteq \Aut_{\TT_A} (M) \cong \Aut_{\TT_A}
            (\TT_A/\p^s) \cong (\TT_A/\p^s)^\times.
        \]
        Since $\TT_A$ is isomorphic to an order of a number field,
        $\Gal(K/\QQ)$ is abelian. Since $A$ has good reduction away from $N$,
        $K/\QQ$ is unramified away from $p, N$.
    \end{proof}

    \begin{lemma}
        The number field $K$ is a subfield of $\QQ(\mu_{p^2}, \mu_N)$.
    \end{lemma}
    \begin{proof}
        By assumption, $\p$ is principally generated by some $\alpha\in \TT_A$. We
        have that $\ker \alpha \cap A[\p^2], \alpha(A[\p^2]) \subseteq A[\p]$ so
        restricting to $M$, we have the sequence
        \begin{equation}
            \label{eq:M_Ap}
            \begin{tikzcd}
                0 \arrow[r] &
                M\cap A[\p] \arrow[r] &
                M \arrow[r] &
                M\cap A[\p] \arrow[r] &
                0
            \end{tikzcd} 
        \end{equation}
        Since $M\subsetneq A[\p]$, $M\cap A[\p]$ is either $C[p]$ or
        $\Sigma[p]$. 

        Let $F=\QQ(\mu_p, \mu_N)$. Then $\Gal(\QQbar/F)$ acts trivially $pM$,
        so elements of $\Gal(KF/F)$ are identified with elements $a$ of 
        $(\ZZ/p^2)^\times$ such that $ap\equiv p \pmod{p^2}$, or equivalently,
        $a\equiv 1 \pmod{p}$. The elements with this property form a cyclic
        subgroup of degree $p$ so $[KF:F]=1$ or $p$.

        We have 
        \[
            \Gal(\QQ(\mu_p ^{n_1}, \mu_N ^{n_2})/F \cong \ZZ/p^{n_1-1} \times
            \ZZ/N^{n_2-1}.
        \]
        Since $[KF:F]=1$ or $p$, $KF\subseteq F(\mu_p ^{n_1})$ is a subfield of
        degree 1 or $p$. In either case, $K\subseteq KF \subseteq \QQ(\mu_p ^2,
        \mu_N)$.
    \end{proof}
\end{proof}




\section{Eisenstein Isogenies}%
\label{sec:eisenstein_isogenies}


\begin{proposition}
    Let $A$ be a simple abelian subvariety of $J_0(N)$ with $N$ prime. Suppose
    $\p\subseteq \TT_A$ is an Eisenstein prime with residue characteristic
    $p>2$. Suppose $p$ divides $\mathrm{num}((N-1)/12)$ exactly. Let $M$ be a
    $\TT_A[G_\QQ]$-module supported, as a $\TT_A$-module, only on $\p$.
\end{proposition}

\begin{lemma}
    \label{lem:abelian_extension}
    Let $K=\QQ(M)$. Then $K$ is an abelian extension unramified away from $p$
    and $N$ so $K\subseteq \QQ(\mu_{p^{n_1}N^{n_2}})$.
\end{lemma}
\begin{proof}
    We first establish that $K$ is an abelian extension. Since $\p$ is a DVR,
    by localizing at $\p$,
    \[
        M \cong M_\p \cong (\TT_A)_\p/\p^{s_1}\times \ldots\times
        (\TT_A)_\p/\p^{s_r},
    \]
    for some $s_i\geq 1$. Since $M[\p]\subsetneq A[\p]$, $r=\dim_{(\TT_A)/\p}
    M[\p]<\dim_{(\TT_A)/\p} A[\p]=2$. Therefore, $M\cong (\TT_A)/\p^{s_1}$.

    The elements of $\TT_A$ are defined over $\QQ$, so
    \[
        \Gal(K/\QQ) \subseteq \Aut_{\TT_A}((\TT_A)/\p^{s_1}) =
        ((\TT_A)/\p^{s_1})^\times.
    \]
    Therefore, since $\TT_A$ is isomorphic to an order in a number field, $K$
    is an abelian extension. Moreover, $K$ is unramified away from $p$ and $N$
    since $A$ has good reduction away from $N$.
\end{proof}

\begin{lemma}
    We have $K\subseteq \QQ(\mu_{p^2N})$.
\end{lemma}
\begin{proof}
    Recall $\p$ is principally generated by $\alpha$. Following the second
    paragraph of~\cite{mazur:eisenstein}, we have the exact sequence
    \[
        \begin{tikzcd}
            0 \arrow[r] &
            A[\p] \arrow[r] &
            A[\p^2] \arrow[r, "\alpha"] &
            A[\p] \arrow[r] &
            0.
        \end{tikzcd}
    \]
    Since $M\subseteq A[\p^2]$, this induces the exact sequence
    \begin{equation}
        \label{eq:McapAp}
        \begin{tikzcd}
            0 \arrow[r] &
            M\cap A[\p] \arrow[r] &
            M \arrow[r, "\alpha"] &
            M\cap A[\p] \arrow[r] &
            0.
        \end{tikzcd}
    \end{equation}
    We have that $A[\p]\cong C[p]\times \Sigma[p]$ so $M\cap A[\p]$ is either
    $C[p]$ or $\Sigma[p]$. In either case, $\# M\cap A[\p] = p$ and $pM=M\cap
    A[\p]$.
    
    Let $F=\QQ(\mu_{pN})$. Then $\Gal(\QQbar/F)$ acts trivially on $pM$ in
    \eqref{eq:McapAp}, so $\Gal(KF/F)$ is in the subgroup of units of $(\ZZ/p^2
    \ZZ)$ which satisfies $ap\equiv p \pmod{p^2}$. The subgroup of such units
    form a cyclic subgroup of order $p$ in $(\ZZ/p^2\ZZ)^\times$. Hence, $KF/F$
    is an abelian extension of degree $1$ or $p$, unramified away from $p$ and
    $N$. So $KF\subseteq \QQ(\mu_{p^{n_1}N^{n_2})}$ for some $n_1,n_2\geq 1$.
    We have
    \[
        \Gal(\QQ(\mu_{p^{n_1}N^{n_2}})/F) \cong 
        \ZZ/p^{n_1-1} \times \ZZ/N^{n_2-1}.
    \]
    Since $KF$ is an extension of degree dividing $p$ and $p\nmid N$,
    $K\subseteq KF \subseteq \QQ(\mu_{p^2N})$, as desired.
\end{proof}


\section{Combining Eisenstein and non-Eisenstein parts}%
\label{sec:combining_eisenstein_and_non_eisenstein_parts}

\begin{corollary}
    Suppose all things. Suppose $\varphi:A\to A'$ is an odd-degree isogeny.
    Then $A'\isom A/A[X]$ with $X=X_C\oplus X_\Sigma$ for some cyclic subgroups
    $X_C\subseteq C$ and $X_\Sigma\subseteq \Sigma$.
\end{corollary}
\begin{proof}
    Let $M=\ker\varphi$. By Theorem~\ref{}, $M$ is a $\TT_A[G_\QQ]$-module. Let
    $S_e = \{\p \in \Supp_{\TT_A}: \p \mid \I\}$ and $S_{ne} = \{\p \in
    \Supp_{\TT_A}: \p \nmid \I\}$ be the sets of Eisenstein and non-Eisenstein
    primes in the support of $M$. Let $M_e = M \cap (\oplus_{\p \in S_e}
    A[\p^\infty])$ and $M_{ne} = M\cap (\oplus_{\p \in S_{ne}} A[\p^\infty])$.
    We have $M = M_e = M_{ne}$.

    By~\ref{}, $M\sim M_{ne}$. By~\ref{}, $M_{ne}\sim A[X]$ for some
    $X\subseteq A[\I]_\odd$. We have that that $A[\I]_\odd = C_\odd\oplus
    \Sigma_\odd$. Since $C_\odd$ is constant and $\Sigma_\odd$ is of
    $\mu$-type, the $G_\QQ$-submodules, $X$, of $A[\I]_\odd$ are of the form 
    $X=X_C\oplus X_\Sigma$ for some cyclic subgroups
    $X_C\subseteq C$ and $X_\Sigma\subseteq \Sigma$.
\end{proof}

\section{Computational Results}

Here are computational results.



\end{document}


\chapter{Tables}
\numberwithin{table}{chapter}
\begin{table}%
    \label{table:split}
    \centering
    \caption{Table of nontrivial totally split Jacobians along with dimension}
    \label{split_table}
    \begin{tabular}{rr}
        \toprule
        $N$ & $\dim J_0(N)$ \\
        \midrule
        11 & 1 \\
        14 & 1 \\
        15 & 1 \\
        17 & 1 \\
        19 & 1 \\
        20 & 1 \\
        21 & 1 \\
        22 & 2 \\
        24 & 1 \\
        26 & 2 \\
        27 & 1 \\
        28 & 2 \\
        30 & 3 \\
        32 & 1 \\
        33 & 3 \\
        34 & 3 \\
        36 & 1 \\
        37 & 2 \\
        \bottomrule
    \end{tabular}
    \begin{tabular}{rr}
        \toprule
        $N$ & $\dim J_0(N)$ \\
        \midrule
        38 & 4 \\
        40 & 3 \\
        42 & 5 \\
        44 & 4 \\
        45 & 3 \\
        48 & 3 \\
        49 & 1 \\
        50 & 2 \\
        52 & 5 \\
        54 & 4 \\
        56 & 5 \\
        57 & 5 \\
        60 & 7 \\
        64 & 3 \\
        66 & 9 \\
        72 & 5 \\
        75 & 5 \\
        76 & 8 \\
        \bottomrule
    \end{tabular}
    \begin{tabular}{rr}
        \toprule
        $N$ & $\dim J_0(N)$ \\
        \midrule
        80 & 7 \\
        84 & 11 \\
        90 & 11 \\
        96 & 9 \\
        99 & 9 \\
        100 & 7 \\
        108 & 10 \\
        112 & 11 \\
        114 & 17 \\
        120 & 17 \\
        121 & 6 \\
        128 & 9 \\
        132 & 19 \\
        144 & 13 \\
        150 & 19 \\
        168 & 25 \\
        180 & 25 \\
        192 & 21 \\
        \bottomrule
    \end{tabular}
    \begin{tabular}{rr}
        \toprule
        $N$ & $\dim J_0(N)$ \\
        \midrule
        198 & 29 \\
        200 & 19 \\
        216 & 25 \\
        240 & 37 \\
        288 & 33 \\
        300 & 43 \\
        336 & 53 \\
        360 & 57 \\
        384 & 49 \\
        396 & 61 \\
        400 & 43 \\
        432 & 55 \\
        576 & 73 \\
        600 & 97 \\
        720 & 121 \\
        1152 & 161 \\
        1200 & 205 \\
            & \\
        \bottomrule
    \end{tabular}
\end{table}

\begin{table}%
    \label{tab:rank_zero_bound}
    \centering
    \caption{Bound on $[J_0(N)(\QQ)_\tor:C(N)(\QQ)]$}
    \begin{tabular}{rr}
        \toprule
        $N$ & bound \\
        \midrule
        84 & 8 \\
        90 & 16 \\
        96 & 64 \\
        120 & 512 \\
        132 & 32 
        \bottomrule
    \end{tabular}
    \begin{tabular}{rr}
        \toprule
        $N$ & bound \\
        \midrule
        144 & 64 \\
        150 & 16 \\
        168 & 512 \\
        180 & 128 \\
        & 
        \bottomrule
    \end{tabular}
\end{table}
% (11, 1)
% (14, 1)
% (15, 1)
% (17, 1)
% (19, 1)
% (20, 1)
% (21, 1)
% (24, 1)
% (27, 1)
% (32, 1)
% (36, 1)
% (49, 1)
% (22, 1)
% (26, 1)
% (28, 1)
% (50, 1)
% (30, 1)
% (33, 1)
% (34, 1)
% (40, 1)
% (45, 1)
% (48, 1)
% (64, 1)
% (38, 1)
% (44, 1)
% (54, 1)
% (42, 1)
% (52, 1)
% (56, 1)
% (72, 1)
% (75, 1)
% (60, 1)
% (80, 1)
% (100, 1)
% (76, 1)
% (66, 1)
% (96, 64)
% (108, 1)
% (84, 8)
% (90, 16)
% (144, 64)
% (120, 512)
% (132, 32)
% (150, 16)
% (168, 512)
% (180, 128)



\bibliographystyle{amsalpha}
\bibliography{biblio}
\end{document}
