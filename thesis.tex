\documentclass{article}
\bibliographystyle{amsalpha}
\include{modabvar/macros}
\usepackage{url}
\usepackage{hyperref}
\usepackage{booktabs}

\author{Kevin Lui}
\title{Explicit Isogenies of $J_0(N)$}
\begin{document} 
\maketitle
\tableofcontents 
\listoftables

\newpage
\section{Introduction}%
\label{sec:intro}


In this thesis, we compute explicit with isogenies of $J_0(N)$. 

We say a Jacobian is totally split if it is $\QQ$-isogenous to a product of
elliptic curves. In Section~\ref{sec:totally_split}, we present techniques for
computing the rational torsion subgroup of totally splt $J_0(N)$.

In Section~\ref{sec:enum_isogenies}, we present techniques enumerating the
$\QQ$-isogeny class of $J_0(N)$ under various conditions.


\section{Background and random facts about $J_0(N)$}%
\label{sec:back}

\section{Totally Split Jacobians}%
\label{sec:totally_split}

A Jacobian is said to be \emph{totally split} if it is $\QQ$-isogenous to a
product of elliptic curves. The classification and construction of totally
split Jacobians of genus 2 curves has been
well-studied~\cite{bruin-doerksen:split_genus_two,kuhn:split_genus_two}
and large rational torsion subgroup of totally split Jacobians of genus 2 and 3
curves have been found~\cite{howe-leprevost-poonen:large}. The original
motivation of this project was to compute the rational torsion subgroup for as
many $J_0(N)$ as possible. The first $N$ for which Sage fails is $J_0(30)$
which happens to be a product of 3 elliptic curves. The author and his adviser
were able to compute the rational torsion subgroup using the fact that
$J_0(30)$ is totally split, the fact that rational torsion subgroups of
elliptic curves can be computed, and Galois cohomology. The goal of this
section is to prove that there are finitely many totally split $J_0(N)$, to
give a general (but totally impratical) method for computing the rational
torsion subgroup, and to present some more practical techniques for computing
the rational torsion subgroup. 



\subsection{Provably enumerating the set of totally split modular Jacobians}

The modular Jacobian $J_0(N)$ is totally split if and only if all cuspforms of
level $N$ have rational Hecke coefficients. We expect this to be rare. In fact,
Ralph Greenberg quickly gave an argument proving the set of totally split
$J_0(N)$ is finite. However, his argument did not give an effective method of
enumeration. The goal of this subsection is to provably enumerate the set of
$J_0(N)$ that are totally split. There are 71 of them that are nontrivial.

We have that $J_0(N)$ is totally split if and only if its dimension is equal to
the number of (modular) elliptic curves of conductor $N$. If $N$ is less than
the upper limit of conductors in the Cremona Database, then determining whether
$J_0(N)$ is totally split is a quick computation since there is a closed-form
formula for $\dim J_0(N)$.  

\begin{lemma}
    \label{lemma:good_primes}
    The only possible primes $p$ so that $J_0(p)$ is a totally split Jacobian are
    \[
        2, 3, 5, 7, 11, 13, 17, 19, 37.
    \]
\end{lemma}
\begin{proof}
    Let $J=J_0(p)$ be a totally split Jacobian of prime level. If $\dim J=0$,
    then $J$ is clearly totally split so assume $\dim J>0$. Then $J\cong
    \prod_f E_f$ with $E_f$'s elliptic curves of conductor $p$. Let $n$ denote
    the order of the rational cuspidal subgroup of $J$ which is, as mentioned
    above, known to be the numerator of $(p-1)/12$. By Emerton's proof of
    Stein's refined Eisenstein conjecture~\cite[Theorem B]{emerton:optimal}, if
    $l$ divides $n$, then $l$ divides the order of $E_f(\QQ)$ for some elliptic
    factor of $J$. But elliptic curves of prime conductor do not have much
    rational torsion.

    In particular, Miyawaki~\cite{miyawaki:ell_prime} enumerates all curves of prime
    power conductor with odd-order rational torsion. The largest prime
    conductor here being 37. This implies that if $p>37$, then $\#C(\QQ)$ must
    be a power of 2 so $p=2^a 3^b + 1$ for some $a\geq 0$ and $b\in \{0,1\}$.
    We now show $a\leq 2$.

    If $a>2$, then $C(\QQ)$ has an order 2 element which implies some $E_i$ has
    a rational 2-torsion point. As a result of Setzer~\cite[Theorem
    2]{setzer:ell_prime}, $p=17$ or $p=u^2+64$ for some integer $u$. We split into 2
    cases to show that $p$ is never $u^2+64$.

    Suppose $b=0$. Then $p$ is a Fermat prime and thus a Fermat number. Outside
    of $3$ and $5$, the recursive formula for Fermat numbers and an induction
    argument shows that the last digit of Fermat numbers is always $7$. But the
    only possible last digits of $u^2+64$ are $0, 3, 4, 5, 8, 9$.

    Suppose $b=1$. Then $2^a\cdot 3 = u^2+63$. This implies $3$ divides $u$ so
    $9$ divides $u^2$. But now the right-hand side is divisible by $9$ while the
    left is not.

    In conclusion, we know that if $J_0(p)$ is a totally split Jacobian, then $p\leq
    37$. A computer search then gives us the list found in the statement of the
    lemma.
\end{proof}

Call a positive integer $N$ \emph{good}, if $J_0(N)$ is totally split and
\emph{bad} otherwise. The simple factors of $J_0(N)$ are also factors of
$J_0(MN)$ for any positive integer $M$. Therefore, if $N$ is bad, then so is
any multiple of $N$. This allows us to prune branches while doing a search for
good integers on the divisiblity tree of the integers.

\begin{algorithm}{Enumerating Good Integers}
    \label{alg:find_split}
    Given a finite list of good primes $S$, this algorithm will halt and
    return the list of all good integers whose prime divisors lie in $S$
    if and only if this list is finite.
\end{algorithm}
\begin{enumerate}
    \item{} [Initialize] 
        Set $i=1$ and $M_1=S$. Here $M_i$ will represent the set of all good
        integers whose sum of exponents in the prime factorization is $i$.
    \item{} [Find prime multiples of $M_i$ are that good]
        Set
        \[
            M_{i+1}=\{pN: p\in S, N\in M_i, pN \text{ good}\}.
        \]
        If $M_{i+1}$ is non-empty, increment $i$ and repeat this step.
    \item [Return] 
        Return $\bigcup_i M_i$.
\end{enumerate}

\begin{theorem}
    There are 71 integers $N$ for which $J_0(N)$ is a totally split Jacobian of
    positive dimension. They are given in Table~\ref{split_table}.
\end{theorem}
\begin{proof}
    We run Algorithm~\ref{alg:find_split} on the good primes found in
    Lemma~\ref{lemma:good_primes} and it halts (luckily before reaching the end
    of the Cremona database).
\end{proof}

\begin{table}%
    \label{table:split}
    \centering
    \caption{Table of nontrivial totally split Jacobians along with dimension}
    \label{split_table}
    \begin{tabular}{rr}
        \toprule
        $N$ & $\dim J_0(N)$ \\
        \midrule
        11 & 1 \\
        14 & 1 \\
        15 & 1 \\
        17 & 1 \\
        19 & 1 \\
        20 & 1 \\
        21 & 1 \\
        22 & 2 \\
        24 & 1 \\
        26 & 2 \\
        27 & 1 \\
        28 & 2 \\
        30 & 3 \\
        32 & 1 \\
        33 & 3 \\
        34 & 3 \\
        36 & 1 \\
        37 & 2 \\
        \bottomrule
    \end{tabular}
    \begin{tabular}{rr}
        \toprule
        $N$ & $\dim J_0(N)$ \\
        \midrule
        38 & 4 \\
        40 & 3 \\
        42 & 5 \\
        44 & 4 \\
        45 & 3 \\
        48 & 3 \\
        49 & 1 \\
        50 & 2 \\
        52 & 5 \\
        54 & 4 \\
        56 & 5 \\
        57 & 5 \\
        60 & 7 \\
        64 & 3 \\
        66 & 9 \\
        72 & 5 \\
        75 & 5 \\
        76 & 8 \\
        \bottomrule
    \end{tabular}
    \begin{tabular}{rr}
        \toprule
        $N$ & $\dim J_0(N)$ \\
        \midrule
        80 & 7 \\
        84 & 11 \\
        90 & 11 \\
        96 & 9 \\
        99 & 9 \\
        100 & 7 \\
        108 & 10 \\
        112 & 11 \\
        114 & 17 \\
        120 & 17 \\
        121 & 6 \\
        128 & 9 \\
        132 & 19 \\
        144 & 13 \\
        150 & 19 \\
        168 & 25 \\
        180 & 25 \\
        192 & 21 \\
        \bottomrule
    \end{tabular}
    \begin{tabular}{rr}
        \toprule
        $N$ & $\dim J_0(N)$ \\
        \midrule
        198 & 29 \\
        200 & 19 \\
        216 & 25 \\
        240 & 37 \\
        288 & 33 \\
        300 & 43 \\
        336 & 53 \\
        360 & 57 \\
        384 & 49 \\
        396 & 61 \\
        400 & 43 \\
        432 & 55 \\
        576 & 73 \\
        600 & 97 \\
        720 & 121 \\
        1152 & 161 \\
        1200 & 205 \\
            & \\
        \bottomrule
    \end{tabular}
\end{table}

\subsection{Enumerating rational torsion is algorithmic}

\begin{proposition}
    Suppose $A$ is a totally split subvariety of $J_0(N)$ consisting of
    $J_0(L)$-optimal curves in their isogeny class. Then we can compute
    the following data:
    \begin{enumerate}
        \item
            A number field containing $\QQ(A[n])$ for any $n$.
        \item
            Let $n$ be a positive integer and $L$ be a number field containing
            $\QQ(A[n])$. The action of $\Gal(L/\QQ)$ on $A[n]$ can be
            determined.
        \item 
            The rational homology representation of elements of $A(K)_\tor$ for any
            number field $K$.
    \end{enumerate}
    Here the elements of $A[n]$ are represented by rational homology modulo
    integral homology.
\end{proposition}
\begin{proof}
    We will proceed by induction on the dimension, $d$, of $A$. We first
    consider the case $d=1$. This is the elliptic curve case.
    \begin{enumerate}
        \item 
            Let $n$ be a positive integer. Then using division polynomials, we
            can determine a number field $L$ that contains $\QQ(A[n])$.
        \item
            Let $n$ be a positive integer and $L$ be a number field containing
            $\QQ(A[n])$. Let $x\in A[n]$. This is on a point on rational
            homology. The determine the action of Galois, it suffices to
            identify $x$ as a point on the Weierstrass equation. To do this, we
            can modular symbols/integration pairing to identify $x$ as a point
            on the period lattice. We now numerically map $x$ to a point on the
            Weierstrass equation using the elliptic logarithm map. Lutz-Nagell
            determines what precision we will need.
        \item
            Let $K$ be a number field. Using reduction mod $p$, we can find an
            $m$ such that $A(K)_\tor \subseteq A[m]$. Using (1), we can define
            a number field $L$ that contains $\QQ(A[m])$. Then using (2), we
            can compute
            \[
                A(K)_\tor = A[m]^{\Gal(L/K)}.
            \]
    \end{enumerate}

    Now assume the result for all appropriate subvarieties of dimension $d\leq
    l$. Let $A$ be of dimension $l+1$ and write $A=B+C$, where $B$ is of
    dimension $l$ and $C$ is of dimension $1$.  We have the exact sequence
    \[
        0\to B\cap C \to B\times C \to A,
    \]
    where we identify $B\cap C$ as a subgroup of $B\times C$ via the
    anti-diagonal embedding. So $A=(B\times C)/(B\cap C)$. Let $r$ be the
    exponent of $B\cap C$. 
    \begin{enumerate}
        \item 
            For any integer $n$, $A[n] \subseteq B[nr]+C[nr]$. So a number
            field containing $\QQ(A[n])$ is the compositum  of the number
            fields containing $\QQ(B[nr])$ and $\QQ(C[nr])$ which can be
            determined by the inductive hypothesis.
        \item
            The Galois action can be determined on $A[n]$ by viewing $A[n]$ as
            a subgroup of $(B[nr]\times C[nr])/(B\cap C)$.
        \item
            Using reduction mod $p$, there exists an integer $m$ such that
            $A(K)_\tor\subseteq A[m]$. Using (1), we can find a number field
            $L$ that contains $\QQ(A[m])$. Then we use (2), to compute
            \[
                A(K)_\tor = A[m]^{\Gal(L/K)}.
            \]
    \end{enumerate}
\end{proof}


\section{Enumerating Isogenies}%
\label{sec:enum_isogenies}


\bibliography{modabvar/biblio}
\end{document}
