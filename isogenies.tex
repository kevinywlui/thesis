\documentclass{article}

\bibliographystyle{amsalpha}
 

% macros.tex
\usepackage{amsmath}
\usepackage{amsfonts}
\usepackage{amssymb}
\usepackage{amsthm}

\usepackage{url}


% You change everything, by adding \usepackage{times} to the document
% Preamble. Now all the roman letters will be set in times and all the
% sans serif stuff will be set in Helvetica. If you don't like times,
% you can try the packages: palatcm, charter, helvet, palatino, avant,
% newcent and bookman
% If you want to change explicitly to a certain font, use the command
% \fontfamily{XYZ}\selectfont whereby XYZ can be set to: pag for Adobe
% AvantGarde, pbk for Adobe Bookman, pcr for Adobe Courier, phv for
% Adobe Helvetica, pnc for Adobe NewCenturySchoolbook, ppl for Adobe
% Palatino, ptm for Adobe Times Roman, pzc for Adobe ZapfChancery
\newcommand{\courier}{\fontfamily{pcr}\selectfont}



\newcommand{\edit}[1]{\footnote{[[#1]]}\marginpar{\hfill {\sf[[\thefootnote]]}}}
%\newcommand{\edit}[1]{{\sl\small [[Todo: #1]]}}


%\author{William~A. Stein}

\newcommand{\Hbar}{\overline{H}}

\newcommand{\myhead}[3]{
\par\noindent
{Version #2}
\vspace{10ex}
\par\noindent
{\bf \LARGE #1}\\
\vspace{3ex}
\par\noindent
{\large W.\thinspace{}A. Stein}\\
{\small Department of Mathematics, Harvard University}\vspace{1ex}\\
#3     
\vspace{2ex}\par
}

\newcommand{\myheadauth}[3]{
\par\noindent
{Version #2}
\vspace{10ex}
\par\noindent
{\bf \LARGE #1}\\
\vspace{3ex}
\par\noindent
#3     
\vspace{5ex}\par
}

\usepackage{xspace}  % to allow for text macros that don't eat space 
\newcommand{\SAGE}{{\sf Sage}\xspace}
\newcommand{\sage}{\SAGE}
\newcommand{\gzero}{\Gamma_0(N)}
\newcommand{\esM}{\overline{\sM}}
\newcommand{\sM}{\boldsymbol{\mathcal{M}}}
\newcommand{\sS}{\boldsymbol{\mathcal{S}}}
\newcommand{\sB}{\boldsymbol{\mathcal{B}}}       
\newcommand{\bA}{\mathbb{A}}
\newcommand{\cK}{\mathcal{K}}
\newcommand{\Adual}{A^{\vee}}
\newcommand{\Bdual}{B^{\vee}}
\newcommand{\kr}[2]{\left(\frac{#1}{#2}\right)}

\newcommand{\defn}[1]{{\em #1}}
\newcommand{\solution}[1]{\vspace{1em}%
  \par\noindent{\bf Solution #1.} }
\newcommand{\todo}[1]{\noindent$\bullet$ {\small \textsf{#1}} $\bullet$\\}
\newcommand{\done}[1]{\noindent {\small \textsf{Done: #1}}\\}
\newcommand{\danger}[1]{\marginpar{\small \textsl{#1}}}
\renewcommand{\div}{\mbox{\rm div}}
\DeclareMathOperator{\GCD}{GCD}
\DeclareMathOperator{\Supp}{Supp}
\DeclareMathOperator{\CH}{CH}
\DeclareMathOperator{\sss}{ss}
\renewcommand{\ss}{\sss}
\DeclareMathOperator{\red}{red}
\DeclareMathOperator{\xgcd}{xgcd}
\DeclareMathOperator{\Kol}{Kol}
\DeclareMathOperator{\can}{can}
\DeclareMathOperator{\Cl}{Cl}
\DeclareMathOperator{\Mod}{Mod}
\DeclareMathOperator{\chr}{char}
\DeclareMathOperator{\charpoly}{charpoly}
\DeclareMathOperator{\cris}{cris}
\DeclareMathOperator{\dR}{dR}
\DeclareMathOperator{\Fil}{Fil}
\DeclareMathOperator{\ind}{ind}
\DeclareMathOperator{\im}{im}
\DeclareMathOperator{\oo}{\infty}
\DeclareMathOperator{\abs}{abs}
\DeclareMathOperator{\lcm}{lcm}
\DeclareMathOperator{\cores}{cores}
\DeclareMathOperator{\coker}{coker}
\DeclareMathOperator{\image}{image}
\DeclareMathOperator{\prt}{part}
\DeclareMathOperator{\proj}{proj}
\DeclareMathOperator{\Br}{Br}
\DeclareMathOperator{\Ann}{Ann}
\DeclareMathOperator{\End}{End}
\DeclareMathOperator{\Tan}{Tan}
\DeclareMathOperator{\Eis}{Eis}
\newcommand{\unity}{\mathbb{1}}
\DeclareMathOperator{\Pic}{Pic}
\DeclareMathOperator{\Tate}{Tate}
\DeclareMathOperator{\Vol}{Vol}
\DeclareMathOperator{\Vis}{Vis}
\DeclareMathOperator{\Reg}{Reg}
%\DeclareMathOperator{\myRes}{Res}
%\newcommand{\Res}{\myRes}
\DeclareMathOperator{\Res}{Res}
\newcommand{\an}{{\rm an}}
\DeclareMathOperator{\rank}{rank}
\DeclareMathOperator{\Sel}{Sel}
\DeclareMathOperator{\Mat}{Mat}
\DeclareMathOperator{\BSD}{BSD}
\DeclareMathOperator{\id}{id}
\DeclareMathOperator{\dz}{dz}
%\DeclareMathOperator{\Re}{Re}
\renewcommand{\Re}{\mbox{\rm Re}}
\DeclareMathOperator{\Imm}{Im}
\renewcommand{\Im}{\Imm}
\DeclareMathOperator{\Selmer}{Selmer}
\newcommand{\pfSel}{\widehat{\Sel}}
\newcommand{\qe}{\stackrel{\mbox{\tiny ?}}{=}}
\newcommand{\isog}{\simeq}
\newcommand{\e}{\mathbf{e}}
\newcommand{\bN}{\mathbf{N}}

% ---- SHA ----
\DeclareFontEncoding{OT2}{}{} % to enable usage of cyrillic fonts
  \newcommand{\textcyr}[1]{%
    {\fontencoding{OT2}\fontfamily{wncyr}\fontseries{m}\fontshape{n}%
     \selectfont #1}}
\newcommand{\Sha}{{\mbox{\textcyr{Sh}}}}

%\font\cyr=wncyr10 scaled \magstep 1
%\font\cyr=wncyr10

%\newcommand{\Sha}{{\cyr X}}
\newcommand{\Shaan}{\Sha_{\mbox{\tiny \rm an}}}
\newcommand{\TS}{Shafarevich-Tate group}

\newcommand{\Gam}{\Gamma}
\newcommand{\X}{\mathcal{X}}
\newcommand{\cH}{\mathcal{H}}
\newcommand{\cA}{\mathcal{A}}
\newcommand{\cF}{\mathcal{F}}
\newcommand{\cG}{\mathcal{G}}
\newcommand{\cJ}{\mathcal{J}}
\newcommand{\cL}{\mathcal{L}}
\newcommand{\cV}{\mathcal{V}}
\newcommand{\cO}{\mathcal{O}}
\newcommand{\cQ}{\mathcal{Q}}
\newcommand{\cX}{\mathcal{X}}
\newcommand{\ds}{\displaystyle}
\newcommand{\M}{\mathcal{M}}
\newcommand{\E}{\mathcal{E}}
\renewcommand{\L}{\mathcal{L}}
\newcommand{\J}{\mathcal{J}}
\DeclareMathOperator{\new}{new}
\DeclareMathOperator{\Morph}{Morph}
\DeclareMathOperator{\old}{old}
\DeclareMathOperator{\Sym}{Sym}
\DeclareMathOperator{\Symb}{Symb}
%\newcommand{\Sym}{\mathcal{S}{\rm ym}}
\newcommand{\dw}{\delta(w)} 
\newcommand{\dwh}{\widehat{\delta(w)}}      
\newcommand{\dlwh}{\widehat{\delta_\l(w)}} 
\newcommand{\dash}{-\!\!\!\!-\!\!\!\!-\!\!\!\!-} 
\DeclareMathOperator{\tor}{tor}  
\newcommand{\Frobl}{\Frob_{\ell}}
\newcommand{\tE}{\tilde{E}}
\renewcommand{\l}{\ell}
\renewcommand{\t}{\tau}
\DeclareMathOperator{\cond}{cond}
\DeclareMathOperator{\Spec}{Spec}
\DeclareMathOperator{\Div}{Div}
\DeclareMathOperator{\Jac}{Jac}
\DeclareMathOperator{\res}{res}
\DeclareMathOperator{\Ker}{Ker}
\DeclareMathOperator{\Coker}{Coker}
\DeclareMathOperator{\sep}{sep}
\DeclareMathOperator{\sign}{sign}
\DeclareMathOperator{\unr}{unr}
\newcommand{\sat}{\mathrm{sat}}
\newcommand{\N}{\mathcal{N}}
\newcommand{\U}{\mathcal{U}}
\newcommand{\Kbar}{\overline{K}}
\newcommand{\Lbar}{\overline{L}}
\newcommand{\gammabar}{\overline{\gamma}}
\newcommand{\q}{\mathbf{q}}
%\renewcommand{\star}{\times}
\newcommand{\gM}{\mathfrak{M}}
\newcommand{\gA}{\mathfrak{A}}
\newcommand{\gP}{\mathfrak{P}}
\newcommand{\bmu}{\boldsymbol{\mu}}
\newcommand{\union}{\cup}
\newcommand{\Tl}{T_{\ell}}
\newcommand{\into}{\rightarrow}
\newcommand{\onto}{\twoheadrightarrow}%  Surjection arrow

\newcommand{\meet}{\cap}
\newcommand{\cross}{\times}
\DeclareMathOperator{\md}{mod}
\DeclareMathOperator{\toric}{toric}
\DeclareMathOperator{\tors}{tors}
\DeclareMathOperator{\Frac}{Frac}
\DeclareMathOperator{\corank}{corank}
\newcommand{\rb}{\overline{\rho}}
\newcommand{\ra}{\rightarrow}
\newcommand{\xra}[1]{\xrightarrow{#1}}
\newcommand{\hra}{\hookrightarrow}
\newcommand{\la}{\leftarrow}
\newcommand{\lra}{\longrightarrow}
\newcommand{\riso}{\xrightarrow{\sim}}
\newcommand{\da}{\downarrow}
\newcommand{\ua}{\uparrow}
\newcommand{\con}{\equiv}
\newcommand{\Gm}{\mathbb{G}_m}
\newcommand{\pni}{\par\noindent}
\newcommand{\set}[1]{\{#1\}}
\newcommand{\iv}{^{-1}}
\newcommand{\alp}{\alpha}
\newcommand{\bq}{\mathbf{q}}
\newcommand{\cpp}{{\tt C++}}
\newcommand{\tensor}{\otimes}
\newcommand{\bg}{{\tt BruceGenus}}
\newcommand{\abcd}[4]{\left(
        \begin{smallmatrix}#1&#2\\#3&#4\end{smallmatrix}\right)}
\newcommand{\mthree}[9]{\left(
        \begin{matrix}#1&#2&#3\\#4&#5&#6\\#7&#8&#9
        \end{matrix}\right)}
\newcommand{\mtwo}[4]{\left(
        \begin{matrix}#1&#2\\#3&#4
        \end{matrix}\right)}
\newcommand{\vtwo}[2]{\left(
        \begin{matrix}#1\\#2
        \end{matrix}\right)}
\newcommand{\smallmtwo}[4]{\left(
        \begin{smallmatrix}#1&#2\\#3&#4
        \end{smallmatrix}\right)}
\newcommand{\twopii}{2\pi{}i{}}  
\newcommand{\eps}{\varepsilon}
\newcommand{\vphi}{\varphi}
\newcommand{\gp}{\mathfrak{p}}
\newcommand{\W}{\mathcal{W}}
\newcommand{\oz}{\overline{z}}
\newcommand{\Zpstar}{\Zp^{\star}}
\newcommand{\Zhat}{\widehat{\Z}}
\newcommand{\Zbar}{\overline{\Z}}
\newcommand{\Zl}{\Z_{\ell}}
\newcommand{\comment}[1]{}
\newcommand{\Q}{\mathbb{Q}}
\newcommand{\QQ}{\mathbb{Q}}
\newcommand{\GQ}{G_{\Q}}
\newcommand{\R}{\mathbb{R}}
\newcommand{\RR}{\mathbb{R}}
\newcommand{\PP}{\mathbb{P}}
\newcommand{\D}{{\mathbf D}}
\newcommand{\cC}{\mathcal{C}}
\newcommand{\cD}{\mathcal{D}}
\newcommand{\cP}{\mathcal{P}}
\newcommand{\cS}{\mathcal{S}}
\newcommand{\Sbar}{\overline{S}}
\newcommand{\K}{{\mathbb K}}
\newcommand{\C}{\mathbb{C}}
\newcommand{\CC}{\mathbb{C}}
\newcommand{\Cp}{{\mathbb C}_p}
\newcommand{\Sets}{\mbox{\rm\bf Sets}}
\newcommand{\bcC}{\boldsymbol{\mathcal{C}}}
\renewcommand{\P}{\mathbb{P}}
\newcommand{\Qbar}{\overline{\Q}}
\newcommand{\QQbar}{\overline{\Q}}
\newcommand{\kbar}{\overline{k}}
\newcommand{\dual}{\bot}
\newcommand{\T}{\mathbb{T}}
\newcommand{\TT}{\mathbb{T}}
\newcommand{\calT}{\mathcal{T}}
\newcommand{\cT}{\mathcal{T}}
\newcommand{\cbT}{\mathbb{\mathcal{T}}}
\newcommand{\cU}{\mathcal{U}}
\newcommand{\Z}{\mathbb{Z}}
\newcommand{\ZZ}{\mathbb{Z}}
\newcommand{\F}{\mathbb{F}}
\newcommand{\FF}{\mathbb{F}}
\newcommand{\Fl}{\F_{\ell}}
\newcommand{\Fell}{\Fl}
\newcommand{\Flbar}{\overline{\F}_{\ell}}
\newcommand{\Flnu}{\F_{\ell^{\nu}}}
\newcommand{\Fbar}{\overline{\F}}
\newcommand{\Fpbar}{\overline{\F}_p}
\newcommand{\fbar}{\overline{f}}
\newcommand{\Qp}{\Q_p}
\newcommand{\Ql}{\Q_{\ell}}
\newcommand{\Qell}{\Q_{\ell}}
\newcommand{\Qlbar}{\overline{\Q}_{\ell}}
\newcommand{\Qlnr}{\Q_{\ell}^{\text{nr}}}
\newcommand{\Qlur}{\Q_{\ell}^{\text{ur}}}
\newcommand{\Qltm}{\Q_{\ell}^{\text{tame}}}
\newcommand{\Qv}{\Q_v}
\newcommand{\Qpbar}{\Qbar_p}
\newcommand{\Zp}{\Z_p}
\newcommand{\Fp}{\F_p}
\newcommand{\Fq}{\F_q}
\newcommand{\Fqbar}{\overline{\F}_q}
\newcommand{\Ad}{Ad}
\newcommand{\adz}{\Ad^0}
\renewcommand{\O}{\mathcal{O}}
\newcommand{\A}{\mathcal{A}}
\newcommand{\Og}{O_{\gamma}}
\newcommand{\isom}{\cong}
\newcommand{\ncisom}{\approx}   % noncanonical isomorphism
\DeclareMathOperator{\ab}{ab}
\DeclareMathOperator{\alg}{alg}
\DeclareMathOperator{\Aut}{Aut}
\DeclareMathOperator{\Frob}{Frob}
\DeclareMathOperator{\Fr}{Fr}
\DeclareMathOperator{\Ver}{Ver}
\DeclareMathOperator{\Norm}{Norm}
\DeclareMathOperator{\Ind}{Ind}
\DeclareMathOperator{\norm}{norm}
\DeclareMathOperator{\disc}{disc}
\DeclareMathOperator{\ord}{ord}
\DeclareMathOperator{\GL}{GL}
\DeclareMathOperator{\PSL}{PSL}
\DeclareMathOperator{\PGL}{PGL}
\DeclareMathOperator{\Gal}{Gal}
\DeclareMathOperator{\SL}{SL}
\DeclareMathOperator{\SO}{SO}
\DeclareMathOperator{\WC}{WC}
\newcommand{\galq}{\Gal(\Qbar/\Q)}
\newcommand{\rhobar}{\overline{\rho}}
\newcommand{\cM}{\mathcal{M}}
\newcommand{\cB}{\mathcal{B}}
\newcommand{\cE}{\mathcal{E}}
\newcommand{\cR}{\mathcal{R}}
\newcommand{\et}{\text{\rm\'et}}

\newcommand{\sltwoz}{\SL_2(\Z)}
\newcommand{\sltwo}{\SL_2}
\newcommand{\gltwoz}{\GL_2(\Z)}
\newcommand{\mtwoz}{M_2(\Z)}
\newcommand{\gltwoq}{\GL_2(\Q)}
\newcommand{\gltwo}{\GL_2}
\newcommand{\gln}{\GL_n}
\newcommand{\psltwoz}{\PSL_2(\Z)}
\newcommand{\psltwo}{\PSL_2}
\newcommand{\h}{\mathfrak{h}}
\renewcommand{\a}{\mathfrak{a}}
\newcommand{\p}{\mathfrak{p}}
\newcommand{\m}{\mathfrak{m}}
\newcommand{\trho}{\tilde{\rho}}
\newcommand{\rhol}{\rho_{\ell}}
\newcommand{\rhoss}{\rho^{\text{ss}}}
\DeclareMathOperator{\tr}{tr}
\DeclareMathOperator{\order}{order}
\DeclareMathOperator{\ur}{ur}
\DeclareMathOperator{\Tr}{Tr}
\DeclareMathOperator{\Hom}{Hom}
\DeclareMathOperator{\Mor}{Mor}
\DeclareMathOperator{\HH}{H}
\renewcommand{\H}{\HH}
\DeclareMathOperator{\Ext}{Ext}
\DeclareMathOperator{\Tor}{Tor}
\newcommand{\smallzero}{\left(\begin{smallmatrix}0&0\\0&0
                        \end{smallmatrix}\right)}
\newcommand{\smallone}{\left(\begin{smallmatrix}1&0\\0&1
                        \end{smallmatrix}\right)}

\newcommand{\pari}{{\sc Pari}}
\newcommand{\magma}{{\sc Magma}}
\newcommand{\hecke}{{\sc Hecke}}
\newcommand{\lidia}{{\sc LiDIA}}

%%%% Theoremstyles
\theoremstyle{plain}
\newtheorem{theorem}{Theorem}[section]
\newtheorem{proposition}[theorem]{Proposition}
\newtheorem{corollary}[theorem]{Corollary}
\newtheorem{claim}[theorem]{Claim}
\newtheorem{lemma}[theorem]{Lemma}
\newtheorem{hypothesis}[theorem]{Hypothesis}
\newtheorem{conjecture}[theorem]{Conjecture}

\theoremstyle{definition}
\newtheorem{definition}[theorem]{Definition}
\newtheorem{question}[theorem]{Question}
\newtheorem{idea}[theorem]{Idea}
\newtheorem{project}[theorem]{Project}
\newtheorem{problem}[theorem]{Problem}
\newtheorem{openproblem}[theorem]{Open Problem}
\newtheorem{challenge}[theorem]{Challenge}

%\theoremstyle{remark}
\newtheorem{goal}[theorem]{Goal}
\newtheorem{remark}[theorem]{Remark}
\newtheorem{remarks}[theorem]{Remarks}
\newtheorem{example}[theorem]{Example}
\newtheorem{exercise}[theorem]{Exercise}

\numberwithin{equation}{section}
\numberwithin{figure}{section}
\numberwithin{table}{section}


% bulleted list environment
\newenvironment{bulletlist}
   {
      \begin{list}
         {$\bullet$}
         {
            \setlength{\itemsep}{.5ex}
            \setlength{\parsep}{0ex}
            \setlength{\parskip}{0ex}
            \setlength{\topsep}{.5ex}
         }
   }
   {
      \end{list}
   }
%end newenvironment

% bulleted list environment
\newenvironment{dashlist}
   {
      \begin{list}
         {---}
         {
            \setlength{\itemsep}{.5ex}
            \setlength{\parsep}{0ex}
            \setlength{\parskip}{0ex}
            \setlength{\topsep}{.5ex}
         }
   }
   {
      \end{list}
   }
%end newenvironment

% numbered list environment
\newcounter{listnum}
\newenvironment{numlist}
   {
      \begin{list}
            {{\em \thelistnum.}}{
            \usecounter{listnum}
            \setlength{\itemsep}{.5ex}
            \setlength{\parsep}{0ex}
            \setlength{\parskip}{0ex}
            \setlength{\topsep}{.5ex}
         }
   }
   {
      \end{list}
   }
%end newenvironment

\newcommand{\hd}[1]{\vspace{1ex}\noindent{\bf #1} }
\newcommand{\nf}[1]{\underline{#1}} 
\newcommand{\cbar}{\overline{c}}

\DeclareMathOperator{\rad}{rad}

\theoremstyle{definition}
\newtheorem{algor}[theorem]{Algorithm}
\newenvironment{algorithm}[1]{%
\begin{algor}[#1]\index{{\bf Algorithm}!#1}
}%
{\end{algor}}

\newenvironment{steps}%
{\begin{enumerate}\setlength{\itemsep}{0.1ex}}{\end{enumerate}}

\usepackage{color}
\usepackage{cprotect}
\usepackage{listings} 
\lstdefinelanguage{Sage}[]{Python}
{morekeywords={True,False,sage,singular},
sensitive=true}
\lstset{
  showtabs=False,
  showspaces=False,
  showstringspaces=False,
  commentstyle={\ttfamily\color{dredcolor}},
  keywordstyle={\ttfamily\color{dbluecolor}\bfseries},
  stringstyle ={\ttfamily\color{dgraycolor}\bfseries},
  language = Sage,
  basicstyle={\small \ttfamily},
  aboveskip=1em,
  belowskip=1em,
  backgroundcolor=\color{lightyellow},
  frame=single
}
\definecolor{lightyellow}{rgb}{1,1,.86}
\definecolor{dblackcolor}{rgb}{0.0,0.0,0.0}
\definecolor{dbluecolor}{rgb}{.01,.02,0.7}
\definecolor{dredcolor}{rgb}{0.8,0,0}
\definecolor{dgraycolor}{rgb}{0.30,0.3,0.30}
\definecolor{graycolor}{rgb}{0.35,0.35,0.35}
\newcommand{\dblue}{\color{dbluecolor}\bf}
\newcommand{\dred}{\color{dredcolor}\bf}
\newcommand{\dblack}{\color{dblackcolor}\bf}
\newcommand{\gray}{\color{graycolor}}

\newcommand{\dbd}[1]{\langle#1\rangle}   % make a diamond bracket d symbol

%%% Local Variables: 
%%% mode: latex
%%% TeX-master: t
%%% End: 



\renewcommand{\q}{\mathfrak{q}}
\renewcommand{\old}{\mathrm{old}}
\renewcommand{\tor}{\mathrm{tor}}
\newcommand{\f}{\mathfrak{f}}
\newcommand{\I}{\mathcal{I}}
\newcommand{\odd}{\mathrm{odd}}




\usepackage{microtype}
\usepackage{tikz-cd}
\renewcommand{\q}{\mathfrak{q}}
\newcommand{\f}{\mathfrak{f}}
\newcommand{\I}{\mathcal{I}}
\newcommand{\odd}{\mathrm{odd}}
\tikzset{labl/.style={anchor=south, rotate=90, inner sep=.5mm}}

\usepackage{algpseudocode}

\begin{document}

\tableofcontents
    
\section{Enumerating Isogenies of Prime Level Subvarieties}

The goal of this section is, under mild conditions, to give an algorithm to
enumerate the odd-degree $\QQ$-isogeny class of simple abelian subvarieties
$A_f$ of $J_0(N)$ for $N$ prime, up to isomorphism. Given a $\QQ$-isogeny
$\psi:A_f\to A'$ with kernel $K$, so
\[
    0 \to K \to A_f \overset{\psi}{\to} A' \to 0.
\]
Since the isogeny is defined over $\QQ$, $K$ is a finite $G_\QQ$-submodule of
$A_f(\QQbar)$ which determines $A'$, up to isomorphism. Conversely, for every
$G_\QQ$-submodule $K$ of $A_f(\QQbar)$, there exists an isogeny of $A_f$ with
kernel $K$. Therefore, we will enumerate the odd-degree isogeny class of $A_f$,
by enumerating the finite odd-order $G_\QQ$-submodules of $A_f(\QQbar)$, up to
isomorphism of the image of the corresponding isogeny. 

We now fix notation, terminology, assumptions.
\begin{itemize}
    \item{} [Isomorphisms and Isogenies]
        Unless otherwise stated, all isomorphisms and isogenies are defined
        over $\QQ$.
    \item{} [Jacobian notation]
        Let $J = J_0(N)$ with $N$ prime. To avoid trivialities, assume $\dim
        J>0$ or equivalently, $N=11$ or $N\geq 17$. Let $\TT$ be the Hecke
        algebra of $J$ and $\I\subseteq \TT$ be the Eisenstein ideal.
    \item{} [Abelian subvariety notation]
        Let $A=A_f\subseteq J$ be a simple abelian subvariety of $J$. We will
        fix a single simple abelian subvariety $A$ and its isogeny class will
        be our object of study. We have that $A$ is $\TT$-invariant so let
        $\TT_A$ be the image of $\TT$ in $\End(A)$ and $\I_A$ be the image of
        the Eisenstein ideal in $\End(A)$. 
    \item{} [Number theory of $\TT_A$ notation] 
        Let $K_f$ be the Hecke eigenvalue field of $f$ so $K_f$ is a number
        field of degree equal to $\dim A$. Moreover, $\TT_A$ is isomorphic to
        an order in $K_f$. We will implicitly fix some embedding of $\TT_A$ and
        treat $\TT_A$ as a subset of $K_f$. Let $\O_K$ be the ring of integers
        of $K_f$ and $\f$ be the conductor of $\O_K$ over $\TT_A$.
    \item{} [Galois-Hecke group ring notation]
        Let $G_\QQ=\Gal(\QQbar/\QQ)$. The Hecke operators are defined over
        $\QQ$, so $\TT_A$ commutes with $G_\QQ$ and we can form the group ring
        $\TT[G_\QQ]=\TT_A[G_\QQ]$.
    \item{} [Equivalence relation notation]
        For finite $G_\QQ$-submodules $M$ and $M'$, we define a relationship
        $M\sim M'$ if and only if $A/M \isom A/M'$. For any $G_\QQ$-submodule
        $P$ of $A(\QQbar)$, let $\M(P)$ be set of equivalence classes of finite
        $G_\QQ$-submodules of $A(\QQbar)_\odd$ with a representative that is a
        submodule of $P$. So the goal of this chapter can be restated as
        enumerating a set of representatives of $\M(A(\QQbar)_\odd)$.
    \item{} [Sets of primes notation]
        Let $S_0=\{\m\in \Spec\TT[\frac 12]: \m \nmid
        \mathfrak{f}\mathcal{I}\}$ and $S_1 = \Spec\TT[\frac 12]\setminus S_0$.
        The best fact about $S_0$ is that its elements are invertible and the
        best fact about $S_1$ is that it is finite.
    \item{} [$S_0$ and $S_1$ decomposition notation]
        For an $\TT[G_\QQ]$-module $M$ of $A(\QQbar)_\odd$, let $M_{S_i}$ be the largest
        $\TT[G_\QQ]$-submodule of $M$ supported on $S_i$ so $M_{S_i}=M[\prod_{\p\in S_i}
        \p^\infty]$. Observe that $M = M_{S_0}\oplus M_{S_1}$.
    \item{} [Odd]
        A module is said to be odd if it is finite with odd cardinality.
\end{itemize}

The goal of this section is Algorithm~\ref{alg:odd_isogeny_class} which
enumerates the odd isogeny class of $A$ when $\TT_A$ is integrally closed. From
experimental data this conditions seems for be mild, at least for small primes.
In particular, there are 450 simple abelian varieties of prime level less than
1000 but only 42 are not integrally closed.

The idea will be as follows. Let $M$ be a odd $G_\QQ$-submodule of
$A(\QQbar)_\odd$. In
subsection~\ref{sub:finite_odd_order_galois_modules_are_hecke}, we will show
that $M$ is an $\TT[G_\QQ]$-module. This is useful because the $\TT$-structure is more
easily understood than the $G_\QQ$-structure. For example, we are now able to
talk about the support of $M$ as a $\TT$-module. Using
Corollary~\ref{cor:bound_support}, we can give an explicit set $S$ such that
every odd $G_\QQ$-module is equivalent to one supported on $S$. In other words,
$\M(A(\QQbar)_\odd)=\M(A[Q^\infty])$, where $Q$ is some ideal with
$\Supp(Q)=S$. Using Algorithm~\ref{}, we can compute the $G_\QQ$-submodules of
$\M(A[Q^r])$ for any $r$ and Proposition~\ref{prop:stop_looking} produces a $k$
such that $\M(A[Q^k])=\M(A[Q^\infty])=\M(A(\QQbar))$. In summary, we have the
following algorithm.
\begin{algorithm}{Odd isogeny class}%
    \label{alg:odd_isogeny_class}
    Let $A$ be a simple abelian subvariety with $\TT_A$ integrally closed. This
    algorithm will enumerate the odd isogeny class of $A$ by giving a set of
    representatives of $\M(A(\QQbar)_\odd)$.
    \begin{enumerate}
        \item{} [Class group representatives]
            Compute a set of odd integral representatives $H=\{\q\}$ of
            $\Cl(\TT_A)$. Let $Q = (\prod_{\q\in H})(\prod_{\p\in S_1} \p)$. 
        \item{} [Initialize search]
            Set $r=1$ and $X_{r-1}=\emptyset$.
        \item{} [$G_\QQ$-submodules of $A[Q^r]$]
            Use Section~\ref{sub:enumerating_isogenies} to give a set, $X_r$,
            of representatives for $\M(A[Q^r])$.
        \item{} [Done?]
            By Corollary~\ref{prop:stop_looking}, if for all $x\in X_r$, $x\sim
            y$ for some $y\in X_{r-1}$, then $X_r$ is a set of representatives
            for $\M(A(\QQbar)_\odd)$. If this is not the case, increment $r$
            and repeat the last step.
        \item{} [Quotient and output]
            Output $A/M$ for $M\in X_r$.
    \end{enumerate}
\end{algorithm}

\subsection{Finite odd-order Galois Modules are Hecke}%
\label{sub:finite_odd_order_galois_modules_are_hecke}

The goal of this subsection is to prove every finite odd $G_\QQ$-submodule $M$
of $A(\QQbar)$ is a Hecke module (Theorem~\ref{thm:G_modules_are_Hecke}). The
Galois action of $J(\QQbar)$ has been extensively studied by
Mazur~\cite{mazur:eisenstein}, so we weaken our hypothesis to $M$ a
$G_\QQ$-submodule of $J(\QQbar)_\odd$. 

It suffices to prove $M$ is a $\TT$-module for each $G_\QQ$-composition factor $V$
of $M[\ell^\infty]$ for $\ell>2$. The irreducibility of $V$ implies that it is
$\ell$-torsion. A theorem of Ribet~\cite[Proposition 2]{ribet:mult_p_finite}
shows that $\TT/\ell \TT$ is generated by $T_p$ for primes $p\nmid \ell N$.
By~\cite[\S 14]{mazur:eisenstein}, these are exactly the unramified primes of
the Galois representation of $V$. We then reduce modulo $p$ and use
Eichler-Shimura to deduce the $\TT$-stability of $V$ from its $G_\QQ$-stability.

\begin{lemma}[Mazur]\label{lemma:cherry_street}
    Let $V$ be an irreducible $\TT[G_\QQ]$-subquotient of $J[\ell^\infty]$. Then
    $\Ann_\TT V$ is a maximal ideal $\m$ of $\TT$. Moreover, $V$ is
    $\TT[G_\QQ]$-isomorphic to a subquotient of $J[\m]$.

    % This is lemma is given in~\cite[\S 14, pg. 112], we are simply elaborating
    % on the details.
\end{lemma}
\begin{proof}
    This is given in~\cite[\S 14, pg. 112]{mazur:eisenstein}.
\end{proof}

\begin{theorem}[Mazur]\label{thm:irreducible_G_sub}
    Let $V$ be an irreducible $G_\QQ$-subquotient of $J[\ell^\infty]$. Then either
    \begin{itemize}
        \item
            $V\cong_{G_\QQ} \ZZ/\ell$ or $V\cong_{G_\QQ} \mu_\ell$ and is unramified away
            from $\ell$, or
        \item 
            $V\cong_{G_\QQ} J[\m]$ for some non-Eisenstein maximal $\m$ and is
            unramified away from $\ell N$.
    \end{itemize}
    In any case, $V$ is unramified away from $\ell N$.
\end{theorem}
\begin{proof}
    This is essentially~\cite[\S 14]{mazur:eisenstein}.

    Let $V$ be an irreducible $\TT[G_\QQ]$-module. By Lemma~\ref{lemma:cherry_street},
    $\m=\Ann_\TT(V)$ is maximal and $V$ is $\TT[G_\QQ]$-isomorphic to a subquotient of
    $J[\m]$. Either $\m$ is Eisenstein or it is not.
    \begin{enumerate}
        \item
            If $\m$ is Eisenstein, then the action of $\TT$ factors through
            $\ZZ$, so $V$ is irreducible as a $G_\QQ$-module. By~\cite[Proposition
            14.1]{mazur:eisenstein}, $J[\m]$ has a $G_\QQ$-composition series
            consisting of $\ZZ/\ell$ and $\mu_\ell$ (this is what Mazur calls
            admissible) so $V$ is isomorphic as $G_\QQ$-module to either $\ZZ/\ell$
            or $\mu_\ell$.
        \item
            If $\m$ is non-Eisenstein, then by~\cite[Proposition
            14.2]{mazur:eisenstein}, $J[\m]$ is irreducible as a $G_\QQ$-module
            so $V\cong_{G_\QQ} J[\m]$, which is unramified away from $\ell N$
            by~\cite[Theorem 6.7]{deligne-serre}.
    \end{enumerate}
\end{proof} 

\begin{theorem}\label{thm:G_modules_are_Hecke}
    Suppose $M$ is an odd $G_\QQ$-module. Then $M$ is $\TT$-stable so $M$ is
    an $\TT[G_\QQ]$-module.
\end{theorem}
\begin{proof}
    It suffices to prove this for each $\ell$-primary part. Let $\ell>2$ and
    assume $M\subseteq J[\ell^\infty]$. Let
    \[
        0 = M_0 \subsetneq \ldots \subsetneq M_n = M
    \]
    be an $G_\QQ$-composition series of $M$ with composition factors $X_i =
    M_i/M_{i-1}$. We proceed by induction on $n$ with the base
    case being the trivial $n=0$ case. 
    
    Assume $M_{s-1}$ is an $\TT[G_\QQ]$-module. We will show $M_s$ is an
    $\TT[G_\QQ]$-module. Since $M_{s-1}$ is an $\TT[G_\QQ]$-module, for each
    $t\in \TT$, we have a well-defined map $t:X_s\to J(\QQbar)/M_{s-1}$. The
    goal is to show $t(X_s)\subseteq X_s$ for all $t\in \TT$.
    By~\cite[Proposition 2]{ribet:mult_p_finite}, $\TT/\ell \TT$ is generated
    by $T_p$ for $p\nmid \ell N$ so it suffices to show $T_p(X_s)\subseteq X_s$
    for prime $p\nmid \ell N$.

    Fix a prime $p\nmid \ell N$. We have that $X_s$ is unramified at $p$ and
    that $J$ has good reduction at $p$. There exists a finite extension $K$ of
    $\QQ$ unramified at $p$ such that the $G_\QQ$-action on $X_s$ factors
    through $\Gal(K/\QQ)$. As in the proof of~\cite[Lemma
    12.6.2]{ribet-stein:mod}, the reduction map induces an isomorphism
    \[
        \tau:J(\QQbar)[\ell^\infty] \riso J_{/\F_p} (\Fpbar)[\ell^\infty].
    \]
    Under this isomorphism the natural $\TT$-action on $J(\QQ)$ maps to the
    natural $\TT$-action on $J_{/\F_p}$. Moreover, the $\Frob_\p$ action on
    $X_s$ is given by the absolute Frobenius map, $F$ on $J_{/\F_p}$. By
    Eicher-Shimura, $T_p = F+p/F\in \End(J_{/\F_p})$ so
    \[
    \tau(T_p X_s) 
    = T_p\tau(X_s) 
    = (F+p/F)\tau(X_s)
    = \tau((\Frob_\p+p/\Frob_\p)X_s)
    \subseteq \tau(X_s)
    \]
    hence, $T_p X_s\subseteq X_s$, as desired.
\end{proof}


\subsection{Non-Eisenstein modules are kernels of Hecke}%
\label{sub:non_eisenstein_modules_are_kernels_of_hecke}

In light of Section~\ref{sub:finite_odd_order_galois_modules_are_hecke}, all
odd $G_\QQ$-submodules of $J(\QQbar)$ are $\TT$-modules. We will now
use this fact freely and will often refer to the $\TT$-support of an odd
$G_\QQ$-submodule.

The goal now is to identity the odd non-Eisenstein $G_\QQ$-submodules of
$J(\QQbar)_\odd$ by their $\TT$-annihilators. Using
Theorem~\ref{thm:G_modules_are_Hecke}, we can already do this in the
irreducible case.
\begin{corollary}
    Let $\m$ be a non-Eisenstein prime of odd residue characteristic, if $M$ is
    a nonzero finite irreducible $G_\QQ$-submodule of $J(\QQbar)$ supported
    only on $\m$, then $M=J[\m]$.
\end{corollary}
\begin{proof}
    By Lemma~\ref{lemma:cherry_street}, the annihilator of $M$ is $\m$. Hence,
    $M\subseteq J[\m]$ but $J[\m]$ is an irreducible
    $G_\QQ$-module~\cite[Proposition 14.2]{mazur:eisenstein} so $M=J[\m]$.
\end{proof}

The general case will follow from the work of David Helm~\cite{helm:jacobian}.
Helm considers the case of Jacobians of Shimura curves but the relevant results
still apply mutatis mutandis.

\begin{theorem}[{\cite[Corollary 4.8]{helm:jacobian}}]%
    \label{thm:non_eisenstein_kernel_hecke}
    Let $M$ be a finite $G_\QQ$-module supported only on the non-Eisenstein
    primes of $\TT$ of odd residue characteristic with $\TT$-annihilator $I$.
    Then $M=J[I]$.

    Moreover, since $A$ is both $G_\QQ$ and $\TT$ invariant, if $M$ is a finite
    $G_\QQ$-module supported only on the non-Eisenstein primes of odd residue
    characteristic of $\TT_A$ with $\TT_A$-annihilator $I$. Then $M=A[I]$.
\end{theorem}

We have that $V(I)=\Supp_\TT(M)$ and $M\subseteq J[I]$ so suffices to prove
$J[I]_\m = M_\m$ for each non-Eisenstein prime of odd residue characteristic.
So let $\m$ be a non-Eisenstein prime of residue characteristic $\ell>2$. Let
$T_\m J\isom \Hom(J[\m^\infty], \QQ_\ell/\ZZ_\ell)$ be the contravariant Tate
module at $\m$ and $\overline{\rho}_\m$ be the Galois representation associated
to $J[\m]^\vee$. Since $\m$ is an odd non-Eisenstein prime,
$\overline{\rho}_\m$ is an irreducible $G_\QQ$-representation of dimension 2
over $k_\m$ that is isomorphic to $J[\m]^\vee$~\cite[Prop.
14.2]{mazur:eisenstein}.

\begin{lemma}[{\cite[Lemma 4.6]{helm:jacobian}}]\label{lemma:finite_index}
    Let $M$ be a $G_\QQ$-stable submodule of $T_\m J$ of finite index. Then
    $M=IT_\m J$ for some ideal $I$ of $\TT$.
\end{lemma}
\begin{proof}
    We proceed by induction on the maximal $G_\QQ$-composition series of $T_\m J/M$
    with the base case being the trivial length zero case. Let 
    \[
        M = M_n \subsetneq M_{n-1} \subsetneq \cdots \subsetneq M_0 = T_\m J
    \]
    be a $G_\QQ$-composition series. By induction, $M_{n-1} = I'T_\m J$ for some
    $I'\subseteq \TT$.

    Consider $\m M_{n-1} + M$. This is a $G_\QQ$-module sitting between $M$ and
    $M_{n-1}$. By Nakayama's lemma, if $\m M_{n-1} M + M = M_{n-1}$, then
    $M=M_{n-1}$ which is a contradiction. Hence, $\m M_{n-1}+M=M$ so $M$
    contains $\m M_{n-1}$ and we can form the quotient.

    The module $M_{n-1}/\m M_{n-1}$ is $G_\QQ$-isomorphic to $(I'/\m
    I')\otimes_{\TT/\m} (T_\m J/\m T_\m J) \cong (I'/\m I')\otimes_{\TT/\m}
    J[\m]^\vee$, where $G_\QQ$ acts trivially on $I'/\m I'$. Let $V$ be the image
    of $M$ in $M_{n-1}/\m M_{n-1}$. Since $V$ is $G_\QQ$-invariant, and
    $J[\m]^\vee$ is irreducible, $V$ is given by $\hat{V}\otimes J[\m]^\vee$
    for some $\TT/\m$-subspace $\hat{V}$ of $I'/\m I'$. Let $I$ be the preimage
    of $\hat{V}$ in $I'$. Then $IT_m J = M$, since both contain $\m M_{n-1}$
    and map to $V$ modulo $\m M_{n-1}$.
\end{proof}

We now return to the proof of the theorem.

\begin{proof}[proof of Theorem~\ref{thm:non_eisenstein_kernel_hecke}]
    Let $B=J/M$ be the quotient abelian variety equipped with the induced
    $\TT$-action. So the projection $\phi:J \to B$ is an $\TT[G_\QQ]$-isogeny
    with $\ker\phi = M$. For any odd non-Eisenstein prime $\m$, $\phi$ induces
    the exact sequence
    \[ 
        0 \to T_\m B \to T_\m J \to M^\vee _\m \to 0.
    \] 
    In particular, the image of $T_\m B$ under $\phi$ is a finite index
    $\TT[G_\QQ]$-submodule of $T_\m A$. By Lemma~\ref{lemma:finite_index}, we can find
    an ideal $I'$ of $\TT$ such that the image of $T_\m B$ is $I' _\m T_\m A$
    for all odd non-Eisenstein primes $\m$. We have $M^\vee _\m = T_\m J / I'
    T_\m J\cong J[I']_\m ^\vee$. Therefore, by taking annihilators of the dual,
    we have that $I_\m = I'_\m$ and then by taking duals $M_\m = J[I]_\m$, as
    desired.
\end{proof}

\subsection{Enumerating representatives of $A[X]$}

Assume $\TT_A$ is integrally closed and let $X$ be an odd ideal of $\TT_A$. The
goal of this section is to give a set of representatives for $\M(A[X])$ by
first enumerating the $G_\QQ$-submodules of $A[X]$. We
have the decomposition $X=\bigoplus_{\p\in \Supp_{\TT_A} (X)} X[\p^{v_\p(X)}]$
so it suffices to enumerate the $G_\QQ$-submodules of $A[\p^s]$ for any $s\geq
1$. This will now follow from Mazur's study of the Galois action on torsion
points~\cite[\S 14]{mazur:eisenstein}.

\begin{proposition}[Mazur]\label{prop:all_G_subs}
    Let $\p$ be a prime of residue characteristic $\ell>2$.
    \begin{enumerate}
        \item 
            If $\p$ is Eisenstein, then $A[\p]=C_A[\ell]\oplus \Sigma_A[\ell]$,
            where $C_A=C\cap A$ and $\Sigma_A=\Sigma \cap A$ with $C$ and
            $\Sigma$ the cuspidal and Shimura subgroups of $J$. The
            $G_\QQ$-submodules of $A[\p]$ are the direct sum of $\ZZ$-submodules of
            $C_A[\ell]$ and $\ZZ$-submodules of $\Sigma_A[\ell]$.
        \item
            if $\p$ is non-Eisenstein, $A[\p]$ is irreducible~\cite[Prop
            14.2]{mazur:eisenstein} as as a $G_\QQ$-module so the only
            $G_\QQ$-submodules are $0$ and $A[\p]$.
    \end{enumerate}
\end{proposition}
\begin{proof}
    The Eisenstein case is~\cite[Corollary 16.3]{mazur:eisenstein} along with
    the fact that $C[\ell]\isom \ZZ/\ell$ and $\Sigma[\ell]\isom \mu\ell$.

    The non-Eisenstein case is~\cite[Propositon 14.2]{mazur:eisenstein}.
\end{proof}

Let $a\in \p^{s-1}\setminus \p^s$ so $a$ generates $\p^{s-1}/\p^s$ as a
$k_\p$-space. There exists a $\TT_A[G_\QQ]$-injection given by
$\phi_s:A[\p^s]/A[\p^{s-1}]\to A[\p]$. The $G_\QQ$-submodules, $M$, of
$A[\p^s]$ are then the $\ZZ$-submodules of $A[\p^s]$ such that $M\cap
A[\p^{s-1}]$ and $\phi_s(M)$ are $G_\QQ$-submodules. This yields the following
algorithm.

\begin{algorithm}{Enumerating $G_\QQ$-submodules of {$A[X]$}}
    Given an odd ideal $X$ of $\TT_A$. This algorithm will output a set of
    representatives for $\M(A[X])$.
    \begin{enumerate}
        \item{} [Factor $X$]
            Compute the factorization $X=\prod \p_i ^{e_i}$.
        \item{} [Non-Eisenstein part]
            For each non-Eisenstein $\p_i$, let $W_i=\{A[\p^j]: 0 \leq j
            \leq e_i\}$.
        \item{} [Eisenstein part]
            For each Eisenstein $\p_i$, set $V_1$ to be the set of
            $G_\QQ$-submodules of $A[\p]$. For $j=2,\ldots,e_i$, set $V_j$ to
            be the set of $\ZZ$-submodules, $M$, of $A[\p^j]$ such that $M\cap
            A[\p^{j-1}]\in V_{j-1}$ and $\phi(M)\in V_1$.
        \item{} [Combine]
            Combine the $W_i$'s to form $W=\{\sum_i M_i: M_i \in W_i\}$.
        \item{} [Output representatives]
            Use Algorithm~\ref{isom_testing} to produce a set of
            representatives of $\M(A[X])$ of elements of $W$.
    \end{enumerate}
\end{algorithm}

\subsection{Bounding support and valuations}%
\label{sub:bounding_support_and_valuations}

The goal of this section is to give an ideal $Q$ such that
$\M(A[Q^\infty])=\M(A(\QQbar)_\odd)$ (Proposition~\ref{prop:bound_support}). In
light of Faltings' Isogeny Theorem, there exists $k$ such that
$\M(A[Q^k])=\M(A(\QQbar)_\odd)$. We give a method for finding this $k$ as
Proposition~\ref{prop:stop_looking}. We now have all the necessary ingredients
for Algorithm~\ref{alg:odd_isogeny_class}.

\begin{lemma}\label{lem:com_alg}
    Let $a,b\in \TT_A$, $\p$ such the localization $(\TT_A)_\p$ is a DVR, and
    $M$ an finite odd $G_\QQ$-submodule of $A(\QQbar)$ with $Z=\Ann_{\TT_A} M$.
    Then
    \begin{enumerate}
        \item 
            If $v_\p(Z) = k$, then $M[\p^\infty]\subseteq A[\p^k]$.
        \item
            We have $b^{-1}(A[Z])=A[bZ]$. This holds locally at $\p$ as well so
            $b_\p ^{-1} A[Z]_\p = A[bZ]_\p$.
        \item
            Let $r=v_\p(b)-v_\p(a)+k$ and $\phi=a\circ b^{-1}$. If $r\leq
            0$, then $\phi(M)[\p^\infty]=0$ so $\phi(M)_\p=0$.
            
            If $r>0$, then
            $\phi(M)[\p^\infty]\subseteq A[\p^r]$. Here $b^{-1}$ is the
            preimage of $b$.
    \end{enumerate}
    We will take on the convention that for negative $r$, $A[\p^r]=0$. So we can
    write the last part as $\phi(M)[\p]\subseteq A[\p^r]$.
\end{lemma}
\begin{proof}
    \mbox{}
    \begin{enumerate}
        \item 
            Suppose $x\in M[\p^\infty]$ so $x\in A[\p^\infty]$. Moreover,
            $v_\p(\Ann_{\TT_A} x) \leq v_\p(Z) = k$ so $x\in A[\p^k]$.
        \item
            We have $x\in b^{-1}(A[Z])$ if and only if $bx \in A[Z]$ if and
            only if $x\in A[bZ]$.
        \item
            Suppose $r\leq 0$, then $\phi_\p \in Z_\p$ so
            \[
                \phi(M)[p^\infty]\subseteq \phi(M[\p^\infty])
                \subseteq \phi(A[\p^k]) = 0.
            \]
            Suppose $r>0$ and then $x\in \phi(M)[\p^\infty] =
            \phi(M[\p^\infty])\subseteq (a\circ
            b^{-1})(A[\p^k])=A[\p^r]$.
    \end{enumerate}
\end{proof}


\begin{lemma}\label{lem:principal_gives_iso}
    Let $C, D$ be nonzero ideals of $\TT_A$, $a,b\in \TT_A$ be nonzero, and $aC
    = bD$. Then there exists an isomorphism $\phi:A/A[D]\to A/A[C]$ such that
    $\phi\circ b=a$.
\end{lemma}
\begin{proof}
    Our argument will now reference the following diagram.
    \[
        \begin{tikzcd}
            A/A[bD] 
            \arrow[r, "\sim", "b"']
            \ar[equal]{d}
            &
            A/A[D]
            \ar[d, "\sim" labl, "\phi"]
            \\
            A/A[a C]
            \arrow[r, "\sim", "a"']
            &
            A/A[C]
        \end{tikzcd}
    \]
    We first establish the $b$ and $a$ isomorphism. We have the exact
    sequence
    \[
        \begin{tikzcd}
            0
            \arrow[r]
            &
            (A/A[D])[b]
            \arrow[hookrightarrow]{r}
            &
            A/A[D]
            \arrow[twoheadrightarrow]{r}{b}
            &
            A/A[D]
            \arrow[r]
            &
            0
        \end{tikzcd}.
    \]
    We have that $x\in (A/A[D])[b]$ if and only if $bx \in A[D]$ if and only if
    $x \in A[bD]$. Hence, $(A/A[D])[b]=A[bD]/A[D]$ so $b:A/A[bD]\xra{\sim}
    A[D]$. A similar argument applies for $a$. 

    We can now chase the diagram to obtain an isomorphism $\phi:A/A[D]\riso
    A/A[C]$, as desired.
\end{proof}

Let $H=\{C_i\}$ be a set of integral representatives for $\Cl(\TT_A)$ and
${S=S_1\cup\bigcup_{C\in H}V(C)}$.
\begin{proposition}%
    \label{prop:bound_support}
    Every finite odd $G_\QQ$-submodule is equivalent to one supported on $S$.
\end{proposition}
\begin{proof}
    Let $M$ be a finite odd $G_\QQ$-submodule. Let $M=M_{S'}\oplus M_S$ be the
    direct sum decomposition of $M$ with $\Supp M_X \subseteq X$. By
    Theorem~\ref{thm:non_eisenstein_kernel_hecke}, there exists an ideal
    $I$ of $\TT_A$ with $V(I)\subseteq S'$. Since $S'$ consists of only
    invertible primes, $I$ is also invertible. So there exists nonzero $a,b\in
    \TT_A$ such that $bI=aC$. By Lemma~\ref{}, there exists an isomorphism
    $\phi:A/A[I]\to A/A[C]$ such that $\phi\circ b = a$. Let $\psi:A/A[I]\to
    A/(A[I]+M_S)$ be the isogeny given by quotienting by $M_S/A[I]$. Then
    $\phi$ induces an isomorphism $\phi':A/(A[I]+M_S)\to A/(A[C]+\phi(M_S))$.

    It is clear that $\Supp_{\TT_A}(A[C])\subseteq S$ so it remains to show
    that $\phi(M_S)_\p =0$ for all $\p \in S'$. Let $\p\in S'$, then
    $v_\p(b)-v_\p(a)+v_\p(\Ann(M_S))=v_\p(C)-v_\p(I)+v_\p(\Ann(M_S)) \leq 0$
    since $V(C),V(\Ann(M_S))\subseteq S$. By Lemma~\ref{lem:com_alg},
    $\phi(M_S)_\p = 0$. Therefore, $A[C]+\phi(M_S)$ is supported on $S$.
\end{proof}

From henceforth, we assume that $\TT_A$ is integrally closed. We can now choose
$H=\{C_i\}$ to be a set of integral representatives for $\Cl(\TT_A)$ of odd
norm. Let $Q=\lcm(\{C:C\in H\}\cup \{\p:\p\in S_1\})$. We have that $V(Q)=S$
so, in light of Proposition~\ref{prop:bound_support},
$\M(A[Q^\infty])=\M(A(\QQbar)_\odd)$. Isogeny class are finite so there exists
some $k$ so that $\M(A[Q^k])=\M(A(\QQbar)_\odd)$. We could have chosen any
ideal with support $S$, but the ideal $Q$ here has the right valuation to
concisely state the next proposition.

\begin{proposition}%
    \label{prop:stop_looking}
    Suppose $\TT_A$ is integrally closed. Suppose $\M(A[Q^k])=\M(A[Q^{k+1}])$
    then $\M(A[Q^k])=\M(A[Q^\infty])$.

    In terms of isogenies, this claim states that if $G_\QQ$-submodules of
    $A[Q^{k+1}]$ produce no new odd-isogenies not found in $A[Q^k]$, then
    $G_\QQ$-submodules of $A[Q^k]$ give the entire odd-isogeny class.
\end{proposition}
\begin{proof}
    Every isogeny can be factored into a composition of irreducible isogenies.
    Let $A'=A/M$ be a abelian variety with $M$ a $G_\QQ$-submodule of $A[Q^k]$. We
    will show that the image of $A'$ under an irreducible isogeny is equivalent
    to $A/Y$ for some $Y\in A[Q^{k+1}]$. This suffices because when
    $A[Q^k]=A[Q^{k+1}]$, we obtain the full isogeny class.

    Let $M'\subseteq A(\QQbar)_\odd$ be an extension of $M$ by an irreducible
    $G_\QQ$-module $Y$. Let $\m$ be the annihilator of $Y$. If $\m\in S$, then
    $M\subseteq A[\m Q^k]\subseteq A[Q^{k+1}]$ and we are done. So assume
    $\m\notin S$ so $\m$ is an odd Eisenstein prime and $Y=A[\m]$
    (Theorem~\ref{thm:irreducible_G_sub}). Since $\m\notin \Supp_{\TT_A}(M)$,
    $M' = A[\m]\oplus M$. As in Proposition~\ref{prop:bound_support}, there is
    an isomorphism $\phi':A/(A[I]+M_S)\to A/(A[C]+\phi(M_S))$ with $\phi\circ b
    = a$. The goal is to show $A[C]+\phi(M)\subseteq A[Q^k]$. Since
    $A[C]\subseteq A[Q^{k+1}]$ so it suffices to show $\phi(M)\subseteq
    A[Q^{k+1}]$.
    
    As in Proposition~\ref{prop:bound_support}, $\Supp\phi(M)\subseteq S$. For all
    $\p\in S$ (remember that $\m \notin S$), $v_\p(b)-v_\p(a) =
    v_\p(C)-v_\p(\m)=v_\p(C)$ so by Lemma~\ref{lem:com_alg},
    $\phi(M)\subseteq A[Q^{k+1}]$, as desired.
\end{proof}

\subsection{Enumerating isogenies}%
\label{sub:enumerating_isogenies}

In this section, we give the final ingredient for enumerating the isogeny class
of $A$. We continue assuming $\TT_A$ is integrally closed. Using
Proposition~\ref{prop:bound_support}, we can compute an ideal $Q$ of $\TT_A$
such that $\M(A[Q^\infty])=\M(A(\QQbar)_\odd)$.
It remains to show how to enumerate
the $G_\QQ$-submodules of $A[Q^r]$. We have the direct sum decomposition
\[
    A[Q^r]=\bigoplus_{\p\in V(Q)} A[\p^{rv_\p(Q)}]
\]
so it suffices to describe how to enumerate the $G_\QQ$-submodules of $A[\p^s]$ for
any $s$. We will do this by induction. Mazur handles the $s=1$ case.




For the inductive step, we use a trick of Mazur~\cite[pg.
112]{mazur:eisenstein}. Let $s\geq 2$, and
$a_1,\ldots,a_t\in \p^s$ be a set of generators whose cosets generate
$\p^{s-1}/\p^s$ as a $k_\p$-vector space. Then there is an $\TT[G_\QQ]$-inclusion
\begin{align*}
    \phi_s: A[\p^s]/A[\p^{s-1}] & \to A[\p]^t, \\
    x+A[\p^{s-1}] & \mapsto (a_1x,\ldots,a_tx).
\end{align*}
The $G_\QQ$-submodules of $A[\p^s]$ are the $\ZZ$-submodules $X$ of $A[\p^s]$ with
the property $\phi(X)$ is a $G_\QQ$-submodule of $A[\p]^t$ and $X\cap A[\p^{s-1}]$
is a $G_\QQ$-submodule of $A[\p^{s-1}]$. So by induction, we can determine the
$G_\QQ$-submodules of $A[\p^s]$ for any $s$. 

\subsection{Table}%
\label{sub:table}

BEHOLD MY AWESOME TABLE.

\bibliography{biblio}
\end{document}
% \begin{example}[A Non-Eisenstein isogeny]
%     Let $A,B,C$ be the simple abelian subvarieties of $J_0(499)$ in decreasing
%     dimension. Let $K$ be the Hecke eigenvalue field of the newform associated
%     to $A$. Using Sage, we determine that $[\O_K:\TT_A]=3$. Moreover, by a
%     theorem of Mazur~\cite{}, $\TT=\End(J_0(499)$. There is a 2-isogeny from
%     $A\times (B+C)$ to $J_0(499)$, therefore, $\TT_A=\End(A)$. So by a theorem
%     of Shimura, there is a rational 3-isogeny from $A\to A'$ with
%     $\End(A')=\O_K$. This isogeny is not Eisenstein as $3$ does not divide $\#
%     \TT/\I = \mathrm{num} \left(\frac{499-1}{12}\right)$.
% \end{example}
