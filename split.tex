\chapter{Totally Split Jacobians}%
\label{chap:totally_split}

A Jacobian is said to be \emph{totally split} if it is $\QQ$-isogenous to a
product of elliptic curves. 

The classification and construction of totally
split Jacobians of genus 2 curves has been
well-studied~\cite{bruin-doerksen:split_genus_two,kuhn:split_genus_two} and
large rational torsion subgroup of totally split Jacobians of genus 2 and 3
curves have been found~\cite{howe-leprevost-poonen:large}. The original
motivation of this project was to compute the rational torsion subgroup for as
many $J_0(N)$ as possible. The first $N$ for which Sage fails is $J_0(30)$
which happens to be a product of 3 elliptic curves. The author and his adviser
were able to compute the rational torsion subgroup using the fact that
$J_0(30)$ is totally split, the fact that rational torsion subgroups of
elliptic curves can be computed, and Galois cohomology. The goal of this
chapter is to see how far we can push these techniques. In particular, we will
show that there are finitely many totally split $J_0(N)$, give a general (but
totally impractical) method for computing the rational torsion subgroup, and
present some more practical techniques for computing the rational torsion
subgroup.


\section{Provably enumerating the set of totally split $J_0(N)$}
\label{sec:provably_enumerating}

The modular Jacobian $J_0(N)$ is totally split if and only if all newforms of
level dividing $N$ have rational Hecke coefficients. We expect this to be rare.
In fact, there are only finitely many totally split $J_0(N)$. Ralph Greenberg
quickly gave an argument on the way to UW Number Theory Seminar lunch proving
the set of totally split $J_0(N)$ is finite but his argument did not give an
effective method of enumeration. Moreover, at Sage Days 87, Alyson Dienes
suggested proving this using asymptotic bounds.
\begin{proposition}%
    \label{prop:totally_split}
    The set of totally split $J_0(N)$ is finite.
\end{proposition}
\begin{proof}
    If $J_0(N)$ is a product of elliptic curves then the dimension is
    exactly equal to the number of elliptic factors.
    \begin{itemize}
        \item
            By~\cite[Thm. 6]{martin:dimension}, the dimension is bounded below
            by $\frac{1}{12}N + O(\sqrt{N}\log\log N)$.
        \item
            By~\cite[Cor. 2]{brumer-silverman:number}, the number of elliptic
            curve factors of $J_0(N)$ is bounded above by $O(N^{1/2+\epsilon})$
            for any $\epsilon>0$.
    \end{itemize}
    Asymptotically, the lower bound will surpass the upper bound so there are
    finitely many totally split $J_0(N)$.
\end{proof}
Neither of their arguments gave an effective method of enumeration. The goal of
this section is to provably enumerate the set of $J_0(N)$ that are totally
split. There are 71 of totally split $J_0(N)$ that are nontrivial (see
Table~\ref{tab:split}).

We have that $J_0(N)$ is totally split if and only if its dimension is equal to
the number of (modular) elliptic curves of conductor $N$. If $N$ is less than
the upper limit of conductors in the Cremona Database, then the hard work has
been done and determining whether $J_0(N)$ is totally split is a quick
computation since there is a closed-form formula for $\dim J_0(N)$.

Call a positive integer $N$ \emph{good}, if $J_0(N)$ is totally split and
\emph{bad} otherwise. The simple factors of $J_0(N)$ are isogenous to simple
factors of $J_0(MN)$ for any positive integer $M$. Therefore, if $N$ is bad,
then so is any multiple of $N$. 

In Lemma~\ref{lem:good_primes}, we will enumerate all good primes. We can now
do a search on the divisibility tree of the positive integers supported on the
good primes. Moreover, we can prune a branch whenever we encounter a bad
integer during our search. Theorem~\ref{thm:finite_totally_split} asserts that
all branches are eventually pruned so this search yields all good integers.

\begin{lemma}
    \label{lem:good_primes}
    The only possible primes $p$ where $J_0(p)$ is totally split are
    \[
        2, 3, 5, 7, 11, 13, 17, 19, 37.
    \]
\end{lemma}
\begin{proof}
    Let $J=J_0(p)$ be a totally split Jacobian of prime level. If $\dim J=0$,
    then $J$ is clearly totally split so assume $\dim J>0$. Then $J\cong
    \prod_f E_f$ with $E_f$'s elliptic curves of conductor $p$. Let $n$ denote
    the order of the rational cuspidal subgroup of $J$ which is, as mentioned
    above, known to be the numerator of $(p-1)/12$. By Emerton's proof of
    Stein's refined Eisenstein conjecture~\cite[Theorem B]{emerton:optimal}, if
    $l$ divides $n$, then $l$ divides the order of $E_f(\QQ)$ for some elliptic
    factor of $J$. But elliptic curves of prime conductor do not have much
    rational torsion.

    In particular, Miyawaki~\cite{miyawaki:ell_prime} enumerates all curves of prime
    power conductor with odd-order rational torsion. The largest prime
    conductor here being 37. This implies that if $p>37$, then $\#C(\QQ)$ must
    be a power of 2 so $p=2^a 3^b + 1$ for some $a\geq 0$ and $b\in \{0,1\}$.
    We now show $a\leq 2$.

    If $a>2$, then $C(\QQ)$ has an order 2 element which implies some $E_i$ has
    a rational 2-torsion point. As a result of Setzer~\cite[Theorem
    2]{setzer:ell_prime}, $p=17$ or $p=u^2+64$ for some integer $u$. We split into 2
    cases to show that $p$ is never $u^2+64$.

    Suppose $b=0$. Then $p$ is a Fermat prime and thus a Fermat number. Outside
    of $3$ and $5$, the recursive formula for Fermat numbers and an induction
    argument shows that the last digit of Fermat numbers is always $7$. But the
    only possible last digits of $u^2+64$ are $0, 3, 4, 5, 8, 9$.

    Suppose $b=1$. Then $2^a\cdot 3 = u^2+63$. This implies $3$ divides $u$ so
    $9$ divides $u^2$. But now the right-hand side is divisible by $9$ while the
    left is not.

    In conclusion, we know that if $J_0(p)$ is a totally split Jacobian, then $p\leq
    37$. A computer search then determines which primes less than or equal to
    $37$ are totally split.
\end{proof}

The following procedure is a breath-first search.
\begin{algorithm}{Enumerating Good Integers}
    \label{alg:find_split}
    Let $S$ be the set of primes found in Lemma~\ref{lem:good_primes}. This
    procedure will halt (see Theorem~\ref{thm:finite_totally_split}) and return
    the list of good integers.
\end{algorithm}
\begin{enumerate}
    \item{} [Initialize]
        \label{step:initialize}
        Set $i=1$ and $M_1=S$. Here $M_i$ will represent the set of all good
        integers with $i$ prime factors, counting multiplicity.
    \item{} [Find prime multiples of $M_i$ are that good]
        Set
        \[
            M_{i+1}=\{pN: p\in S, N\in M_i, pN \text{ good}\}.
        \]
        If $M_{i+1}$ is non-empty, increment $i$ and repeat this step.
    \item{} [Return]
        Return $\bigcup_i M_i$.
\end{enumerate}
\begin{theorem}
    \label{thm:finite_totally_split}
    There are 71 integers $N$ for which $J_0(N)$ is a totally split Jacobian of
    positive dimension. They are given in Table~\ref{tab:split}.
\end{theorem}
\begin{proof}
    We run Algorithm~\ref{alg:find_split} and it terminates after find 71
    integers $N$ for which $J_0(N)$ is totally split (luckily before reaching
    the end of the Cremona database). This proves the finiteness of totally
    split $J_0(N)$ without using Proposition~\ref{prop:totally_split} but with
    the added benefit of provably enumerating the totally split $J_0(N)$.
\end{proof}

\section{Enumerating rational torsion is algorithmic}
\label{sec:enumerating_is_algorithmic}

In this section, we give a completely impractical algorithm to compute the
rational torsion subgroup of a totally split Jacobian $J_0(N)$ just to show it
is algorithmic.
\begin{proposition}
    Suppose $A$ is a totally split abelian subvariety of $J=J_0(N)$ Then we can
    compute the following data:
    \begin{enumerate}
        \item
            A number field containing $\QQ(A[n])$ for positive integer $n$.
        \item
            Let $n$ be a positive integer and $L$ be a number field containing
            $\QQ(A[n])$. The action of $\Gal(L/\QQ)$ on $A[n]$.
        \item
            The $K$-rational torsion points of $A(K)_\tor$.
    \end{enumerate}
    In particular, we can compute the rational torsion subgroup of any totally
    split Jacobian.
\end{proposition}
\begin{proof}
    Recall that the abelian subvarieties are represented by giving a submodule
    of the integral homology~\ref{sec:defining_data}. Suppose $A$ is
    1-dimensional subvariety of $J$. Then by~\ref{alg:weierstrass}, we can
    compute an elliptic curve $E_A$ given in Weierstrass defining equation and
    an isomorphism $\Phi_A:A_\tor\to (E_A)_\tor$.

    We will proceed by induction on the dimension, $d$, of $A$. We first
    consider the case $d=1$.
    \begin{enumerate}
        \item
            Let $n$ be a positive integer. Then using division polynomials, we
            can compute a number field $L$ containing $\QQ(A[n])$.
        \item
            Let $n$ be a positive integer and $L$ be a number field containing
            $\QQ(A[n])$. The Galois action on the points of $E_A[n]$ is given
            by applying the Galois action to each coordinate. Using $\Phi_A$
            and the action of $\Gal(L/\QQ)$ on $E_A$, we can explicit determine
            the action of $\Gal(L/\QQ)$ on $A[n]$.
        \item
            Let $K$ be a number field. Using reduction mod $p$, there exists an
            integer $m$, such that that $A(K)_\tor \subseteq A[m]$. Using (1),
            we can define a number field $L$ that contains $\QQ(A[m])$. Then
            using (2), we can compute
            \[
                A(K)_\tor = A[m]^{\Gal(L/K)}.
            \]
    \end{enumerate}

    Now assume we can compute (1)-(3) for any totally split abelian subvariety
    of dimension less than $k$.

    Let $A$ be of dimension $k+1$ and write $A=B+C$, where $B$ is of
    dimension $k$ and $C$ is of dimension $1$.  We have the exact sequence
    \[
        0\to B\cap C \to B\times C \to A,
    \]
    where we identify $B\cap C$ as a subgroup of $B\times C$ via the
    anti-diagonal embedding. So $A=(B\times C)/(B\cap C)$. Let $r$ be the
    exponent of $B\cap C$.
    \begin{enumerate}
        \item
            For any integer $n$, $A[n] \subseteq B[nr]+C[nr]$. So a number
            field containing $\QQ(A[n])$ is the compositum  of the number
            fields containing $\QQ(B[nr])$ and $\QQ(C[nr])$ which can be
            computed by the inductive hypothesis.
        \item
            The Galois action can be determined on $A[n]$ by viewing $A[n]$ as
            a subgroup of $(B[nr]\times C[nr])/(B\cap C)$.
        \item
            Using reduction mod $p$, there exists an integer $m$ such that
            $A(K)_\tor\subseteq A[m]$. Using (1), we can find a number field
            $L$ that contains $\QQ(A[m])$. Then we use (2), to compute
            \[
                A(K)_\tor = A[m]^{\Gal(L/K)}.
            \]
    \end{enumerate}
\end{proof}

\section{Strategies for computing the rational torsion subgroup}
\label{sec:strategies_for_rational_torsion}

The Generalized Ogg Conjecture is equivalent to the assertion that
$[J_0(N)(\QQ)_\tor:C_N(\QQ)]=1$. In this section, we give strategies for
bounding this index when $J_0(N)$ is a rank-0 totally split Jacobian. We are
able to compute generators for $C_N(\QQ)$ in our presentation~\ref{} and the
group order. Therefore, when $[J_0(N)(\QQ):C_N(\QQ)]=1$, we are able to compute
generators for $J_0(N)(\QQ)_\tor$ in our presentation~\ref{} and the group
order.

Using these strategies, we are able to able to verify the Generalized Ogg
Conjecture for all but 9 rank-0 $J_0(N)$. In these 9 cases, we are
able to bound the index $[J_0(N)(\QQ)_\tor:C_N(\QQ)]$ by a power of 2
(Table~\ref{}).

\subsection{Upper bound}%
\label{sub:upper_bound}

In this subsection, we give two techniques for giving an upper bound for
$J_0(N)(\QQ)_\tor$. The first technique uses reduction modulo $p$,
Eicher-Shimura, and the Hecke polynomial to obtain a bound on the order. The
second technique gives an ideal, $I^*$, similar to Mazur's Eisenstein ideal, so
that $J_0(N)(\QQ)_\tor\subseteq J_0(N)[I]$. We will be primarily using the
second technique for our computations but we give the first technique to
motivate the second and we will be using the first technique

\subsubsection{Reduction modulo primes}%
\label{ssub:reduction_modulo_primes}
Let $A$ be a modular abelian variety and $K$ be a number field. A common
technique for bounding $A(K)_\tor$ is to reduce modulo completely split primes
$p$ of $K$ of good reduction for $A$. In particular,
by~\cite[Appendix]{katz:torsion}, if $\mathcal{A}$ is the Neron model for $A$
and $p$ a prime completely split of $K$ of good reduction for $A$, then
\[
    A(K)_\tor \hookrightarrow \mathcal{A}_{/p}(\Fp)
\]
so $\# A(K)_\tor \mid \#\mathcal{A}_{/p}(\Fp)$.

The expression $\#\mathcal{A}_{\p}(\Fp)$ is an isogeny invariant and
multiplicative on direct products. So it suffices to describe how to compute
$\mathcal{A}_{/p}(\Fp)$ when $A\subseteq J_0(N)$ is a simple abelian subvariety
of the new subvariety of $J_0(N)$ (for $A\subseteq J_1(N)$, see~\cite[\S
3.5]{agashe-stein:bsd}). Let $F_p$ be the absolute Frobenius at $p$. By the
Eicher-Shimura relations, $T_p = F_p + p/F_p\in \End(\mathcal{A}_{/p})$. Then
\begin{align}
    \label{eq:katz_bound}
    \begin{split}
    \#\mathcal{A}_{/p}(\Fp)
    & = \deg(1-F_p) \\
    & = |\det(1-F_p)| \\
    & = \mathrm{charpoly}(F_p)(1) \\
    & = \mathrm{charpoly}(T_p)(p+1).
    \end{split}
\end{align}
The last term is the characteristic polynomial of the matrix associated to the
Hecke operator $T_p$ which can be computed~\ref{}.

We now return to the case of $K=\QQ$ and will apply the more general number
field case in Chapter~\ref{chap:isogeny_class}. So in the case where $K=\QQ$
and $A=J_0(N)$, we have that for any finite set $S$ of odd prime not dividing
$N$,
\[
    \# J_0(N)(\QQ)_\tor \mid \gcd_{p\in S} \#\mathcal{J}_{/p}(\Fp).
\]
We now give the current \sage (Version 8.7) implementation of this idea by
William Stein.
\begin{algorithm}{Upper bound on rational torsion order}%
    \label{alg:upper_bound_stable}
    Given a modular Jacobian $J_0(N)$, this algorithm outputs an upper bound
    for $\# J_0(N)(\QQ)_\tor$. This algorithm computes successive GCD's until
    it stabilizes for 3 iterations. Of course, increasing the stability
    threshold could lead to be a better bound.
    \begin{enumerate}
        \item{} [Initialize]
            Let $i=0$, $p_0$ be the smallest prime greater than 2 not dividing
            $N$.
        \item{} [Add another factor]
            Use~\ref{} to compute the matrix associated to $T_{p_0}$. Let $m_i
            = \mathrm{charpoly}(T_p)(p+1)$. If $i=0$, set $B_i=m_i$. Otherwise,
            set $B_i=\gcd(B_{i-1}, m_i)$.
        \item{} [Stable?/Output]
            If $i\leq 2$, increment $i$ and return to Step 2. Otherwise, if
            $B_i\neq B_{i-2}$, return to Step 2. Otherwise, $B_i$ has been
            stable for 3 iterations and we output $B_i$.
    \end{enumerate}
\end{algorithm}
There are two disadvantages to Algorithm~\ref{alg:upper_bound_stable}.

The first is that is isogeny invariant so we expect the bound to not be tight.
In Example~\ref{}, we show that there exists an abelian variety isogenous to
$J_0(30)$ with strictly greater rational torsion order. So
Algorithm~\ref{alg:upper_bound_stable} will never give a tight bound even if we
increase the stability threshold. This Jacobian is the first $J_0(N)$ that
\sage (Version 8.7) cannot compute and is the motivating example for this
totally split Jacobian project of this thesis.

The second is the loss of information of the group structure. For example,
suppose for some $A$ and odd primes $p,q$ of good reduction, we have the
group isomorphisms $\mathcal{A}_{/p}(\Fp)\cong \ZZ/2\times \ZZ/2$ and
$\mathcal{A}_{/q}(\F_q)\cong \ZZ/4$. If we look only at the orders, we can only
deduce $\# A_f(\QQ)_\tor\mid 4$. However, the group structure tells us that as
a group $A_f(\QQ)_\tor$ is trivial or $A_f(\QQ)_\tor \cong \ZZ/2$.

\subsubsection{Real Eisenstein Kernel}%
\label{ssub:real_eisenstein_kernel}

This next technique avoids both of the mentioned disadvantages of the first
technique. This technique was discovered by William Stein and will be
elaborated on as part of a forthcoming paper by Hao Chen, the author, and
William Stein. The idea give to construct an ideal $I^*$ so that
$J_0(N)(\QQ)_\tor\subseteq J_0(N)[I^*]$.

For any odd prime $\ell$ of good reduction (so $\ell\nmid 2N$), let
$\eta_\ell = T_\ell - (\ell+1)$.
\begin{lemma}[William Stein]%
    \label{lem:rational_is_eta}
    For any odd prime $\ell$ of good reduction, $J(\QQ)_\tor\subseteq
    J[\eta_\ell]$.
\end{lemma}
\begin{proof}
    Let $\ell$ be an odd prime of good reduction.
    By~\cite[Appendix]{katz:torsion}, there the reduction modulo $\ell$ map
    yields the inclusion $\tau:J(\QQ)_\tor \hookrightarrow
    \mathcal{J}_{/\ell}(\F_\ell)$. Let $F_\ell$ be the absolute Frobenius of of
    $\mathcal{J}_{/\ell}$. By Eicher-Shimura, $T_\ell=F_\ell+\ell/F_\ell$. Let
    $x\in J(\QQ)_\tor$ so $F_\ell(\tau(x))=\tau(x)$. We have
    \[
        \tau(\eta_\ell(x)) = (T_\ell-(\ell+1))\tau(x) =
        (F_\ell+\ell/F_\ell-(\ell+1))\tau(x) = 0.
    \]
    Since $\tau$ is injective $x\in J[\eta_\ell]$.
\end{proof}

Let $I= \langle \eta_\ell:\ell\nmid 2N \rangle$. By
Lemma~\ref{lem:rational_is_eta}, $J(\QQ)_\tor\subseteq J[I]$. Now let $\star$
be the star-involution (\ref{}) so $J(\CC)(1-\star)=J(\RR)$. and $I^*=I +
\langle 1-\star \rangle$. William Stein calls this the Real Eisenstein Ideal.
We have that $J(\CC)[1-\star]=J(\RR)$ so combined with
Lemma~\ref{lem:rational_is_eta}, $J(\QQ)_\tor\subseteq J[I^*]$. Extending $I$
to $I^*$ is an important step in our strategy because it is often the case that
$J[I]$ is strictly larger than $J[I^*]$. For instance, if $J=J_0(N)$ with $N$
prime, then by~\cite[Cor. 16.3]{mazur:eisenstein}, $J[I]$ contains the Shimura
subgroup $\Sigma_N$ which is a finite subgroup of $\mu$-type and order
$\num((N-1)/12)$.

To approximate $J[I^*]$, let $I_r ^* = \langle \eta_\ell:\ell\leq r \rangle$
and define $E_r = J[I_r ^*]$.
We have
\begin{equation}%
    \label{eq:eta_equation}
    C_N(\QQ)\subseteq J(QQ)_\tor \subseteq E_r \subseteq J[I^*].
\end{equation}

\section{General approach to bound}%
\label{sec:galois_cohomology_bounds}

\subsection{Galois cohomology}%
\label{sub:galois_cohomology}

Let $A, E$ be abelian subvarieties of $J$ with $E$ an elliptic curve and
$E\not\subseteq A$. Then
\begin{equation}%
    \label{eq:ses_ab}
    \begin{tikzcd}
        0 \arrow[r] &
        A\cap E \arrow[r,"d"] &
        A\times E \arrow[r,"s"] &
        A+E \arrow[r] &
        0,
    \end{tikzcd}
\end{equation}
where $d(x)=(x,-x)$ and $s(x,y)=(x+y)$. By applying Galois cohomology, we
obtain the long exact sequence:
\begin{equation}%
    \label{eq:les_ab}
    \begin{tikzcd}
        0 \rar 
        &
        (A\cap E)(\QQ) \rar 
        &
        (A\times E)(\QQ) \rar
        \arrow[phantom, ""{coordinate, name=W}]{d}
        &
        (A+E)(\QQ)
        \arrow[
        rounded corners,
        to path={%
            -- ([xshift=2ex]\tikztostart.east)
            |- (W) [near end]\tikztonodes
            -| ([xshift=-2ex]\tikztotarget.west)
            -- (\tikztotarget)
        }
        ]{dll} \\
        &
        H^1(\QQ,A\cap E) \rar["\gamma"] 
        &
        H^1(\QQ,A\times E) \rar 
        & \ldots.
    \end{tikzcd}
\end{equation}
We now have
\begin{equation*}%
    \label{eq:galois_cohomology_order}
    \# (A+E)(\QQ) =
    \frac{\# (A\times E)(\QQ) \# \ker\gamma}{\# (A\cap E)(\QQ)}.
\end{equation*}
We will use this equality to inductively compute the rational torsion order of
$J$. Suppose we are able to compute the rational torsion order of $A$ for some
abelian subvariety $A$. The rational torsion of $E$ can be identified using
Nagell-Lutz. So we are able to compute $\#(A\times E)(\QQ)$ and $\#(A\cap
E)(\QQ)$. The hard part is $\#\ker\gamma$ and in general, we will only be able
to bound this term.

Let $\beta:H^1(\QQ, A\cap E)\to H^1(\QQ, E)$ be the map induced by the
inclusion of $A\cap E$ into $E$. Then $\ker\gamma\subseteq \ker\beta$. We have
that $A\cap E$ is some finite Galois group of $E$ and it is often the case that
$A\cap E=E[n]$ for some positive integer $n$. In this case, by Kummer Theory,
we have that
\[
    \#\ker\gamma\leq \#\ker\beta = \frac{\# E(\QQ)}{\# (nE(\QQ))}.
\]
Putting this altogether, when $A\cap E=E[n]$ for some positive integer $n$,
\begin{equation}
    \label{eq:galois_cohomology_bound}
    \# (A+E)(\QQ) \leq
    \frac{\# (A\times E)(\QQ)\#E(\QQ)}{\# (A\cap E)(\QQ)\#(nE(\QQ))}. 
\end{equation}

This now yields an inductive algorithm for computing the rational torsion order
of any rank-0 abelian totally split subvariety $X$ of $J_0(N)$.

\subsection{Reducing to smaller abelian subvarieties}%
\label{sub:reducing_to_smaller_abelian_subvarieties}

We now take advantage to \eqref{eq:eta_equation} to show that it often suffices
to verify the Generalized Ogg Conjecture on some abelian subvarieties of $J$.
\begin{proposition}
    Let $E$ be finite subset of $J_0(N)(\QQ)_\tor$ containing $C_N(\QQ)$ (for
    example $E=E_r$~\eqref{eq:stein_eta_group}). Let $x_1,\ldots,x_r\in E$ be a
    set of representatives of $E/(C_N(\QQ))$. Suppose for each $i$, there
    exists an abelian subvariety $A_i$ of $J_0(N)$ satisfying the Generalized
    Ogg Conjecture and containing $x_i$. Then $C_N(\QQ)=J_0(N)(\QQ)_\tor$ so
    $J_0(N)$ satisfies the Generalized Ogg Conjecture.
\end{proposition}
\begin{proof}
    We already have $C_N(\QQ)\subseteq J_0(N)(\QQ)_\tor$ (\ref{}). To show the
    reverse inclusion, let $x\in J_0(N)(\QQ)_\tor\subseteq E$. Then $x\in
    x_i+C_N(\QQ)$ for some representative $x_i$. This implies $x_i\in
    J_0(N)(\QQ)_\tor$. But $A_i$ satisfies the Generalized Ogg Conjecture so
    $x_i\in C_N(\QQ)$. Hence, $x\in C_N(\QQ)$.
\end{proof}

We will give an example of using this trick in Example~\ref{}.

\subsection{Summary of techniques}%
\label{sub:summary_of_techniques} 

Using Algorithm~\ref{alg:final_bound}, we have the following theorem.
\begin{theorem}%
    \label{thm:verification_rank_zero}
    There are 45 totally split rank-0 Jacobians $J_0(N)$ (See
    Table~\ref{tab:split}). The Generalized Ogg Conjecture has been verified
    for all but 9 such $N$'s given by
    \[
        84,90,96,120,132,144,150,168,180
    \]
    In these cases, they are bounded by a power of 2. See
    Table~\ref{tab:rank_zero_bound}.
\end{theorem}

\begin{algorithm}%
    \label{alg:final_bouind}
    Given a totally split rank-0 Jacobian $J=J_0(N)$ of dimension $k$, this
    algorithm will output a bound on $[J(\QQ)_\tor:C_N(\QQ)]$, where $C_N$ is
    the cuspidal subgroup of $J$.
    \begin{enumerate}
        \item{} [Upper bound by $E_{50}$]
            Set $r=50$ in~\eqref{eq:stein_eta_group} and compute $E=E_{50}$.
        \item{} [Set of representatives]
            Compute a list of elements $x_1,\ldots,x_s\in E_{50}$ that form a
            set of representatives for $E_{50}/C(N)$.
        \item{} [Decompose $J$]
            Use Algorithm~\ref{} to decompose $J$ into $J=\sum_{i=1} ^k E_j$,
            where $E_j$ are elliptic curves. 
        \item{} [Reduce to smaller abelian varieties]
            Let $V$ be the set of abelian subvarieties obtained by taking sums
            of $E_j$'s. Let each $x_i$, find $A_i\in V$ of minimal dimension
            such that $x_i\in A_i$.
        \item{} [Galois cohomology]
            Use Subsection~\ref{sub:galois_cohomology} to verify the
            Generalized Ogg Conjecture on each $A_i$.
    \end{enumerate}
\end{algorithm}

\section{Examples}

