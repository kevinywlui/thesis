\documentclass[thesis.tex]{subfiles}

\begin{document}
    
\chapter{Enumerating the Odd Isogenies Class of Prime Level Subvarieties}%
\label{chap:isogeny_class}

Let $N$ be a prime number so that $J_0(N)$ is non-trivial so $N=11$ or $N\geq
17$. Let $A$ be a simple abelian subvariety of $J_0(N)$. The goal of this
chapter is to, under certain conditions, enumerate the $\QQ$-isomorphism
classes of abelian varieties isogenous to $A$ by an odd-degree $\QQ$-isogeny.
We will call this the \emph{odd-degree isogeny class} of $A$.

More precisely, let $\TT$ be the Hecke algebra of $J_0(N)$ and $\TT_A$ be the
image of $\TT$ in $\End(A)$. There exists a newform $f=\sum a_n q^n$ associated
$A$. By~\cite[Prop. 7.14]{shimura:intro}, $\TT_A$ is isomorphic to an order of
the Hecke eigenvalue field, $K_f=\QQ(\ldots,a_n,\ldots)$. The goal of this
Chapter is to enumerate the odd-degree isogeny class of $A$ when $\TT_A$ is
integrally closed, $\Cl(\TT_A)$ is trivial, and when the hypothesis of
Corollary~\ref{cor:eisenstein} holds.

Unless otherwise stated, in this chapter, all abelian varieties, isogenies, and
isomorphisms are defined over $\QQ$.

The image of an odd-degree isogeny, $\varphi:A\to A'$, is determined, up to
isomorphism, by its kernel, $M$, which is a finite $G_\QQ$-submodule of $A(\QQbar)$.
We begin by showing every finite $G_\QQ$-submodule of $A(\QQbar)_\odd$ is a
$\TT_A[G_\QQ]$-module (Proposition~\ref{prop:G_modules_are_hecke}). This is
useful because the $\TT_A$-structure is more easily understood than the
$G_\QQ$-structure. Let $\I_A$ be the image of the Eisenstein ideal $\I$ into
$\End(A)$. A prime of $\TT_A$ is Eisenstein if it divides $\I_\A$ and is
non-Eisenstein otherwise. Let $\P_e$ be the set of Eisenstein primes of
odd-residue characteristic and $\P_{ne}$ be the set of non-Eisenstein primes of
odd-residue characteristic. We will use a theorem of Frank Calegari
(Theorem~\ref{thm:frank}) to prove that the image of an odd-degree isogeny is
isomorphic to one whose kernel is supported, as a $\TT_A$-module, only on
$\P_e$. Finally, we adapt the work of Krzysztof Klosin and Mihran Papikian to
enumerate the isomorphic classes of abelian varieties isogenous to $A$ by an
isogeny supported on $\P_e$.

\section{Finite odd-order Galois Modules are Hecke}%
\label{sec:finite_odd_order_galois_modules_are_hecke}

The goal of this section is to prove every finite odd-order $G_\QQ$-submodule
$M$ of $A(\QQbar)$ is a Hecke module
(Proposition~\ref{prop:G_modules_are_hecke}). The Galois action of
$J(\QQbar)_\odd$ has been extensively studied by Mazur~\cite{mazur:eisenstein},
so we weaken our hypothesis to $M$ a finite $G_\QQ$-submodule of
$J(\QQbar)_\odd$. This is allowed because $A$ is $\TT[G_\QQ]$-stable.

It suffices to prove $M$ is $\TT$-stable for each $G_\QQ$-composition factor $V$
of $M[\ell^\infty]$ for $\ell>2$. The irreducibility of $V$ implies that it is
$\ell$-torsion. Ribet~\cite[Proposition 6.1]{ribet:semistable_gal} shows that
$\TT/\ell \TT$ is generated by $T_p$ for primes $p\nmid \ell N$. So we reduce
modulo $p$ for $p\nmid \ell N$, and use Eichler-Shimura to derived its
$\TT$-stability from its $G_\QQ$-stability.

\begin{proposition}\label{prop:G_modules_are_hecke}
    Suppose $M$ is a finite odd-order $G_\QQ$-submodule of $J_0(N)$, with $N$
    prime. Then $M$ is a $\TT[G_\QQ]$-module.
\end{proposition}
\begin{proof}
    It suffices to show $M$ is $\TT$-stable for each $\ell$-primary part. Let
    $\ell>2$ and assume $M\subseteq J[\ell^\infty]$. Let
    \[
        0 = M_0 \subsetneq \ldots \subsetneq M_n = M
    \]
    be an $G_\QQ$-composition series of $M$ with composition factors $X_i =
    M_i/M_{i-1}$. We proceed by induction on $n$ with the base
    case being the trivial $n=0$ case.

    Assume $M_{s-1}$ is an $\TT[G_\QQ]$-module. We will show $M_s$ is an
    $\TT[G_\QQ]$-module. Since $M_{s-1}$ is an $\TT[G_\QQ]$-module, for each
    $t\in \TT$, we have a well-defined map $t:X_s\to J(\QQbar)/M_{s-1}$. The
    goal is to show $t(X_s)\subseteq X_s$ for all $t\in \TT$.
    By~\cite[Proposition 2]{ribet:mult_p_finite}, $\TT/\ell \TT$ is generated
    by $T_p$ for $p\nmid \ell N$ so it suffices to show $T_p(X_s)\subseteq X_s$
    for prime $p\nmid \ell N$.

    Fix a prime $p\nmid \ell N$. Since $p$ does not divide $N$, $J$ has good
    reduction at $p$. Fix a place $\p$ over $p$. The reduction map yields an
    isomorphism~\cite[Theorem 1, Lemma 2]{serre-tate}
    \[
        \tau:J(\QQbar)[\ell^\infty] \riso J_{/\F_p} (\Fpbar)[\ell^\infty]
    \]
    sending $\Frob_\p$ to $F_p$, where $F_p$ is the absolute Frobenius on
    $J_{/\F_p}$. Under this isomorphism the natural $\TT$-action on $J(\QQ)$
    maps to the natural $\TT$-action on $J_{/\F_p}$~\cite[\S
    5.2]{ribet-stein:serre}. By Eicher-Shimura, $T_p = F+p/F\in
    \End(J_{/\F_p})$ so
    \[
        \tau(T_p X_s)
        = T_p\tau(X_s)
        = (F+p/F)\tau(X_s)
        = \tau((\Frob_\p+p/\Frob_\p)X_s)
        \subseteq \tau(X_s)
    \]
    hence, $T_p X_s\subseteq X_s$, as desired.
\end{proof}


\section{Non-Eisenstein modules are kernels of Hecke}%
\label{sec:non_eisenstein_modules_are_kernels_of_hecke}

In the previous section, we show that every finite odd-order $G_\QQ$-submodules of
$J(\QQbar)$ and $A(\QQbar)$ are $\TT[G_\QQ]$-modules and
$\TT_A[G_\QQ]$-modules, respectively. The goal now is to show if $M$ is a
finite-odd order $\TT_A[G_\QQ]$-submodule $A(\QQbar)$ supported only on the
non-Eisenstein primes as a $\TT_A$-module then $M=A[\Ann_{\TT_A} M]$. As in the
previous section, since $A$ is $\TT[G]$-invariant, we can instead take the
hypothesis that $M$ is a finite-odd order $\TT[G_\QQ]$-submodule of $J(\QQbar)$
supported as a $\TT$-module only on the non-Eisenstein primes, then
$M=J[\Ann_{\TT} M]$.

If $M$ is any $\TT[G_\QQ]$-submodule of $J(\QQbar)$, it is always the case that
$M\subseteq J[\Ann_{\TT} M]$. It is for the reverse inclusion where we need to
use the fact that $M$ is supported only on the non-Eisenstein primes of
odd-residue characteristic. The crucial fact we use about non-Eisenstein primes
is that, by~\cite[Prop. 14.2]{mazur:eisenstein}, if $\m$ is a non-Eisenstein of
odd residue characteristic, then $J[\m]$ is an irreducible $G_\QQ$-module.

We are already able to show that $M=J[\Ann_{\TT} M]$ when $M$ is irreducible as
a $\TT[G_\QQ]$-submodule.
\begin{proposition}
    Let $\m$ be a non-Eisenstein prime of odd residue characteristic, if $M$ is
    a nonzero finite irreducible $\TT[G_\QQ]$-submodule of $J(\QQbar)$
    supported only on $\m$ as a $\TT$-module, then $M=J[\m]$.
\end{proposition}
\begin{proof}
    Let $\a=\Ann_{\TT}(M)$. We will first show that $\a=\m$ by showing $\a$ is
    maximal. Let $e$ be the exponent of $M$ as an abelian group. We have that
    $e\in \a$ and $\TT/e\TT$ is a finite ring so it suffices to show that $\a$
    is a prime ideal. Suppose $x,y\not\in \Ann_{\TT}(M)$. Then $yM$ is a nonzero
    $\TT[G_\QQ]$-submodule of the irreducible module $M$ so $yM=M$. Similarly, $xyM = xM$ is
    nonzero. Therefore, $yx\notin \Ann_{\TT}(M)$ so $\a=\m$.

    Now $M\subseteq J[\m]$ but $J[\m]$ is an irreducible
    $G_\QQ$-module~\cite[Proposition 14.2]{mazur:eisenstein} so $M=J[\m]$.
\end{proof}

The general case will follow from a trivial adaptation of the work of David
Helm~\cite{helm:jacobian}. Helm considers the case of Jacobians, $J$, of
Shimura curves. One of the key inputs into Helm's proof is that for the maximal
ideals $\m$ in question is that if $T_\m J$ is the contravariant $\m$-adic Tate
module, then $T_\m J/\m T_\m J\cong J[\m]^\vee$ is dimension two over $\TT/\m$
and is irreducible as a $G_\QQ$-module. By~\cite[Prop. 14.2]{mazur:eisenstein},
this is also the case for $J=J_0(N)$ with $N$ prime and $\m$ a non-Eisenstein
prime of odd residue characteristic.

\begin{theorem}[{\cite[Corollary 4.8]{helm:jacobian}}]%
    \label{thm:non_eisenstein_kernel_hecke}
    Let $M$ be a finite odd-order $G_\QQ$-module supported, as a $\TT$-module,
    only on the non-Eisenstein primes. If $I=\Ann_\TT(M)$, then $M=J[I]$.

    Moreover, since $A$ is both $\TT[G_\QQ]$-invariant, if $M$ is a finite
    odd-order $G_\QQ$-submodule of $A(\QQbar)$ supported, as a $\TT_A$-module,
    only on the non-Eisenstein primes. If $I=\Ann_{\TT_A}(M)$, then $M=A[I]$.
\end{theorem}
\begin{proof}
Since $M\subseteq J[I]$ and $\Supp_{\TT} M = \Supp_{\TT} J[I]$, it suffices to
prove $J[I]_\m = M_\m$ for each non-Eisenstein prime of odd residue
characteristic. So let $\m$ be a non-Eisenstein prime of odd residue
characteristic. We start by reviewing the contravariant Tate modules of $J$ and
proving Lemma~\ref{lem:finite_index}.

Let $T_\m J\isom \Hom(J[\m^\infty], \QQ_\ell/\ZZ_\ell)$ be the contravariant
Tate module at $\m$ and $\overline{\rho}_\m$ be the Galois representation
associated to $J[\m]^\vee$. Since $\m$ is an odd non-Eisenstein prime,
$\overline{\rho}_\m$ is an irreducible $G_\QQ$-representation of dimension 2
over $k_\m$ that is isomorphic to $J[\m]^\vee$~\cite[Prop.
14.2]{mazur:eisenstein}.

\begin{lemma}[{\cite[Lemma 4.6]{helm:jacobian}}]\label{lem:finite_index}
    Let $R$ be a $G_\QQ$-stable submodule of $T_\m J$ of finite index. Then
    $R=IT_\m J$ for some ideal $I$ of $\TT$.
\end{lemma}
\begin{proof}
    We proceed by induction on the maximal $G_\QQ$-composition series of $T_\m J/R$
    with the base case being the trivial length zero case. Let
    \[
        R = R_n \subsetneq R_{n-1} \subsetneq \cdots \subsetneq R_0 = T_\m J
    \]
    be a $G_\QQ$-composition series. By induction, $R_{n-1} = I'T_\m J$ for some
    $I'\subseteq \TT$.

    Consider $\m R_{n-1} + R$. This is a $G_\QQ$-module sitting between $R$ and
    $R_{n-1}$. By Nakayama's lemma, if $\m R_{n-1} R + R = R_{n-1}$, then
    $R=R_{n-1}$ which is a contradiction. Hence, $\m R_{n-1}+R=R$ so $R$
    contains $\m R_{n-1}$ and we can form the quotient.

    The module $R_{n-1}/\m R_{n-1}$ is $G_\QQ$-isomorphic to $(I'/\m
    I')\otimes_{\TT/\m} (T_\m J/\m T_\m J) \cong (I'/\m I')\otimes_{\TT/\m}
    J[\m]^\vee$, where $G_\QQ$ acts trivially on $I'/\m I'$. Let $V$ be the image
    of $R$ in $R_{n-1}/\m R_{n-1}$. Since $V$ is $G_\QQ$-invariant, and
    $J[\m]^\vee$ is irreducible, $V$ is given by $\hat{V}\otimes J[\m]^\vee$
    for some $\TT/\m$-subspace $\hat{V}$ of $I'/\m I'$. Let $I$ be the preimage
    of $\hat{V}$ in $I'$. Then $IT_m J = R$, since both contain $\m R_{n-1}$
    and map to $V$ modulo $\m R_{n-1}$.
\end{proof}


Let $B=J/M$ be the quotient abelian variety. Since $M$ is a $\TT$-module,
we may equipped with a $\TT$-action. So the projection $\phi:J \to B$ is an
$\TT[G_\QQ]$-isogeny with $\ker\phi = M$. For any odd non-Eisenstein prime
$\m$, $\phi$ induces the exact sequence
\[
    0 \to T_\m B \to T_\m J \to M^\vee _\m \to 0.
\]
In particular, the image of $T_\m B$ under $\phi$ is a finite index
$\TT[G_\QQ]$-submodule of $T_\m A$. By Lemma~\ref{lem:finite_index}, we can find
an ideal $I'$ of $\TT$ such that the image of $T_\m B$ is $I' _\m T_\m A$
for all odd non-Eisenstein primes $\m$. We have $M^\vee _\m = T_\m J / I'
T_\m J\cong J[I']_\m ^\vee$. Therefore, by taking annihilators of the dual,
we have that $I_\m = I'_\m$ and then by taking duals $M_\m = J[I]_\m$, as
desired.
\end{proof}

\section{Bounding non-Eisenstein part}%
\label{sec:bounding_non_eisenstein_part}

Recall that a prime of $\TT_A$ is non-Eisenstein if it does not divides the
Eisenstein ideal $\TT_A$. There are infinitely many non-Eisenstein primes. The
goal of this section is to bound isomorphism classes of $A/X$ with
$\Supp_{\TT_A} X\subseteq \P_{ne}$ by $\Cl(\TT_A)$. In particular, we will
prove a theorem of Frank Calegari.

Let $\H$ be a set of integral representatives for $\Cl(\TT_A)$ coprime to
$2\I_A$. Suppose $M$ is $\TT_A$-supported only on the primes of $\P_{ne}$. By
Theorem~\ref{thm:non_eisenstein_kernel_hecke}, $M=A[\a]$ with $\a=\Ann_{\TT_A}
M$. Now there exists $s,t\in \TT_A$ and $\b \in \H$, such that $s\a=t\b$. We
will show that $\varphi=s\circ t^{-1}\in \TT_A\otimes \QQ$ is well-defined and
yields an isomorphism $A/M\to A/A[\b]$. Lastly, we will make some remarks about
the $\p$-adic valuation of $\varphi$ that we will use in
Section~\ref{sec:bounding_support_and_valuations} 

\begin{theorem}[Frank Calegari]
    \label{thm:frank}
    Let $A\subseteq J_0(N)$ be a simple abelian subvariety with $N$ prime.
    Suppose $\TT_A$ is integrally closed. Let $M$ be a $\TT_A[G_\QQ]$-submodule
    of $A(\QQbar)$ supported only on the non-Eisenstein primes of odd residue
    characteristic. Then there exists $\varphi\in \TT_A\otimes \QQ$ and $\b\in
    H$, such that, there is an isomorphism
    \[
        \varphi:A/M\to A/A[\b].
    \] 
    Moreover, $v_\p(\varphi)=0$ if $\p\in \P_e$ and $v_\p(\varphi)\leq
    v_\p(\b)$ if $\p\in \P_{ne}$.
\end{theorem}
\begin{proof}
    By Theorem~\ref{thm:non_eisenstein_kernel_hecke}, if $\a=\Ann_{\TT_A} M$,
    then $M=A[\a]$. We have that there exists $s,t\in \TT_A$ and $\b\in H$,
    such that $s\a=t\b$. We will justify the following commutative diagram.
    \[
        \begin{tikzcd}
            A/A[s\a]
            \arrow[r, "\sim", "s"']
            \ar[equal]{d}
            &
            A/A[\a]
            \ar[d, "\sim" labl, "\phi"]
            \\
            A/A[t\b]
            \arrow[r, "\sim", "t"']
            &
            A/A[\b],
        \end{tikzcd}
    \]
    
    We will first establish the isomorphism $s:A/A[s\a]\to A/A[\a]$.
    \[
        \begin{tikzcd}
            0 \arrow[r]
            &
            (A/A[\a])[s]
            \arrow[hookrightarrow]{r}
            &
            A/A[\a]
            \arrow[twoheadrightarrow]{r}{s}
            &
            A/A[\a]
            \arrow[r]
            &
            0
        \end{tikzcd}.
    \]
    We have that $x\in (A/A[\a])[b]$ if and only if $bx \in A[\a]$ if and only if
    $x \in A[s\a]$. Hence, $(A/A[\a])[b]=A[s\a]/A[\a]$ so $b:A/A[s\a]\xra{\sim}
    A[\a]$. A similar argument applies for $a:A/A[t\b]\to A/A[\b]$.

    We have that
    \[
        v_\p(\varphi)=v_\p(t)-v_\p(s) = v_\p(\a)-v_\p(\b).
    \]
    If $\p\in P_e$, then $v_\p(C), v_\p(D)=0$. Therefore, $v_\p(\varphi)=0$ if
    $\p\in \P_e$ and $v_\p(\varphi)\leq v_\p(\b)$ if $\p\in \P_{ne}$.
\end{proof}



\section{Bounding support and valuations}%
\label{sec:bounding_support_and_valuations}

Let $A\subseteq J_0(N)$ be a simple abelian subvariety with $N$ prime. Suppose
$\TT_A$ is integrally closed. Suppose $\psi:A\to A'$ be an odd-isogeny. Then
$M=\ker\psi$ is a $G_\QQ$-submodule of $A(\QQbar)_\odd$. By
Proposition~\ref{prop:G_modules_are_hecke}, $M$ is a $\TT_A[G_\QQ]$-module so
we can decompose $M$ as $M_{ne}\oplus M_e$, where $M_{ne}$ and $M_e$ are
$\TT_A[G_\QQ]$-submodules supported, as $\TT_A$-modules, only on $\P_{ne}$ and
$\P_e$, respectively. Using Theorem~\ref{thm:frank}, $A/M_{ne} \isom A/A[\b]$
for some $\b\in \H$. The goal of this section is to show $A/M=A/(M_{ne}\oplus
M_e) \isom A/(A[\b]\oplus X_e)$ while controlling the difference between $M_e$
and $X_e$.


\begin{proposition}%
    \label{prop:control_supp}
    Let $A\subseteq J_0(N)$ be a simple abelian subvariety with $N$ prime.
    Suppose $\TT_A$ is integrally closed. Suppose $\psi:A\to A'$ be an
    odd-isogeny. Let $M, M_{ne}, M_e$ be as above. Then
    \[
        A'\isom A/(A[\b]\oplus X_e),
    \]
    where $\Supp_{\TT_A} X_e\subseteq \P_e$. Moreover, for each $\p\in \P_e$,
    let $e_\p$ be the smallest nonnegative integer such that
    $M_e[\p^{\infty}]\subseteq A[\p^{e_\p}]$. Then, for each $\p\in \P_e$, we
    have $X_e[\p^\infty]\subseteq A[\p^{e_\p}]$.
\end{proposition}
\begin{proof}
    By Theorem~\ref{thm:frank}, there exists $\varphi\in \TT_A\otimes \QQ$ and
    $\b\in \H$, such that $\varphi:A/M_{ne}\to A/A[\b]$ is an isomorphism with
    $v_\p(\varphi)=0$ if $\p\in \P_e$ and $v_\p(\varphi)\leq v_\p(\b)$ if
    $\p\in \P_{ne}$.

    Let $\psi:A/M_{ne}\to A/(M_{ne}+M_e)$ be the isogeny corresponding to
    quotienting by $M_e$. Let $\psi':A/A[\b]\to A(A[\b]+\varphi(M_e))$. We have
    that $\varphi$ is also an isomorphism from $A/(M_{ne}+M_e)\to
    A(A[\b]+\varphi(M_e))$. 
    \[
        \begin{tikzcd}
            A/M_{ne}
            \arrow[r, "\psi"]
            \arrow[d, "\varphi"]
            &
            A/(M_{ne}+M_e)
            \arrow[d, "\varphi'"]
            \\
            A/A[\b]
            \arrow[r, "\psi'"]
            &
            A/(A[\b]+\varphi(M_e)).
        \end{tikzcd}
    \] 
    Since $\varphi(M_e)$ is $\TT_A$-module, we can write it as $X_{ne}\oplus
    X_e$. Let $\p\in \P_{ne}$. Since $M_e$ is supported away from $\p$ and
    $v_\p(\varphi)\leq v_\p(\b)$, $\varphi(M_e)[\p^\infty]\subseteq A[\b]$.
    Therefore, $A[\b] + \varphi(M_e)=A[\b] + X_{e}$. Now if $\p\in \P_e$
    instead, we have that $v_\p(\varphi)=0$ and $M_e[\p^\infty]\subseteq
    A[\p^\infty]$ so $X_e[\p^\infty]=\varphi(M_e)[\p^\infty]\subseteq
    \varphi(A[\p^{e_\p}])\subseteq A[\p^{e_\p}]$.
\end{proof}

As a corollary, we can give an explicit set of primes such that %TODO
\begin{corollary}
    Let $\P_\H$ be the union of the set of primes dividing elements of $\H$ with
    the set of Eisenstein prime of odd-residue characteristic. If $A'$ is
    isogenous to $A$ by an odd-degree isogeny, $A'\cong A/X$ for some
    $\TT_A[G]$-submodule of $A(\QQbar)$ supported on $\H$.
\end{corollary}

\section{Eisenstein part}%
\label{sec:eisenstein_part}

In this section, we will follow the work of Krzysztof Klosin and Mihran
Papikian to enumerate the isomorphic classes of abelian varieties isogenous to
$A$ by an isogeny supported on $\P_e$.

\begin{proposition}{{\cite[Prop. 4.5]{klosin-papikian:ribet}}}%
    \label{prop:eisenstein_cyclic}
    Let $A\subseteq J_0(N)$ be a simple abelian subvariety with $N$ prime. Let
    $\p$ be an Eisenstein prime of $\TT_A$ of odd residue characteristic $p$.
    Suppose there exists $\ell\equiv 1\pmod{p^2N}$ such that $p^3\nmid
    \mathcal{A}_{/\ell}(\FF_\ell)$, where $\mathcal{A}_{/\ell}$ is the
    reduction of the Neron model of $A$ at $\ell$. Suppose that $M\subseteq
    A[\p^\infty]$. If $A[\p]\not\subseteq M$, then $M\subseteq A[\p]$.
\end{proposition}
\begin{proof}
    We will assume $A[\p]\not\subseteq M$ and $M\not\subseteq A[\p]$, and
    derive a contradiction by showing $p^3\mid \#A(\F_\ell)$ for all rational
    primes $\ell \equiv 1\pmod{p^2N}$.

    Let $K=\QQ(M)$. By~\cite[Cor. 16.3]{mazur:eisenstein}, $A[\p]=C[p]\oplus
    \Sigma[p]$. Since $\Sigma[p]$ is of $\mu$-type, $\QQ(A[\p])=\QQ(\mu_p)$.
    Let $F=\QQ(\mu_{pN})$ and $K=\QQ(M)$. Then both $A[\p]$ and $M$ are
    constant over $KF$. Since $A[\p]\not\subseteq M$ and $M\not\subseteq
    A[\p]$, we have that $p^3\mid A(K)_\tor$. The goal now is to show that if
    $\ell\equiv 1\pmod{p^2N}$ is a rational prime then $\ell$ is completely
    split over $K$. We will do this by showing $K \subseteq \QQ(\mu_{p^2 N})$.

    \begin{lemma}[{\cite[Lem. 4.6]{klosin-papikian:ribet}}]
        The number field $K=\QQ(M)$ is an abelian extension of $\QQ$ unramified
        away from $p, N$.
    \end{lemma}
    \begin{proof}
        Since $M$ is supported only on $\p$ as a $\TT_A$-module, we may view
        $M$ as a $(\TT_A)_\p$-module. Since $M$ is finite and $(\TT_A)_\p$ is a
        DVR\@, we have
        \[
            M\cong (\TT_A)_\p / \p^{s_1} \times\cdots \times (\TT_A)_\p /
            \p^{s_r}\cong (\TT_A)/\p^{s_1} \times \cdots \times
            (\TT_A)/\p^{s_r}
        \]
        for some $s_1,\ldots,s_r\geq 0$. Since $\dim_{\TT_A/\p} A[\p]=2$ and
        $M[\p]\cong (\TT_A / \p)^r \subsetneq A[\p]$ so $r=1$. Therefore,
        $M\cong \TT_A/\p^{s_1}$.

        Recall the elements of $\TT_A$ are defined over $\QQ$ so they commute
        with elements of $G_\QQ$, therefore,
        \[
            \Gal(K/\QQ)\subseteq \Aut_{\TT_A} (M) \cong \Aut_{\TT_A}
            (\TT_A/\p^{s_1}) \cong (\TT_A/\p^{s_1})^\times.
        \]
        Since $\TT_A$ is isomorphic to an order of a number field,
        $\Gal(K/\QQ)$ is abelian. Since $A$ has good reduction away from $N$,
        $K/\QQ$ is unramified away from $p, N$.
    \end{proof}

    \begin{lemma}
        The number field $K$ is a subfield of $\QQ(\mu_{p^2}, \mu_N)$.
    \end{lemma}
    \begin{proof}
        By assumption, $(\TT_A)_\p$ is a DVR\@, so, as a $(\TT_A)/\p$-space,
        $\p/\p^2$ is generated by some $\alpha\in \p$. This yields the exact
        sequence
        \[
            \begin{tikzcd}
                0 \arrow[r] &
                A[\p] \arrow[r] &
                A[\p^2] \arrow[r, "\alpha"] &
                A[\p]
            \end{tikzcd}.
        \]
        By restricting to $M$, we have
        \begin{equation}
            \label{eq:M_Ap}
            \begin{tikzcd}
                0 \arrow[r] &
                M\cap A[\p] \arrow[r] &
                M \arrow[r] &
                M\cap A[\p] \arrow[r] &
                0
            \end{tikzcd} 
        \end{equation}

        Since $M\subsetneq A[\p]$, $M\cap A[\p]$ is either $C[p]$ or
        $\Sigma[p]$. As a $\ZZ$-module, $M\cong \ZZ/p^2$. Therefore,
        $\Gal(\QQbar/F)$ acts trivially on $pM=M\cap A[\p]$. So $\Gal(KF/F)$
        can be identified with the subgroup $\{a\in (\ZZ/p^2):ap\equiv p
        \pmod{p^2}\}\subseteq (\ZZ/p^2)$ of order $p$ so $[KF:F]=1$ or $p$.

        We have 
        \[
            \Gal(\QQ(\mu_p ^{n_1}, \mu_N ^{n_2})/F \cong \ZZ/p^{n_1-1} \times
            \ZZ/N^{n_2-1}.
        \]
        Since $[KF:F]=1$ or $p$, $KF\subseteq F(\mu_p ^{n_1})$ is a subfield of
        degree 1 or $p$. In either case, $K\subseteq KF \subseteq \QQ(\mu_p ^2,
        \mu_N)$.
    \end{proof}

    Now suppose $\ell\equiv 1\pmod{p^2 N}$ is a rational prime. Then $\ell$ is
    completely split over $K$ so, as in Section~\ref{sub:upper_bound},
    by~\cite[Appendix]{katz:galois}, there is an inclusion
    \[
        A(K)_\tor\hookrightarrow \mathcal{A}_{/\F_\ell}(\F_\ell).
    \]
    Since $p^3\mid \#A(K)_\tor$, $\p^3\mid \#\mathcal{A}_{/\F_\ell}(\F_\ell)$,
    as desired.
\end{proof}

\begin{corollary}
    \label{cor:eisenstein}
    Let $A$ be a simple abelian subvariety of $J_0(N)$ with $N$ prime. Suppose
    $\TT_A$ is integrally closed. Suppose that for all Eisenstein primes $\p$
    of odd residue characteristic $p$ with $A[\p]$ nontrivial, $\p$ is
    principal and there exists a rational prime $\ell\equiv 1\pmod{p^2N}$ such
    that $p^3\nmid \#\mathcal{A}_{/\F_\ell}(/\F_\ell)$. Then if $\varphi:A\to
    A'$ is an isogeny with $\Supp_{\TT_A} \ker\varphi\subseteq \P_e$, then
    \[
        A'\isom A/M
    \]
    with $\Supp_{\TT_A}(M)\subseteq \P_e$ and $M[\p^\infty]=(C_N\cap A)[p]$ or
    $(\Sigma_N\cap A)[p]$ for all $\p\in \P_e$. Here $C_N$ and $\Sigma_N$ are
    the cuspidal and Shimura subgroups of $J_0(N)$.
\end{corollary}
\begin{proof}
    We begin by simplifying $\varphi$ so that we can apply
    Proposition~\ref{prop:eisenstein_cyclic}. Suppose that for some $\p\in
    \P_e$, $0\neq A[\p]\subseteq \ker\varphi$. By assumption, $\p$ is
    principally generated by some $\alpha\in \TT_A$. The isogeny $\varphi$ now
    factors as $\varphi=\varphi'\circ \alpha$. We have
    $\Im\varphi'=\Im\varphi=A$ and the strict containment of kernels
    $\ker\varphi'\subsetneq \ker\varphi$. If needed, we can repeat this
    process, so we may assume that $\varphi':A\to A'$ is an isogeny with
    $\Supp_{\TT_A}\ker\varphi'\subseteq \P_e$ and $A[\p]\not\subseteq
    \ker\varphi'$ for any $\p\in \P_e$ with $A[\p]\neq 0$.

    Let $M=\ker\varphi'$. Now by Proposition~\ref{prop:eisenstein_cyclic},
    $M[\p^\infty]\subsetneq A[\p]$ so $M[\p^\infty]=(C_N\cap A)[p]$ or
    $(\Sigma_N\cap A)[p]$ for all $\p\in \P_e$.
\end{proof}


\section{Combining Eisenstein and non-Eisenstein parts}%
\label{sec:combining_eisenstein_and_non_eisenstein_parts}


\begin{theorem}%
    \label{thm:combine}
    Suppose $A$ is a simple abelian subvariety of $J_0(N)$ with $N$ prime.
    Suppose $\TT_A$ is integrally closed and $\#\Cl(\TT_A)=1$. Suppose that for
    all Eisenstein primes $\p$ of odd residue characteristic $p$ with $A[\p]$
    nontrivial, there exists a rational prime $\ell\equiv 1\pmod{p^2N}$ such
    that $p^3\nmid \#\mathcal{A}_{/\F_\ell}(/\F_\ell)$. Then if $\varphi:A\to
    A'$ is an isogeny with $\Supp_{\TT_A} \ker\varphi\subseteq \P_e$, then
    \[
        A'\isom A/M
    \]
    with $\Supp_{\TT_A}(M)\subseteq \P_e$ and $M[\p^\infty]=(C_N\cap A)[p]$ or
    $(\Sigma_N\cap A)[p]$ for all $\p\in \P_e$. Here $C_N$ and $\Sigma_N$ are
    the cuspidal and Shimura subgroups of $J_0(N)$.
\end{theorem}
\begin{proof}
    Since $\TT_A$ is principal, by Proposition~\ref{prop:control_supp}, we have
    that $A'\isom A/X$ with $X$ a $\TT_A[G_\QQ]$-submodule of $A(\QQbar)$ such
    that $\Supp_{\TT_A}X\subseteq \P_e$. And now Corollary~\ref{cor:eisenstein}
    yields the desired result.
\end{proof}

Under these hypothesis, we obtain a set $S$ of abelian varieties such that if
$A'\sim A$ is a odd-degree isogeny, then $A'\isom A''$ for some $A''$ in $S$.
Now by applying isomorphism testing, we obtain the odd-degree isogeny class of
$A$. 


% \section{Further Questions}

% A few questions arise naturally from our results. In this chapter, we take on
% the hypothesis that $\TT_A$ is integrally closed, $\Cl(\TT_A)$ is trivial, and
% the hypothesis of Corollary~\ref{cor:eisenstein}.

% Suppose $\TT_A$ is not integrally closed. Then the valuation argument we use in
% Theorem~\ref{thm:frank} will no longer apply. However, we can still bound the
% support of the kernel of the isogenies. In particular, let $\H'$ be a set of
% integral representatives for $\Cl(\TT_A)$ and $\P'$ be the set of primes
% dividing the elements of $H$ and the conductor of $\TT_A$ in its integral
% closure. Then by a similar argument to Theorem~\ref{thm:frank}, we can prove
% \begin{proposition}
%     Suppose $A$ is a simple abelian subvariety of $J_0(N)$ with $N$ prime.
%     Let $\psi:A\to A'$ be an odd-degree isogeny. Then
%     \[
%         A'\cong A/X
%     \]
%     where $X$ is a $\TT_A[G_\QQ]$-submodule of $A(\QQbar)$ supported, as a
%     $\TT_A$-module, on the elements of $\P'\cup \P_e$.
% \end{proposition}
% Another challenge that arise under this hypothesis is enumerating the
% $G_\QQ$-submodule of $A[\p^s]$ when $\p$ is a non-Eisenstein prime. For
% simplicity, assume $s=2$. Following
% the second paragraph of~\cite[\S 14]{mazur:eisenstein}, let $a_1,\ldots,a_r\in
% \p$ generate $\p/\p^2$ as a $(\TT_A/\p)$-vector space. We have the
% $\TT_A[G_\QQ]$-injection, $A[\p^2]/A[\p]\to A[\p]^r$ given by $x+A[\p]\mapsto
% (a_1x,\ldots,a_rx)$. If $(\TT_A)_\p$ is a DVR\@, then $r=1$ and the
% $G_\QQ$-submodule of $A[\p^2]$ are $A[\p^k]$ for $k=0,1,2$. This trick no
% longer holds if $(\TT_A)_\p$ is not a DVR\@.

% In all of our experiments, we could not a find a case where $\TT_A$ is
% integrally closed but $\Cl(\TT_A)$ is non-trivial. The Hecke eigenvalue field
% $\TT_A\otimes \QQ$ is totally real so the folklore claims that $\Cl(\TT_A)$
% will often be trivial. However, this could still be potentially interesting as
% the infinitude of number fields with trivial class group is an open problem. 

% Lastly, we have not been able to find 

\end{document}
