\documentclass[thesis.tex]{subfiles}

\begin{document}
    
\chapter{Enumerating the Odd Isogenies Class of Prime Level Subvarieties}%
\label{chap:isogeny_class}

Let $N$ be a prime number so that $J_0(N)$ is non-trivial so $N=11$ or $N\geq
17$. Let $A$ be a simple abelian subvariety of $J_0(N)$. The goal of this
chapter is to, under certain conditions, enumerate the $\QQ$-isomorphism
classes of abelian varieties isogenous to $A$ by an odd-degree $\QQ$-isogeny.
We will call this the \emph{odd-degree isogeny class} of $A$.

More precisely, let $\TT$ be the Hecke algebra of $J_0(N)$ and $\TT_A$ be the
image of $\TT$ in $\End(A)$. There exists a newform $f=\sum a_n q^n$ associated
$A$. By~\cite[Prop. 7.14]{shimura:intro}, $\TT_A$ is isomorphic to an order of
the Hecke eigenvalue field, $K_f=\QQ(\ldots,a_n,\ldots)$. The goal of this chapter
is to enumerate the odd-degree isogeny class of $A$ when $\TT_A$ is integrally
closed and (when Proposition~\ref{} works).

Unless otherwise stated, in this chapter, all abelian varieties, isogenies, and
isomorphisms are defined over $\QQ$.

The image of an isogeny is determined by its kernel up to isomorphism. So we
define an equivalence relation on the set of finite odd-order
$G_\QQ$-submodules of $A(\QQbar)$ given by $M_1\sim M_2$ if and only if
$A/M_1\isom A/M_2$. For any odd-order $G_\QQ$-submodule $X$ of $A(\QQbar)$, let
$\M(X)$ be the set of equivalence classes of finite odd-order
$G_\QQ$-submodules of $A(\QQbar)$ with a representative that is a submodule of
$X$. Let $A(\QQbar)_\odd$ be the odd-torsion part of $A(\QQbar)$. The goal of
this chapter can then be restated as enumerating a set of representatives for
$\M(A(\QQbar)_\odd)$ when $\TT_A$ is integrally closed and (when
Proposition~\ref{} works). 

We begin by showing every finite $G_\QQ$-submodule of $A(\QQbar)_\odd$ is a
$\TT_A$-module (Proposition~\ref{prop:G_modules_are_hecke}). This is useful
because the $\TT_A$-structure is more easily understood than the
$G_\QQ$-structure. For instance, we are now able to talk about the
$\TT_A$-support of finite odd-order $G_\QQ$-submodules. We then show that every
finite odd-order $G_\QQ$-submodule is equivalent to one support on $S$
(\ref{prop:bound_support}). Now we construct an ideal $Q$ of $\TT_A$ such that
$S$ is the set of primes dividing $Q$. So $\M(A[Q^\infty])=\M(A(\QQbar)_\odd$.
We will then give an method for determining an integer $k$ such that
$\M(A[Q^k])=\M(A[Q^\infty])=\M(A(\QQbar)_\odd)$
(Proposition~\ref{prop:bound_support}). Lastly, we give an algorithm for
enumerating the finite $G_\QQ$-submodules of $\M(A[Q^k])$. In summary, we have
the following algorithm.

\begin{algorithm}{Odd isogeny class}%
    \label{alg:odd_isogeny_class}
    Let $A$ be a simple abelian subvariety with $\TT_A$ integrally closed. This
    algorithm will enumerate the class of abelian varieties isomorphic
    isogenous to $A$ by an odd-degree isogeny.
    \begin{enumerate}
        \item{} [Class group representatives]
            Compute a set of odd integral representatives $H=\{C_i\}$ of
            $\Cl(\TT_A)$. Let $Q = (\prod_{C_i\in H} C_i)(\prod_{\p\in E} \p)$,
            where $E$ is the set of primes dividing the Eisenstein ideal.
        \item{} [Initialize search]
            Set $r=1$ and $X_{r-1}=\emptyset$.
        \item{} [$G_\QQ$-submodules of $A[Q^r]$]
            Use Algorithm~\ref{alg:enumerating_AX} to give a set, $X_r$, of
            representatives for $\M(A[Q^r])$.
        \item{} [Done?]
            By Corollary~\ref{prop:stop_looking}, if for all $x\in X_r$, $x\sim
            y$ for some $y\in X_{r-1}$, then $X_r$ is a set of representatives
            for $\M(A(\QQbar)_\odd)$. If this is not the case, increment $r$
            and repeat the last step.
        \item{} [Quotient and output]
            Output $A/M$ for $M\in X_r$.
    \end{enumerate}
\end{algorithm}

\section{Finite odd-order Galois Modules are Hecke}%
\label{sec:finite_odd_order_galois_modules_are_hecke}

The goal of this section is to prove every finite odd-order $G_\QQ$-submodule $M$
of $A(\QQbar)$ is a Hecke module (Proposition~\ref{prop:G_modules_are_hecke}). The
Galois action of $J(\QQbar)$ has been extensively studied by
Mazur~\cite{mazur:eisenstein}, so we weaken our hypothesis to $M$ a
$G_\QQ$-submodule of $J(\QQbar)_\odd$. This is allowed because $A$ is
$\TT[G_\QQ]$-stable.

It suffices to prove $M$ is $\TT$-stable for each $G_\QQ$-composition factor $V$
of $M[\ell^\infty]$ for $\ell>2$. The irreducibility of $V$ implies that it is
$\ell$-torsion. Ribet~\cite[Proposition 6.1]{ribet:semistable_gal} shows that
$\TT/\ell \TT$ is generated by $T_p$ for primes $p\nmid \ell N$. So we reduce
modulo $p$ for $p\nmid \ell N$, and use Eichler-Shimura to derived its
$\TT$-stability from its $G_\QQ$-stability.

\begin{proposition}\label{prop:G_modules_are_hecke}
    Suppose $M$ is a finite odd-order $G_\QQ$-submodule of $J_0(N)$, with $N$
    prime. Then $M$ is a $\TT[G_\QQ]$-module.
\end{proposition}
\begin{proof}
    It suffices to show $M$ is $\TT$-stable for each $\ell$-primary part. Let
    $\ell>2$ and assume $M\subseteq J[\ell^\infty]$. Let
    \[
        0 = M_0 \subsetneq \ldots \subsetneq M_n = M
    \]
    be an $G_\QQ$-composition series of $M$ with composition factors $X_i =
    M_i/M_{i-1}$. We proceed by induction on $n$ with the base
    case being the trivial $n=0$ case.

    Assume $M_{s-1}$ is an $\TT[G_\QQ]$-module. We will show $M_s$ is an
    $\TT[G_\QQ]$-module. Since $M_{s-1}$ is an $\TT[G_\QQ]$-module, for each
    $t\in \TT$, we have a well-defined map $t:X_s\to J(\QQbar)/M_{s-1}$. The
    goal is to show $t(X_s)\subseteq X_s$ for all $t\in \TT$.
    By~\cite[Proposition 2]{ribet:mult_p_finite}, $\TT/\ell \TT$ is generated
    by $T_p$ for $p\nmid \ell N$ so it suffices to show $T_p(X_s)\subseteq X_s$
    for prime $p\nmid \ell N$.

    Fix a prime $p\nmid \ell N$. Since $p$ does not divide $N$, $J$ has good
    reduction at $p$. Fix a place $\p$ over $p$. The reduction map yields an
    isomorphism~\cite[Theorem 1, Lemma 2]{serre-tate}
    \[
        \tau:J(\QQbar)[\ell^\infty] \riso J_{/\F_p} (\Fpbar)[\ell^\infty]
    \]
    sending $\Frob_\p$ to $F_p$, where $F_p$ is the absolute Frobenius on
    $J_{/\F_p}$. Under this isomorphism the natural $\TT$-action on $J(\QQ)$
    maps to the natural $\TT$-action on $J_{/\F_p}$~\cite[\S
    5.2]{ribet-stein:serre}. By Eicher-Shimura, $T_p = F+p/F\in
    \End(J_{/\F_p})$ so
    \[
        \tau(T_p X_s)
        = T_p\tau(X_s)
        = (F+p/F)\tau(X_s)
        = \tau((\Frob_\p+p/\Frob_\p)X_s)
        \subseteq \tau(X_s)
    \]
    hence, $T_p X_s\subseteq X_s$, as desired.
\end{proof}


\section{Non-Eisenstein modules are kernels of Hecke}%
\label{sec:non_eisenstein_modules_are_kernels_of_hecke}

In light of Section~\ref{sec:finite_odd_order_galois_modules_are_hecke}, all
finite odd-order $G_\QQ$-submodules of $J(\QQbar)$ are $\TT$-modules. We will
now use this fact freely and will often refer to the $\TT$-support of a finite
odd-order $G_\QQ$-submodule.

The goal now is to identify the finite odd-order $G_\QQ$-submodules of
$J(\QQbar)_\odd$ supported, as a $\TT$-module, on the non-Eisenstein primes by
their $\TT$-annihilators. Using Proposition~\ref{prop:G_modules_are_hecke}, we
can already do this in the irreducible case.

\begin{proposition}
    Let $\m$ be a non-Eisenstein prime of odd residue characteristic, if $M$ is
    a nonzero finite irreducible $G_\QQ$-submodule of $J(\QQbar)$ supported
    only on $\m$ as a $\TT$-module, then $M=J[\m]$.
\end{proposition}
\begin{proof}
    By Lemma~\ref{lem:cherry_street}, the $\TT$-annihilator of $M$ is $\m$.
    Hence, $M\subseteq J[\m]$ but $J[\m]$ is an irreducible
    $G_\QQ$-module~\cite[Proposition 14.2]{mazur:eisenstein} so $M=J[\m]$.
\end{proof}

The general case will follow from a mild adaptation of the work of David
Helm~\cite{helm:jacobian}. Helm considers the case of Jacobians, $J$, of
Shimura curves. One of the key inputs into Helm's proof is that for the maximal
ideals $\m$ in question is that if $T_\m J$ is the contravariant $\m$-adic Tate
module, then $T_\m J/\m T_\m J\cong J[\m]^\vee$ is dimension two over $\TT/\m$
and irreducible as a $G_\QQ$-module. By~\cite[Prop. 14.2]{mazur:eisenstein},
this is also the case for $J=J_0(N)$ with $N$ prime and $\m$ a non-Eisenstein
prime of odd residue characteristic.

\begin{theorem}[{\cite[Corollary 4.8]{helm:jacobian}}]%
    \label{thm:non_eisenstein_kernel_hecke}
    Let $M$ be a finite odd-order $G_\QQ$-module supported, as a $\TT$-module,
    only on the non-Eisenstein primes. If $I=\Ann_\TT(M)$, then $M=J[I]$.

    Moreover, since $A$ is both $\TT[G_\QQ]$-invariant, if $M$ is a finite
    odd-order $G_\QQ$-submodule of $A(\QQbar)$ supported, as a $\TT_A$-module,
    only on the non-Eisenstein primes. If $I=\Ann_{\TT_A}(M)$, then $M=A[I]$.
\end{theorem}
\begin{proof}
Since $M\subseteq J[I]$ and $\Supp_{\TT} M = \Supp_{\TT} J[I]$, it suffices to
prove $J[I]_\m = M_\m$ for each non-Eisenstein prime of odd residue
characteristic. So let $\m$ be a non-Eisenstein prime of odd residue
characteristic. We start by reviewing the contravariant Tate modules of $J$ and
proving Lemma~\ref{lem:finite_index}.

Let $T_\m J\isom \Hom(J[\m^\infty], \QQ_\ell/\ZZ_\ell)$ be the contravariant
Tate module at $\m$ and $\overline{\rho}_\m$ be the Galois representation
associated to $J[\m]^\vee$. Since $\m$ is an odd non-Eisenstein prime,
$\overline{\rho}_\m$ is an irreducible $G_\QQ$-representation of dimension 2
over $k_\m$ that is isomorphic to $J[\m]^\vee$~\cite[Prop.
14.2]{mazur:eisenstein}.

\begin{lemma}[{\cite[Lemma 4.6]{helm:jacobian}}]\label{lem:finite_index}
    Let $R$ be a $G_\QQ$-stable submodule of $T_\m J$ of finite index. Then
    $R=IT_\m J$ for some ideal $I$ of $\TT$.
\end{lemma}
\begin{proof}
    We proceed by induction on the maximal $G_\QQ$-composition series of $T_\m J/R$
    with the base case being the trivial length zero case. Let
    \[
        R = R_n \subsetneq R_{n-1} \subsetneq \cdots \subsetneq R_0 = T_\m J
    \]
    be a $G_\QQ$-composition series. By induction, $R_{n-1} = I'T_\m J$ for some
    $I'\subseteq \TT$.

    Consider $\m R_{n-1} + R$. This is a $G_\QQ$-module sitting between $R$ and
    $R_{n-1}$. By Nakayama's lemma, if $\m R_{n-1} R + R = R_{n-1}$, then
    $R=R_{n-1}$ which is a contradiction. Hence, $\m R_{n-1}+R=R$ so $R$
    contains $\m R_{n-1}$ and we can form the quotient.

    The module $R_{n-1}/\m R_{n-1}$ is $G_\QQ$-isomorphic to $(I'/\m
    I')\otimes_{\TT/\m} (T_\m J/\m T_\m J) \cong (I'/\m I')\otimes_{\TT/\m}
    J[\m]^\vee$, where $G_\QQ$ acts trivially on $I'/\m I'$. Let $V$ be the image
    of $R$ in $R_{n-1}/\m R_{n-1}$. Since $V$ is $G_\QQ$-invariant, and
    $J[\m]^\vee$ is irreducible, $V$ is given by $\hat{V}\otimes J[\m]^\vee$
    for some $\TT/\m$-subspace $\hat{V}$ of $I'/\m I'$. Let $I$ be the preimage
    of $\hat{V}$ in $I'$. Then $IT_m J = R$, since both contain $\m R_{n-1}$
    and map to $V$ modulo $\m R_{n-1}$.
\end{proof}


Let $B=J/M$ be the quotient abelian variety. Since $M$ is a $\TT$-module,
we may equipped with a $\TT$-action. So the projection $\phi:J \to B$ is an
$\TT[G_\QQ]$-isogeny with $\ker\phi = M$. For any odd non-Eisenstein prime
$\m$, $\phi$ induces the exact sequence
\[
    0 \to T_\m B \to T_\m J \to M^\vee _\m \to 0.
\]
In particular, the image of $T_\m B$ under $\phi$ is a finite index
$\TT[G_\QQ]$-submodule of $T_\m A$. By Lemma~\ref{lem:finite_index}, we can find
an ideal $I'$ of $\TT$ such that the image of $T_\m B$ is $I' _\m T_\m A$
for all odd non-Eisenstein primes $\m$. We have $M^\vee _\m = T_\m J / I'
T_\m J\cong J[I']_\m ^\vee$. Therefore, by taking annihilators of the dual,
we have that $I_\m = I'_\m$ and then by taking duals $M_\m = J[I]_\m$, as
desired.
\end{proof}

\section{Bounding non-Eisenstein part}%
\label{sec:bounding_non_eisenstein_part}

Recall that a prime of $\TT_A$ is non-Eisenstein if it does not divides the
Eisenstein ideal $\TT_A$. There are infinitely many non-Eisenstein primes. The
goal of this section is to the bound the images of isogenies of $A$ supported
on the non-Eisenstein by the class group of $\TT_A$ when $\TT_A$ is integrally
closed. In particular, we will prove a theorem of Frank Calegari.

Let $\H$ be a set of integral representatives for $\Cl(\TT_A)$ coprime to
$2\I_A$.

\begin{theorem}[Frank Calegari]
    \label{thm:frank}
    Let $A\subseteq J_0(N)$ be a simple abelian subvariety with $N$ prime.
    Suppose $\TT_A$ is integrally closed. Let $M$ be a $\TT_A[G_\QQ]$-submodule
    of $A(\QQbar)$ supported only on the non-Eisenstein primes of odd residue
    characteristic. Then there exists $\varphi\in \TT_A\otimes \QQ$ and $C\in
    H$, such that, there is an isomorphism
    \[
        \varphi:A/M\to A/A[C].
    \] 
    Moreover,
    \[
        v_\p(\varphi) =
        \begin{cases}
            0            & \text{if } \p\in \P_e, \\
            \geq v_\p(C) & \text{if } \p\in \P_{ne}.
        \end{cases}
    \]
\end{theorem}
\begin{proof}
    By Theorem~\ref{thm:non_eisenstein_kernel_hecke}, if $D=\Ann_{\TT_A} M$,
    then $M=A[D]$. We have that there exists $a,b\in \TT_A$ and $C\in H$, such
    that $aC=bD$. We will justify the following commutative diagram.
    \[
        \begin{tikzcd}
            A/A[bD]
            \arrow[r, "\sim", "b"']
            \ar[equal]{d}
            &
            A/A[D]
            \ar[d, "\sim" labl, "\phi"]
            \\
            A/A[a C]
            \arrow[r, "\sim", "a"']
            &
            A/A[C],
        \end{tikzcd}
    \]
    
    We will first establish the isomorphism $b:A/A[bD]\to A/A[D]$.
    \[
        \begin{tikzcd}
            0
            \arrow[r]
            &
            (A/A[D])[b]
            \arrow[hookrightarrow]{r}
            &
            A/A[D]
            \arrow[twoheadrightarrow]{r}{b}
            &
            A/A[D]
            \arrow[r]
            &
            0
        \end{tikzcd}.
    \]
    We have that $x\in (A/A[D])[b]$ if and only if $bx \in A[D]$ if and only if
    $x \in A[bD]$. Hence, $(A/A[D])[b]=A[bD]/A[D]$ so $b:A/A[bD]\xra{\sim}
    A[D]$. A similar argument applies for $a:A/A[aC]\to A/A[C]$.

    We have that
    \[
        v_\p(\varphi)=v_\p(a)-v_\p(b) = v_\p(D)-v_\p(C).
    \]
    If $\p\in P_e$, then $v_\p(C), v_\p(D)=0$ so
    \[
        v_\p(\varphi) =
        \begin{cases}
            0             & \text{if } \p\in \P_e \\
            \geq -v_\p(C) & \text{if } \p\in \P_{ne}.
        \end{cases}
    \] 
\end{proof}


\section{Bounding support and valuations}%
\label{sec:bounding_support_and_valuations}

Let $A\subseteq J_0(N)$ be a simple abelian subvariety with $N$ prime. Suppose
$\TT_A$ is integrally closed. Suppose $\psi:A\to A'$ be an odd-isogeny. Then
$M=\ker\psi$ is a $G_\QQ$-submodule of $A(\QQbar)_\odd$. By
Proposition~\ref{prop:G_modules_are_hecke}, $M$ is a $\TT_A[G_\QQ]$-module so
we can decompose $M$ as $M_{ne}\oplus M_e$, where $M_{ne}$ and $M_e$ are
$\TT_A[G_\QQ]$-submodules supported, as $\TT_A$-modules, only on $\P_{ne}$ and
$\P_e$, respectively. As we saw in Section~\ref{sec:eisenstein_isogenies},
$M_{ne}\sim A[C]$. The goal of this section is to show $M=M_{ne}\oplus M_e\sim
A[C]\oplus Y_e$, where $Y_e$ does not differ from $M_e$ too much.

\begin{proposition}
    Let $A\subseteq J_0(N)$ be a simple abelian subvariety with $N$ prime.
    Suppose $\TT_A$ is integrally closed. Suppose $\psi:A\to A'$ be an
    odd-isogeny. Let $M, M_{ne}, M_e$ be as above. Then
    \[
        A'\isom A/(A[C]\oplus Y_e),
    \]
    where $\Supp_{\TT_A} Y_e\subseteq \P_e$. Moreover, if $e_j$'s are exponents
    such that $M_e\subseteq \sum_{\p_j\in \P_e} A[\p_j ^{e_j}]$ then
    $Y_e\subseteq \sum_{\p_j \in \P_e} A[\p_j ^{e_j}]$.
\end{proposition}
\begin{proof}
    By Theorem~\ref{thm:frank}, there exists $\varphi\in \TT_A\otimes \QQ$ and
    $C\in \H$, such that $\varphi:A/M_{ne}\to A/A[C]$ is an isomorphism with
    \[
        v_\p(\varphi) =
        \begin{cases}
            0            & \text{if } \p\in \P_e, \\
            \geq v_\p(C) & \text{if } \p\in \P_{ne}.
        \end{cases}
    \]
    Let $\psi:A/M_{ne}\to A/(M_{ne}+M_e)$ be the isogeny corresponding to
    quotienting by $M_e$. Let $\psi':A/A[C]\to A(A[C]+\varphi(M_e))$. We have
    that $\varphi$ is also an isomorphism from $A/(M_{ne}+M_e)\to
    A(A[C]+\varphi(M_e))$. 
    \[
        \begin{tikzcd}
            A/M_{ne}
            \arrow[r, "\psi"]
            \arrow[d, "\varphi"]
            &
            A/(M_{ne}+M_e)
            \arrow[d, "\varphi'"]
            \\
            A/A[C]
            \arrow[r, "\psi'"]
            &
            A/(A[C]+\varphi(M_e)).
        \end{tikzcd}
    \] 
    Since $\varphi(M_e)$ is $\TT_A$-module, we can write it as $X_{ne}\oplus
    X_e$. Let $\p\in \P_{ne}$. Since $v_\p(\varphi)\geq v_\p(C)$, by
    Lemma~\ref{}, $X_{ne}\subseteq A[C]$ so $A[C]+\varphi(M_e) = A[C]\oplus
    X_e$. Now let $\p\in \P_e$. Since $v_\p(\varphi)=0$, by Lemma~\ref{}, we
    have that $Y_e\subseteq \sum_{\p_j \in \P_e} A[\p_j ^{e_j}]$, as desired.
\end{proof}

\begin{corollary}
    Let $M$ be a finite odd-order $G_\QQ$-submodule of $A(\QQbar)$. Then $M\sim
    X$ for some $\TT_A[G_\QQ]$-submodule of $A(\QQbar)$ supported on $\P_\H\cup
    \P_e$, where $\P_\H$ is the set of primes dividing $C$ for some $C\in \H$
    and $\P_e$ is the set of Eisenstein primes of odd residue characteristic.
\end{corollary}

\section{Eisenstein part}%
\label{sec:eisenstein_part}

The goal now is to be able enumerate the isogenies supported on the Eisenstein
ideals.

\begin{proposition}{{\cite[Prop. 4.5]{klosin-papikian:ribet}}}%
    \label{prop:eisenstein_cyclic}
    Let $A\subseteq J_0(N)$ be a simple abelian subvariety with $N$ prime. Let
    $\p$ be an Eisenstein prime of $\TT_A$ of residue characteristic $p>2$ with
    $p$ dividing $\mathrm{num}((N-1)/12)$ exactly. Suppose $\p$ is principally
    generated by $\alpha$. Suppose that $M\subseteq A[\p^\infty]$.
    If $A[\p]\not\subseteq M$, then $M\subseteq A[\p]$.
\end{proposition}
\begin{proof}
    We will assume $A[\p]\not\subseteq M$ and $M\not\subseteq A[\p]$, and
    derive a contradiction by showing $p^3\mid \#A(\F_\ell)$ for all rational
    primes $\ell \equiv 1\pmod{p^2N}$.

    By~\cite[Cor. 16.3]{mazur:eisenstein}, $A[\p]=C[p]\oplus \Sigma[p]$. Since
    $\Sigma[p]$ is of $\mu$-type, $\QQ(A[\p])=\QQ(\mu_p)$. Let
    $F=\QQ(\mu_{pN})$ and $K=\QQ(M)$. Then both $A[\p]$ and $M$ are constant
    over $KF$. Since $A[\p]\not\subseteq M$ and $M\not\subseteq A[\p]$, we have
    that $p^3\mid A(K)_\tor$. The goal now is to show that $KF\subseteq
    \QQ(\mu_{p^2 N})$.
    
    \begin{lemma}[{\cite[Lem. 4.6]{klosin-papikian:ribet}}]
        The number field $K=\QQ(M)$ is an abelian extension unramified away
        from $p, N$.
    \end{lemma}
    \begin{proof}
        Since $M$ is supported only on $\p$ as a $\TT_A$-module, we may view
        $M$ as a $(\TT_A)_\p$-module. Since $M$ is finite and $(\TT_A)_\p$ is a
        DVR\@, we have
        \[
            M\cong (\TT_A)_\p / \p^{s_1} \times\cdots \times (\TT_A)_\p /
            \p^{s_r}\cong (\TT_A)/\p^{s_1} \times \cdots \times
            (\TT_A)/\p^{s_r}
        \]
        for some $s_1,\ldots,s_r\geq 0$. Since $\dim_{\TT_A/\p} A[\p]=2$ and
        $M[\p]\cong (\TT_A / \p)^r \subsetneq A[\p]$ so $r=1$. Therefore,
        $M\cong \TT_A/\p^{s_1}$.

        Recall the elements of $\TT_A$ are defined over $\QQ$ so they commute
        with elements of $G_\QQ$, therefore,
        \[
            \Gal(K/\QQ)\subseteq \Aut_{\TT_A} (M) \cong \Aut_{\TT_A}
            (\TT_A/\p^{s_1}) \cong (\TT_A/\p^{s_1})^\times.
        \]
        Since $\TT_A$ is isomorphic to an order of a number field,
        $\Gal(K/\QQ)$ is abelian. Since $A$ has good reduction away from $N$,
        $K/\QQ$ is unramified away from $p, N$.
    \end{proof}

    \begin{lemma}
        The number field $K$ is a subfield of $\QQ(\mu_{p^2}, \mu_N)$.
    \end{lemma}
    \begin{proof}
        By assumption, $(\TT_A)_\p$ is a DVR\@, so, as a $(\TT_A)/\p$-space,
        $\p/\p^2$ is generated by some $\alpha\in \p$. This yields the exact
        sequence
        \[
            \begin{tikzcd}
                0 \arrow[r] &
                A[\p] \arrrow[r] &
                A[\p^2] \arrow[r, "\alpha"] &
                A[\p]
            \end{tikzcd}.
        \]
        By restricting to $M$, we have
        \begin{equation}
            \label{eq:M_Ap}
            \begin{tikzcd}
                0 \arrow[r] &
                M\cap A[\p] \arrow[r] &
                M \arrow[r] &
                M\cap A[\p] \arrow[r] &
                0
            \end{tikzcd} 
        \end{equation}

        Since $M\subsetneq A[\p]$, $M\cap A[\p]$ is either $C[p]$ or
        $\Sigma[p]$. As a $\ZZ$-module, $M\cong \ZZ/p^2$. Therefore,
        $\Gal(\QQbar/F)$ acts trivially on $pM=M\cap A[\p]$. So $\Gal(KF/F)$
        can be identified with the subgroup $\{a\in (\ZZ/p^2):ap\equiv p
        \pmod{p^2}\}\subseteq (\ZZ/p^2)$ of order $p$ so $[KF:F]=1$ or $p$.

        We have 
        \[
            \Gal(\QQ(\mu_p ^{n_1}, \mu_N ^{n_2})/F \cong \ZZ/p^{n_1-1} \times
            \ZZ/N^{n_2-1}.
        \]
        Since $[KF:F]=1$ or $p$, $KF\subseteq F(\mu_p ^{n_1})$ is a subfield of
        degree 1 or $p$. In either case, $K\subseteq KF \subseteq \QQ(\mu_p ^2,
        \mu_N)$.
    \end{proof}
\end{proof}

\begin{corollary}
    Suppose $A$ is a simple abelian subvariety of $J_0(N)$ with $N$ prime.
    Suppose $\TT_A$ is integrally closed and $\#\Cl(\TT_A)=1$. Let $n_\odd$ be
    the odd part of $\mathrm{numerator}((N-1)/12)$. Suppose that for each prime
    $p\mid n_\odd$, there exists a prime $\ell\equiv 1 \pmod{p^2N}$, such that,
    $p^3\nmid \#A(\F_\ell)$. Then if $\psi:A\to A'$ is an odd-degree isogeny
    with $\ker\psi$ supported only on the Eisenstein primes. Then
    \[
        A'\isom A/X,
    \]
    where $X=\Sum X_p$, where $X_p$ is either $C[p]$ or $\Sigma[p]$.
\end{corollary}
\begin{proof}
    We will first simplify $\psi$. Suppose that for some Eisenstein prime, $\p$
    of odd-residue characteristic, $A[\p]$ is non-trivial and $A[\p]\subseteq
    X$. The ideal $\p$ is principally generated by some $\alpha\in \TT_A$. The
    isogeny $\psi$ now factors as $\psi=\psi'\circ \alpha$ with $\Im \psi'=\Im
    \psi = A'$. We can now replace $\psi$ with $\psi'$. We can repeat this
    process, if needed, until we obtain an isogeny $\psi:A\to A'$ whose kernel
    does not contain $A[\p]\neq 0$ for any Eisenstein prime, $\p$, of odd-residue
    characteristic.

    Since $X$ is supported, as a $\TT_A$-module, on the elements of $\P_e$, we
    have
    \[
        X = \oplus_{\p\in \P_e} X\cap A[\p^\infty].
    \]
    And now by~\ref{TODO}, $X\cap A[\p^\infty]$ is either $C[p]$ or
    $\Sigma[p]$, as desired.
\end{proof}


\section{Combining Eisenstein and non-Eisenstein parts}%
\label{sec:combining_eisenstein_and_non_eisenstein_parts}

\begin{theorem}
    Suppose $A$ is a simple abelian subvariety of $J_0(N)$ with $N$ prime.
    Suppose $\TT_A$ is integrally closed and $\#\Cl(\TT_A)=1$. Let $n_\odd$ be
    the odd part of $\mathrm{numerator}((N-1)/12)$. Suppose that for each prime
    $p\mid n_\odd$, there exists a prime $\ell\equiv 1 \pmod{p^2N}$, such that,
    $p^3\nmid \#A(\F_\ell)$. Then if $\psi:A\to A'$ is an odd-degree isogeny
    \[
        A'\isom A/X
    \]
    where $X=X_C\oplus X_\Sigma$ where $X_C\subseteq A\cap C$ and $X_\Sigma
    \subseteq A\cap \Sigma$ are any subgroups.
\end{theorem}
\begin{proof}
    By Theorem~\ref{thm:frank}, $A'\isom A/Y$ with $X$ a $G_\QQ$-submodule of
    $A(\QQbar)$ supported only on the non-Eisenstein primes of odd-residue
    characteristic. And now by Proposition~\ref{}, $Y\sim X$
\end{proof}

\section{Computational Results}

Here are computational results.



\end{document}
