\chapter{Enumerating the Odd Isogenies Class of Prime Level Subvarieties}%
\label{chap:isogeny_class}

Let $N$ be a prime number so that $J_0(N)$ is non-trivial so $N=11$ or $N\geq
17$. Let $A$ be a simple abelian subvariety of $J_0(N)$. The goal of this
chapter is to, under certain conditions, enumerate the $\QQ$-isomorphism
classes of abelian varieties isogenous to $A$ by an odd-degree $\QQ$-isogeny.
We will call this the \emph{odd-degree isogeny class} of $A$.

More precisely, let $\TT$ be the Hecke algebra of $J_0(N)$ and $\TT_A$ be the
image of $\TT$ in $\End(A)$. There exists a newform $f=\sum a_n q^n$ associated
$A$. By~\cite[Prop. 7.14]{shimura:intro}, $\TT_A$ is isomorphic to an order of
the Hecke eigenvalue field - $\QQ(\ldots,a_n,\ldots)$. The goal of this chapter
is the enumerate the odd-degree isogeny class of $A$ when $\TT_A$ is integrally
closed.

The image of an isogeny is determined by its kernel up to isomorphism. So we
define an equivalence relation on the set of finite odd-order
$G_\QQ$-submodules of $A(\QQbar)$ given by $M_1\sim M_2$ if and only if
$A/M_1\isom A/M_2$. For any odd-order $G_\QQ$-submodule $X$ of $A(\QQbar)$, let
$\M(X)$ be the set of equivalence classes of finite odd-order
$G_\QQ$-submodules of $A(\QQbar)$ with a representative that is a submodule of
$X$. Let $A(\QQbar)_\odd$ be the odd-torsion part of $A(\QQbar)$. The goal of
this chapter can then be restated as enumerating a set of representatives for
$\M(A(\QQbar)_\odd)$ when $\TT_A$ is integrally closed. From experimental data,
this condition appears to be mild. In particular, there are 450 simple abelian
varieties of prime level less than 1000 and for only 42 of these, the Hecke
algebra is not integrally closed.

% The idea will be as follows. Assume $A$ is a simple abelian subvariety of
% $J_0(N)$ with $\TT_A$ integral closed. Let $H=\{C_i\}$ be a set of odd-norm
% integral representatives for $\Cl(\TT_A)$. Let $D$ be the set of primes of
% $\TT_A$ dividing $C_i$ for some $C_i\in H$, $E$ be the set of primes dividing
% the Eisenstein ideal, and $S=D\cup E$.

We begin by showing every finite $G_\QQ$-submodule of $A(\QQbar)_\odd$ is a
$\TT_A$-module (Proposition~\ref{prop:G_modules_are_hecke}). This is useful
because the $\TT_A$-structure is more easily understood than the
$G_\QQ$-structure.\ For instance, we are now able to talk about the
$\TT_A$-support of finite odd-order $G_\QQ$-submodules. We then show that every
finite odd-order $G_\QQ$-submodule is equivalent to one support on $S$
(\ref{prop:bound_support}). Now we construct an ideal $Q$ of $\TT_A$ such that
$S$ is the set of primes dividing $Q$. So $\M(A[Q^\infty])=\M(A(\QQbar)_\odd$.
We will then give an method for determining an integer $k$ such that
$\M(A[Q^k])=\M(A[Q^\infty])=\M(A(\QQbar)_\odd)$
(Proposition~\ref{prop:bound_support}). Lastly, we give an algorithm for
enumerating the finite $G_\QQ$-submodules of $\M(A[Q^k])$. In summary, we have
the following algorithm.

\begin{algorithm}{Odd isogeny class}%
    \label{alg:odd_isogeny_class}
    Let $A$ be a simple abelian subvariety with $\TT_A$ integrally closed. This
    algorithm will enumerate the class of abelian varieties isomorphic
    isogenous to $A$ by an odd-degree isogeny.
    \begin{enumerate}
        \item{} [Class group representatives]
            Compute a set of odd integral representatives $H=\{C_i\}$ of
            $\Cl(\TT_A)$. Let $Q = (\prod_{C_i\in H} C_i)(\prod_{\p\in E} \p)$,
            where $E$ is the set of primes dividing the Eisenstein ideal.
        \item{} [Initialize search]
            Set $r=1$ and $X_{r-1}=\emptyset$.
        \item{} [$G_\QQ$-submodules of $A[Q^r]$]
            Use Algorithm~\ref{alg:enumerating_AX} to give a set, $X_r$, of
            representatives for $\M(A[Q^r])$.
        \item{} [Done?]
            By Corollary~\ref{prop:stop_looking}, if for all $x\in X_r$, $x\sim
            y$ for some $y\in X_{r-1}$, then $X_r$ is a set of representatives
            for $\M(A(\QQbar)_\odd)$. If this is not the case, increment $r$
            and repeat the last step.
        \item{} [Quotient and output]
            Output $A/M$ for $M\in X_r$.
    \end{enumerate}
\end{algorithm}

\section{Finite odd-order Galois Modules are Hecke}%
\label{sec:finite_odd_order_galois_modules_are_hecke}

The goal of this subsection is to prove every finite odd $G_\QQ$-submodule $M$
of $A(\QQbar)$ is a Hecke module (Proposition~\ref{prop:G_modules_are_hecke}). The
Galois action of $J(\QQbar)$ has been extensively studied by
Mazur~\cite{mazur:eisenstein}, so we weaken our hypothesis to $M$ a
$G_\QQ$-submodule of $J(\QQbar)_\odd$.

It suffices to prove $M$ is $\TT$-stable for each $G_\QQ$-composition factor $V$
of $M[\ell^\infty]$ for $\ell>2$. The irreducibility of $V$ implies that it is
$\ell$-torsion. Ribet~\cite[Proposition 6.1]{ribet:semistable_gal} shows that
$\TT/\ell \TT$ is generated by $T_p$ for primes $p\nmid \ell N$. So we reduce
modulo $p$ for $p\nmid \ell N$, and use Eichler-Shimura to derived its
$\TT$-stability from its $G_\QQ$-stability.

\begin{proposition}\label{prop:G_modules_are_hecke}
    Suppose $M$ is a finite odd-order $G_\QQ$-submodule of $J_0(N)$, with $N$
    prime. Then $M$ is a $\TT[G_\QQ]$-module.
\end{proposition}
\begin{proof}
    It suffices to show $M$ is $\TT$-stable for each $\ell$-primary part. Let
    $\ell>2$ and assume $M\subseteq J[\ell^\infty]$. Let
    \[
        0 = M_0 \subsetneq \ldots \subsetneq M_n = M
    \]
    be an $G_\QQ$-composition series of $M$ with composition factors $X_i =
    M_i/M_{i-1}$. We proceed by induction on $n$ with the base
    case being the trivial $n=0$ case.

    Assume $M_{s-1}$ is an $\TT[G_\QQ]$-module. We will show $M_s$ is an
    $\TT[G_\QQ]$-module. Since $M_{s-1}$ is an $\TT[G_\QQ]$-module, for each
    $t\in \TT$, we have a well-defined map $t:X_s\to J(\QQbar)/M_{s-1}$. The
    goal is to show $t(X_s)\subseteq X_s$ for all $t\in \TT$.
    By~\cite[Proposition 2]{ribet:mult_p_finite}, $\TT/\ell \TT$ is generated
    by $T_p$ for $p\nmid \ell N$ so it suffices to show $T_p(X_s)\subseteq X_s$
    for prime $p\nmid \ell N$.

    Fix a prime $p\nmid \ell N$. So $J$ has good reduction at $p$. Fix a place
    $\p$ over $p$. The reduction map
    yields an isomorphism~\cite[Theorem 1, Lemma 2]{serre-tate}
    \[
        \tau:J(\QQbar)[\ell^\infty] \riso J_{/\F_p} (\Fpbar)[\ell^\infty]
    \]
    sending $\Frob_\p$ to $F_p$, where $F_p$ is the absolute Frobenius on
    $J_{/\F_p}$. Under this isomorphism the natural $\TT$-action on $J(\QQ)$
    maps to the natural $\TT$-action on $J_{/\F_p}$~\cite[\S
    5.2]{ribet-stein:serre}. By Eicher-Shimura, $T_p = F+p/F\in
    \End(J_{/\F_p})$ so
    \[
        \tau(T_p X_s)
        = T_p\tau(X_s)
        = (F+p/F)\tau(X_s)
        = \tau((\Frob_\p+p/\Frob_\p)X_s)
        \subseteq \tau(X_s)
    \]
    hence, $T_p X_s\subseteq X_s$, as desired.
\end{proof}


\section{Non-Eisenstein modules are kernels of Hecke}%
\label{sec:non_eisenstein_modules_are_kernels_of_hecke}

In light of Section~\ref{sec:finite_odd_order_galois_modules_are_hecke}, all
odd-order $G_\QQ$-submodules of $J(\QQbar)$ are $\TT$-modules. We will now
use this fact freely and will often refer to the $\TT$-support of an odd
$G_\QQ$-submodule.

The goal now is to identify the odd-order $G_\QQ$-submodules of
$J(\QQbar)_\odd$ supported away from $E$ by their $\TT$-annihilators. Using
Proposition~\ref{prop:G_modules_are_hecke}, we can already do this in the
irreducible case.
\begin{corollary}
    Let $\m$ be a non-Eisenstein prime of odd residue characteristic, if $M$ is
    a nonzero finite irreducible $G_\QQ$-submodule of $J(\QQbar)$ supported
    only on $\m$, then $M=J[\m]$.
\end{corollary}
\begin{proof}
    By Lemma~\ref{lem:cherry_street}, the annihilator of $M$ is $\m$. Hence,
    $M\subseteq J[\m]$ but $J[\m]$ is an irreducible
    $G_\QQ$-module~\cite[Proposition 14.2]{mazur:eisenstein} so $M=J[\m]$.
\end{proof}

The general case will follow from the work of David Helm~\cite{helm:jacobian}.
Helm considers the case of Jacobians, $J$, of Shimura curves. One of the key
inputs into Helm's proof is that for the maximal ideals $\m$ in question is
that if $T_\m J$ is the contravariant $\m$-adic Tate module, then $T_\m J/\m
T_\m J\cong J[\m]^\vee$ is dimension two over $\TT/\m$ and irreducible as a
$G_\QQ$-module. By~\cite[Prop. 14.2]{mazur:eisenstein}, this is also the case
for $J=J_0(N)$ and $\m$ a non-Eisenstein prime of odd residue characteristic.

\begin{theorem}[{\cite[Corollary 4.8]{helm:jacobian}}]%
    \label{thm:non_eisenstein_kernel_hecke}
    Let $M$ be a finite $G_\QQ$-module supported only on the non-Eisenstein
    primes of $\TT$ of odd residue characteristic with $\TT$-annihilator $I$.
    Then $M=J[I]$.

    Moreover, since $A$ is both $G_\QQ$ and $\TT$ invariant, if $M$ is a finite
    $G_\QQ$-module supported only on the non-Eisenstein primes of odd residue
    characteristic of $\TT_A$ with $\TT_A$-annihilator $I$. Then $M=A[I]$.
\end{theorem}

We have that $\Supp_\TT(M)$ is the set of primes lying for $I$ and $M\subseteq
J[I]$ so suffices to prove $J[I]_\m = M_\m$ for each non-Eisenstein prime of
odd residue characteristic. So let $\m$ be a non-Eisenstein prime of residue
characteristic $\ell>2$. Let $T_\m J\isom \Hom(J[\m^\infty],
\QQ_\ell/\ZZ_\ell)$ be the contravariant Tate module at $\m$ and
$\overline{\rho}_\m$ be the Galois representation associated to $J[\m]^\vee$.
Since $\m$ is an odd non-Eisenstein prime, $\overline{\rho}_\m$ is an
irreducible $G_\QQ$-representation of dimension 2 over $k_\m$ that is
isomorphic to $J[\m]^\vee$~\cite[Prop. 14.2]{mazur:eisenstein}.

\begin{lemma}[{\cite[Lemma 4.6]{helm:jacobian}}]\label{lem:finite_index}
    Let $M$ be a $G_\QQ$-stable submodule of $T_\m J$ of finite index. Then
    $M=IT_\m J$ for some ideal $I$ of $\TT$.
\end{lemma}
\begin{proof}
    We proceed by induction on the maximal $G_\QQ$-composition series of $T_\m J/M$
    with the base case being the trivial length zero case. Let
    \[
        M = M_n \subsetneq M_{n-1} \subsetneq \cdots \subsetneq M_0 = T_\m J
    \]
    be a $G_\QQ$-composition series. By induction, $M_{n-1} = I'T_\m J$ for some
    $I'\subseteq \TT$.

    Consider $\m M_{n-1} + M$. This is a $G_\QQ$-module sitting between $M$ and
    $M_{n-1}$. By Nakayama's lemma, if $\m M_{n-1} M + M = M_{n-1}$, then
    $M=M_{n-1}$ which is a contradiction. Hence, $\m M_{n-1}+M=M$ so $M$
    contains $\m M_{n-1}$ and we can form the quotient.

    The module $M_{n-1}/\m M_{n-1}$ is $G_\QQ$-isomorphic to $(I'/\m
    I')\otimes_{\TT/\m} (T_\m J/\m T_\m J) \cong (I'/\m I')\otimes_{\TT/\m}
    J[\m]^\vee$, where $G_\QQ$ acts trivially on $I'/\m I'$. Let $V$ be the image
    of $M$ in $M_{n-1}/\m M_{n-1}$. Since $V$ is $G_\QQ$-invariant, and
    $J[\m]^\vee$ is irreducible, $V$ is given by $\hat{V}\otimes J[\m]^\vee$
    for some $\TT/\m$-subspace $\hat{V}$ of $I'/\m I'$. Let $I$ be the preimage
    of $\hat{V}$ in $I'$. Then $IT_m J = M$, since both contain $\m M_{n-1}$
    and map to $V$ modulo $\m M_{n-1}$.
\end{proof}

We now return to the proof of Theorem~\ref{thm:non_eisenstein_kernel_hecke}.

\begin{proof}[proof of Theorem~\ref{thm:non_eisenstein_kernel_hecke}]
    Let $B=J/M$ be the quotient abelian variety equipped with the induced
    $\TT$-action. So the projection $\phi:J \to B$ is an $\TT[G_\QQ]$-isogeny
    with $\ker\phi = M$. For any odd non-Eisenstein prime $\m$, $\phi$ induces
    the exact sequence
    \[
        0 \to T_\m B \to T_\m J \to M^\vee _\m \to 0.
    \]
    In particular, the image of $T_\m B$ under $\phi$ is a finite index
    $\TT[G_\QQ]$-submodule of $T_\m A$. By Lemma~\ref{lem:finite_index}, we can find
    an ideal $I'$ of $\TT$ such that the image of $T_\m B$ is $I' _\m T_\m A$
    for all odd non-Eisenstein primes $\m$. We have $M^\vee _\m = T_\m J / I'
    T_\m J\cong J[I']_\m ^\vee$. Therefore, by taking annihilators of the dual,
    we have that $I_\m = I'_\m$ and then by taking duals $M_\m = J[I]_\m$, as
    desired.
\end{proof}

\section{Enumerating representatives of $A[X]$}

Assume $\TT_A$ is integrally closed and let $X$ be an odd ideal of $\TT_A$. The
goal of this section is to give a set of representatives for $\M(A[X])$ by
first enumerating the $G_\QQ$-submodules of $A[X]$. We
have the decomposition $X=\bigoplus_{\p\in \Supp_{\TT_A} (X)} X[\p^{v_\p(X)}]$
so it suffices to enumerate the $G_\QQ$-submodules of $A[\p^s]$ for any $s\geq
1$. This will now follow from Mazur's study of the Galois action on torsion
points~\cite[\S 14]{mazur:eisenstein}.

\begin{proposition}[Mazur]\label{prop:all_G_subs}
    Let $\p$ be a prime of residue characteristic $\ell>2$.
    \begin{enumerate}
        \item
            If $\p$ is Eisenstein, then $A[\p]=C_A[\ell]\oplus \Sigma_A[\ell]$,
            where $C_A=C\cap A$ and $\Sigma_A=\Sigma \cap A$ with $C$ and
            $\Sigma$ the cuspidal and Shimura subgroups of $J$. The
            $G_\QQ$-submodules of $A[\p]$ are the direct sum of $\ZZ$-submodules of
            $C_A[\ell]$ and $\ZZ$-submodules of $\Sigma_A[\ell]$.
        \item
            if $\p$ is non-Eisenstein, $A[\p]$ is irreducible~\cite[Prop
            14.2]{mazur:eisenstein} as as a $G_\QQ$-module so the only
            $G_\QQ$-submodules are $0$ and $A[\p]$.
    \end{enumerate}
\end{proposition}
\begin{proof}
    The Eisenstein case is~\cite[Corollary 16.3]{mazur:eisenstein} along with
    the fact that $C[\ell]\isom \ZZ/\ell$ and $\Sigma[\ell]\isom \mu\ell$.

    The non-Eisenstein case is~\cite[Propositon 14.2]{mazur:eisenstein}.
\end{proof}

Let $a\in \p^{s-1}\setminus \p^s$ so $a$ generates $\p^{s-1}/\p^s$ as a
$k_\p$-space. There exists a $\TT_A[G_\QQ]$-injection given by
$\phi_s:A[\p^s]/A[\p^{s-1}]\to A[\p]$. The $G_\QQ$-submodules, $M$, of
$A[\p^s]$ are then the $\ZZ$-submodules of $A[\p^s]$ such that $M\cap
A[\p^{s-1}]$ and $\phi_s(M)$ are $G_\QQ$-submodules. This yields the following
algorithm.

\begin{algorithm}{Enumerating $G_\QQ$-submodules of {$A[X]$}}%
    \label{alg:enumerating_AX}
    Given an odd ideal $X$ of $\TT_A$. This algorithm will output a set of
    representatives for $\M(A[X])$.
    \begin{enumerate}
        \item{} [Factor $X$]
            Compute the factorization $X=\prod \p_i ^{e_i}$.
        \item{} [Non-Eisenstein part]
            For each non-Eisenstein $\p_i$, let $W_i=\{A[\p^j]: 0 \leq j
            \leq e_i\}$.
        \item{} [Eisenstein part]
            For each Eisenstein $\p_i$, set $V_1$ to be the set of
            $G_\QQ$-submodules of $A[\p]$. For $j=2,\ldots,e_i$, set $V_j$ to
            be the set of $\ZZ$-submodules, $M$, of $A[\p^j]$ such that $M\cap
            A[\p^{j-1}]\in V_{j-1}$ and $\phi(M)\in V_1$.
        \item{} [Combine]
            Combine the $W_i$'s to form $W=\{\sum_i M_i: M_i \in W_i\}$.
        \item{} [Output representatives]
            Use isomorphism testing 
            %TODO
            % (Algorithm~\ref{alg:isom_testing})
            to produce a set of representatives of $\M(A[X])$ of elements of
            $W$.
    \end{enumerate}
\end{algorithm}

\section{Bounding support and valuations}%
\label{sec:bounding_support_and_valuations}

The goal of this section is to give an ideal $Q$ such that
$\M(A[Q^\infty])=\M(A(\QQbar)_\odd)$ (Proposition~\ref{prop:bound_support}). In
light of Faltings' Isogeny Theorem, there exists $k$ such that
$\M(A[Q^k])=\M(A(\QQbar)_\odd)$. We give a method for finding this $k$ as
Proposition~\ref{prop:stop_looking}. We now have all the necessary ingredients
for Algorithm~\ref{alg:odd_isogeny_class}.

\begin{lemma}\label{lem:com_alg}
    Let $a,b\in \TT_A$, $\p$ such the localization $(\TT_A)_\p$ is a DVR, and
    $M$ an finite odd $G_\QQ$-submodule of $A(\QQbar)$ with $Z=\Ann_{\TT_A} M$.
    Then
    \begin{enumerate}
        \item
            If $v_\p(Z) = k$, then $M[\p^\infty]\subseteq A[\p^k]$.
        \item
            We have $b^{-1}(A[Z])=A[bZ]$. This holds locally at $\p$ as well so
            $b_\p ^{-1} A[Z]_\p = A[bZ]_\p$.
        \item
            Let $r=v_\p(b)-v_\p(a)+k$ and $\phi=a\circ b^{-1}$. If $r\leq
            0$, then $\phi(M)[\p^\infty]=0$ so $\phi(M)_\p=0$.

            If $r>0$, then
            $\phi(M)[\p^\infty]\subseteq A[\p^r]$. Here $b^{-1}$ is the
            preimage of $b$.
    \end{enumerate}
    We will take on the convention that for negative $r$, $A[\p^r]=0$. So we can
    write the last part as $\phi(M)[\p]\subseteq A[\p^r]$.
\end{lemma}
\begin{proof}
    \mbox{}
    \begin{enumerate}
        \item
            Suppose $x\in M[\p^\infty]$. Then $x\in A[\p^\infty]$. Moreover,
            $v_\p(\Ann_{\TT_A} x) \leq v_\p(Z) = k$ so $x\in A[\p^k]$.
        \item
            We have $x\in b^{-1}(A[Z])$ if and only if $bx \in A[Z]$ if and
            only if $x\in A[bZ]$.
        \item
            Suppose $r\leq 0$. Then $\phi_\p \in Z_\p$ so
            \[
                \phi(M)[p^\infty]\subseteq \phi(M[\p^\infty])
                \subseteq \phi(A[\p^k]) = 0.
            \]
            Suppose $r>0$. Then $x\in \phi(M)[\p^\infty] =
            \phi(M[\p^\infty])\subseteq (a\circ
            b^{-1})(A[\p^k])=A[\p^r]$.
    \end{enumerate}
\end{proof}


\begin{lemma}\label{lem:principal_gives_iso}
    Let $C, D$ be nonzero ideals of $\TT_A$, $a,b\in \TT_A$ be nonzero, and $aC
    = bD$. Then there exists an isomorphism $\phi:A/A[D]\to A/A[C]$ such that
    $\phi\circ b=a$.
\end{lemma}
\begin{proof}
    Our argument will now reference the following diagram.
    \[
        \begin{tikzcd}
            A/A[bD]
            \arrow[r, "\sim", "b"']
            \ar[equal]{d}
            &
            A/A[D]
            \ar[d, "\sim" labl, "\phi"]
            \\
            A/A[a C]
            \arrow[r, "\sim", "a"']
            &
            A/A[C]
        \end{tikzcd}
    \]
    We first establish the $b$ and $a$ isomorphism. We have the exact
    sequence
    \[
        \begin{tikzcd}
            0
            \arrow[r]
            &
            (A/A[D])[b]
            \arrow[hookrightarrow]{r}
            &
            A/A[D]
            \arrow[twoheadrightarrow]{r}{b}
            &
            A/A[D]
            \arrow[r]
            &
            0
        \end{tikzcd}.
    \]
    We have that $x\in (A/A[D])[b]$ if and only if $bx \in A[D]$ if and only if
    $x \in A[bD]$. Hence, $(A/A[D])[b]=A[bD]/A[D]$ so $b:A/A[bD]\xra{\sim}
    A[D]$. A similar argument applies for $a$.

    We can now chase the diagram to obtain an isomorphism $\phi:A/A[D]\riso
    A/A[C]$, as desired.
\end{proof}

Recall $\TT_A$ is assumed to be integrally closed. Let $H=\{C_i\}$ be a set of
odd-norm integral representatives for $\Cl(\TT_A)$. Let $D$ be the set of
primes of $\TT_A$ dividing $C_i$ for some $C_i\in H$, $E$ be the set of primes
dividing the Eisenstein ideal, and $S=D\cup E$.

\begin{proposition}%
    \label{prop:bound_support}
    Every finite odd-order $G_\QQ$-submodule is equivalent to one supported on
    $S$.
\end{proposition}
\begin{proof}
    Let $M$ be a finite odd $G_\QQ$-submodule. Let $M=M_{S'}\oplus M_S$ be the
    direct sum decomposition of $M$ with $\Supp M_X \subseteq X$. By
    Theorem~\ref{thm:non_eisenstein_kernel_hecke}, there exists an ideal
    $I$ of $\TT_A$ with $V(I)\subseteq S'$. Since $S'$ consists of only
    invertible primes, $I$ is also invertible. So there exists nonzero $a,b\in
    \TT_A$ such that $bI=aC$. By Lemma~\ref{lem:principal_gives_iso}, there
    exists an isomorphism $\phi:A/A[I]\to A/A[C]$ such that $\phi\circ b = a$.
    Let $\psi:A/A[I]\to A/(A[I]+M_S)$ be the isogeny given by quotienting by
    $M_S/A[I]$. Then $\phi$ induces an isomorphism $\phi':A/(A[I]+M_S)\to
    A/(A[C]+\phi(M_S))$.

    It is clear that $\Supp_{\TT_A}(A[C])\subseteq S$ so it remains to show
    that $\phi(M_S)_\p =0$ for all $\p \in S'$. Let $\p\in S'$, then
    $v_\p(b)-v_\p(a)+v_\p(\Ann(M_S))=v_\p(C)-v_\p(I)+v_\p(\Ann(M_S)) \leq 0$
    since $V(C),V(\Ann(M_S))\subseteq S$. By Lemma~\ref{lem:com_alg},
    $\phi(M_S)_\p = 0$. Therefore, $A[C]+\phi(M_S)$ is supported on $S$.
\end{proof}

From henceforth, we assume that $\TT_A$ is integrally closed. We can now choose
$H=\{C_i\}$ to be a set of integral representatives for $\Cl(\TT_A)$ of odd
norm. Let $Q=\lcm(\{C:C\in H\}\cup \{\p:\p\in S_1\})$. We have that $V(Q)=S$
so, in light of Proposition~\ref{prop:bound_support},
$\M(A[Q^\infty])=\M(A(\QQbar)_\odd)$.

\begin{proposition}%
    \label{prop:stop_looking}
    Suppose $\TT_A$ is integrally closed. Suppose $\M(A[Q^k])=\M(A[Q^{k+1}])$
    then $\M(A[Q^k])=\M(A[Q^\infty])$.

    In terms of isogenies, this claim states that if $G_\QQ$-submodules of
    $A[Q^{k+1}]$ produce no new odd-isogenies not found in $A[Q^k]$, then
    $G_\QQ$-submodules of $A[Q^k]$ give the entire odd-isogeny class.
\end{proposition}
\begin{proof}
    Every isogeny can be factored into a composition of irreducible isogenies.
    Let $A'=A/M$ be a abelian variety with $M$ a $G_\QQ$-submodule of $A[Q^k]$. We
    will show that the image of $A'$ under an irreducible isogeny is equivalent
    to $A/Y$ for some $Y\in A[Q^{k+1}]$. This suffices because when
    $A[Q^k]=A[Q^{k+1}]$, we obtain the full isogeny class.

    Let $M'\subseteq A(\QQbar)_\odd$ be an extension of $M$ by an irreducible
    $G_\QQ$-module $Y$. Let $\m$ be the annihilator of $Y$. If $\m\in S$, then
    $M\subseteq A[\m Q^k]\subseteq A[Q^{k+1}]$ and we are done. So assume
    $\m\notin S$ so $\m$ is an odd Eisenstein prime and $Y=A[\m]$
    (Theorem~\ref{thm:irreducible_G_sub}). Since $\m\notin \Supp_{\TT_A}(M)$,
    $M' = A[\m]\oplus M$. As in Proposition~\ref{prop:bound_support}, there is
    an isomorphism $\phi':A/(A[I]+M_S)\to A/(A[C]+\phi(M_S))$ with $\phi\circ b
    = a$. The goal is to show $A[C]+\phi(M)\subseteq A[Q^k]$. Since
    $A[C]\subseteq A[Q^{k+1}]$ so it suffices to show $\phi(M)\subseteq
    A[Q^{k+1}]$.

    As in Proposition~\ref{prop:bound_support}, $\Supp\phi(M)\subseteq S$. For all
    $\p\in S$ (remember that $\m \notin S$), $v_\p(b)-v_\p(a) =
    v_\p(C)-v_\p(\m)=v_\p(C)$ so by Lemma~\ref{lem:com_alg},
    $\phi(M)\subseteq A[Q^{k+1}]$, as desired.
\end{proof}

\section{Eisenstein Isogenies}%
\label{sec:eisenstein_isogenies}

The goal of this section is to prove the following proposition which limits the
possible Eisenstein isogenies so just the boring ones.
\begin{proposition}
    Let $A$ be a simple subvariety of $J=J_0(N)$ with $N$ prime with $\TT_A$
    integrally closed. Let $\p$ be an Eisenstein prime of odd residue
    characteristic $p$ such that $A[\p]\neq 0$. Suppose $A[\p]=J[\p]$ (This
    might always be true). Let $M\subseteq A[\p^s]$ be a $\TT_A[G_\QQ]$-module.
    If $M \not\subseteq A[\p]$, then $M\subseteq A[\p]$.
\end{proposition}
Before beginning the prime, we will need the following definition by Mazur.
\begin{definition}
    A prime $\ell$ is a \emph{good prime} relative to $(p, N)$ if 
    \begin{enumerate}
        \item
            $\ell$ is not a $p$-th power modulo $N$,
        \item
            $\ell-1\not\equiv 0 \pmod{p}$
    \end{enumerate}
\end{definition}
Mazur~\cite[Prop 16.1]{mazur:eisenstein} proves that if $\p$ is an Eisenstein
prime of odd residue characteristic $p$. Then the Eisenstein ideal, $\I \TT_\p$
is locally, at $\p$, generated by $p$ and $\eta_\ell = T_\ell - (\ell+1)$,
where $\ell$ is any good prime.

\begin{proof}
    We will assume $M\not\subseteq A[\p]$ and $M\not\subseteq A[\p]$ and reach
    a contradiction. Since $M[\p^2]\subseteq A[\p^2]$ also satisfies the
    hypothesis, it suffices to arrive at a contradiction after replacing $M$ by
    $M[\p^2]$. Let $K = \QQ(M)$. We will first show that $K$ is an abelian
    extension.
    \begin{lemma}
        Let $K=\QQ(M)$. Then $K$ is an abelian extension of $\QQ$.
    \end{lemma}
    \begin{proof}
        We first consider $M$ as a finite $(\TT_A)_\p$-module. By assumption,
        $(\TT_A)_\p$ is a DVR so
        \[
            M \cong \TT/\p^{s_1}\times \cdots\times \TT/\p^{s_r}.
        \]
        Since $\dim_{\TT/\p} A[\p] = 2$ and $M[\p]\cong (\TT/\m)^r\subseteq
        A[\p]$, we must have $r=1$ so $M\cong \TT/\p^s$ for some $s$.

        The action of $\TT_A$ and $G_\QQ$ commute, we have
        \[
            \Gal(K/\QQ)\subseteq \Aut_\TT(M)\cong \Aut(\TT/\p^s) \cong
            (\TT/\p^s)^\times
        \]
        which is commutative so $K$ is an abelian extension. Since $K$ is
        abelian, by Chebotarev Density, there exists a good prime $\ell$
        (recall this also implies $A$ has good reduction at $\ell$) that
        totally splits over $K$. Therefore,
        \[
            M \subseteq J(K)_\tor \subseteq J[\eta_\ell].
        \]
        But this implies $M_\p\subseteq J[\eta_\ell]_\p = C_p \oplus
        \Sigma_p\subseteq A[\p]$, where $C_p$ and $\Sigma_p$ are the
        $p$-primary parts of the cuspidal and Shimura subgroups. This yields
        a contradiction.
    \end{proof}
\end{proof}
