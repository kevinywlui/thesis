\chapter{Algorithms for modular abelian varieties}%
\label{chap:algorithms}

In this chapter, we will review algorithms on modular abelian varieties. For
the most part, the goal is to reduce the computation problems to linear algebra
on modular symbol spaces.

\section{Defining Data}%
\label{sec:defining_data}

In this section, we will describe how to explicitly represent modular abelian
varieties in a way suitable for linear algebraic computations. We represent
modular abelian varieties explicitly as follows. Let $A$ be a modular abelian
variety and $\varphi:A\to J$ a finite degree morphism. Let $B$ be the image of
$A$ in $J$. Then there is an isogeny $B$ to $A$ with kernel $G$ such that
$A\cong B/G$.
\begin{equation}
    \label{eq:defining_data}
    \begin{tikzcd}
        &
        &
        A \arrow[d]\arrow[dr,"\varphi"] &
        &
        &
        \\
        0 \arrow[r] &
        G \arrow[r] &
        B \arrow[r] &
        J \arrow[r] &
        0
    \end{tikzcd}
\end{equation}
So we can represent any modular abelian variety $J$ by giving $G\subseteq
B\subseteq J$ all defined over $\QQ$.

\begin{itemize}
    \item
        We will represent $J$ by giving a modular
        symbol basis for $H_1(J, \ZZ)$ and $H_1(J, \QQ) = H_1(J, \ZZ)\otimes
        \QQ$ (Section~\ref{sec:modular_symbols}).
    \item
        We will represent $B$ as an abelian subvariety of $J$ as follows. The
        inclusion of $B\subseteq J$ induces an inclusion of rational homology
        $V=H_1(B, \QQ)$ into $H_1(J, \QQ)$ and $B$ is determined by this
        inclusion. Therefore, we specify $B$ by a basis in reduced echelon form
        for the subspace $V$. Of course, not every subspace of $H_1(J, \QQ)$
        corresponds to an abelian subvariety. In
        Algorithm~\ref{sec:decomp_verify}, we give a method for determining
        exactly when a subspace of $H_1(J, \QQ)$ corresponds to the rational
        homology of an abelian subvariety.
    \item
        We will represent $G$ as a finite subgroup of $B$ as follows. Let
        $\Lambda=V\cap H_1(J, \ZZ)$. Then the torsion subgroup of $B$ is given
        by $B(\CC)_\tor=V/\Lambda$. Therefore, we represent $G$ as specifying a
        Hermite normal form basis for the lattice $L$ with $\Lambda \subseteq L
        \subseteq V$.
\end{itemize}

Therefore, we can represent any modular abelian variety $A$ with the triplet
$(L, V, J)$ with the properties $J=J_1(N)$ for some $N$, $V\subseteq H_1(J,
\QQ)$, $L$ a lattice containing $V\cap H_1(J, \ZZ)$. Since the $\QQ$-span of
$L$ is $V$, $V$ can be recover from $L$ so $A$ can also be specified by $(L,
J)$.

\section{Modular Symbols}
\label{sec:modular_symbols}

In this section, we give a brief overview of modular symbols. Modular symbols
give a finite presentation for the homology group $H_1(X_0(N), \ZZ)$.

\section{Natural Maps}%
\label{sec:alg_natural_maps}
The Hecke operators~\cite[\S 8.3]{stein:modform}, degeneracy maps~\cite[\S
8.6]{stein:modform}, and star involution~\cite[\S 8.5]{stein:modform} are
useful maps defined on modular symbols which, in turn, induce maps on the
Jacobian.

\subsection{Decomposition and Verification of an Abelian Subvariety}
\label{sec:decomp_verify}

As mentioned in Section~\ref{sec:defining_data}, not every subspace $V$ of
$H_1(J, \QQ)$ corresponds to an abelian subvariety, $B$, of $J$, meaning
$B=(V,J)$. In this section, we will give an algorithm for verifying when $V$
corresponds to an abelian subvariety. If $V$ does correspond to a subvariety
$B$ of $J$, this algorithm will also give a decomposition of $B$ into simple
subvarieties.


\begin{algorithm}{Decomposing and Verifying Abelian Subvarieties}
    \label{alg:decomp_and_verify_subvarieties}
    Let $J=J_H(N)$ and $V$ be a subspace of $H_1(J, \QQ)$. If $V$ corresponds
    to a subvariety $A$ of $J$, then this algorithm will return a decomposition
    into simples $X_i=(V_i, J)$ of $A$, otherwise, this algorithm will return
    `not a subvariety'. Let $\T=\{T_1,\ldots\}$ be the Hecke algebra of $J$.
    \begin{enumerate}
        \item{} [Decompose into potential Hecke eigenspaces]
            We first decompose $V$ into a direct sum $V=W_1\oplus \cdots \oplus
            W_r$ so that, for each $W_i$, there exists a newform $f_i$ of level
            dividing $N$, such that either $W_i$ either corresponds to an
            abelian subvariety isogenous to a power of $A_{f_i}$ or is not an
            abelian subvariety and we can return `not a subvariety'.
            \begin{enumerate}
                \item{} [Initialize]
                    Set $j=1$.
                \item{} [Decompose into eigenspaces]
                    Compute the simultaneous eigenspaces $U_{j,1},\ldots,U_{j,
                    n_j}$ of $\{T_1,\ldots,T_j\}$. If $\sum_{k=1} ^{n_j}
                    U_{j, k}\neq V$, return `not a subvariety'.
                \item{} [Compare with newforms]
                    Let $S_{j,k}$ be the set of newforms whose $\ell$th Fourier
                    coefficient agrees with the eigenvalue of $T_\ell$ on
                    $U_{j, k}$ for $l=1,\ldots,j$. If $S_{j, k}$ is empty for
                    any $k=1,\ldots,n_j$, return `not a subvariety'. If $S_{j,k}$ is not
                    singleton for any $k=1,\ldots, n_j$, increment $j$ and
                    return to (b). Otherwise, $S_{j,k}$ is singleton for all
                    $k$ so let $S_{j,k}=\{f_k\}$ and $W_k=U_{j,k}$.
            \end{enumerate}
        \item{} [Is isogenous to power?]
            We now verify that each $W_i$ is isogenous to a power of $A_{f_i}$
            and give its decomposition. The $W_i$'s are pairwise non-isogenous
            so we will do this individually for each $W_i$ and consider just a
            single pair $W$ and $A_f$ with $f$ being a newform of level $L$.
            Let $d_1,\ldots,d_r:J_H(L)\to J_H(N)$ be the full collection of
            degeneracy maps from $J_H(L)$ to $J_H(N)$.  Set $U=\{0\}$.
            \begin{enumerate}
                \item{} [Image of $\QQ$-combination of $d_j$'s?]
                    Choose any $v\in V\setminus U$. By
                    Proposition~\ref{prop:integral_degen}, if $W$ is an abelian
                    subvariety then there exists $q_1,\ldots,q_r\in \QQ$
                    such that $v\in \Im \left(\sum_{i=1} ^r q_i
                    \delta_i\right)$. If this is not the case, return
                    `NO'. Let $R = \Im \left(\sum_{i=1} ^r q_i
                    \delta_i\right)$. If $R\not\subset V$, return `not a
                    subvariety'.
                \item{} [Full space?]
                    Replace $U$ with $U+R$. If $V=U$, we are done. Otherwise,
                    return to (b).
            \end{enumerate}
    \end{enumerate}
\end{algorithm}

\section{Cuspidal subgroup and rational cuspidal subgroup}

In this section, we will review the computation of the cuspidal and rational
cuspidal subgroups. The Generalized Ogg Conjecture states that the rational
cuspidal subgroup of $J_0(N)$ is equal to $J_0(N)(\QQ)_\tor$. Consequently, the
computation of these subgroups will be a crucial input to bounding the rational
torsion subgroup (Section~\ref{sec:galois_cohomology_bounds}). 

The cusps of $X_H(N)$ are the equivalence classes of $\P^1(\QQ)$ under
$\Gamma_H(N)$. We will denote the elements by $[x/y]_{\Gamma_H(N)}$. The
\defn{cuspidal subgroup} $C$ of $J$ is the subgroup of degree-0 divisors
$(\alpha)-(\beta)$ where $\alpha, \beta$ are cusps on $J$. More generally, if
$A=(L, V, J)$ is a modular abelian variety that is the quotient of $B$ by $L$.
Then the cuspidal subgroup of $A$ is defined to be $(C\cap B)/L$, where $C$ is
the cuspidal subgroup of $J$.

The Galois structure of the cuspidal subgroup is well-understood~\cite[\S
1.3]{stevens:thesis}. The points of the cuspidal subgroup are
$\QQ(\mu_N)$-rational under the following Galois action. There is an abstract
group homomorphism $\Gal(\QQ(\mu_N)/\QQ)\cong \ZZ/N\ZZ)^*$, let
\[
    \sigma_d\in \Gal(\Q(\mu_N)/\Q):\mu_N\mapsto \mu_N ^d
\]
then
\[
    \sigma_d([x/y]_{\Gamma_1})=[x/d'y]_{\Gamma_1},
\]
where $dd'\equiv 1 \mod{N}$. This explicit description of the Galois action
\emph{rational cuspidal subgroup} $C(\QQ):=C^{\Gal(\QQ(\mu)/\QQ)}$.

\section{Shimura subgroup}

In this section, we will compute the Shimura subgroup. This subgroup will be
used extensively (combine with previous?)

Let $\Sigma(N)$ be the Shimura subgroup of $J_0(N)$ so $\Sigma(N)$ is the
kernel of the natural map $J_0(N)\to J_1(N)$. To compute $\Sigma(N)$, we will
use a theorem by Ribet. We will choose some odd prime $p$ coprime to $N$,
by~\cite[Prop. 1]{ribet:raising}, $\Sigma(N)=\ker(d_1-d_p)$, where
$d_1,d_p:J_0(N)\to J_0(pN)$ are the degeneracy maps corresponding to $1,p$.

\section{Simple abelian subvarieties as image of degeneracies}

In this section, we will an algorithm

\begin{algorithm}{Simple abelian subvarieties as image of degeneracies}%
    \label{alg:simple_degen}
    Given a simple abelian subvariety $A\subseteq J_0(N)$. This algorithm
    gives a newform $f$ and a linear combination of degeneracy maps
    $\phi=\sum n_i d_i:J_0(L)\to J_0(N)$ such that $\phi(A_f)=A$ with $L$ the
    level of $f$.
    \begin{enumerate}
        \item{} [Decompose]
            Use Algorithm~\ref{alg:decomp_and_verify_subvarieties} to obtain
            the decomposition $J_0(N)=\sum_f V_f$, where the sum runs over
            newforms $f$ of level dividing $N$ and $V_f$ is the sum of all
            abelian subvarieties of $J_0(N)$ isogenous to $A_f$.
        \item{} [Determine newform]
            Find the newform $f$ of level $L$ so that $A\subseteq V_f$.
        \item{} [Basis under degeneracy]
            Let $\{b_1,\ldots,b_r\}$ be a $\ZZ$-basis for $\Lambda_{A_f}$ and
            $\{d_1,\ldots,d_s\}$ be the set of degeneracy maps $d_j:J_0(L)\to
            J_0(N)$.
        \item{} [Solve system]
            Let $x$ be any nonzero element element of $\Lambda_A$. Find
            $c_{ij}\in \QQ$ such that $x=\sum c_{ij} d_j(b_i)$.
        \item{} [Output]
            Let $i', j'$ be such that $c_{i'j'}\neq 0$. Let
            $(n_1,\ldots,n_s)\in \ZZ^s$ be the vector obtained by
            clearing denominators from $(c_{i'1},\ldots,c_{i's})$.
    \end{enumerate}
\end{algorithm}


\begin{algorithm}{Weierstrass equation of 1-dimension abelian subvariety}%
    \label{alg:weierstrass}
    Given a simple 1-dimensional subvariety $E$ of $J_0(N)$. This algorithm
    returns the Weierstrass equation for $E$.
    \begin{enumerate}
        \item{}
            [Isogeny from optimal subvariety] Use
            Algorithm~\ref{alg:simple_degen} to find a newform $f$ of level
            $L$ and isogenies $\phi:A_f\to E$.
        \item{}
            [Weierstrass of optimal subvariety] The Cremona tables contain the
            Weierstrass equations for optimal subvarieties of $J_0(L)$ so in
            particular, for $A_f$.
        \item{}
            [Kernel] Compute the kernel $M$ of $\phi$ and use the complex
            exponential map to identify $M$ as a set of points, $M'$ on the
            Weierstrass equation for $A_f$.
        \item{}
            [Velu's formulas] Output the Weierstrass equation of $A_f/M'$ using
            Velu's formulas.
    \end{enumerate}
\end{algorithm}



% \section{Homomorphism spaces}
%TODO: take from modabvar?

\section{Isogeny and isomorphism testing}

In this section, we give an algorithm for determining when a pair of simple
modular abelian varieties are isomorphic.

\begin{algorithm}{Isomorphism testing}%
    \label{alg:isomorphism_testing}
    Given simple modular abelian varieties $A, B$, this algorithm determine if
    $A, B$ are isomorphic. If so, this algorithm will also return an
    isomorphism between $A, B$.
    \begin{enumerate}
        \item{}
            [Isogenous?] Use Algorithm~\ref{} to find newforms $f,g$ and
            isogenies $\phi_f:A_f\to A$ and $\phi_g:A_g\to B$. If $f\neq g$,
            return `not isomorphic'. Otherwise, let $\phi:B\to A=\phi_g ^\vee
            \circ \phi_f$.
        \item{}
            [Square degree?] The composition $\phi^\vee \circ \phi:B\to B$ is
            the multiplication by $d$ map where $d$ is the degree of $\phi$. If
            $d$ is not square, return `not isomorphic'.
        \item{}
            [Endomorphism algebra] Use Algorithm~\ref{} to find a number field
            $K$, an order $\O\subseteq K$, and an isomorphism $\tau:\End(A)\to
            \O$.
        \item{}
            [$\Hom(A, B)$] Use Algorithm~\ref{} to compute $\Hom(A,B)$.
        \item{}
            [Image under $\phi$] Compute the image $H_f$ of $\Hom(A, B)$ in
            $\End(A)$ by composing with $f$.
        \item{}
            [Norm equation] Find solutions $x_1,\ldots,x_r$, up to units in
            $\O$, to the norm equation $\Norm_{\O} (x) = \pm \sqrt{d}$. If
            there are no solutions, return `not isomorphic'.
        \item{}
            [Isomorphic?] For each solution $x_i$, determine if $x_i\in H_f$,
            if so, $x_i \circ f^{-1}$ is an isomorphism from $A\to B$. If
            $x_i\notin H_f$ for all $x_i$, then return `not isomorphic'.
    \end{enumerate}
\end{algorithm}
% \begin{proof}
%     prove this
% \end{proof}
