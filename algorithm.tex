\chapter{Algorithms for modular abelian varieties}%
\label{chap:algorithms}

In this chapter, we will review algorithms on modular abelian varieties. For
the most part, the goal is to reduce the computation problems to linear algebra
on modular symbol spaces.

\section{Defining Data}%
\label{sec:defining_data}

We begin by giving an explicit description of modular abelian varieties that is
suitable for linear algebraic computations. We represent modular abelian
varieties explicitly as follows. Let $A$ be a modular abelian variety and
$\varphi:A\to J$ a finite degree morphism. Let $B$ be the image of $A$ in $J$.
By dualizing, there is an isogeny $B$ to $A$ with kernel $G$ such that $A\isom
B/G$.
\begin{equation*}
    \label{eq:defining_data}
    \begin{tikzcd}
        0 \arrow[r]
            &
            G \arrow[r]
            &
            B \arrow[r]\arrow[rd]
            &
            A \arrow[d, "\varphi"]
            \arrow[r]
            &0
            \\
            &
            &
            &
            J
            &
    \end{tikzcd}
\end{equation*}
So we can represent any modular abelian variety $J$ by giving $G\subseteq
B\subseteq J$ all defined over $\QQ$.

\begin{itemize}
    \item
        We will represent $J$ by giving a modular
        symbol basis for $H_1(J, \ZZ)$ and $H_1(J, \QQ) = H_1(J, \ZZ)\otimes
        \QQ$ (Section~\ref{sec:modular_symbols}).
    \item
        We will represent $B$ as an abelian subvariety of $J$ as follows. The
        inclusion of $B\subseteq J$ induces an inclusion of rational homology
        $V=H_1(B, \QQ)$ into $H_1(J, \QQ)$ and $B$ is determined by this
        inclusion. Therefore, we specify $B$ by a basis in reduced echelon form
        for the subspace $V$. Of course, not every subspace of $H_1(J, \QQ)$
        corresponds to an abelian subvariety. In
        Algorithm~\ref{sec:decomp_verify}, we give a method for determining
        exactly when a subspace of $H_1(J, \QQ)$ corresponds to the rational
        homology of an abelian subvariety.
    \item
        We will represent $G$ as a finite subgroup of $B$ as follows. Let
        $\Lambda=V\cap H_1(J, \ZZ)$. Then the torsion subgroup of $B$ is given
        by $B(\CC)_\tor=V/\Lambda$. Therefore, we represent $G$ as specifying a
        Hermite normal form basis for the lattice $L$ with $\Lambda \subseteq L
        \subseteq V$.
\end{itemize}

Therefore, we can represent any modular abelian variety $A$ with the triplet
$(L, V, J)$, denoted $A\sim (L, V, J)$, with the properties $J=J_1(N)$
for some $N$, $V\subseteq H_1(J, \QQ)$, $L$ a lattice containing $V\cap H_1(J,
\ZZ)$. Since the $\QQ$-span of $L$ is $V$, $V$ can be recovered from $L$ so $A$
can also be specified by $(L, J)$, denote $A\sim (L, J)$.

In this Chapter, it'll be convenient to allow other Jacobian besides $J_1(N)$.
If $H$ is congruence subgroup with $\Gamma_1(N)\subseteq H\subseteq
\Gamma_0(N)$, we will denote $\Jac(X_H(N))$ by $J_H(N)$. For the purposes of
this thesis, we usually take $J_H(N)$ to mean $J_0(N)$ or $J_1(N)$.

\section{Modular Symbols}
\label{sec:modular_symbols}

The computational tools presented in this chapter will be built on top of the
theory of modular symbols. As mentioned in Section~\ref{sec:defining_data},
modular symbols give a finite presentation of $H_1(J, \ZZ)$. For instance, a
$\ZZ$-basis for $H_1(J_0(15), \ZZ)$ is given by the set of Manin symbols
$\{(1,8), (1,9)\}$. For this thesis, we will not need to delve into the
underlying theory of modular symbols and we will take it as a blackbox. We
instead refer the reader to~\cite[\S 3, \S 8, \S 9]{stein:modform}.

Similarly, the Hecke operators~\cite[\S 8.3]{stein:modform}, the degeneracy
maps~\cite[\S 8.6]{stein:modform}, and the star involution~\cite[\S
8.5]{stein:modform} can all be defined via modular symbols and thus induce maps
on modular Jacobians.

\section{Finite Subgroups}

The goal of this section is to establish some background for computing with
finite subgroups of modular abelian varieties. We begin by explaining how we
will present the data of a finite subgroup. We then go over some basic
arithmetic performed on finite subgroup. Lastly, we will discuss computing the
cuspidal and Shimura subgroup which will be used extensively in the upcoming
chapters.

\subsection{Defining Data}

Let $A=(V, L, J)$ be a modular abelian variety. A finite subgroup $G$ of $A$
can be specified by giving a defining lattice $\mathcal{L}$ such that
$\mathcal{L}/L = G$.

Given 2 finite subgroups $G_1=(\mathcal{L}_1, A)$ and $G_2=(\mathcal{L}_2, A)$,
a map $\varphi: G_1\to G_2$ can be given as a map on the defining lattices.

\subsection{Intersection of Finite Subgroups}
\label{sec:finitegroup_intersection}

Let $G=(\mathcal{L}_1, A)$ and $H=(\mathcal{L}_2, A)$ be finite subgroups of an
modular abelian variety $A=(L, V, J)$. Let $\mathcal{L}_1 ' = \mathcal{L}_1+L$
and $\mathcal{L}_2 ' = \mathcal{L}_2 + L$. Then the intersection $G\cap H$ is
the group $(\mathcal{L}, A)$, where $\mathcal{L}=\mathcal{L}_1\cap
\mathcal{L}_2 \cap V$.

\subsection{Sums of Finite Subgroups}

Let $G=(\mathcal{L}_1, A)$ and $H=(\mathcal{L}_2, A)$ be finite subgroups of
$A=(L, V, J)$. The sum is given by $G+H=(\mathcal{L}_1 + \mathcal{L}_2+L, A)$.

\subsection{Quotients of Finite Subgroups}

Let $G=(\mathcal{L}_1, A)$ and $H=(\mathcal{L}_2, A)$ be finite subgroups of
$A=(L, V, J)$ with $H\subseteq G$ so $\mathcal{L}_2\subseteq \mathcal{L}_1$.
The quotient is given by $G/H=(\mathcal{L}_1/\mathcal{L}_2, A/H)$, where the
computation of $A/H$ is given in
Section~\ref{sub:quotienting_by_finite_subgroup}.

\subsection{Cuspidal subgroup and rational cuspidal subgroup}%
\label{sub:cuspidal}

In this section, we will review the computation of the cuspidal and rational
cuspidal subgroups. 

% The Generalized Ogg Conjecture states that the rational
% cuspidal subgroup of $J_0(N)$ is equal to $J_0(N)(\QQ)_\tor$. Consequently, the
% computation of the rational cuspidal subgroup will be a crucial input to
% Chapter~\ref{chap:totally_split} where we attempt to verify the Generalized Ogg
% Conjecture for totally split $J_0(N)$. Moreover, along with the Shimura
% subgroup (Section~\ref{sec:shimura_subgroup}), when $N$ is prime, the cuspidal
% subgroup will be cyclic Galois subgroups of $J_0(N)(\QQbar)_\tor$ and will
% hence correspond to nontrivial isogenies. This idea will be applied in
% Chapter~\ref{chap:isogeny_class}, where we attempt to enumerate the odd-degree
% isogeny of simple subvarieties of $J_0(N)$ for $N$ prime.

The cusps of $X_H(N)$ are the equivalence classes of $\P^1(\QQ)$ under
$\Gamma_H(N)$. We will denote the elements by $[x/y]_{\Gamma_H(N)}$. The
\defn{cuspidal subgroup} $C_N$ of $J_H(N)$ is the subgroup of degree-0 divisors
$(\alpha)-(\beta)$ where $\alpha, \beta$ are cusps on $J$. More generally, if
$A=(L, V, J)$ is a modular abelian variety that is the quotient of $B$ by $L$.
Then the cuspidal subgroup of $A$ is defined to be $(C_N\cap B)/L$, where $C_N$ is
the cuspidal subgroup of $J_H(N)$.

The computational of $C_N$ is given in~\cite[\S 3.8]{stein:phd} and we will now
discuss the computation of the rational cuspidal subgroup. The Galois structure
of the cuspidal subgroup is well-understood~\cite[\S 1.3]{stevens:thesis}. The
points of the cuspidal subgroup are $\QQ(\mu_N)$-rational under the following
Galois action. There is an abstract group homomorphism
$\Gal(\QQ(\mu_N)/\QQ)\cong \ZZ/N\ZZ)^*$, let
\[
    \sigma_d\in \Gal(\Q(\mu_N)/\Q):\mu_N\mapsto \mu_N ^d
\]
then
\[
    \sigma_d([x/y]_{\Gamma_1})=[x/d'y]_{\Gamma_1},
\]
where $dd'\equiv 1 \mod{N}$. This explicit description of the Galois action
\emph{rational cuspidal subgroup} $C_N:=C_N ^{\Gal(\QQ(\mu)/\QQ)}$.

\subsection{Shimura subgroup}%
\label{sub:shimura}

In this section, we will explain how to compute the Shimura subgroup of
$J_0(N)$. The \defn{Shimura subgroup}, $\Sigma_N$, is the kernel of the natural
map $J_0(N)\to J_1(N)$. This will be useful in
Chapter~\ref{chap:totally_split}, where Moreover, the Shimura subgroup is a
Galois subgroup of $J_0(N)(\QQ)_\tor$. This fact will be used in
Chapter~\ref{chap:isogeny_class} where we enumerate the rational isogeny
classes.

Let $\Sigma_N$ be the Shimura subgroup of $J_0(N)$ so $\Sigma_N$ is the kernel
of the natural map $J_0(N)\to J_1(N)$. To compute $\Sigma_N$, we will use a
theorem by Ribet. We will choose some odd prime $p$ coprime to $N$,
by~\cite[Prop. 1]{ribet:raising}, $\Sigma_N=\ker(d_1-d_p)$, where $\delta_1
^*,\delta_p ^*:J_0(N)\to J_0(pN)$ are the degeneracy maps corresponding to
$1,p$. The computation of the kernel is given in~\ref{sub:kernel}.

\section{Morphisms}%
\label{sec:morphisms}

The goal of this section is to establish some background for computing with
morphisms between modular abelian varieties.

\subsection{Defining Data}%
\label{sub:defining_data}

Let $A=(L, V, J)$ and $B=(L', V', J')$ be modular abelian varieties and
$\varphi:A\to B$ a map of abelian varieties. Then $\varphi$ induces a map on rational
homology and is also completely determined by the map on rational homology. So
we will defined $\varphi:A\to B$ by giving $\varphi_V:V\to V'$. For brevity, we will
use $\varphi$ to denote the maps on defining lattices, rational homology, and
abelian varieties.

\subsection{Kernel}%
\label{sub:kernel}

Let $\varphi:A\to B$ be a morphism between modular abelian varieties $A=(L, V, J)$
and $B=(L', V', J')$. Let $V_K=\ker \varphi_V$ and $L_K=L\cap V_K$. Then the
kernel $K$ of $\varphi$ is the extension of the abelian variety $K^0 = (L_K, V_K,
J)$ by the finite component group $\varphi^{-1}(L')/L$.

\subsection{Image}%
\label{sub:image}

Let $\varphi:A\to B$ be a morphism between modular abelian varieties $A=(L, V, J)$
and $B=(L', V', J')$. The image of $\varphi$ in $B$ is the abelian subvariety
$\varphi(A)=(L'', \varphi(V), J')$, where $L''=\varphi(V)\cap L'$.


\subsection{Quotienting by finite subgroup}%
\label{sub:quotienting_by_finite_subgroup}

Let $A=(L, V, J)$ be a modular abelian variety with a finite subgroup
$G=\mathcal{L}, A)$. There is an isogeny $\vphi_G: A\to A'=(L+\mathcal{L}, V,
J)$ with kernel exactly $G$ given by the identity map on rational homology.


\section{Decomposition and Verification of Abelian Subvarieties}
\label{sec:decomp_verify}

\subsection{Decomposition of $J_H(N)$}%
\label{sub:decomposition_of_j_}

The decomposition of $J_H(N)$ is equivalent to the decomposition of the
cuspidal modular symbol space of $\Gamma_H(N)$. This is discussed in~\cite[\S
9]{stein:modform}.

\subsection{Simple abelian subvarieties}%
\label{sub:simple_abelian_subvarieties}

\begin{algorithm}{Simple abelian subvarieties as image of degeneracies}%
    \label{alg:simple_degen}
    Given a simple abelian subvariety $A$ of $J=J_H(N)$, this algorithm returns
    a newform $f$ of level $L$, and an isogeny $\varphi:A_f ^\vee\to A$, where
    $A_f ^\vee\subseteq J_H(L)$ is the optimal subvariety attached to $f$.
    \begin{enumerate}
        \item{} [Decompose $J$]
            Use Algorithm~\ref{alg:decomp_J} to obtain the decomposition
            $J=\sum_f V_f$, where the sum runs over newforms $f$ of level
            dividing $N$ and $V_f$ is the sum of all abelian subvarieties of
            $J_H(N)$ isogenous to $A_f$.
        \item{} [Determine newform]
            Find the newform $f$ of level $L$ so that $A\subseteq V_f$.
        \item{} [Basis under degeneracy]
            Let $\{b_1,\ldots,b_r\}$ be a $\ZZ$-basis for the defining lattice
            of $A_f^\vee$ and $\{\delta_1,\ldots,\delta_s\}$ be the set of degeneracy
            maps $\delta_j:J_H(L)\to J_H(N)$.
        \item{} [Solve system]
            Let $x$ be any nonzero element of the defining lattice of
            $A_f^\vee$. Find $c_{ij}\in \QQ$ such that $x=\sum c_{ij}
            d_j(b_i)$.
        \item{} [Clear denominators]
            Let $i', j'$ be such that $c_{i'j'}\neq 0$. Let
            $(n_1,\ldots,n_s)\in \ZZ^s$ be the vector obtained by
            clearing denominators from $(c_{i'1},\ldots,c_{i's})$.
        \item{} [Output]
            The desired isogeny is now given by $\varphi=\sum_{j=1} ^s n_j
            \delta_j|_{A_f ^\vee}$.
    \end{enumerate}
\end{algorithm}

Let $B$ be an abelian subvariety of $J$. The defining data of $B$ is given by
$(L, V, J)$, where $V=H_1(B,\QQ)$ and $L = H_1(B, \ZZ)$. Since $L = V\cap
H_1(J, \ZZ)$, the abelian subvariety $B$ is completely determined by the
subspace $V\subseteq H_1(J, \QQ)$, denoted $B\sim (V, J)$. As mentioned in
Section~\ref{sec:defining_data}, not every subspace $V$ of $H_1(J, \QQ)$
corresponds to an abelian subvariety. The following algorithm will determine
when $V$ does correspond to an abelian subvariety. When $V$ does correspond to
an abelian subvariety $B$, we will also give a decomposition of $B$ into simple
abelian subvarieties.

\begin{algorithm}{Decomposing and Verifying Abelian Subvarieties}
    \label{alg:decomp_and_verify_subvarieties}
    Let $J=J_H(N)$ and $V$ be a subspace of $H_1(J, \QQ)$. If $V$ corresponds
    to a subvariety $B$ of $J$, then this algorithm will return a decomposition
    into simple abelian subvarieties $X_i=(V_i, J)$ of $B$, otherwise, this
    algorithm will return `not a subvariety'.
    \begin{enumerate}
        \item{} [Decompose $J$]
            Use Algorithm~\ref{alg:decomp_J} to decompose $J$ as $J=\sum_f
            X_f$, where $X_f=(W_f, J)$ is the sum of all abelian subvarieties
            of $J$ isogenous to $A_f$, where $A_f$ is the optimal quotient
            attached to $f$.
        \item{} [Intersect with $V$]
            Set $V_f=V\cap W_f$. We have that $V$ corresponds to an abelian
            subvariety if and only if each $V_f$ corresponds to an abelian
            subvariety. So we will explain the rest of our algorithm for just a
            single $V_f$ with $f$ a newform of level $L$.
        \item{} [Build up to $V_f$]
            If $V_f$ does correspond to an abelian subvariety $B_f$, then $B_f$
            must decompose as a product of simple abelian varieties isogenous
            to $A_f$. Using Proposition~\ref{prop:integral_degen}, $V_f$
            corresponds to an abelian subvariety if and only if $V_f=\sum V_i$,
            where each $V_i$ is the image of an integral linear combination of
            $\delta_j:A_f ^\vee \to J_0(N)$.
            \begin{enumerate}
                \item{} [Initiate]
                    Set $S_0=0$ and $i=0$.
                \item{} [Add a $V_i$]
                    Pick some $x\in V_f\setminus S_i$. Use Algorithm~\ref{} to
                    determine if there exists $V_i$ such that $x\in V_i$, where
                    $V_i$ is the image of an integral linear combination of
                    $\delta_j:A_f ^\vee \to J_0(N)$. If not, then $V_f$ does
                    not correspond to an abelian subvariety and we are done.
                \item{} [Done?]
                    Set $S_{i+1} = S_i + V_i$. If $S_{i+1}=V_f$, we are done
                    and we output $X_i=(V_i, J)$. Otherwise, increment $i$ and
                    return to the last step.
            \end{enumerate}
    \end{enumerate}
\end{algorithm}

By applying the previous algorithm to 1-dimension abelian subvarieties, we can
recover the Weierstrass equation.

\begin{algorithm}{Weierstrass equation of 1-dimension abelian subvariety}%
    \label{alg:weierstrass}
    Given an elliptic curve subvariety $E$ of $J_0(N)$. This algorithm
    returns the Weierstrass equation for $E$.
    \begin{enumerate}
        \item{}
            [Isogeny from optimal subvariety] Use
            Algorithm~\ref{alg:simple_degen} to find a newform $f$ of level
            $L$ and an isogeny $\varphi:A_f ^\vee\to E$.
        \item{}
            [Weierstrass of optimal subvariety] The Cremona tables contain the
            Weierstrass equations for optimal subvarieties of $J_0(L)$ so in
            particular, for $A_f$.
        \item{}
            [Kernel] Compute the kernel $M$ of $\varphi$ and use the complex
            exponential map to identify $M$ as a set of points, $M'$ on the
            Weierstrass equation for $A_f ^\vee$.
        \item{}
            [Velu's formulas] Output the Weierstrass equation of $A_f ^\vee/M'$ using
            Vela's formulas.
    \end{enumerate}
\end{algorithm}

% TODO: abelian subvariety need not be Hecke stable...
% \begin{algorithm}{Decomposing and Verifying Abelian Subvarieties}
%     \label{alg:decomp_and_verify_subvarieties}
%     Let $J=J_H(N)$ and $V$ be a subspace of $H_1(J, \QQ)$. If $V$ corresponds
%     to a subvariety $B$ of $J$, then this algorithm will return a decomposition
%     into simple abelian subvarieties $X_i=(V_i, J)$ of $B$, otherwise, this
%     algorithm will return `not a subvariety'.
%     \begin{enumerate}
%         \item{} [$\ZZ$-basis for Hecke algebra]
%             Let $\TT$ be the Hecke algebra of $J$ and $s$ be the Sturm bound of
%             $\TT$. By~\cite[Appendix]{lario-schoof}, $\{T_1,T_2,\ldots, T_s\}$
%             spans $\TT$ as a $\ZZ$-module. So we can compute a $\ZZ$-basis
%             $\mathcal{B}$ of $\TT$.
%         \item{} [Decompose into potential Hecke eigenspaces]
%             We first decompose $V$ into a direct sum $V=W_1\oplus \cdots \oplus
%             W_r$ so that, for each $W_i$, there exists a newform $f_i$ of level
%             dividing $N$, such that either $W_i$ either corresponds to an
%             abelian subvariety isogenous to a power of $A_{f_i}$ or is not an
%             abelian subvariety and we can return `not a subvariety'.
%             \begin{enumerate}
%                 \item{} [Initialize]
%                     Set $j=1$.
%                 \item{} [Decompose into eigenspaces]
%                     Compute the simultaneous eigenspaces $U_{j,1},\ldots,U_{j,
%                     n_j}$ of $\{T_1,\ldots,T_j\}$. If $\sum_{k=1} ^{n_j}
%                     U_{j, k}\neq V$, return `not a subvariety'.
%                 \item{} [Compare with newforms]
%                     Let $S_{j,k}$ be the set of newforms whose $\ell$th Fourier
%                     coefficient agrees with the eigenvalue of $T_\ell$ on
%                     $U_{j, k}$ for $l=1,\ldots,j$. If $S_{j, k}$ is empty for
%                     any $k=1,\ldots,n_j$, return `not a subvariety'. If $S_{j,k}$ is not
%                     singleton for any $k=1,\ldots, n_j$, increment $j$ and
%                     return to (b). Otherwise, $S_{j,k}$ is singleton for all
%                     $k$ so let $S_{j,k}=\{f_k\}$ and $W_k=U_{j,k}$.
%             \end{enumerate}
%         \item{} [Is isogenous to power?]
%             We now verify that each $W_i$ is isogenous to a power of $A_{f_i}$
%             and give its decomposition. The $W_i$'so are pairwise non-isogenous
%             so we will do this individually for each $W_i$ and consider just a
%             single pair $W$ and $A_f$ with $f$ being a newform of level $L$.
%             Let $d_1,\ldots,d_r:J_H(L)\to J_H(N)$ be the full collection of
%             degeneracy maps from $J_H(L)$ to $J_H(N)$. Set $U=\{0\}$ and 
%             \begin{enumerate}
%                 \item{} [Image of $\QQ$-combination of $d_j$'s?]
%                     Choose any $v\in V\setminus U$. By
%                     Proposition~\ref{prop:integral_degen}, if $W$ is an abelian
%                     subvariety then there exists $q_1,\ldots,q_r\in \QQ$
%                     such that $v\in \Im \left(\sum_{i=1} ^r q_i
%                     \delta_i\right)$. If this is not the case, return
%                     `NO'. Let $R = \Im \left(\sum_{i=1} ^r q_i
%                     \delta_i\right)$. If $R\not\subset V$, return `not a
%                     subvariety'.
%                 \item{} [Full space?]
%                     Replace $U$ with $U+R$. If $V=U$, we are done. Otherwise,
%                     return to (B).
%             \end{enumerate}
%         \item{} [Output]
%     \end{enumerate}
% \end{algorithm}

% In the case of a elliptic curve subvariety $E$ of $J_0(N)$, we can also return
% the Weierstrass equation for $E$. This is 


\section{Homomorphism spaces}

Let $A, B$ be simple modular abelian varieties. In this section, we give
algorithms for computing with homomorphism spaces between $A, B$. This
algorithms will be crucial for determining when $A$ and $B$ are isomorphic. 

We begin by giving an algorithm for computing a $\ZZ$-basis for $\Hom(A,B)$.
By applying this algorithm to the case when $A=B$, we obtain an algorithm for
computing a $\ZZ$-basis for $\End(A)$. Of course, $\End(A)$ also has the
structure of a ring that is isomorphic to an order in a number field. So we
will also give an algorithm for determining an order $\O$ isomorphic to
$\End(A)$, as well as maps to and from $\O$.

\subsection{Extending by $\QQ$}%
\label{sub:extending_by_qq_}

Let $A, B$ be simple modular abelian varieties both of dimension $d$. If $A$ is
not isogenous $B$, then $\Hom(A, B)=0$ and we are done. If $A$ is isogenous to
$B$, then $\Hom(A,B)\otimes \QQ \cong \End(A)\otimes \QQ\cong K_f$, where $K_f$
is the Hecke eigenvalue field of a newform $f$ associated to the simple modular
abelian variety $A$ (~\cite{shimura}). The goal of this section is to prove
Proposition~\ref{prop:extend_QQ} which allows us to recover $\Hom(A,B)$ from
$\Hom(A,B)\otimes \QQ$.

After choosing a basis for $\Lambda_1 = H_1(A, \ZZ)$ and $H_1(B, \ZZ)$, 
\[
    \Hom(\Lambda_1, \Lambda_2) \cong (\ZZ^{(2d)})^2.
\]
\begin{proposition}%
    \label{prop:extend_QQ}
    Let $A, B$ be simple abelian varieties over $\QQ$, let $\Lambda_1=H_1(A,
    \ZZ)$, and let $\Lambda_2=H_1(B, \ZZ)$. Embed $\Hom(A, B)$ into
    $\Hom(\Lambda_1, \Lambda_2)$ by the action on homology. Then
    \[
        \Hom(A, B) = 
        (\Hom(A, B) \otimes \QQ)\cap \Hom(\Lambda_1, \Lambda_2)
    \]
    where the intersection takes place in $\Hom(A,B)\otimes \QQ$.
\end{proposition}

We first prove a lemma that will be used in the proof of
Proposition~\ref{prop:extend_QQ}.

\begin{lemma}%
    \label{lem:aut}
    If $x\in \CC$ is fixed by every element of $\Aut(\CC/\QQ)$, then $x\in
    \QQ$.
\end{lemma}
\begin{proof}
    Suppose $x$ is transcendental, then there is a field automorphism
    $\sigma:\QQbar(x)\to\QQbar(x)$ given by $x\mapsto x+1$. This automorphism
    extends to an automorphism of $\CC$ that does not fix $x$. Therefore, $x$
    must be algebraic and by standard Galois theory, $x\in \QQ$.
\end{proof}

\begin{proof}[Proof of Proposition~\ref{prop:extend_QQ}]
    An element of $\Hom(A, B)$ is certainly an element of $\Hom(A, B)\otimes
    \QQ$. Moreover, an element of $\Hom(A, B) \subseteq \Hom_\CC (A,B)$ is a
    complex linear map from $\Tan(A_\CC)\to \Tan(B_\CC)$ that sends $\Lambda_1$
    to $\Lambda_2$ so it must be in $\Hom(\Lambda_1,\Lambda_2)$. This
    establishes the forward inclusion.

    Conversely, suppose $\varphi\in (\Hom(A, B)\otimes \QQ)\cap \Hom(\Lambda_1,
    \Lambda_2)$. Then there exists a positive integer $n$ such that $n\varphi n
    \Hom(A, B)$. Hence, $n\varphi\in \Hom(A,B)\subseteq \Hom_\CC(A, B)$ is a
    complex linear map from $\Tan(A_\CC)$ to $\Tan(B_\CC)$. Hence,
    $\varphi=(1/n)n\varphi$ is also a complex linear map from $\Tan(A_\CC)$ to
    $\Tan(B_\CC)$. By assumption, $\varphi$ also maps $\Lambda_1$ to
    $\Lambda_2$ so $\varphi\in \Hom_\CC(A,B)$. It remains to show that
    $\varphi$ is defined over $\QQ$. Let $\sigma\in \Gal(\CC/\QQ)$. Since
    $[n]\varphi\in \Hom(A,B)$, $\sigma([n]\varphi)-[n]\varphi=0$. By
    rearranging, $[n](\sigma\varphi-\varphi)=0$. The image of
    $\sigma\varphi-\varphi$ is either infinite or 0 and the kernel of $[n]$ is
    finite so we must have $\sigma\varphi=\varphi$. By Lemma~\ref{lem:aut},
    $\varphi\in \Hom(A,B)$.
\end{proof}

\subsection{Endomorphism Algebra as Field}%
\label{sub:endomorphism_algebra_as_field}

\begin{algorithm}{Endomorphism Algebra as Field}
    Given a simple abelian variety $A$ over $\QQ$, this algorithm computes a
    number field $F$ and an isomorphism from $\End(A)\otimes \QQ$ to $F$.
    \begin{enumerate}
        \item{} [Isogeny to $A_f ^\vee$]
            Use Algorithm~\ref{simple_deg} to compute an optimal subvariety
            $A_f ^\vee$ and an isogeny $\varphi:A\to A_f ^\vee$. This isogeny
            induces an isomorphism $\End(A)\otimes\QQ$ to $\End(A_f)\otimes
            \QQ$.
        \item{} [Random endomorphism]
            Use Algorithm~\ref{todo} to compute a $\ZZ$-basis $B$ for the Hecke
            algebra, $\TT'$, of $A_f ^\vee$. We then generated a random
            element, $T$, of $\End(A_f)$ by taking a random rational linear
            combination of the elements of $B$.
        \item{} [Is random element primitive?]
            Let $g$ be the minimal polynomial of $T$. If $\dim A = \deg g$,
            then $g$ will be a primitive generator for $\End(A)\otimes \QQ$ as
            a field and we proceed to the next step. Otherwise, return to the
            last step.
        \item{} [Output]
            Let $F$ be a number field generated a root, $\alpha$, of $g$. Let
            $\Psi:\End(A)\otimes \QQ\to F$ be the unique map sending $T$ to
            $\alpha$. We then precompose $\Psi$ with the isomorphism
            $\End(A)\otimes \QQ \to \End(A_f)\otimes \QQ$ to obtain the
            desired isomorphism.
    \end{enumerate}
\end{algorithm}
%TODO: proof

\subsection{Homomorphism space}%
\label{sub:_hom_a_b_}

Let $A, B$ be isogenous simple abelian varieties. The goal of this section is
to compute $\Hom(A,B)$.

\begin{algorithm}[Compute $\Hom(A,B)$]
    Given simple modular abelian varieties $A, B$, this algorithm computes
    $\Hom(A,B)$.
    \begin{enumerate}
        \item{} [Isogenous?]
            Use Algorithm~\ref{alg:isogenous} to determine if $A$ is isogenous
            to $B$. If not, then $\Hom(A,B)$ is trivial and we are done. If so,
            let $\varphi:A\to B$ be an isogeny.
        \item{} [Compute $\End(A)$]
            Use Algorithm~\ref{alg:end_A} to compute the endomorphism ring of
            $A$.
        \item{} [Image of $\End(A)$]
            Compute the image, $H$, of $\End(A)$
        \item{} [Saturate]
            Compute the saturation of $H$ in $\Hom(L_1, L_2)$.
    \end{enumerate}
\end{algorithm}
\begin{proof}
    
\end{proof}



\section{Isogeny and isomorphism testing}

In this section, we give an algorithm for determining when a pair of simple
modular abelian varieties are isomorphic.

\begin{algorithm}{Isomorphism testing}%
    \label{alg:isomorphism_testing}
    Given simple modular abelian varieties $A, B$, this algorithm determine if
    $A, B$ are isomorphic. If so, this algorithm will also return an
    isomorphism between $A, B$.
    \begin{enumerate}
        \item{} [Isogenous?]
            Use Algorithm~\ref{alg:isogenous} to determine if $A$ is isogenous
            to $B$. If not, then $A$ and $B$ are not isomorphic and we are
            done. If so, let $\varphi:A\to B$ be an isogeny.
        \item{}
            [Square degree?] The composition $\varphi^\vee \circ \varphi:B\to B$ is
            the multiplication by $d$ map where $d$ is the degree of $\varphi$. If
            $d$ is not square, return `not isomorphic'.
        \item{}
            [Endomorphism algebra] Use Algorithm~\ref{} to find a number field
            $K$, an order $\O\subseteq K$, and an isomorphism $\tau:\End(A)\to
            \O$.
        \item{}
            [Homomorphism space] Use Algorithm~\ref{} to compute $\Hom(A,B)$.
        \item{}
            [Image under $\varphi$] Compute the image $H_f$ of $\Hom(A, B)$ in
            $\End(A)$ by composing with $f$.
        \item{}
            [Norm equation] Find solutions $x_1,\ldots,x_r$, up to units in
            $\O$, to the norm equation $\Norm_{\O} (x) = \pm \sqrt{d}$. If
            there are no solutions, return `not isomorphic'.
        \item{}
            [Isomorphic?] For each solution $x_i$, determine if $x_i\in H_f$,
            if so, $x_i \circ f^{-1}$ is an isomorphism from $A\to B$. If
            $x_i\notin H_f$ for all $x_i$, then return `not isomorphic'.
    \end{enumerate}
\end{algorithm}
