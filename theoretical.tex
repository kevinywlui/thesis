\chapter{Theoretical Preliminaries}%
\label{chap:prelim}

In this chapter, we will review some theoretical preliminaries needed for
future chapters. 

\section{Modular Abelian Varieties}

In Chapter~\ref{chap:isogeny_class}, we will discuss enumerating the isogeny
class of simple abelian subvarieties of $J_0(N)$. This leads us to define a
class of abelian variety containing the subquotients of $J_0(N)$.

\begin{definition}
    \label{defn:modabvar}
    Let $A$ be an abelian variety over $\QQ$. Then $A$ is \emph{modular} if
    there exists some $N$ and a finite degree morphism $\varphi:A\to J_1(N)$.

    Note that modularity is closed under isogenies, subvarieties, and products.
\end{definition}

The natural quotient of $X_1(N)\to X_0(N)$ induces a finite degree map
$J_0(N)\to J_1(N)$. The kernel of this map, $\Sigma_N$, is finite and is the
Shimura subgroup of $J_0(N)$. Therefore, all abelian varieties isogenous to a
simple subvariety of $J_0(N)$ are modular.

\section{Old subvariety and degeneracy maps}

The goal of this section is to prove Corollary~\ref{cor:elliptic_decomp} which
gives a direct sum decomposition of certain abelian subvarieties of $J_0(N)$.
This will be particularly useful for computing the rational torsion order of
$J_0(N)$ as rational torsion order is multiplicative on direct sums.

\subsection{Degeneracy Maps}%
\label{sub:degeneracy_maps}

Let $L\mid N$, $LM=N$, and $t_1,\ldots,t_r$ be the divisors of $M$ in
increasing order. Then, for each $t_i$, there are degeneracy maps relating the
modular curves, forms, and Jacobians of level $L$ and $N$. For ease of
exposition, we will present the $\Gamma_0(N)$ case. The same arguments
generalize easily to $\Gamma_1(N)$.

We give the more algebraic construction. Recall that the non-cuspidal
points of $X_0(N)$ correspond to elliptic curves with some $N$-torsion data.
The degeneracy map to $X_0(L)$ will forget some of this data. More precisely,
on the non-cuspidal points
\begin{equation}
    \label{eq:degen_moduli}
    \begin{split}
        \delta_t: Y_0(N)  \to Y_0(L) \\
        [ E, G_N ]       \mapsto [ E/G_t, G_L ' ],
    \end{split}
\end{equation}
where $G_N$ is a cyclic subgroup of $E$ of order $N$, $G_t$ is the unique
cyclic subgroup of $G_N$ of order $t$, and $G_L '\subseteq G_N/G_t$ is the
unique subgroup of order $L$. By~\cite[Ghap. 1, Prop. 6.8]{hartshorne},
$\delta_t$ extends uniquely to a map (which we give the same name). By Pic
and Alb functoriality, this induces the maps $\delta_t ^*:J_0(L)\to
J_0(N)$, $\delta_{t*} :J_0(N)\to J_0(L)$, respectively.

\subsection{Optimal Elliptic Curves}%
\label{sub:optimal_elliptic_curves}

Let $E$ be an elliptic curve of conductor $N$. By the Modularity Theorem for
elliptic curves~\cite{breuil-conrad-diamond-taylor}. There exists a surjection
$\varphi:J_0(N)\onto E$. In general, $\ker\varphi$ is not connected. There does
$\varphi':J_0(N)\onto E'$ with $E'$ isogenous to $E$ so that $\ker\varphi'$ is
connected. By multiplicity one, this $E'$ is unique in its isogeny class. This
leads to the definition of optimal quotients.

\begin{definition}%
    \label{defn:optimal_quotient}
    Let $J$ be the Jacobian of a curve. Then an abelian variety $A$ is an
    \emph{optimal quotient} of $J$, if there exists a surjective morphism $J
    \twoheadrightarrow A$ with connected kernel. Equivalently, $A$ is an
    optimal quotient of $J$ is it is the quotient of $J$ by an abelian
    subvariety.
\end{definition}

If $A$ is an optimal quotient of $J$, then $A^\vee$ can be embedded as a
subvariety of $J$. This particularly useful in the case where $J$ is totally
split because both elliptic curves and Jacobians are self-dual.

\begin{proposition}
    \label{prop:optimal}
    Let $A$ be an abelian variety and $J$ a modular Jacobian. Then $A$ is an
    optimal quotient of $J$ if and only if there exists an injection of
    $A^\vee$ into $J$.

    Suppose $A$ is self-dual. Then
    $A$ is an optimal quotient of $J$ if and only if it is an abelian
    subvariety of $J$.
\end{proposition}
\begin{proof}
    Suppose $A$ is an optimal quotient of $J$. Then by dualizing the sequence
    of abelian varieties~\cite[Proposition 2.4.2]{lange-birkenhake:complex}
    \[
        0 \to C \to J \to A \to 0 \qquad
        0\to A^\vee \to J^\vee \to C^\vee \to 0,
    \]
    there is an injection of $A^\vee\hookrightarrow J^\vee$.

    Conversely, if $A^\vee$ injects into $J$. Then by the Poincar\`{e} Reducibility
    Theorem, the quotient of $J$ by $A^\vee$ is again an abelian variety $C$.
    Then by dualizing the sequence of abelian varieties
    \[
        0 \to A^\vee \to J \to C \to 0 \qquad
        0 \to C^\vee \to J^\vee \to A \to 0,
    \]
    there is a surjection of $J^\vee\cong J\twoheadrightarrow A$ with connected
    kernel $C^\vee$.
\end{proof}

In particular, if $f$ is a newform of level $N$, then there is an optimal
quotient $A_f$ of $J_0(N)$ attached to $f$. By dualizing, we obtain an abelian
subvariety $A_f ^\vee\subseteq J_0(N)$. We call this the \emph{optimal abelian
subvariety} attached to $f$.
\begin{definition}
    \label{defn:optimal_subvariety}
    Let $f$ be a newform of level $N$ and $A_f$ be the optimal quotient of
    $J_0(N)$ attached to $f$. We call $A_f^\vee\subseteq J_0(N)$, the
    \emph{optimal abelian subvariety} attached to $f$.
\end{definition}

Using the machinery of modular symbols, Cremona has created large databases of
elliptic curves and their invariants, including the $J_0(N)$-optimal curve in
each isogeny class. In this database, the curves are labeled $NXT$, where $N$
is the conductor, $X$ is the isogeny class within that conductor, and $T$ is
the isomorphism class with that isogeny class. When $T=1$, then that curve is
the $J_0(N)$-optimal curve with its isogeny class. For example, the curve
$15a1$ is the $J_0(N)$-optimal curve within the isogeny class $15a$.

We will soon also be interested in $J_1(N)$-optimal elliptic curves. Though the
$J_0(N)$-optimal and $J_1(N)$-optimal curves often agree, this is not always
the case. We can use Algorithm~\ref{alg:weierstrass} to determine the
Weierstrass equation of the $J_1(N)$-optimal curve. Alternatively, we can use
Stevens Conjecture~\cite[Conjecture II]{stevens:param} which states that within
an isogeny class, the $J_1(N)$-optimal curve is the curve of minimal Faltings
height. This conjecture is still open but some progress has been made. Stein
and Watkins~\cite[\S 3]{stein-watkins:ns} have proved Stevens conjecture for
isogeny classes of prime conductor. Vatsal~\cite[Thm.
1.11]{vatsal:multiplicative} has proved Stevens conjectures for isogeny classes
containing an elliptic curve $E$ such that for some prime $\ell\geq 7$,
$E[\ell]$ is reducible and $E$ is ordinary at $\ell$.

\subsection{Old subvariety}%
\label{sub:old_subvariety}

Let $L$ be a proper divisor of $N$ and $t_1,\ldots,t_r$ be divisors of $N/L$ in
any order. For each divisor $t_i$, the degeneracy maps give a finite-degree
morphism $\delta_{t_i} ^*:J_0(L)\to J_0(N)$. We now gather all the degeneracy
maps coming from $J_0(N)$ to define
\[
    \Phi_L ^N = \prod_{i=1} ^r \delta_{t_i} ^* : J_0(L)^r \to J_0(N).
\]
The \emph{old subvariety} of $J_0(N)$ is $\sum_{L\mid N} \Im\Phi_L ^N$ and the
\emph{$N/L$-old subvariety} is $\Im\Phi_L ^N$.

There is a strong relationship between $\ker \Phi_L ^N$ and
\[
    (\Sigma_L)_0 ^r =\{(x_1,\ldots,x_r)\in \Sigma_L ^r: \sum x_i = 0\}.
\]
Because degeneracy maps agree on the Shimura subgroup~\cite[Theorem
4]{ling-oesterle:shimura}, $(\Sigma_L)_0 ^r \subseteq \ker\Phi_L ^N$. The
reverse equality was established by Ribet when $M$ a prime coprime to $L$ and
was generalized by Ling to:
\begin{theorem}[{\cite[Prop. 1]{ribet:raising}\cite[Thm. 3]{ling:shimura}}]
    \label{thm:ribet-ling}
    Let $L$ and $M$ be relatively prime integers with $M$ squarefree. Let
    \[
        \Phi_L ^N = \prod_{i=1} ^r d_{t_i} ^* : J_0(L)^r \to J_0(LM)
    \]
    be as defined above. Then
    \begin{enumerate}
        \item
            If $L$ is odd or $M$ is prime, then $(\Sigma_L)_0 ^r=\ker\Phi_L ^N$.
        \item
            If $L$ is even and $M$ is not a prime, then $[\ker\Phi_L ^N:
            (\Sigma_L)_0 ^r]$ is a power of 2.
    \end{enumerate}
\end{theorem}

However, when $\ker\Phi_L ^N = (\Sigma_L)_0 ^r$, there is an useful direct sum
decomposition of $\Im\Phi_L ^N$, especially when $J_0(L)$ is an elliptic curve.

\begin{corollary}
    \label{cor:elliptic_decomp}
    Suppose that $\ker\Phi_L ^N = (\Sigma_L)_0 ^r$ and $E=J_0(L)$ is an elliptic
    curve. Then there is a $\QQ$-isomorphism
    \[
        \Im\Phi_L ^N \cong E \times F^{r-1},
    \]
    where $F$ is the $J_1(L)$-optimal curve in the isogeny class of $E$.
\end{corollary}
\begin{proof}
    This follows from Proposition~\ref{prop:decomp} and
    Proposition~\ref{prop:optimal}.
\end{proof}

This decomposition is particularly useful because the $J_0(L)$-optimal curves
are identified in Cremona's table and the $J_1(L)$-optimal curves can be
identified using Steven's conjecture. For example, using this decomposition,
$J_0(22) = J_0(22)_\old \cong E\times F$, where $E$ is the $J_0(11)$-optimal
curve and $F$ is the $J_1(11)$-optimal curve (in this case, $E=J_0(11)$ and
$F=J_1(11)$). So computing the BSD invariants of $J_0(22)$ amounts to computing
the BSD invariants of the elliptic curves $E$ and $F$.

\begin{proposition}
    \label{prop:decomp}
    Suppose $\ker\Phi_L ^N=(\Sigma_L)_0 ^r$. Then there is a $\QQ$-isomorphism
    \[
        \Im\Phi_L ^N \cong J_0(L)\times \Im(J_0(L)\to J_1(L))^{r-1}.
    \]
\end{proposition}
\begin{proof}
    Let $e$ be the exponent of $\Sigma_L$. For $i=1,\ldots,r-1$, let $m_i$ be
    integers so that $m_i \equiv 1 \pmod{e}$. Define $D_i:J_0(L)\to J_0(N)$ by
    \[
        D_i =
        \begin{cases}
            \delta_i ^*                   & \text{if } i = 1\\
            \delta_i ^* - m_i \delta_{i-1} ^* & \text{if } 2\leq i \leq r.
        \end{cases}
    \]
    We first show $\Im\Phi_L ^N = \bigoplus_{i=1} ^r \Im D_i$.

    Define $\Phi_L ^N =\prod_{i=1} ^r D_i : J_0(L)^r \to J_0(N)$. We have that
    $\Phi_L ^N = \Phi_L ^N \circ T$, where
    \begin{align*}
        T:J_0(L)^r & \to J_0(L)^r \\
        (x_1,x_2,\ldots,x_r)&\mapsto (x_1-m_2x_2,x_2-m_3 x_3,\ldots,x_{r-1}-m_r
        x_r, x_r).
    \end{align*}
    The matrix associated to $T$ is consisted of $1$'s along the diagonal and
    $m_i$'s along the superdiagonal. The determinant is 1 so
    \[
        \sum_{i=1} ^r \Im D_i = \Im \Phi_L ^N.
    \]

    The goal is to now show that this sum is direct. Let $y_1,\ldots,y_r\in
    J_0(L)$. Suppose
    \begin{equation}
        \label{eq:phiLprime}
        \Phi_L ^N (y_1,\ldots,y_r) =D_1(y_1)+ \cdots +D_r(y_r)=0.
    \end{equation}
    Then that $T(y_1,\ldots,y_r)\in \ker\Phi_L ^N=(\Sigma_L)_0 ^r$.  This
    immediately implies that $y_r\in \Sigma_L$ and then, by repeated
    back-substitution, $y_i\in \Sigma_L$ for $i=1,\ldots,r$. Since $m_i \equiv
    1 \pmod{e}$,
    \[
        T(y_1,\ldots,y_r)=(y_1-y_2,y_2-y_3,\ldots,y_{n-1}-y_n,y_n).
    \]
    Since $T(y_1,\ldots,y_r)\in (\Sigma_L)_0 ^r$,
    \[
        (y_1-y_2)+(y_2-y_3)+\cdots+(y_{n-1}-y_n)+y_n =0.
    \]
    This implies that $y_1=0$ so $D_1(y_1)=0$. Moreover, for $i=2,\ldots,r$,
    \begin{equation*}
        D_i(y_i)
            = \delta_{i-1} ^*(-y_i) + \delta_i ^* (y_i)
            = \Phi_L ^N (0,\ldots,0,-y_i,y_i,0,\ldots,0)
            = 0,
    \end{equation*}
    where the last equality follows from the fact
    $(0,\ldots,0,-y_i,y_i,0,\ldots,0)\in (\Sigma_L)_0 ^r$. Therefore, the terms
    in~\eqref{eq:phiLprime} are trivial so $\sum \Im D_i$ is direct.

    It remains to show $D_1(J_0(L))\cong J_0(L)$ and $D_i(J_0(L))\cong
    \Im(J_0(L)\to J_1(L))$ for $i\geq 2$. Notice that $D_1(x)=\Phi_L
    ^N(x,0,\ldots,0)$ and $D_i(x)=\Phi_L ^N(\ldots,-m_ix,x,\ldots)$. Since
    $\ker\Phi_L ^N=(\Sigma_L)_0 ^r$, $\ker D_1=0$ and $\ker D_i=\Sigma_L$ for
    $i\geq 2$, as desired.
\end{proof}

\section{Subvarieties of $J_0(N)$}%
\label{sec:subvarieties_of_j_0_n_}

In this section, we will prove that every simple abelian subvariety of $J_0(N)$
is the image of degeneracy map and discuss some interesting questions arising
from this. This fact will be extensively used in the decomposition algorithms
of Chapter~\ref{chap:algorithms}

\begin{proposition}
    \label{prop:integral_degen}
    Let $A$ be a simple abelian subvariety of $J_0(N)$. There exists a divisor
    $L$ of $N$ and a newform $f$ of level $L$ such that $A_f \sim A$. Let
    $\delta_1 ^*,\ldots,\delta_r ^*$ be the full collection of degeneracy maps
    from $J_0(L)$ to $J_0(N)$. Then there exists integers $n_1,\ldots,n_r$ such
    that $S:=\sum n_i \delta_i ^*|_{A_f ^\vee}: A_f^\vee\to A$ is an isogeny
    from $A_f^\vee$ to $A$. Note that $S$ is defined over $\QQ$.
\end{proposition}
\begin{proof}
    Let $V_f=\sum_{i=1} ^r \delta_i(A_f ^\vee)$ and $\Phi:(A_f ^\vee)^r \to
    V_f$ be defined by $\Phi(x_1,\ldots,x_r) = \delta_1 ^*(x_1)+\cdots+\delta_r
    ^*(x_r)$. Let $K_f$ be the Fourier coefficient field of $f$. Since $A$ is
    an abelian subvariety of $V_f$, there exists $M\in \End(V)\otimes \QQ \cong
    M_{r\times r}(\End(A_f ^\vee)\otimes\QQ) = M_{r\times r} (K_f)$ such that
    $\Im M = A$. Let $i:A_f ^\vee\to (A_f ^\vee)^r$ be the inclusion map into
    the first coordinate. Then there exists $U\in \Aut((A_f ^\vee)^r)=\GL_r
    (K_f)$ such that,
    \[
        \begin{tikzcd}
            A_f ^\vee\arrow[r,"i"] &
            (A_f ^\vee)^r \arrow[r, dotted, "U"] &
            (A_f ^\vee) ^r \arrow[r, "D"] &
            V_f \arrow[r, "M"] &
            A,
        \end{tikzcd}
    \]
    the map $T:=M \circ \Phi \circ U\circ i:A_f ^\vee\to A\in \Hom_0(A_f, A)$
    is nonzero. Since degeneracy maps are $K_f$-linear, there exists
    coefficients $a_1,\ldots,a_r\in K_f$ such that $T = \sum a_i \delta_i ^*$.
    Now there exists $b\in \ZZ^*$ such that $T':=bT\in \Hom(A_f ^\vee, A)$ is
    nonzero and hence an isogeny. As complex tori, $A_f\isom \CC^n/\Lambda_{A_f
    ^\vee}$ and $A\isom \CC^n/\Lambda_A$. Since $T'(\Lambda_{A_f ^\vee})\subset
    \Lambda_A$, $T'=\sum q_i \delta_i ^*$ for $q_i\in \QQ$. Finally, there
    exists $w\in \ZZ^*$ such that $S:=wT'=\sum n_i \delta_i ^*$ with $n_i\in \ZZ$.
\end{proof}

\section{Connectedness of Hecke Algebra}%
\label{sec:connectedness_of_hecke_algebra}

Mazur~\cite[Prop. 10.6]{mazur:eisenstein} proves that the Hecke Algebra
$\TT$ for $J_0(N)$ with $N$ prime is connected. This was done by showing any
direct product decomposition of $J_0(N)$ with $N$ prime contradicts the
irreducibility of the $\theta$-divisor. This was surprising to the author
as~\ref{cor:elliptic_decomp} gives a direct sum decomposition of $J_0(22)$. In
this section, we dissect Mazur's proof and give a mild generalization.
Moreover, we will explain why Mazur's argument fails in the composite case.

In this section, let $V_A$ be the sum of all abelian subvarieties of $J_0(N)$
isogenous to $A$. Let $V_f=V_{A_f}$.

\subsection{General results for semistable Jacobians}

We will begin with some fairly general results. So assume $J$ is a semistable
Jacobian defined over $\QQ$ that is possibly not $J_0(N)$. By~\cite[Corollary
1.4]{ribet:endo}, isogenies, endomorphisms, and abelian subvarieties (a priori
defined over $\QQbar$) are defined over $\QQ$. In this section, we use this
fact freely and will make no reference to the field of definition.

\begin{theorem}
    \label{thm:theta_irred}
    Any Jacobian $J$ taken with its principal polarization cannot be decomposed
    into a nontrivial direct sum of principally polarized abelian varieties.
\end{theorem}
\begin{proof}
    Any such decomposition will give a decomposition of the $\Theta$-divisor
    attached to $J$ which contradicts the irreducibility of the
    $\Theta$-divisor~\cite[\S 4(a)]{kempf:riemann}.
\end{proof}

\begin{lemma}
    \label{lem:decomp_isogeny}
    Suppose $J$ decomposes nontrivially as the direct sum of abelian subvarieties
    $A\oplus B$. Then $A$ must share an isogenous factor with $B$.
\end{lemma}
\begin{proof}
    We proceed via contradiction. Suppose $A$ and $B$ share no isogenous
    factors. Let $\lambda:J\to \hat{J}$ be the principal polarization induced
    by its $\Theta$-divisor. Since $A$ shares no isogenous factors with $B$,
    $\lambda(A)=\hat{A}$ so $\lambda|_A$ is a polarization of $A$. Similarly,
    $\lambda|_B$ is a polarization of $B$. This now contradicts
    Theorem~\ref{thm:theta_irred}.
\end{proof}

\begin{lemma}
    \label{lem:faithful}
    Let $R$ be a ring acting faithfully on $J$. Let $S=\{A_1,\ldots,A_k\}$ be a
    set of representatives of the isogeny class of subvarieties of $J$. Suppose
    that for all $A\in S$, and every idempotent $r\in R$, either $rV_A=0$ or
    $rV_A=V_A$. Then $\Spec R$ is connected.
\end{lemma}
\begin{proof}
    Recall that $\Spec R$ is connected if and only if $R$ contains an
    idempotent $r$ different from $0$ or $1$. We proceed via by contradiction.
    Let $r\in R$ be an idempotent different from $0$ or $1$. Then $K = rK
    \oplus (1-r) K$ is a decomposition of $K$ into subvarieties. Moreover, this
    decomposition is nontrivial because $R$ acts faithfully.

    Let $S_1=\{A\in S:rV_A=V_A\}$ and $S_2=\{A\in S:(1-r)V_A=V_A\}$. Observe
    that $r$ and $1-r$ kill every element of $S_2$ and $S_1$, respectively, so
    $S=S_1\sqcup S_2$. We can rewrite the previous decomposition as
    \[
        J
        = rJ \oplus (1-r)J
        = \left(\sum_{A\in S_1} V_A \right)
        \oplus \left(\sum_{A\in S_2} V_A \right).
    \]
    The big summands share no isogenous factors which contradicts
    Lemma~\ref{lem:decomp_isogeny}.
\end{proof}

\subsection{Application to $J_0(N)$}

The goal now is to apply Lemma~\ref{lem:faithful} to the case of semistable
$J$ and subrings of the Hecke algebra $\TT$. Recall that the action of the
Hecke algebra on $J$ is faithful.

\begin{proposition}\label{prop:new_jacobian_connected}
    Suppose that $J=J_{\new}$. Then $\Spec\TT$ is connected.
\end{proposition}
\begin{proof}
    When all subvarieties are new, they appear with multiplicity 1 so the
    conditions of Lemma~\ref{lem:faithful} are automatic.
\end{proof}

\begin{proposition}
    Suppose $\TT'$ is the anemic Hecke algebra for $J$. Then $\Spec\TT'$ is
    connected.
\end{proposition}
\begin{proof}
    By Lemma~\ref{lem:faithful}, it suffices to show that for any newform $f$,
    and $r\in \TT'$, $rV_{A_f}=V_{A_f}$ or $rV_{A_f}=0$. Fix a newform $f$ of
    level $L$ and $r\in \TT'$. Let $s$ be the number of divisors of $N/L$. We
    now abuse notation by overloading $T_\ell$ and $r$ as operators on $A_f
    ^\vee$, $(A_f ^\vee)^s$, and $J$. Following the notation
    of~\ref{sub:old_subvariety}, let $\Phi_f = (\Phi_L ^N) |_{(A_f ^\vee)^s}:
    (A_f ^vee)^s \to J_0(N)$.

    In the Formulaire section of~\cite{ribet:old}, for $\ell\nmid N$,
    $T_\ell\circ \phi=\phi\circ T_\ell$ when $N$ is prime. However, this is
    also true with $N$ squarefree since $T_\ell$ commutes with the pushforward
    and pullback of any degeneracy map $\delta_d ^*$ with $d\mid N$. So for any
    prime $\ell\nmid N$,
    \[
        T_\ell(\Phi_f(x_1,\ldots,x_s))
        = \Phi_f(T_\ell(x_1),\ldots,T_\ell(x_s)).
    \]
    It follows that
    \[
        r(\Phi_f(x_1,\ldots,x_s))
        = \Phi_f(rx_1,\ldots,rx_s).
    \]
    Therefore,
    \[
        rV_f = r(\Phi_f(A_f ^\vee)) = \Phi_f(rA_f^\vee,\ldots,rA_f^\vee)
    \]
    but $A_f ^\vee$ is simple so either $rA_f ^\vee=A_f ^\vee$ or $rA_f
    ^\vee=0$. Therefore, by Lemma~\ref{lem:faithful}, $\Spec\TT'$ is connected.
\end{proof}

\begin{example}
    When $N$ is prime, the argument Mazur~\cite[Prop. 10.6]{mazur:eisenstein}
    gives is essentially the same as
    Proposition~\ref{prop:new_jacobian_connected}. We see that this fails for
    $J=J_0(22)$ since $J=d_1(J_0(11))+d_2(J_0(11))$ so the factors appear with
    multiplicity greater than 1. So one of the key difference is the
    multiplicities in which the simple factors appear.

    Using \sage, we can show that the Hecke algebra of $J_0(22)$ is isomorphic
    to $\ZZ[i]$ and is thus connected. On the other hand,
    by~\ref{cor:elliptic_decomp}, $J_0(22)$ is the direct product of elliptic
    curves so the endomorphism ring is not connected.
\end{example}


% \begin{proposition}
%     \label{prop:star_eisenstein}
%     Suppose $(L, N)$ satisfies the $\star$-condition. Then $\ker S\subseteq
%     \Sigma(L)$. In particular, this isogeny is Eisenstein.
% \end{proposition}
% \begin{proof}
%     Suppose $x\in \ker S$. Then $(n_1 x, \ldots, n_r x)\in \ker\Phi_L ^N =
%     \Sigma(L)_0 ^r$. Hence, $n_i x\in \Sigma(L)$ for all $i$. Since
%     $\gcd(n_1,\ldots,n_r)=1$, $x\in \Sigma(L)$.
% \end{proof}

% \begin{question}
%     \label{question:iso}
%     Let $A$ be a simple subvariety of $J_0(N)$ for some $N$. Let $\mathcal{I}$
%     be the isogeny class of $A$. For all $B\in \mathcal{I}$, is $B$ a
%     subvariety of $J_0(M)$ for some $M$?
% \end{question}

% \begin{example}
%     Suppose $I$ is a Neumann-Setzer isogeny class of elliptic curves of
%     conductor $p$. Then there are exactly 2 elliptic curves $E_0$ and $E_1$,
%     where $E_0$ is $J_0(p)$-optimal and $E_1$ is $J_1(p)$-optimal
%     (Stein-Watkins). Viewing $E_0$ as a subvariety of $J_0(p)$ and taking $q$
%     to be a prime distinct from $p$, we have
%     \[
%         E_1 \isom \Im (d_1-d_q)(E_0)\subseteq J_0(pq),
%     \]
%     where $d_1, d_q:J_0(p)\to J_0(pq)$ are the natural degeneracy maps. This
%     gives an affirmative answer to Question~\ref{question:iso} in the
%     Neumann-Setzer isogeny class case.
% \end{example}


% \begin{question}
%     Let $M$ be a $\Gal(\QQbar/\QQ)$-submodule of $J_0(L)$. Does there exists
%     $N$ and an integral linear combination of degeneracy maps $S:=\sum n_i d_i$
%     such that $\ker S=M$?
% \end{question}

% \begin{question}
%     Is $\ker S$ always Eisenstein? This is true under the $\star$-condition.
% \end{question}

% \subsection{Endomorphism Ring}

% Let $A$ be a simple abelian variety so $A\sim A_f$ for some newform $f=\sum
% a_n q^n$. Let $K_f=\QQ(\ldots,a_n,\ldots)$ be the Hecke eigenvalue field of
% $f$. By~\cite[Prop. 7.14]{shimura:intro}, the endomorphism algebra
% $\End(A_f)\otimes \QQ$ is isomorphic to $K_f$ and is generated as a
% $\QQ$-vector space by the Hecke operators. It follows that both the
% endomorphism ring and the Hecke algebra of $A_f$ are isomorphic to orders of
% $K_f$. Since $A\sim A_f$, the endomorphism ring of $A$ is also isomorphic to an
% order of $K_f$.
