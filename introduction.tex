\chapter{Introduction}%
\label{chap:intro}

Let $X_0(N)$ be the modular curve whose non-cuspidal points parameterize
complex elliptic curve with some additional $N$-torsion data. Let
$J_0(N)=\Jac(X_0(N))$ be the Jacobian variety of $X_0(N)$. The study of
$J_0(N)$ has yielded many deep results in arithmetic geometry. Two notable
examples are:
\begin{itemize}
    \item 
        Mazur~\cite{mazur:eisenstein} bounds the possible group structures for
        the rational torsion subgroup of an elliptic curve over $\QQ$ by
        understanding the rational torsion subgroup of $J_0(N)$.
    \item
        The Modularity Theorem for elliptic curves over $\QQ$ states that for
        every elliptic curve $E$, there is a surjective map $J_0(N)\to E$,
        where $N$ is the conductor of $E$. As a step towards proving Fermat's
        Last Theorem, Taylor and
        Wiles~\cite{wiles:fermat}\cite{taylor-wiles:fermat} prove the
        Modularity Theorem for semistable stable elliptic curves. The full
        Modularity Theorem was later established following the completion
        of~\cite{breuil-conrad-diamond-taylor}.
\end{itemize}

The goal of this thesis is to study the computational aspects of $J_0(N)$. The
most interesting work will be presented in Chapter~\ref{chap:totally_split} and
Chapter~\ref{chap:isogeny_class}. But before arriving at these chapters, we
will lay down some theoretical and algorithmic preliminaries in
Chapter~\ref{chap:prelim} and Chapter~\ref{chap:algorithms}.

\section{Rational torsion points}%
\label{sec:rational_torsion_points}

As part of his landmark paper bounding the possible group structures for the
rational torsion subgroup of an elliptic curve. As a key step, Mazur proves the
Ogg Conjecture:
\begin{theorem}[{\cite[Thm. 1]{mazur:eisenstein}}]
    Let $N$ be prime and $C_N$ be the cuspidal subgroup of $J_0(N)$. Then
    \[
        J_0(N)(\QQ)_\tor = C_N(\QQ).
    \]
\end{theorem}
A natural generalization is to allow $N$ to be composite. This is the
Generalized Ogg Conjecture.
\begin{conjecture}[Generalized Ogg Conjecture]
    Let $C_N$ be the cuspidal subgroup of $J_0(N)$. Then
    \[
        J_0(N)(\QQ)_\tor = C_N(\QQ).
    \]
\end{conjecture}
There been some progress towards this conjecture, particularly, away from $2$
and $3$.
\begin{enumerate}
    \item 
        When $N$ is prime, $J_0(N)(\QQ)_\tor =C(N)(\QQ)$.~\cite[Thm.
        1]{mazur:eisenstein}
    \item
        When $p\geq 5$ is a prime and $r$ a positive integer,
        \[
            J_0(p^r)(\QQ)[q^\infty] = C(p^r)(\QQ)[q^\infty]
        \]
        for any prime $q\nmid 6p$. If $r=2$, the result holds for any prime
        $q\nmid 2p$.~\cite[Thm. 4]{ling:cuspidal_pr}
    \item
        When $N$ is a positive squarefree integer,
        \[
            J_0(N)(\QQ)[q^\infty]=C(N)(\QQ)[q^\infty]
        \]
        for $q\nmid 6$.~\cite[Thm. 3.6.2]{ohta:rational_torsion_2}
    \item
        When $p>3$,
        \[
            J_0(3p)(\QQ)[3^\infty] = C(3p)(\QQ)[3^\infty]
        \]
        unless $p\equiv 1 \pmod{9}$ and $3^{\frac{p-1}{3}} \equiv 1
        \pmod{9}$.~\cite{yoo:rational}
    \item
        When $N$ is any positive integer,
        \[
            J_0(N)(\QQ)[q^\infty]=0
        \]
        for $q\nmid 6N\pi(N)$, where $\pi(N) = \prod_{p\mid N}
        (p^2-1)$.~\cite[Thm. 1.2]{ren:rational}
\end{enumerate}

The original motivation of the rational torsion subgroup project is to compute
the rational torsion subgroup for $J_0(N)$ for as many $N$'s as possible. The
first $J_0(N)$ for which \sage (Version 8.7) cannot compute is $J_0(30)$ which
happens to be a product of 3 elliptic curves. The author and his advisor were
able to compute the rational torsion subgroup using the fact that $J_0(30)$ is
totally split, the fact that rational torsion subgroups of elliptic curves can
be computed, and Galois cohomology.

The goal of this project is to see how far we can push these techniques. In
particular, we are able to provably enumerate the set of totally split $J_0(N)$
(Theorem~\ref{thm:finite_totally_split}). There are 71 nontrivial $N$'s for
which $J_0(N)$ is totally split. Of these 71 abelian varieties, there are 46
$N$'s for which $J_0(N)$ has algebraic rank 0. We will be able to verify the
Generalized Ogg Conjecture for all but $9$ abelian varieties. For those $9$
abelian varieties, we are able to bound the discrepancy
$[J_0(N)(\QQ)_\tor:C_N(\QQ)]$ by a power of 2.


\section{Rational isogeny class}%
\label{sec:rational_isogeny_class}

By Faltings' isogeny theorem, the rational isogeny class of any abelian variety
is finite. We would now like to be able to  enumerate the rational isogeny
class of any modular abelian variety (Defintion~\ref{defn:modabvar}) within our
computational framework (Chapter~\ref{chap:algorithms}). In particular, if $A$
is a modular abelian variety, we wish to give a set of finite $G_\QQ$-subgroups
of $A(\QQbar)$, $\{M_1,\ldots,M_r\}$ such that $\{A/M_1,\ldots,A/M_r\}$ is a
set of pairwise non-isomorphic abelian varieties.

The goal of enumerating the rational isogeny class of any modular abelian
variety seems hopelessly difficult. We will need to take on a few hypothesis.
In particular, when $A$ is a simple abelian subvariety of $J_0(N)$ with
integrally closed Hecke algebra $\TT_A\subseteq \End(A)$, Frank Calegari had an
idea to bound the images of isogenies supported only on the non-Eisenstein
primes of odd-residue characteristic by the class group of $\TT_A$. So we will
take on these hypothesis and attempt to enumerate the odd-degree isogeny class
of $A$ when $\TT_A$ is integrally closed. From our computational experiments,
when $\TT_A$ is integrally closed, the class group is often trivial so, in
these cases, it remains the enumerate the Eisenstein isogenies. We then adapt
the work of Krzysztof Klosin and Mihran Papikian to enumerate the isogenies
supported on the Eisenstein primes of odd-residue characteristic.
